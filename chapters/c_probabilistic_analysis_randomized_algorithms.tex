\startcomponent c_probabilistic_analysis_randomized_algorithms

\chapter{Probabilistic Analysis and Randomized Algorithms}

\section{The hiring problem}

\startEXERCISE
證明:假設在 HIRE-ASSISTANT 算法的第 4 行中,
我們總能確定哪一個應聘者更佳,
則意味着我們知道應聘者排名的全部次序。

\CLRSH{HIRE-ASSISTANT(n)}
\startCLRS
best = 0	// candidate 0 is a least-qualified dummy candidate
for i = 1 to n
	interview candidate i
	if candidate i is better than candidate best
		best = i
		hire candidate i
\stopCLRS
\stopEXERCISE
\startANSWER
全序是一種包含了所有關系的偏序\m{\forall a,b \in A: a R b \text{或} b R a}。
一個關系若是偏序關系,他必須是自反的、反對稱的以及可傳遞的。

令這個關系爲“同樣好或更好”:
\startigBase[1]
\item 自反:顯然,任何人與自身相比都滿足次關系;
\item 反對稱:如果 \m{A} 比 \m{B} 好,那麼 \m{B} 不會比 \m{A} 好;
\item 可傳遞:如果 \m{A} 比 \m{B} 好, \m{B} 又比 \m{C} 好,那麼 \m{A} 比 \m{C} 好。
\stopigBase
到此可知,此關系是偏序關系。

又由於我們假設可以比較任意兩個應聘者,也就說包含了所有關系,即全序。
\stopANSWER

\startEXERCISE \DIFFICULT
利用 \ALGO{RANDOM(0,1)} 實現過程 \ALGO{RANDOM(a,b)}。
作爲 \m{a} 和 \m{b} 的函數,其運行時間是多少?
\stopEXERCISE
\startANSWER
令 \m{n=b-a},算法爲:
\startigBase[n]
\item 找到最小整數 \m{c},使得 \m{2^c\ge n} (\m{c=\lceil\ln{n}\rceil});
\item 調用 \ALGO{RANDOM(0,1)} \m{c} 次,得到一個 \m{c} 比特的二進制數 \m{r};
\item 如果 \m{r>n},跳轉到上一步;
\item 否則返回 \m{a+r}。
\stopigBase
這會在範圍 \m{[a, b]} 內均勻的生成一個隨機數。
但是,第二步可能會重復多次。
有 \m{p=n/2^c} 概率不會重復第二步。
由幾何分布可知有 \m{1/p} 的概率會重復第二步,即 \m{2^c/n}。
由於每次執行第二步,需要調用 \ALGO{RANDOM(0,1)} \m{c} 次,運行時間爲:
\startformula
O(\frac{2^c}{n} c) = O(\frac{\ln(b-a)2^{\ln(b-a)}}{b-a}) = O(\ln(b-a))
\stopformula
\stopANSWER

\startEXERCISE \DIFFICULT
假設你希望以各爲 1/2 的概率輸出 0 和 1。
已有一個可以輸出 0 或 1 的過程 \ALGO{BIASED-RANDOM} 可用。
此過程輸出 1 的概率爲 \m{p},輸出 0 的概率爲 \m{1-p},
其中 \m{0<p<1},但 \m{p} 的值未知。
請給出一個利用 \ALGO{BIASED-RANDOM} 作爲子程式的算法,返回一個無偏的結果,
即各以 1/2 的概率返回 0 和 1。
作爲 \m{p} 的函數,此算法的運行時間爲多少?
\stopEXERCISE
\startANSWER
簡單!
\startigBase[n]
\item 生成兩個隨機數 \m{x} 和 \m{y};
\item 如果這兩個數不同,則返回 \m{x};
\item 否則,重復第一步。
\stopigBase
\m{x=0, y=1} 和 \m{x=1, y=0} 的概率相同。
第二步直接返回的概率是 \m{2pq}。
重試次數的期望值是 \m{1/(2pq)},運行時間爲 \m{O(\frac{1}{p(1-p)})}。
\stopANSWER

\section{Indicatorrandom variables}

\startEXERCISE
在 \ALGO{HIRE-ASSISTANT} 中,假設應聘者以隨機順序出現,
你只僱傭一次的概率是多少?正好僱傭 \m{n} 次的概率是多少?
\stopEXERCISE
\startANSWER
當最好的應聘者出現在第一個時,我們只需僱傭一次,其概率爲 \m{1/n} (一共有 \m{n!} 種全排列,而最好的出現在第一個的全排列共 \m{(n-1)!} 種)。
而正好僱傭 \m{n} 次,則需應聘者必須以增序出現,概率爲 \m{1/n!}。
\stopANSWER

\startEXERCISE
在 \ALGO{HIRE-ASSISTANT} 中,假設應聘者以隨機順序出現,
你正好僱傭兩次的概率是多少?
\stopEXERCISE
\startANSWER
如果正好僱傭了兩次,那麼假設第一個僱傭的人級別爲 \m{i},
第二次僱傭的人級別肯定爲 \m{n},且 \m{i < n}。

比第一個僱傭的人級別高的應聘者有 \m{n-i} 個,
這些人中級別爲 \m{n} 的那個排在最前面,其概率爲 \m{1/(n-i)}
(我們可以忽略其他應聘者,他們不影響概率)。
因此在第一個應聘者級別爲 \m{i},則正好僱傭兩次的概率爲:
\startformula
\Pr{T_i} = \frac{1}{n}\frac{1}{n-i}
\stopformula
其中第一部分反映的是級別爲 \m{i} 的應聘者排在第一個的概率。

正好僱傭兩次的概率爲:
\startformula\startmathalignment
\NC \Pr{T} \NC= \sum_{i=1}^{n-1}\Pr{T_i} \NR
\NC        \NC= \sum_{i=1}^{n-1}\frac{1}{n}\frac{1}{n-i} \NR
\NC        \NC= \frac{1}{n} \sum_{i=1}^{n-1}\frac{1}{i} \NR
\NC        \NC= \frac{1}{n} (\lg(n-1) + O(1)) \NR
\stopmathalignment\stopformula
\stopANSWER

\startEXERCISE
利用指示器隨機變量計算擲 \m{n} 個骰子之和的期望值。
\stopEXERCISE
\startANSWER
如果是單個骰子,期望值 \m{X_i} 爲:
\startformula\startmathalignment
\NC E[X_k] \NC= \sum_{i=0}^6 i \Pr{X_k = i} \NR
\NC        \NC= \frac{1 + 2 + 3 + 4 + 5 + 6}{6} \NR
\NC        \NC= \frac{21}{6} \NR
\NC        \NC= 3.5 \NR
\stopmathalignment\stopformula

對於多個骰子,則:
\startformula\startmathalignment
\NC E[X] \NC= E[\sum_{i=1}^nX_i] \NR
\NC      \NC= \sum_{i=1}^n E[X_i] \NR
\NC      \NC= \sum_{i=1}^n 3.5 \NR
\NC      \NC= 3.5 \cdot n \NR
\stopmathalignment\stopformula
\stopANSWER

\startEXERCISE
利用指示器隨機變量來求解如下{\EMP 帽子核對問題}(hat-check problem):
\m{n} 位顧客,每個人都將自己的帽子交給服務生。
服務生以隨機次序將帽子歸還給顧客。
請問拿到自己帽子的顧客數目的期望值是多少?
\stopEXERCISE
\startANSWER
每位顧客拿回自己帽子的概率都是 \m{1/n}。
記 \m{X_i} 爲第 \m{i} 個顧客拿回自己帽子的事件。
因此:
\startformula
E[X] = E[X_1 + X_2 + \ldots + X_n]
         = \sum_{i=1}^n E[X_i]
         = \sum_{i=1}^n \frac{1}{n}
         = 1
\stopformula
\stopANSWER

\startEXERCISE
設 \m{A[1..n]} 是由 \m{n} 個不同的數構成的數列。
如果 \m{i < j} 且 \m{A[i] > A[j]},則稱 \m{(i,j)} 爲 \m{A} 的一個{\EMP 逆序對}(inversion)。
(參見\refproblem{inversion} 中更多關於逆序對的例子)
假設 \m{A} 的元素構成 \m{\langle1,2,\ldots,n\rangle} 上的一個均勻隨機排列,
請用指示器隨機變量來計算其中逆序對的數目期望值。
\stopEXERCISE
\startANSWER
記 \m{X_{ij}} 爲 \m{i} 和 \m{j} 爲逆序對的情況。
每個逆序對發生的概率爲 \m{1/2},
因爲有 \m{\binom{n}{2}} 種可能,而大的那個在前面的概率爲 \m{\frac{1}{n(n-1)}}。
兩式相乘得 \m{1/2}。由此可得 \m{E[X_{ij}] = 1/2}。
\startformula\startmathalignment
\NC E[X] \NC= \sum_{i=1}^{n-1} \sum_{j=i+1}^n E[X_{ij}] \NR
\NC      \NC= \sum_{i=1}^{n-1} \sum_{j=i+1}^n \frac{1}{2} \NR
\NC      \NC= \frac{1}{2} \sum_{i=1}^{n-1} \sum_{j=i+1}^n 1 \NR
\NC      \NC= \frac{1}{2} \sum_{i=1}^{n-1} (n-i) \NR
\NC      \NC= \frac{1}{2} \sum_{i=1}^{n-1} i \NR
\NC      \NC= \frac{n(n-1)}{4} \NR
\NC      \NC= \binom{n}{2}/2 \NR
\stopmathalignment\stopformula
\stopANSWER

\section{Randomized algorithms}

\startEXERCISE
Marceau 教授不同意引理 5.5 證明中所用的循環不變式。
他對第一次迭代之前循環不變式是否爲真提出質疑。
他的理由是,我們可以很容易宣稱一個空數列不包含 0 排列,
從而第一次迭代之前循環不變式無效。
請重寫過程 \ALGO{RANDOMIZE-IN-PLACE},
使得循環不變式適用於第一次迭代之前的非空子數列,
並爲你的過程修改引理 5.5 的證明。
\stopEXERCISE
\startANSWER
變通一下,其實我們可以在循環之前,隨機選擇一個元素來替代第一個。
這樣不變式對於 1 排列是有效的。
\stopANSWER

\startEXERCISE
Kelp 教授決定寫一個過程來隨機產生除恆等排列(identity permutation)外的任意排列。
他提出了如下過程:

\CLRSH{PERMUTE-WITHOUT-IDENTITY(A)}
\startCLRS
n = A.length
for i = 1 to n - 1
	swap A[i] with A[RANDOM(i + 1, n)]
\stopCLRS
這段代碼實現了 Kelp 教授的意圖嗎?
\stopEXERCISE
\startANSWER
不能。因爲這個過程改變了所有元素的位置。
我們無法得到恆等排列,但同時也無法使得任何一個元素處於原有位置。
\stopANSWER

\startEXERCISE
假設我們不是將元素 \m{A[i]} 與子數列 \m{A[i..n]} 中的一個隨機元素交換,
而是將他與數列任意位置上的隨機元素交換:

\CLRSH{PERMUTE-WITH-ALL(A)}
\startCLRS
n = A.length
for i = 1 to n
	swap A[i] with A[RANDOM(1, n)]
\stopCLRS
這段代碼會產生一個均勻隨機排列嗎?爲什麼?
\stopEXERCISE
\startANSWER
不會。直覺上來講,有 \m{n^n} 種不同方式,但只有 \m{n!} 種組合。
由於 \m{n!} 無法整除 \m{n^n},不可能是均勻分布
(爲什麼無法整除?這就是直覺。對於 \m{n>2}, \m{n-1} 可以整除 \m{n!},但卻不能整除 \m{n^n})。
參見 \goto{Coding Horror}[url(http://blog.codinghorror.com/the-danger-of-naivete/)]。
\stopANSWER

\startEXERCISE
Armstrong 教授建議用下面的過程來產生一個均勻隨機排列:

\CLRSH{PERMUTE-BY-CYCLIC(A)}
\startCLRS
n = A.length
let B[1..n] be a new array
offset = RANDOM(1, n)
for i = 1 to n
	dest = i + offset
	if dest > n
		dest = dest - n
	B[dest] = A[i]
return B
\stopCLRS

請說明每個元素 \m{A[i]} 出現在 \m{B} 中任何特定位置的概率是 \m{1/n}。
然後通過說明排列結果不是均勻隨機排列,證明 Armstrong 教授錯了。
\stopEXERCISE
\startANSWER
如果 \m{j \equiv \text{offset} + i \pmod{n}},則 \m{A[i]} 就會出現在 \m{B[j]} 處,其概率爲 \m{1/n}。
但他不能生成所有排列,他所生成的排列都是數列 \m{A} 循環右移得到。
\stopANSWER

\startEXERCISE \DIFFICULT
證明:在過程 \ALGO{PERMUTE-BY-SORTING} 的數列 \m{P} 中,所有元素都唯一的概率至少是 \m{1 - 1/n}。
\stopEXERCISE
\startANSWER
令 \m{\Pr\{j\}} 爲索引爲 \m{j} 的元素唯一的概率。
如果有 \m{n^3} 個元素,則 \m{\Pr\{j\} = 1 - \frac{j-1}{n^3}}。
\startformula\startmathalignment
\NC \Pr\{1 \cap 2 \cap 3 \cap \ldots\}
       \NC= \Pr\{1\} \cdot \Pr\{2 | 1\} \cdot \Pr\{3 | 1 \cap 2\} \cdots \NR
\NC    \NC= 1 (1 - \frac{1}{n^3})
            (1 - \frac{2}{n^3})
            (1 - \frac{3}{n^3})
            \cdots \NR
\NC    \NC\ge 1 (1 - \frac{n}{n^3})
            (1 - \frac{n}{n^3})
            (1 - \frac{n}{n^3})
            \cdots \NR
\NC    \NC\ge (1 - \frac{1}{n^2})^n \NR
\NC    \NC\ge 1 - \frac{1}{n} \NR
\stopmathalignment\stopformula
你問最後一步如何推導出來的? \m{(1-x)^n \ge 1 - nx}。
\stopANSWER

\startEXERCISE
請解釋如何實現算法 \ALGO{PERMUTE-BY-SORTING},
以處理兩個或更多優先級相同的情形。
也就是說,即使有兩個或更多優先級相同,你的算法也應該產生一個均勻隨機排列。
\stopEXERCISE
\startANSWER
愚蠢的算法需要用愚蠢的方式解決。只要重新生成優先級重試就可以了。
\stopANSWER

\startEXERCISE
假設我們希望創建集合 \m{\{1,2,3,\ldots,n\}} 的一個{\EMP 隨機樣本(random sample)},
即一個具有 \m{m} 個元素的子集 \m{S},其中 \m{0\le m\le n},
使得以同樣的概率創建每個 \m{m} 子集。
一種方法是對 \m{i = 1,2,\ldots,n} 設 \m{A[i] = i},
調用 \ALGO{RANDOMIZE-IN-PLACE(A)},
然後取最前面的 \m{m} 個數列元素。
這種方法會調用 \m{n} 次過程 \ALGO{RANDOM}。
如果 \m{n} 比 \m{m} 大的多,
只調用 \ALGO{RANDOM} 少量次數,就能創建一個隨機樣本。
請說明下面遞迴過程返回 \m{\{1,2,3,\ldots,n\}} 的一個隨機 \m{m} 子集 \m{S},
其中每個 \m{m} 子集是等概率的,
然後只調用 \m{m} 次 \ALGO{RANDOM}。

\CLRSH{RANDOM-SAMPLE(m,n)}
\startCLRS
if m == 0
	return ∅
else S = RANDOM-SAMPLE(m - 1, n - 1)
	i = RANDOM(1, n)
	if i ∈ S
		S = S ∪ {n}
	else S = S ∪ {i}
	return S
\stopCLRS
\stopEXERCISE
\startANSWER
每種組合的概率應爲 \m{1/\binom{n}{m}}。
我們來證明下面的命題:

\ALGO{RANDOM-SAMPLE(m,n)} 返回均勻分布的組合。

在 \m{m} 上進行歸納。 \m{m} 爲 \m{1} 或 \m{0} 時顯然是成立的。
現在假定 \m{m-1} 時命題成立,看看 \m{m} 會發生什麼。

遞迴調用 \m{m - 1} 返回的樣本時均勻分布的。
然後有兩種情況,新的 \m{m} 子集是否包含 \m{n}。

如果包含 \m{n} (概率: \m{m/n}),則其概率爲:
\startformula
\frac{m}{n}\frac{1}{\binom{n-1}{m-1}} = 1/\binom{n}{m}
\stopformula

如果不包含 \m{n} (概率: \m{(n-m)/n}),則其概率爲:
\startformula
\frac{n-m}{n}\frac{1}{\binom{n-1}{m}} = 1/\binom{n}{m}
\stopformula
\stopANSWER

\section{Probabilistic analysis and further uses of indicator random variables}

\startEXERCISE
屋子裏至少要有多少人,才能使得某人與你生日相同的概率不小於 \m{1/2}?
至少要有多少人,才能使得至少兩人生日爲 7 月 4 日的概率大於 \m{1/2}?
\stopEXERCISE
\startANSWER
任一人生日與己不同的概率爲 \m{(n-1)/n},
則 \m{k} 個人生日都與己不同的概率爲 \m{(\frac{n-1}{n})^k}。
利用互補事件求 \m{k}:
\startformula\startmathalignment[n=1]
1 - (\frac{n-1}{k})^k \ge \frac{1}{2} \NR
(\frac{n-1}{k})^k \le \frac{1}{2} \NR
k \lg(\frac{n-1}{n}) \ge \lg\frac{1}{2} \NR
k = \frac{\log(1/2)}{\log(364/365)} \approx 263 \NR
\stopmathalignment\stopformula

另外一個問題:
\startformula\startmathalignment
\NC \Pr\{\text{至少兩人生日爲 7 月 4 日}\} \NC=
        1 - \Pr\{\text{只有 1 人生日爲 7 月 4 日}\} - Pr\{\text{沒有人生日爲 7 月 4 日}\} \NR
\NC \NC= 1 - \frac{k}{n}(\frac{n-1}{n})^{k-1} - (\frac{n-1}{n})^k \NR
\NC \NC= 1 - (\frac{n-1}{n})^{k-1}(\frac{n+k-1}{n}) \NR
\stopmathalignment\stopformula
計算可得 \m{k} 爲 115。
\stopANSWER

\startEXERCISE
假設我們將球投到 \m{b} 個箱子裏,直到某個箱子中有兩個球。
每一次投擲都是獨立的,並且每個球落入任一箱子的機會均等。
請問投擲次數的期望值是多少?
\stopEXERCISE
\startANSWER
本質還是生日問題,更多討論參見 \goto{wikipedia}[url(http://en.wikipedia.org/wiki/Birthday_problem#Average_number_of_people)]。
\stopANSWER

\startEXERCISE \DIFFICULT
生日悖論的分析中,要求各人生日彼此獨立是否很重要?
或者,是否只要兩兩成對獨立就足夠了?
證明你的答案。
\stopEXERCISE
\startANSWER
成對獨立足夠了。對於(5.6)之後的推導,有此即可。
\stopANSWER

\startEXERCISE \DIFFICULT
一次聚會需要邀請多少人,才能讓其中 3 人的生日很可能相同?
\stopEXERCISE
\startANSWER
感覺不大對勁:(

令 \m{X_{ijk}} 爲 \m{i}、 \m{j}、 \m{k} 三人生日相同時間的指示器隨機變量,
其概率爲 \m{1/n^2}。則:
\startformula\startmathalignment
\NC E[X] \NC= \sum_{i=1}^n\sum_{j=i+1}^n\sum_{k=j+1}^nX_{ijk} \NR
\NC      \NC= \sum_{i=1}^n\sum_{j=i+1}^n\sum_{k=j+1}^n\frac{1}{n^2} \NR
\NC      \NC= \binom{n}{3}\frac{1}{n^2} \NR
\NC      \NC= \frac{k(k-1)(k-2)}{6n^2} \NR
\stopmathalignment\stopformula
解得 \m{k = 94}。
\stopANSWER

\startEXERCISE \DIFFICULT
一個長度爲 \m{k} 的字串,其中所有字符均選自一個元素個數爲 \m{n} 的集合,
那麼此字串構成一個 \m{k} 排列的概率是多少?
此問題與生日悖論有何關聯?
\stopEXERCISE
\startANSWER
\startformula
\Pr\{k\text{-perm in }n\} = 1 \cdot
                                 \frac{n-1}{n} \cdot
                                 \frac{n-2}{n} \cdots
                                 \frac{n-k+1}{n}
      = \frac{(n-1)!}{(n-k)!n^k}
\stopformula
這是生日問題的互補事件,即 \m{k} 個人生日各不相同。
\stopANSWER

\startEXERCISE \DIFFICULT
假設將 \m{n} 個球投入 \m{n} 個箱子裏,其中每次投球相互獨立,
並且每個球落入任一箱子的機會均等。
空箱子的數目期望值是多少?
正好有一個球的箱子數目期望值是多少?
\stopEXERCISE
\startANSWER
當 \m{n} 足夠大時,兩個答案都漸進於 \m{n/e}。
首先來看空箱子的數目:

令 \m{X_i} 代表的事件爲:第 \m{i} 個箱子爲空:
\startformula
\Pr\{X_i\} = \left(\frac{n-1}{n}\right)^n
                = \left(1 - \frac{1}{n}\right)^n
                \approx \frac{1}{e}
\stopformula
其期望值爲:
\startformula
E[X] = \sum_{i=1}^n E[X_i] = \frac{n}{e}
\stopformula

箱子裏只有一個球的情況類似,其概率爲:
\startformula
\Pr\{Y_i\} = n\frac{1}{n}\left(\frac{n-1}{n}\right)^{n-1}
                = \left(\frac{n-1}{n}\right)^{n-1} \approx \frac{1}{e}
\stopformula
期望值一樣。

參見 \goto{stackexchange}[url(http://math.stackexchange.com/questions/545920/expectation-of-throwing-n-balls-into-n-bins)]。
\stopANSWER

%e5.4-7
\startEXERCISE \DIFFICULT
为使特征序列长度的下界变得更精确,
请说明在 \m{n} 次硬币的公平抛掷中,
不出现比 \m{\lg{n} - 2\lg\lg{n}} 更长的连续正面特征序列的概率小于 \m{1/n}。
\stopEXERCISE

\startANSWER
\TODO{需要證明}
\stopANSWER

\startPROBLEM(Probabilstic counting)
利用一個 \m{b} 位的計數器,
我們一般只能計數到 \m{2^b -1}。
而用 R.Morris 的{\EMP 概率記數法},
我們可以計數到一個大得多的值,
代價是精度有所損失。

對 \m{i=0,1,\ldots,2^b-1},令計數器的值 \m{i} 表示 \m{n_i} 的計數,
其中 \m{n_i} 構成了一個非負的遞增序列。
假設計數器初值爲 \m{0},表示計數 \m{n_0 = 0}。
INCREMENT 運算單元工作在一個計數器上,
他以概率的方式包含值 \m{i}。
如果 \m{i = 2^b-1},則該運算單元報告溢出錯誤;
否則,該運算單元以概率 \m{1/(n_{i+1}-n_i)} 把計數器增加 \m{1},
以概率 \m{1-1/(n_{i+1}-n_i)} 保持計數器不變。

對所有的 \m{i\ge 0},若選擇 \m{n_i = i},此計數器就退化爲一個普通的計數器。
若選擇 \m{n_i=2^{i-1}(i>0)},或者 \m{n_i=F_i}(第 \m{i} 個 Fibonacci 數,參見\refsection{notationfunction}),
則會出現更多有趣的情形。

對於這個問題,假設 \m{n_{2^b-1}} 已足夠大,發生溢出錯誤的概率可以忽略。

\startigBase[a]
\item 請說明在執行 \m{n} 次 INCREMENT 運算後,計數器所表示的數期望值正好是 \m{n};

\startANSWER
假設第 \m{j} 次 INCREMENT 前,計數器的值爲 \m{i},表示 \m{n_i}。
如果計數器增加,那麼其值會增大 \m{n_{i+1} - n_i}。
其概率爲 \m{1/(n_{i+1} - n_i)},並且:
\startformula\startmathalignment
\NC E[X_j] \NC= 0 \cdot \Pr\{\text{stays same}\} + 1 \cdot \Pr\{\text{increases}\} \NR
\NC        \NC= 0 \cdot \left(1 - \frac{1}{n_{i+1} - n_i}\right) +
                1 \cdot \left((n_{i+1} - n_i) \cdot \frac{1}{n_{i+1} - n_i}\right) \NR
\NC        \NC= 1 \NR
\stopmathalignment\stopformula
這是每次 INCREMENT 的期望值,由於有 \m{n} 次 INCREMNT,其期望值爲 \m{n}。
\stopANSWER

\item 分析計數器表示的計數的方差依賴於 \m{n_i} 序列。
我們來看一個簡單情形:對所有 \m{i\ge 0}, \m{n_i = 100i}。
在執行了 \m{n} 次 INCREMENT 運算後,請估計計數器所表示數的方差。

\startANSWER
單次 INCREMENT 運算的協方差:
\startformula\startmathalignment
\NC Var[X_j] \NC= E[X_j^2] - E^2[X_j] \NR
\NC           \NC= \left(0^2 \cdot \frac{99}{100} + 100^2 \frac{1}{100}\right) - 1 \NR
\NC           \NC= 99 \NR
\stopmathalignment\stopformula

\m{n} 次運算的協方差:
\startformula
Var[X] = Var[X_1 + X_2 + \ldots + X_n] = \sum_{i=1}^n Var[X_i] = 99n
\stopformula
\stopANSWER
\stopigBase
\stopPROBLEM

\startPROBLEM(Searching an unsorted array)
本題將分析三個算法,他們在一個包含 \m{n} 個元素的無序數列 \m{A} 中查找一個值 \m{x}。

考慮如下的隨機策略:隨機挑選 \m{A} 中的一個下標 \m{i}。
如果 \m{A[i] = x},則終止;否則,繼續挑選 \m{A} 中的一個新的隨機下標。
重復隨機挑選下標,知道找到一個下標 \m{j},使得 \m{A[j] = x},
或者知道我們已檢查過 \m{A} 中的所有元素。
注意,我們每次都是從全部下標的結合中挑選,於是可能會不止一次地檢查某個元素。
\startigBase[a]
\item 請寫出過程 \ALGO{RANDOM-SEARCH} 的僞碼實現上述策略。
確保當 \m{A} 中所有下標都被挑選過時,你的算法應停止。

\startANSWER
\CLRSH{RANDOM-SEARCH(x, A, n)}
\startCLRS
v = ∅
while |∅| ≠ n
	i = RANDOM(1, n)
	if A[i] = x
		return i
	else:
		v = v ∩ i
return ␀
\stopCLRS
其中可以多種方式實現 \m{v}——如散列表、樹或位圖。位圖的時間、空間性能應當都是最好的。
\stopANSWER

\item 假定恰好有一個下標 \m{i} 使得 \m{A[i] = x}。
在我們找到 \m{x} 和 \ALGO{RANDOM-SEARCH} 結束之前,
必須挑選 \m{A} 下標的數目期望值是多少?

\startANSWER
參考 \goto{Bernoulli trials}[url(http://en.wikipedia.org/wiki/Bernoulli_trial)],期望值爲 \m{n}。
\stopANSWER

\item 假設有 \m{k\ge 1} 個下標 \m{i} 使得 \m{A[i] = x},
推廣你對上一項的解答。
在找到 \m{x} 或 \ALGO{RANDOM-SEARCH} 結束之前,
必須挑選 \m{A} 的下標的數目期望值是多少?
你的答案應該是 \m{n} 和 \m{k} 的函數。

\startANSWER
答案爲 \m{n/k}。
\stopANSWER

\item 假設沒有下標 \m{i} 使得 \m{A[i] = x},
在檢查完 \m{A} 的所有元素或 \ALGO{RANDOM-SEARCH} 結束之前,
我們必須挑選 \m{A} 的下標的數目期望值是多少?

\startANSWER
答案爲 \m{n(\ln{n} + \O(1))},參見節 5.4.2。
\stopANSWER

\stopigBase

現在考慮一個確定性的線性查找算法,我們稱之爲 \ALGO{DETERMINISTIC-SEARCH}。
具體地說,這個算法在 \m{A} 中順序查找 \m{x},
考慮 \m{A[1]}、 \m{A[2]}、 \m{A[3]}、 \m{\ldots}、 \m{A[n]},
直到找到 \m{A[i] = x},或者到達數列的末尾。
假設輸入數列的所有排列都是等可能的。
\startigBase[a,continue]
\item 假設恰好有一個下標 \m{i} 使得 \m{A[i] = x}。
\ALGO{DETERMINISTIC-SEARCH} 的平均運行時間是多少?
最壞運行時間又是多少?

\startANSWER
最壞運行時間爲 \m{n}。平均運行時間爲 \m{(n+1)/2}。
\stopANSWER

\item 假設有 \m{k\ge 1} 個下標 \m{i} 使得 \m{A[i] = x},推廣你對上一項的解答。
\ALGO{DETERMINISTIC-SEARCH} 的平均運行時間是多少?最壞運行時間又是多少?
你的答案應該是 \m{n} 和 \m{k} 的函數。

\startANSWER
最壞運行時間爲 \m{n-k+1}。
平均運行時間爲 \m{(n+1)/(k+1)}。
令 \m{X_i} 爲指示器隨機變量,代表第 \m{i} 個元素匹配,則 \m{\Pr\{X_i\} = 1/(k+1)}。
令 \m{Y} 爲指示器隨機變量,代表在前 \m{n-k+1} 個元素中找到了匹配項(\m{\Pr{\{Y\}}=1})。
因此:
\startformula\startmathalignment
\NC E[X] \NC= E[X_1 + X_2 + \ldots + X_{n-k} + Y] \NR
\NC      \NC= 1 + \sum_{i=1}^{n-k}E[X_i] \NR
\NC      \NC= 1 + \frac{n-k}{k+1} \NR
\NC      \NC= \frac{n+1}{k+1} \NR
\stopmathalignment\stopformula
\stopANSWER

\item 假設沒有下標 \m{i} 使得 \m{A[i] = x},
\ALGO{DETERMINISTIC-SEARCH} 的平均運行時間是多少?最壞運行時間又是多少?

\startANSWER
兩種情況都是 \m{n}。
\stopANSWER
\stopigBase

最後,考慮一個隨機算法 \ALGO{SCRAMBLE-SEARCH},
他縣將輸入數列隨機變換排列,然後在變換後的數列上,運行上面的確定性線性查找算法。
\startigBase[a,continue]
\item 設 \m{k} 是滿足 \m{A[i] = x} 的下標的數目,
請給出在 \m{k=0} 和 \m{k=1} 兩種情況下,
算法 \ALGO{SCRAMBLE-SEARCH} 最壞情形的運行時間和運行時間期望值。
推廣你的解答以處理 \m{k\ge 1} 的情況。

\startANSWER
與 \ALGO{DETERMINISTIC-SEARCH} 一樣,只是將“平均情況”換成了“期望”。
\stopANSWER

\item 你將會使用 3 種查找算法中的哪一個?解釋你的答案。

\startANSWER
當然是 \ALGO{DETERMINISTIC-SEARCH}。
雖然 \ALGO{SCRAMBLE-SEARCH} 可以有更好的期望值,
但是排列變換需要額外的時間,而這個時間是線性的操作。
在進行排列變換的時間李,我們可能已經掃描完了整個數列並得到了結果。
\stopANSWER

\stopigBase

\stopPROBLEM

\stopcomponent
