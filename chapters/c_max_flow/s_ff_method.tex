\startsection[
  title={The Ford-Fulkerson method},
]

%e26.2-1
\startEXERCISE
證明式 26.6 中的和值等於式 26.7 中的和值。
\stopEXERCISE

\startANSWER
\startformula\startmathalignment
\NC |f\uparrow f'| \NC = \sum_{v\in V_1}f(s,v) - \sum_{v\in V_2}f(v,s)
  + \sum_{v\in V_1\cup V_2}f'(s,v) - \sum_{v\in V_1\cup V_2}f'(v,s) \NR
\NC \NC = \sum_{v\in V}f(s,v) - \sum_{v\in V}f(v,s)
  + \sum_{v\in V}f'(s,v) - \sum_{v\in V}f'(v,s) \NR
\NC \NC = |f| + |f'| \NR
\stopmathalignment\stopformula
\stopANSWER

%e26.2-2
\startEXERCISE
在圖 26-1(b)中,橫跨切割(\m{\{s,v_2,v_4\}}, \m{\{v_1,v_3,t\}})的流是多少?
該切割的容量又是多少?附圖 26-1:

\externalfigure[output/e26_2_2-1]
\stopEXERCISE

\startANSWER
\startformula
f(s,t) = f(s,v_1) + f(v_2,v_1) + f(v_4,v_3) + f(v_4,t) - f(v_3,v_2) = 19
\stopformula

\startformula
f(s,t) = c(s,v_1) + c(v_2,v_1) + c(v_4,v_3) + c(v_4,t) = 31
\stopformula
\stopANSWER

%e26.2-3
\startEXERCISE
在圖 26-1(a)所示的流網絡上演示 \ALGO{EDMONDS-KARP} 的執行過程。
附圖 26-1(a):

\externalfigure[output/e26_2_3-1]
\stopEXERCISE

\startANSWER
\startcombination[nx=2]
{\externalfigure[output/e26_2_3-2]}{}
{\externalfigure[output/e26_2_3-3]}{}
\stopcombination
\startcombination[nx=2]
{\externalfigure[output/e26_2_3-4]}{}
{\externalfigure[output/e26_2_3-5]}{}
\stopcombination
\startcombination[nx=2]
{\externalfigure[output/e26_2_3-6]}{}
{\externalfigure[output/e26_2_3-7]}{}
\stopcombination
\startcombination[nx=2]
{\externalfigure[output/e26_2_3-8]}{}
{}{}
\stopcombination
\stopANSWER

%e26.2-4
\startEXERCISE
在圖 26-6 的例子中,對應圖中所示最大流的最小切割是什麼?
在例子中出現的增廣路徑裏,哪一條路徑抵消了先前被傳輸的流?
\stopEXERCISE

\startANSWER
最小切割爲 \m{(\{s,v_1,v_2,v_4\},\{v_3,t\})}。
(c)中的 \m{(v_1,v_2)} 抵消了(b)中的 \m{(v_2,v_1)}。
(c)中的 \m{(v_2,v_3)} 抵消了(a)中的 \m{(v_3,v_2)}。
\stopANSWER

%e26.2-5
\startEXERCISE
在\refsection{flow_networks} 中,我們通過增加容量無限的邊,
把多源點多匯點的流網絡轉換爲單源點單匯點的流網絡。
證明:
如果原來的網絡容量是有限的,則轉換後的網絡中任何一個流的值都是有限的。
\stopEXERCISE

\startANSWER
因爲我們可以找到一個最小切割,只包含有限個有限權重的邊。
\stopANSWER

%e26.2-6
\startEXERCISE
假定在一個多源點多匯點的流網絡中,
每個源節點 \m{s_i} 生產出恰好 \m{p_i} 個單位的流,
因此, \m{\sum_{v\in V}f(s_i,v)=p_i}。
假定每個匯點 \m{t_j} 恰好消費 \m{q_j} 個單位的流,
因此 \m{\sum_{v\in V}f(v,t_j)=q_j},其中 \m{\sum_{i}p_i = \sum_{j}q_j}。
說明如何在此網絡中尋找最大流 \m{f} (可以先將其轉換爲一個單源點單匯點的流網絡)。
\stopEXERCISE

\startANSWER
新添加源點 \m{s} 和匯點 \m{t}。
邊 \m{(s,s_i)} 的容量爲 \m{c(s,s_i) = p_i}。
邊 \m{(t_j,t)} 的容量爲 \m{c(t_j,t) = q_j}。
\stopANSWER

%e26.2-7
\startEXERCISE
證明引理 26.2。附引理 26.2:

設 \m{G=(V,E)} 爲一個流網絡,
 \m{f} 是圖 \m{G} 的一個流,
設 \m{p} 爲殘存網路 \m{G_f} 中的一條增光路徑。
定義一個函數 \m{f_p: V\times V\rightarrow R} 如下:
\startformula
f_p(u,v)=\startcases
\NC c_f(p) \NC 如果 \m{(u,v)} 在 \m{p} 上 \NR
\NC 0 \NC 否則 \NR
\stopcases
\stopformula
則 \m{f_p} 是殘存網絡 \m{G_f} 中的一個流,
其值爲 \m{|f_p| = c_f(p) > 0}。
\stopEXERCISE

\startANSWER
容量限制:由於 \m{c_f(p)} 是 \m{p} 上所有邊的最小權重,
因此滿足 \m{0\le f_p(u,v) = c_p(u,v) \le c_f(p)}。

流量守恆: \m{\sum_{v\in V}f(v,u) = c_f(p) = \sum_{v\in V}f(u,v)}。
\stopANSWER

%e26.2-8
\startEXERCISE
假定我們重新定義殘存網絡,
進制一切進入源節點 \m{s} 的邊。
算法 \ALGO{FORD-FULKERSON} 所計算出的最大流是否仍然正確?
\stopEXERCISE

\startANSWER
正確。
\stopANSWER

%e26.2-9
\startEXERCISE
假定 \m{f} 和 \m{f'} 都是流網絡 \m{G} 中的流,
計算流 \m{f\uparrow f'}。
加增後的流滿足流量守恆嗎?
滿足容量限制嗎?
\stopEXERCISE

\startANSWER
滿足流量守恆,不滿足容量限制。
\stopANSWER

%e26.2-10
\startEXERCISE
說明在流網絡 \m{G=(V,E)} 中,
如何使用一個最多包含 \m{|E|} 條增廣路徑的序列找到最大流。
(\hint 找到最大流後再確定路徑)
\stopEXERCISE

\startANSWER
先找到最大流,可以得到網絡的最小切割。
在最小切割的基礎上,每次增加一條邊,上限爲 \m{|E|}。
\stopANSWER

%e26.2-11
\startEXERCISE
無向圖的{\EMP 邊連通性}是指使圖變爲非連通圖所需要刪除的最少邊數 \m{k}。
例如,樹的邊連通性爲 \m{1},環路的邊連通性爲 \m{2}。
請說明如何在最多 \m{|V|} 個流網絡上運行最大流算法,
從而確定無向圖 \m{G=(V,E)} 的邊連通性,
這裏的每個流網絡節點數爲 \m{O(V)},邊數爲 \m{O(E)}。
\stopEXERCISE

\startANSWER
先將圖 \m{G} 轉換成帶權有向圖 \m{G'},兩圖頂點相同。
 \m{G} 中的每條邊 \m{(u,v)} 對應 \m{G'} 中的兩條邊 \m{(u,v)} 和 \m{(v,u)}。
 \m{G'} 中所有邊權重均爲 \m{1}。
在 \m{G'} 基礎上構造 \m{|V|} 個網絡流,
分別令每個頂點作爲匯點,源點可任意選取,只要不是匯點即可。
依次求這 \m{|V|} 個流網絡的最大流。
所有最大流的最小值即爲 \m{G} 的邊連通性。
\stopANSWER

%e26.2-12
\startEXERCISE
給定一個流網絡 \m{G}, \m{G} 中包含進入源節點 \m{s} 的邊。
設 \m{f} 爲網絡 \m{G} 中的一個流,
在該流中,其中一條進入源點的邊 \m{(v,s)} 有 \m{f(v,s)=1}。
證明:圖 \m{G} 中必存在另一個流 \m{f'},
滿足 \m{f'(v,s)=0},使得 \m{|f|=|f'|}。
給出一個 \m{O(E)} 時間複雜度的算法,
在給定流 \m{f} 的情況下計算 \m{f'},
這裏假定所有邊的容量都是正整數。
\stopEXERCISE

\startANSWER
在流網絡 \m{G=(V,E)} 中,對於節點 \m{v\in V-s},
如果沒有路徑 \m{s\leadsto v},則 \m{f(v,s)=0}。

令從 \m{s} 可達的節點集合爲 \m{Y},
從 \m{s} 不可達的節點集合爲 \m{Z},有:
\startformula\startmathalignment
\NC \sum_{z\in Z} \sum_{x\in V} f(x,z) \NC = \sum_{z\in Z}\sum_{x\in V} f(z,x) \qquad \text{流量守恆} \NR
\NC \sum_{z\in Z}\sum_{x\in Z}f(x,z) + \sum_{z\in Z}\sum_{x\in Y}f(x,z) \NC
  \sum_{z\in Z}\sum_{x\in Z}f(z,x) + \sum_{z\in Z}\sum_{x\in Y}f(z,x) \NR
\stopmathalignment\stopformula

根據流量守恆有 \m{\sum_{z\in Z}\sum_{x\in Z}f(x,z) = \sum_{z\in Z}\sum_{x\in Z}f(z,x)}。
又由於 \m{\sum_{z\in Z}\sum_{x\in Y} f(x,z) = 0},所以:
對於所有 \m{z\in Z} 和 \m{x\in Y},都有 \m{f(z,x) = 0}。
因此,對於所有 \m{v\in Z} 和 \m{s\in Y},都有 \m{f(v,s) = 0}。

而如果 \m{f(v,s)=1},那麼 \m{v\notin Z},即肯定存在路徑 \m{s\leadsto v}。
從而有環路 \m{s\leadsto v \leadsto s},給其中每條邊都減 \m{1},形成新的流 \m{f'},
不會影響總流量,即 \m{|f| = |f'|}。

首先搜索節點 \m{v},使得 \m{(v,s)\in E} 且 \m{f(v,s)=1},
可以通過 \ALGO{BFS} 搜索路徑 \m{s\leadsto s}。
確保路徑上所有邊 \m{(u,v)} 都有 \m{f(u,v)\ge 1}。

然後將所得路徑上所有邊 \m{(u,v)} 的 \m{f(u,v)} 都減去 \m{1}。得到 \m{f'},返回即可。
\stopANSWER

%e26.2-13
\startEXERCISE
假定我們希望找到流網絡 \m{G} 的一個最小切割,使其邊數是所有最小切割中最少的。
這裏假定 \m{G} 的所有容量都是整數。
說明如何修改 \m{G} 的容量來創建一個新的流網絡 \m{G'},
使得 \m{G'} 中的任一最小切割都是 \m{G} 中邊數最少的最小切割。
\stopEXERCISE

\startANSWER
\m{G'} 的頂點、邊與 \m{G} 的相同,但邊的容量有所區別。
對於所有 \m{(u,v)\in E},有 \m{c_{G'}(u,v) = c_{G}(u,v) + \sigma}。
其中 \m{\sigma} 是一個常數,保證最小切割的邊數最多不會超過 \m{G} 中邊數最小的最小切割。
同時 \m{\sigma} 也不能太大,因爲在減少最小切割邊數的同時可能導致切割不再是最小切割。

\startformula
\sigma = \frac{m}{2|E|}
\stopformula
其中 \m{m} 是 \m{G} 中最小切割與非最小切割的最小差異。
要證明其正確性,我們需要證明:

1) \m{G} 中邊數最少的最小切割是 \m{G'} 中的唯一最小切割。

2) \m{G} 中非最小切割不是 \m{G'} 的最小切割。

先證明(1),令 \m{(S,T),(X,Y)} 是 \m{G} 的最小切割, \m{|(S,T)|<|(X,Y)|},有:
\startformula\startmathalignment
\NC c'(S,T) \NC = c(S,T) + |(S,T)|\sigma \NR
\NC \NC < c(X,Y) + |(X,Y)|\sigma \qquad c(S,T) = c(X,Y) \text{且} |(S,T)| < |(X,Y)| \NR
\NC \NC = c'(X,Y) \NR
\stopmathalignment\stopformula

再證明(2),令 \m{(S,T)} 爲 \m{G} 的最小切割, \m{c(S,T)<c(X,Y)},
我們來證明 \m{c'(X,Y)} 不是 \m{G'} 的最小切割:
\startformula\startmathalignment
\NC c'(S,T) \NC = c(S,T) + |(S,T)|\sigma \NR
\NC \NC \le c(S,T) + |E|\sigma \qquad |(S,T)| \le |E| \NR
\NC \NC = c(S,T) + \frac{m}{2} \NR
\NC \NC \le c(X,Y) \qquad c(X,Y) - c(S,T) \ge m \NR
\NC \NC < c'(X,Y) \NR
\stopmathalignment\stopformula

\stopANSWER

\stopsection
