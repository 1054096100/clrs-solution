\startsection[
  title={Maximum bipartite matching},
]

%e26.3-1
\startEXERCISE
在圖 26-8(c)上運行算法 \ALGO{FORD-FULKERSON},
給出每次流量遞增後的殘存網絡。
將集合 \m{L} 中的節點從上至下編號 1~5,
集合 \m{R} 中的編號從上至下編號 6~9。
對於每次迭代,選擇字典次序最小的增廣路徑。
附圖 26-8(c):

\externalfigure[output/e26_3_1-1]
\stopEXERCISE

\startANSWER
\startcombination[nx=4]
{\externalfigure[output/e26_3_1-2]}{}
{\externalfigure[output/e26_3_1-3]}{}
{\externalfigure[output/e26_3_1-4]}{}
{\externalfigure[output/e26_3_1-5]}{}
\stopcombination
\startcombination[nx=4]
{\externalfigure[output/e26_3_1-6]}{}
{\externalfigure[output/e26_3_1-7]}{}
{\externalfigure[output/e26_3_1-8]}{}
{}{}
\stopcombination
\stopANSWER

%e26.3-2
\startEXERCISE
證明定理 26.10。
附定理 26.10:

如果容量函數 \m{c} 只能取整數值,
則 \ALGO{FORD-FULKERSON} 所生成的最大流 \m{f} 滿足: \m{|f|} 是整數值。
而且,對於所有節點 \m{u} 和 \m{v}, \m{f(u,v)} 的值都是整數。
\stopEXERCISE

\startANSWER
算法 \ALGO{FORD-FULKERSON} 中增廣路徑中邊的最小容量肯定是整數,
流的容量每次的變化都是整數,因此最終的最大流 \m{|f|} 是整數,且任一邊的流量都是整數。
\stopANSWER

%e26.3-3
\startEXERCISE
設 \m{G=(V,E)} 是一個二分圖,
其節點劃分爲 \m{V=L\cup R},設 \m{G'=(V',E')} 爲其對應的流網絡。
在 \ALGO{FORD-FULKERSON} 執行過程中,
對在 \m{G'} 中找出的任意增廣路徑的長度給出一個適當的上界。
\stopEXERCISE

\startANSWER
根據定義,殘存網絡 \m{G'_{f}} 的增廣路徑是 \m{s\leadsto t} 的簡單路徑。
在 \m{G} 中不存在兩端均在 \m{L} 或 \m{R} 中的邊,
 \m{G'} 和 \m{G'_f} 中也沒有這樣的邊。
只有 \m{s} 連接 \m{L} 中的節點, \m{R} 中的節點連接到 \m{t}。
 \m{G'} 中的邊均從 \m{L} 到 \m{R},
而 \m{G'_f} 中的邊則可能從 \m{R} 到 \m{L}。
所以增廣路徑只能是:
\startformula
s\rightarrow L\rightarrow R\rightarrow \ldots \rightarrow L\rightarrow R\rightarrow t
\stopformula
在 \m{L} 和 \m{R} 中來回穿梭,但每個節點只能通過一次。
總節點數爲 \m{2 + 2\times \min(|L|,|R|)}。
因此增廣路徑的長度上限爲 \m{2\times \min(|L|,|R|) + 1}。
\stopANSWER

%e26.3-4
\startEXERCISE\DIFFICULT
{\EMP 完全匹配(perfect matching)}指包含圖中所有節點的匹配。
設 \m{G=(V,E)} 是節點劃分爲 \m{V=L\cup R} 的無向二分圖,
其中 \m{|L|=|R|}。
對於任意 \m{X\subseteq V},定義 \m{X} 的{\EMP 鄰居(neighborhood)}爲:
\startformula
N(X)=\{y\in V:\exists x\in X, (x,y)\in E\}
\stopformula
即集合中的元素由 \m{X} 中的節點可達。
請證明 Hall 定理:

圖 \m{G} 中存在一個完全匹配當且僅當對於每個子集 \m{A\subseteq L},
有 \m{|A|\le |N(A)|}。
\stopEXERCISE

\startANSWER
先證明:如果存在完全匹配,則對於任一 \m{A\subseteq L},有 \m{|A|\le |N(A)|}。

根據完全匹配的定義, \m{L} 中的每個節點都連接到了 \m{R} 中的節點,且互不相同。
即 \m{|A|\le |N(A)|}。

現在看另一面:如果所有 \m{A\subseteq L},都有 \m{|A|\le |N(A)|},
則存在完全匹配。

如果 \m{|L|=|R|=1},很顯然存在完全匹配,
即 \m{G} 中只有一條邊。假定上述假設對 \m{|L|=|R|=1,2,\ldots,n-1} 均成立。
當 \m{|L|=|R|=n} 時,分兩種情況考慮:

第一種情況:對於所有 \m{A\subseteq L},都有 \m{|A|<|N(A)|}。
在 \m{L} 中任取一個節點 \m{u},然後在 \m{R} 中選取 \m{u} 的任一鄰居,
然後將這兩個點從 \m{G} 中移除。
剩下的爲圖 \m{G'},仍然能保證:對於所有 \m{A\subseteq L},都有 \m{|A|\le |N(A)|}。
由於在 \m{G'} 中, \m{|L|=|R|=n-1},根據歸納假設,有完全匹配 \m{M'},
在此基礎上添加 \m{(u,v)},即爲 \m{G} 的完全匹配。

第二種情況:至少存在一個 \m{A\subset L},滿足 \m{|A|=|N(A)|}。
根據歸納假設,在 \m{A} 和 \m{N(A)} 間存在完全匹配 \m{M_A}。
從 \m{G} 中移除 \m{A} 和 \m{N(A)}。
在剩下的 \m{G'} 中,仍然滿足:對於所有 \m{B\subseteq L - A},都有 \m{|B|\le |N(B)|}。
用反證法。假如 \m{|B|>|N(B)|},那麼 \m{|A\cup B|>|N(A\cup B)|},顯然矛盾。
因此, \m{G'} 有完全匹配 \m{M'}。
結合 \m{M_A} 和 \m{M'},就可以得到 \m{G} 的完全匹配。
\stopANSWER

%e26.3-5
\startEXERCISE\DIFFICULT
將圖 \m{G=(V,E)} 中的節點劃分爲 \m{V=L\cup R},
如果二分圖中的每個節點 \m{v} 的度數都是 \m{d},
則稱該二分圖是 {\EMP \m{d} 正則的(\m{d}-regular)}。
對於每個 \m{d} 正則的二分圖,都有 \m{|L|=|R|}。
證明:
每個 \m{d} 正則二分圖的匹配基數都是 \m{|L|}。
(\hint 證明對應的流網絡最小切割的容量爲 \m{|L|})
\stopEXERCISE

\startANSWER
\m{L} 的總出度與 \m{R} 的總入度相同,每個節點的度數又都相同,所以 \m{|L|=|R|}。

最大匹配的基數就是 \m{G'} 中最大流的值。
定義 \m{G'} 的一個流:
\startformula
f(u,v)=\startcases
\NC 1/d \MC \text{如果 \m{(u,v)\in E}} \NR
\NC 1 \MC \text{如果 \m{u=s} 或者 \m{v=t}} \NR
\NC 0 \MC \text{其他} \NR
\stopcases
\stopformula
則 \m{f} 是 \m{G'} 的一個流,其值就是完全匹配的基數。
\stopANSWER

\stopsection
