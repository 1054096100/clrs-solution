\section{Performance of quicksort}

\startEXERCISE
用代入法證明:正如本節開頭所提到的那樣,遞迴式 \m{T(n)=T(n-1)+\Theta(n)} 的解爲:
\startformula
T(n)=\Theta(n^2)
\stopformula
\stopEXERCISE

\startANSWER
令 \m{\Theta(n)} 爲 \m{c_2 n},猜測 \m{T(n)\le c_1 n^2}:
\startformula\startmathalignment
\NC T(n) \NC=   T(n-1) + c_2n \NR
\NC      \NC\le c_1(n-1)^2 + c_2n \NR
\NC      \NC=   c_1n^2 - 2c_1n + c_1 + c_2n \qquad (2c_1 > c_2, n \ge c_1/(2c_1 - c_2))\NR
\NC      \NC\le c_1n^2 \NR
\stopmathalignment\stopformula
\stopANSWER

\startEXERCISE
當數列 \m{A} 包含的元素都具有相同值時, \ALGO{QUICKSORT} 的時間復雜度時什麼?
\stopEXERCISE

\startANSWER
時間復雜度爲 \m{\Theta(n^2)},因爲 \ALGO{PARTITION} 時總有一邊爲空(參見\refexercise{same_partition})。
\stopANSWER

\startEXERCISE
證明:當數列 \m{A} 包含的元素不同,且按降序排列時, \ALGO{QUICKSORT} 的時間復雜度爲 \m{\Theta(n^2)}。
\stopEXERCISE

\startANSWER
\ALGO{PARTITION} 總是返回 \m{p},即總有一邊是空。
遞迴式仍爲 \m{T(n)=T(n-1)+\Theta(n)}。
(即使 {\EMP if} 塊內執行不到, {\EMP for} 循環的執行次數不變,仍是 \m{\Theta(n)}。)
\stopANSWER

\startEXERCISE
銀行一般會按照交易時間記錄某一賬戶的交易情況。
但是,很多人卻希望收到的銀行張但是按照支票號碼的順序排列。
這是因爲,人們通常都是按照支票號碼的順序來開出支票的,
而商人也通常都是根據支票號碼的順序兌付支票。
這一問題是將按交易時間排序的序列轉換成按支票號排序的序列,
他實質上是一個對幾乎有序的輸入序列進行排序的問題。
請證明:在這個問題上, \ALGO{INSERTION-SORT} 的性能往往要優於 \ALGO{QUICKSORT}。
\stopEXERCISE

\startANSWER
\ALGO{INSERTION-SORT} 的時間復雜度爲 \m{\Theta(n+d)},其中 \m{d} 爲逆序對的數目。
在本例中, \m{d} 的值幾乎爲 0,因此插入排序的時間幾乎是線性的。

而在基本有序的情況下,大部分 \m{PARTITION} 都會出現一邊爲空的情況。
從而使得 \ALGO{QUICKSORT} 所用時間復雜度接近 \m{\Theta(n^2)}。
\stopANSWER

\startEXERCISE
假設快速排序的每一層所做的劃分比例都是 \m{1-\alpha : \alpha},
其中 \m{0<\alpha\le 1/2} 且爲常數。
證明:在相應的遞迴樹中,葉子節點的最小深度大約是 \m{-\lg{n}/\lg{\alpha}},
最大深度大約是 \m{-\lg{n}/\lg(1-\alpha)}(無需考慮整數舍入問題)。
\stopEXERCISE

\startANSWER
樹的最小深度是 \m{n\alpha^x\le 1} 的解:
\startformula\startmathalignment[n=1]
\NC n\alpha^x \le 1 \NR
\NC \Downarrow \NR
\NC \alpha^x \le \frac 1 n \NR
\NC \Downarrow \NR
\NC x \ge \log_{\alpha}\frac{1}{n} \NR
\NC \Downarrow \NR
\NC \log_{\alpha}\frac{1}{n} = \log_{1/\alpha}
                             = \frac{\lg{n}}{\lg(1/\alpha)}
                             = - \frac{\lg{n}}{\lg{\alpha}} \NR
\stopmathalignment\stopformula

類似,最大深度爲 \m{\log_{1/(1-\alpha)}n = - \frac{\lg{n}}{\lg(1-\alpha)}}。
\stopANSWER

\startEXERCISE[exercise:const_part_probability]\DIFFICULT
證明:在一個隨機輸入數列上,對於任何常數 \m{0<\alpha\le 1/2},
 \ALGO{PARTITION} 產生比 \m{1-\alpha : \alpha} 更平衡的劃分的概率約爲 \m{1-2\alpha}。
\stopEXERCISE

\startANSWER
將元素劃分成三部分,最小的 \m{\alpha n} 個元素、最大的 \m{\alpha n} 個元素,
還有 \m{(1-2\alpha)n} 個中間值元素。
只有當 \ALGO{PARTITION} 所選的主元處於最後一個集合中時,所的劃分才比 \m{1-\alpha : \alpha} 更平衡。
其概率爲 \m{(1-2\alpha)n / n = (1-2\alpha)}。
\stopANSWER
