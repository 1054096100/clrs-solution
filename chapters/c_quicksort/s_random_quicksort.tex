\startsection[
  reference=section:rand_quicksort,
  title={A randomized version of quicksort},
]

\startEXERCISE
爲什麼我們分析隨機化算法的期望運行時間,而不是其最壞運行時間?
\stopEXERCISE

\startANSWER
因爲最壞情況不會由特定輸入觸發,而時隨機的。
即無法以可信的方式觸發最壞情況,但是他以一定概率包含在期望運行時間內。
\stopANSWER

\startEXERCISE
在 \ALGO{RANDOMIZED-QUICKSORT} 的運行過程中,
在最壞情況下,隨機數生成器 \ALGO{RANDOM} 被調用了多少次?
在最好情況下呢?
以 \m{\Theta} 符號的形式給出答案。
\stopEXERCISE

\startANSWER
最壞情況下,調用 \ALGO{RANDOM} 的次數爲:
\startformula
T(n)=T(n-1)+1=n=\Theta(n)
\stopformula
而最好情況則爲:
\startformula
T(n)=2T(n/2)+1=\Theta(n)
\stopformula
不用驚訝,因爲至少由三分之一的元素都會用作主元。
\stopANSWER

\stopsection
