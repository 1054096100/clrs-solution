\startsection[
  reference=section:desc_quicksort,
  title={Description of quicksort},
]

\startEXERCISE
參照圖 7-1 的方法,說明 \ALGO{PARTITION} 在數列 \m{A = \langle 13, 19, 9, 5, 12, 8, 7, 4, 21, 2, 6, 11 \rangle} 上的操作過程。
\stopEXERCISE

\startANSWER
{\externalfigure[output/e7_1_1-1]}
{\externalfigure[output/e7_1_1-2]}
{\externalfigure[output/e7_1_1-3]}
{\externalfigure[output/e7_1_1-4]}
{\externalfigure[output/e7_1_1-5]}
{\externalfigure[output/e7_1_1-6]}
{\externalfigure[output/e7_1_1-7]}
{\externalfigure[output/e7_1_1-8]}
{\externalfigure[output/e7_1_1-9]}
{\externalfigure[output/e7_1_1-10]}
{\externalfigure[output/e7_1_1-11]}
{\externalfigure[output/e7_1_1-12]}
{\externalfigure[output/e7_1_1-13]}
\stopANSWER

\startEXERCISE[exercise:same_partition]
當數列 \m{A[p..r]} 中的元素都相同時, \ALGO{PARTITION} 返回的 \m{q} 值是多少?
修改 \ALGO{PARTITION},使得當數列 \m{A[p..r]} 中所有元素的值都相同時, \m{q=\lfloor (p+r)/2\rfloor}。
\stopEXERCISE

\startANSWER
元素都相同時返回的值爲 \m{r}。

\CLRSH{PARTITION'(A, p, r)}
\startCLRS
x = A[r]
i = p - 1
for j = p to r - 1
	if A[j] <= x
		i = i + 1
		exchange A[i] with A[j]
i = i + 1
exchange A[i] with A[r]

if i = r
	return ⌊(p + r) / 2⌋
return i
\stopCLRS
\stopANSWER

\startEXERCISE
證明:在規模爲 \m{n} 的子數列上, \ALGO{PARTITION} 的時間復雜度爲 \m{\Theta(n)}。
\stopEXERCISE

\startANSWER
{\EMP for} 循環的次數爲 \m{r - 1 - p = \Theta(n)}。
最壞情況下,每次循環都會執行 {\EMP if} 塊,需要常數時間;循環外的語句也需要常數隨時間。
因此時間復雜度爲 \m{\Theta(n)}。
\stopANSWER

\startEXERCISE
修改 \ALGO{QUICKSORT},使其能以非遞增方式配需。
\stopEXERCISE

\startANSWER
只需修改 \ALGO{PARTITION} 中第 4 行的比較條件。
\stopANSWER

\stopsection
