\startsection[
  title={Querying a binary search tree},
]

%e12.2-1
\startEXERCISE
假設一棵二叉搜索樹中的節點在 1 到 1000 之間,
現在想要查找數值爲 363 的節點。
下面序列中哪個不是查找過的序列?
\startigBase[a]
\item \m{2,252,401,398,330,344,397,363};
\item \m{924,220,911,244,898,258,362,363};
\item \m{925,202,911,240,912,245,363};
\item \m{2,399,387,219,266,382,381,278,363};
\item \m{935,278,347,621,299,392,358,363};
\stopigBase
\stopEXERCISE

\startANSWER
c 中 911 和 912 不符合二叉搜索樹的性質。

e 中 347 和 299 不符合二叉搜索樹的性質。
\stopANSWER

%e12.2-2
\startEXERCISE
寫出 \ALGO{TREE-MINIMUM} 和 \ALGO{TREE-MAXIMUM} 的遞迴版本。
\stopEXERCISE

\startANSWER
\CLRSH{TREE-MINIMUM(x)}
\startCLRS
if x.lchild == NIL
	return x;
else
	TREE-MINIMUM(x.lchild)
\stopCLRS

\CLRSH{TREE-MAXIMUM(x)}
\startCLRS
if x.rchild == NIL
	return x;
else
	TREE-MAXIMUM(x.rchild)
\stopCLRS
\stopANSWER

%e12.2-3
\startEXERCISE
寫出過程 \ALGO{TREE-PREDECESSOR} 的僞碼。
\stopEXERCISE

\startANSWER
\CLRSH{TREE_PREDECESSOR(x)}
\startCLRS
if x.lchild != NIL
	return TREE-MAXIMUM(x.lchild)
y = x.p
while y != NIL and y.lchild == x
	x = y
	y = y.p
return y
\stopCLRS
\stopANSWER

%e12.2-4
\startEXERCISE
Bunyan 教授認爲他發現了一個二叉搜索樹的重要性質。
假設在一棵二叉搜索樹中查找一個關鍵字 k,查找結束於一個樹葉。
考慮三個集合: A 爲查找路徑左邊的關鍵字集合;
 B 爲查找路徑上的關鍵字集合; C 爲查找路徑右邊的關鍵字集合。
 Bunyan 教授聲稱:任何 \m{a\in A}, \m{b\in B} 和 \m{c\in C},一定滿足 \m{a\le b\le c}。
請給出該教授這個論斷的一個可能的最小反例。
\stopEXERCISE

\startANSWER
\externalfigure[output/e12_2_4-1]
查找路徑爲 \m{(1,3,4)\in B}, \m{2\in A},所以 \m{2>1}。
\stopANSWER

%e12.2-5
\startEXERCISE
證明:如果一棵二叉搜索樹中的一個節點有兩個孩子,
那麼他的後繼沒有左孩子,他的前驅沒有右孩子。
\stopEXERCISE

\startANSWER
令此節點爲 x,後繼爲 y,而後繼 y 有左孩子 z,y 和 z 均在 x 的右子樹上,
則 \m{x < z < y},這與 y 爲 x 的後繼相矛盾,所以 x 的後繼沒有左孩子。

同理, x 的前驅沒有右孩子。
\stopANSWER

%e12.2-6
\startEXERCISE
考慮一棵二叉搜索樹 T,其關鍵字互不相同。
證明:如果 T 中一個節點 x 的右子樹爲空,且 x 有一個後繼 y,
那麼 y 一定是 x 的最底層祖先。並且 y 的左孩子也是 x 的祖先。
(注意:每個節點都是他自己的祖先。)
\stopEXERCISE

\startANSWER
若 x 是其後繼 y 的左孩子,那麼 \m{x.key\le y.key},
所以 y 是 x 的最低祖先, y 的左孩子 x 也是 x 的祖先。

若 x 是其後繼 y 的右孩子,那麼 \m{x.key\ge y.key},這明顯與後繼的定義相矛盾。

若 x 是其後繼 y 的左孩子 z 的右孩子,那麼 \m{z.key\le x.key\le y.key},
所以 y 的左孩子 z 是 x 的祖先, y 是 x 的最低祖先。
\stopANSWER

%e12.2-7
\startEXERCISE
對於一棵有 n 個節點的二叉搜索樹,有另一種方法來實現中序遍歷,
先調用 \ALGO{TREE-MINIMUM} 找到這棵樹中的最小元素,
然後再調用 \m{n-1} 次的 \ALGO{TREE-SUCCESSOR}。
證明:該算法的運行時間爲 \m{\Theta(n)}。
\stopEXERCISE

\startANSWER
此方法等同於 \ALGO{INORDER-TREE-WALK}。

在 \m{n-1} 條邊上,通過每條邊最多兩次。
\stopANSWER

%e12.2-8
\startEXERCISE[exercise:cont_successor]
證明:在一棵高度爲 h 的二叉搜索樹中,不論從哪個節點開始,
 k 次連續的 \ALGO{TREE-SUCCESSOR} 調用所需時間爲 \m{O(k+h)}。
\stopEXERCISE

\startANSWER
假設從根結點開始運行 n 步,總時間爲 \m{O(n+h)}。
則從 \m{n-k+1} 到 \m{n} 的連續 k 步開銷爲 \m{O(k+h)}。
\stopANSWER

%e12.2-9
\startEXERCISE
設 T 是一棵二叉搜索樹,其關鍵字互不相同;
設 x 是一個葉子節點, y 爲其父節點。
證明: \m{y.key} 是 T 中大於 \m{x.key} 的最小關鍵字,
或者是 T 中小於 \m{x.key} 的最大關鍵字。
\stopEXERCISE

\startANSWER
按中序遍歷, x 和 y 相鄰。
\stopANSWER

\stopsection
