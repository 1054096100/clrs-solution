\startsection[
  title={Randomly built binary search trees},
]

%e12.4-1
\startEXERCISE
證明等式 12.3。附等式 12.3:
\startformula
\sum_{i=0}^{n-1}\binom{i+3}{3} = \binom{n+3}{4}
\stopformula
\stopEXERCISE

\startANSWER
\startformula\startmathalignment
\NC \sum_{i=0}^{n-1}\binom{i+3}{3}
    \NC = \binom{4}{4} + \sum_{i=1}^{n-1}\binom{i+3}{3} \NR
\NC \NC = \binom{4}{4} + \binom{4}{3} + \sum_{i=2}^{n-1}\binom{i+3}{3} \NR
\NC \NC = \binom{5}{4} + \sum_{i=2}^{n-1}\binom{i+3}{3} \NR
\NC \NC = \binom{6}{4} + \sum_{i=3}^{n-1}\binom{i+3}{3} \NR
\NC \NC = \ldots \NR
\NC \NC = \binom{n+3}{4} \NR
\stopmathalignment\stopformula
\stopANSWER

%e12.4-2
\startEXERCISE
請描述這樣一棵有 n 個節點的二叉搜索樹,
其樹中節點的平均深度爲 \m{\Theta(\lg{n})},
但這棵樹的高度是 \m{\omega(\lg{n})}。
一棵有 n 個節點的二叉搜索樹中節點的平均深度爲 \m{\Theta(\lg{n})},
給出這棵樹高度的一個漸進上界。
\stopEXERCISE

\startANSWER
假設這樣構建二叉樹:給定 n 個節點,取 \m{f(n)} 個節點放在一邊。
然後構建一個二叉樹,使得根結點的左子樹是一個完全二叉樹,
有 \m{n-f(n)-1} 個節點,而右子樹則是 \m{f(n)} 個節點構成的鏈。
後面再處理那 \m{f(n)} 個節點。

根的左子樹平均深度是多少?由於我們只想得到一個漸進界,
讓我們選取一個 n,使得 \m{n-f(n)-1} 正好比 2 的冪小 1,即 \m{2^k-1}。
這種情況下,這部分的總高度爲 \m{1\times 2+2\times 3+4\times 4+8\times 5+\ldots+2^{k-1}\times k}。
總數約爲 \m{2^k\times k},即 \m{(n-f(n))\log{(n-f(n))}},只跟 k 有關。
而對於另外一部分,總高度爲 \m{f(n)^2}。
即整棵樹的平均高度爲 \m{((n-f(n))\log{(n-f(n))} + f(n)^2)/n}。
而樹的高度爲 \m{f(n)},所以我們要使得 \m{f(n)} 最大化,並保持平均高度 \m{O(\log{n})}。

我們需要找到 \m{f(n)},使得:
\startigBase[n]
\item \m{n-f(n)=\Theta(n)},否則分子中的 \m{\log} 項會消失,從而使高度不再是指數級的;
\item \m{f(n)^2/n=O(\log{n})},否則分子中的第二項就太大了。
\stopigBase

如果選擇 \m{f(n)=\Theta(\sqrt(n\log{n}))},則這兩項都得到了最大滿足。
即樹的高度爲 \m{\Theta(\sqrt(n\log{n}))},而平均高度爲 \m{\Theta(\log{n})}。

參見 \goto{stackoverflow}[url(http://stackoverflow.com/questions/9257464/give-an-asymptotic-upper-bound-on-the-height-of-an-n-node-binary-search-tree-in)]。
\stopANSWER

%e12.4-3
\startEXERCISE
說明含有 n 個關鍵字的隨機選擇二叉搜索樹的概念,
這裏每一棵 n 個節點的二叉搜索樹都是等可能的被選擇,
不同於本節中給出的隨機構建二叉搜索樹的概念。
(\hint 當 \m{n=3} 時,列出所有的可能。)
\stopEXERCISE

\startANSWER
排列 \m{213} 和 \m{231} 所構建的二叉樹相同。
\stopANSWER

%e12.4-4
\startEXERCISE
證明:函數 \m{f(x)=2^x} 是凸的。
\stopEXERCISE

\startANSWER
\m{f''(x) > 0}
\stopANSWER

%e12.4-5
\startEXERCISE\DIFFICULT
考慮在 n 個互不相同的輸入數據的序列上執行 \ALGO{RANDOMIZED-QUICKSORT} 操作。
證明:對於任何常數 \m{k>0}, \m{n!} 種輸入排列除了其中的 \m{O(1/n^k)} 種之外,
運行時間都爲 \m{O(n\lg{n})}。
\stopEXERCISE

\startANSWER
\ALGO{QUICKSOR} 生成二叉搜索樹,每層總時間開銷爲 \m{O(n)}。
總時間爲 \m{O(n h_n)},其中 \m{h_n} 爲樹的高度,是一個隨機變量。
 \m{E(h_n) = c\lg{n}}。
 \m{\omega(n\lg{n})} 時間概率爲:
\startformula
 \Pr\left{\lim_{n\rightarrow\infty}\frac{h_{\pi(n)}}{c\lg{n}} = \infty\right}
\stopformula

要證明:
\startformula
\forall\rho
  \lim_{n\rightarrow\infty}n^\rho
  \Pr\left{\lim_{n\rightarrow\infty}\frac{h_{\pi(n)}}{c\lg{n}} < \infty\right}
\stopformula
只需要證明:
\startformula
 \Pr\left{\lim_{n\rightarrow\infty}\frac{h_{\pi(n)}}{c\lg{n}} = 0\right}
\stopformula

其中 \m{h_{\pi(n)}} 表示 \m{1\ldots n} 的排列 \m{\pi(n)} 生成的二叉搜索樹的高度。
事件可等價表示爲:
\startformula\startmathalignment
\NC  \NC \lim_{n\rightarrow\infty}\frac{h_{\pi(n)}}{c\lg{n}} = \infty \NR
\NC \Leftrightarrow \NC \forall K \exists N(K), n > N(K), \frac{h_{\pi(n)}}{c\lg{n}} > K \NR
\NC \Leftrightarrow \NC \bigcap_{K}\left{\frac{h_{\pi(n)}}{c\lg{n}} > K | n>N(K) \right} \NR
\stopmathalignment\stopformula

由 Markov 不等式可得:
\startformula
\Pr\left{ \frac{h_{\pi(n)}}{c\lg{n}} > K \right} < \frac{1}{K}
\stopformula

\startformula
\Pr\left{ \bigcap_{K}\left{\frac{h_{\pi(n)}}{c\lg{n}} > K | n>N(K) \right} \right} < \frac{1}{K} \forall K=0
\stopformula

\stopANSWER

\stopsection
