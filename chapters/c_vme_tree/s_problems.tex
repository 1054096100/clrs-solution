\startsubject[
  title={Problems},
]

%p20-1
\startPROBLEM[problem:cs_veb]
(Space requirements for van Emde Boas trees)
這個問題討論 van Emde Boas 樹的空間需求,
並給出一種修改數據結構的方法,
使其空間需求依賴元素個數 \m{n},
而不是全域大小 \m{u}。
爲簡單起見,假設 \m{\sqrt{u}} 總是整數。

\startigBase[a]\startitem
解釋爲什麼下面的遞歸式表示全域大小爲 \m{u} 的 van Emde Boas 的空間需求 \m{P(u)}。
\startformula
P(u) = (\sqrt{u}+1)P(\sqrt{u}) + \Theta(\sqrt{u})
\stopformula

\stopitem\stopigBase

\startANSWER
\TODO{略。}
\stopANSWER

\startigBase[continue]\startitem
證明:上面遞歸式的解爲 \m{P(u) = O(u)}。
\stopitem\stopigBase

\startANSWER
\TODO{略。}
\stopANSWER

爲了減少空間需求,定義一棵{\EMP 縮減空間的 van Emde Boas 樹}
(reduced-space van Emde Boas tree),或 {\EMP RS-vEB 樹}。
 RS-vEB 樹是一棵 vEB 樹,但作出了如下修改:
\startigNum[1]
\item 屬性 V.cluster 並不是一個指向全域大小爲 \m{\sqrt{u}} 的 vEB 樹的簡單指針數組,
而是一個散列表(見\refchapter{hash_tables}),以動態表的方式來存儲(\refsection{dynamic_tables})。
相應於 V.cluster 的數組版本,而散列表存儲指向全域大小爲 \m{\sqrt{u}} 的 RS-vEB 樹的指針。
於是要查找第 \m{i} 個簇,就在散列表中查找關鍵字 \m{i},
所以可以用在散列表中的簡單搜索找到第 \m{i} 個簇。

\item 散列表只存儲指向非空簇的指針,
在散列表中搜索一個空簇,返回 NIL,表示這個簇爲空。

\item 如果所有的簇爲空,則屬性 V.summary 爲 NIL;
否則, V.summary 指向全域大小爲 \m{\sqrt{u}} 的 RS-vEB 樹。
\stopigNum

因爲散列表使用動態表來實現,所以需要的空間與非空簇的數量成正比。

當需要向空 RS-vEB 樹插入一個元素時,
調用下面的過程創建 RS-vEB 樹,
其中參數 \m{u} 是 RS-vEB 樹的全域大小。

\CLRSH{CREATE-NEW-RS-vEB-TREE(u)}
\startCLRS
allocate a new vEB tree V
V.u = u
V.min = NIL
V.max = NIL
V.summary = NIL
create V.cluster as an empty dynamic hash table
return V
\stopCLRS

\startigBase[continue]\startitem
基於 \ALGO{vEB-TREE-INSERT} 的僞碼實現 \ALGO{RS-vEB-TREE-INSERT(V, x)} 的僞碼,
實現將 \m{x} 插入 RS-vEB 樹 V 中,
並且調用相應的 \ALGO{CREATE-NEW-RS-vEB-TREE}。
\stopitem\stopigBase

\startANSWER
\TODO{略。}
\stopANSWER

\startigBase[continue]\startitem
基於 \ALGO{vEB-TREE-SUCCESSOR} 的僞碼實現 \ALGO{RS-vEB-TREE-SUCCESSOR(V, x)} 的僞碼,
實現返回 \m{x} 在 RS-vEB 樹 V 中的後繼,
或者如果 \m{x} 在 V 中無後繼,則返回 NIL。
\stopitem\stopigBase

\startANSWER
\TODO{略。}
\stopANSWER

\startigBase[continue]\startitem
在簡單均勻散列的假設下,證明:
你實現的 \ALGO{RS-vEB-TREE-INSERT} 和 \ALGO{RS-vEB-TREE-SUCCESSOR} 的期望運行時間爲 \m{O(\lg \lg u)}。
\stopitem\stopigBase

\startANSWER
\TODO{略。}
\stopANSWER

\startigBase[continue]\startitem
假設從不刪除 vEB 樹中的元素,證明: RS-vEB 樹結構的空間需求爲 \m{O(n)},
其中 \m{n} 是存儲在 RS-vEB 樹中的實際元素個數。
\stopitem\stopigBase

\startANSWER
\TODO{略。}
\stopANSWER

\startigBase[continue]\startitem
相比 vEB 樹, RS-vEB 樹具有另一個有點:創建樹的時間較燒。
創建一棵空 RS-vEB 樹需要多長時間?
\stopitem\stopigBase

\startANSWER
\TODO{略。}
\stopANSWER

\stopPROBLEM

%p20-2
\startPROBLEM
(y-fast tries)
本題討論的是 D.Willard 的 y-fast 檢索樹,
他與 van Emde Boas 樹類似。
一個全域大小爲 \m{u} 的 y-fast 檢索樹的 \ALGO{MEMBER}、 \ALGO{MINIMUM}、 \ALGO{MAXIMUM}、
 \ALGO{PREDECESSOR} 和 \ALGO{SUCCESSOR} 操作的最壞情況運行時間爲 \m{O(\lg\lg u)}。
 \ALGO{INSERT} 和 \ALGO{DELETE} 操作的攤還時間爲 \m{O(\lg\lg u)}。
像縮減空間的 van Emde Boas 樹一樣(見\refproblem{cs_veb}),
 y-fast 檢索樹存儲 \m{n} 個元素的空間僅需要 \m{O(n)} 空間。
 y-fast 檢索樹的設計依賴於完全散列(見\refsection{perfect_hashing})。

 假設創建一個完全散列表作爲初步的結構,
 該散列表中不僅包含動態集合中的每個元素,
 而且還包含這些元素的二進制前綴。
 例如,如果 \m{u=16},那麼 \m{\lg u=4},並且 \m{x=13} 在集合中,
 由於 13 的二進制表示爲 1101,
 因此完全散列表應含有串 1、 11、 110 和 1101。
 除了該散列表外,還要創建一個該集合當前元素以升序排列的雙向鏈表。

\startigBase[a]\startitem
這個數據結構的空間需求是多少?
\stopitem\stopigBase

\startANSWER
\TODO{略。}
\stopANSWER

\startigBase[continue]\startitem
如何在 \m{O(1)} 時間內完成 \ALGO{MINIMUM} 和 \ALGO{MAXIMUM} 操作?
如何在 \m{O(\lg\lg u)} 時間內完成 \ALGO{MEMBER}、 \ALGO{PREDECESSOR} 和 \ALGO{SUCCESSOR} 操作?
如何在 \m{O(\lg u)} 時間內完成 \ALGO{INSERT} 和 \ALGO{DELETE} 操作?
\stopitem\stopigBase

\startANSWER
\TODO{略。}
\stopANSWER

爲了將空間需求減少到 \m{O(n)},我們對數據結構作出如下修改:
\startigNum[2]
\item 將 \m{n} 個元素分成大小爲 \m{\lg u} 的 \m{n/\lg u} 個組。
(假設 \m{\lg u} 可以整除 \m{n}。)
第一組由最小的 \m{\lg u} 個元素組成,
第二組由下面 \m{\lg u} 個最小的元素組成,以此類推。

\item 對每個組都設置一個“代表”。
第 \m{i} 組的代表至少與組立的最大元素一樣大,
而且比第 \m{i+1} 組的最小元素要小。
(最後一組的代表可以是最大的可能元素 \m{u-1}。)
注意,代表可能是一個值並不包含在集合中。

\item 把每組的 \m{\lg u} 個元素存儲到一個平衡二叉搜索樹中,比如紅黑樹。
每個代表指向他所在組中的平衡二叉搜索樹,
而且每個平衡二叉搜索樹指向他的代表。

\item 完全散列表僅存儲這些代表,也是用雙向鏈表按升序排列來存儲。
\stopigNum
我們稱這種結構爲一個 y-fast 檢索樹。

\startigBase[continue]\startitem
說明一個 y-fast 檢索樹存儲 \m{n} 個元素的空間需求僅爲 \m{O(n)}。
\stopitem\stopigBase

\startANSWER
\TODO{略。}
\stopANSWER

\startigBase[continue]\startitem
說明使用 y-fast 檢索樹,
如何在 \m{O(\lg \lg u)} 時間內完成 \ALGO{MINIMUM} 和 \ALGO{MAXIMUM} 操作?
\stopitem\stopigBase

\startANSWER
\TODO{略。}
\stopANSWER

\startigBase[continue]\startitem
說明如何在 \m{O(\lg\lg u)} 時間內完成 \ALGO{MEMBER} 操作?
\stopitem\stopigBase

\startANSWER
\TODO{略。}
\stopANSWER

\startigBase[continue]\startitem
說明如何在 \m{O(\lg\lg u)} 時間內完成 \ALGO{PREDECESSOR} 和 \ALGO{SUCCESSOR} 操作?
\stopitem\stopigBase

\startANSWER
\TODO{略。}
\stopANSWER

\startigBase[continue]\startitem
解釋爲什麼 \ALGO{INSERT} 和 \ALGO{DELETE} 操作要耗費 \m{O(\lg\lg u)} 時間?
\stopitem\stopigBase

\startANSWER
\TODO{略。}
\stopANSWER

\startigBase[continue]\startitem
在 y-fast 檢索樹中每組需要精確的 \m{\lg u} 個元素存儲,
試說明如何放鬆這個存儲需求來保證在 \m{O(\lg\lg u)} 攤還時間內完成 \ALGO{INSERT} 和 \ALGO{DELETE} 操作,
並同時不影響其他操作的漸進運行時間。
\stopitem\stopigBase

\startANSWER
\TODO{略。}
\stopANSWER

\stopPROBLEM

\stopsubject%Problems
