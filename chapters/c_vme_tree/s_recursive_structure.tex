\startsection[
  title={A recursive structure},
]

%e20.2-1
\startEXERCISE
寫出 \ALGO{PROTO-vEB-MAXIMUM} 和 \ALGO{PROTO-vEB-PREDECESSOR} 過程的僞碼。
\stopEXERCISE

\startANSWER
\TODO{略。}
\stopANSWER

%e20.2-2
\startEXERCISE
寫出 \ALGO{PROTO-vEB-DELETE} 的僞碼。
通過掃描簇內的相關位,
來更新相應的 \m{summary} 位。
你實現的僞碼最壞運行時間是多少?
\stopEXERCISE

\startANSWER
\TODO{略。}
\stopANSWER

%e20.2-3
\startEXERCISE
爲每個 proto-vEB 結構增加屬性 \m{n},
以給出其所在集合的元素個數,
然後寫出 \ALGO{PROTO-vEB-DELETE} 的僞碼,
要求使用屬性 \m{n} 來確定何時將 \m{summary} 重置爲 0。
你的僞碼最壞運行時間是多少?
由於加入了新的屬性 \m{n},
其他操作要變更嗎?
這些變化是否會影響其運行時間?
\stopEXERCISE

\startANSWER
\TODO{略。}
\stopANSWER

%e20.2-4
\startEXERCISE
修改 proto-vEB 結構,以支持重複關鍵字。
\stopEXERCISE

\startANSWER
\TODO{略。}
\stopANSWER

%e20.2-5
\startEXERCISE
修改 proto-vEB 結構,以支持帶有衛星數據的關鍵字。
\stopEXERCISE

\startANSWER
\TODO{略。}
\stopANSWER

%e20.2-6
\startEXERCISE
寫出一個創建 proto-vEB(u) 結構的僞碼。
\stopEXERCISE

\startANSWER
\TODO{略。}
\stopANSWER

%e20.2-7
\startEXERCISE
試說明如果 \ALGO{PROTO-vEB-MINIMUM} 中的第 9 行被執行,
則 proto-vEB 結構爲空。
\stopEXERCISE

\startANSWER
\TODO{略。}
\stopANSWER

%e20.2-8
\startEXERCISE
假設有這樣一個 proto-vEB 結構,
其中每個簇數列僅有 \m{u^{1/4}} 個元素。
那麼每個操作的運行時間是多少?
\stopEXERCISE

\startANSWER
每個簇的大小爲 \m{u^{1-1/4}},遞迴可知最終簇大小爲 \m{u^{{3/4}^j} = 2}。
因此 \m{j = \log_{4/3}\log_{2}u},
因此運行時間爲 \m{O(\log\log u)}。
\stopANSWER

\stopsection
