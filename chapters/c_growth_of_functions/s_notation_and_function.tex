\startsection[
  reference={section:notationfunction},
  title={Standard notations and common functions},
]

\startEXERCISE
如果 \m{f(n)} 和 \m{g(n)} 均單調遞增,證明函式 \m{f(n) + g(n)} 和 \m{f(g(n))} 也都單調遞增。
另外,如果 \m{f(n)} 和 \m{g(n)} 均非負,則 \m{f(n)\cdot g(n)} 也單調遞增。
\stopEXERCISE
\startANSWER
\m{f(n)} 和 \m{g(n)} 均單調遞增,則:
\startformula\startalign
 \NC f(m) \leq f(n) \NC \quad \text{若 } m \leq n \NR
 \NC g(m) \leq f(n) \NC \quad \text{若 } m \leq n \NR
\stopalign\stopformula
兩式相加得:
\startformula
f(m) + g(m) \leq f(n) + g(n)
\stopformula
即 \m{f(n) + g(n)} 單調遞增。

而
\startformula\startalign
 \NC m \NC \leq n \NR
\NC g(m) \NC \leq g(n) \NR
\NC f(g(m)) \NC \leq f(g(n)) \NR
\stopalign\stopformula
,所以 \m{f(g(n))} 也單調遞增。

由於 \m{f(n)} 和 \m{g(n)} 均非負,兩式直接相乘:
\startformula
f(m) \cdot g(m) \leq f(n) \cdot g(n)
\stopformula
\stopANSWER

\startEXERCISE
證明 \m{a^{\log_bc} = c^{\log_ba}}。
\stopEXERCISE
\startANSWER
\startformula
a^{\log_bc}
= a^{\frac{\log_ac}{\log_ab}}
= (a^{\log_ac})^{\frac{1}{\log_ab}}
= c^{\log_ba}
\stopformula
\stopANSWER

\startEXERCISE[exercise:lg_n_fac]
證明:
\startformula\startalign
 \NC n! \NC = o(n^n) \NR
 \NC n! \NC = \omega(2^n) \NR
 \NC \lg(n!) \NC = \Theta(n\lg n) \NR
\stopalign\stopformula
\stopEXERCISE
\startANSWER
使用 {\em Stirling's approximation}:
\startformula\startalign
\NC \lg(n!) \NC= \lg(\sqrt{2\pi{n}} (\frac{n}{e})^n (1+\Theta(\frac{1}{n}))) \NR
\NC         \NC= \lg\sqrt{2\pi{n}} + \lg(\frac{n}{e})^n + \lg(1+\Theta(\frac{1}{n})) \NR
\NC         \NC= \Theta(\sqrt{n}) + n\lg{\frac{n}{e}} + \lg(\Theta(1) + \Theta(\frac{1}{n})) \NR
\NC         \NC= \Theta(\sqrt{n}) + \Theta(n\lg{n}) + \Theta(\frac{1}{n}) \NR
\NC         \NC= \Theta(n\lg{n}) \NR
\stopalign\stopformula

\startformula
\forall n > 3:\quad
  2^n = \underbrace{2 \cdot 2 \cdot \cdots \cdot 2}_{\text{n 次}}
      < 1 \cdot 2 \cdot \cdots \cdot n
      = n!
      \quad\Rightarrow\quad n! = \omega(2^n)
\stopformula

\startformula
\forall n > 1:\quad
  n! = 1 \cdot 2 \cdot \cdots n
     < \underbrace{n \cdot n \cdot \cdots \cdot n}_{\text{n 次}}
     = n^n
     \quad\Rightarrow\quad n! = o(n^n)
\stopformula
\stopANSWER

\startEXERCISE
函式 \m{\lceil \lg{n} \rceil!} 多項式有界嗎?
函式 \m{\lceil \lg{\lg{n}} \rceil!} 多項式有界嗎?
\stopEXERCISE
\startANSWER
多項式有界的定義:
\startformula
 f(n) \leq c n^k
\stopformula
兩邊同時取對數:
\startformula
 \lg f(n) \leq c_1 k \lg n
\stopformula
即如果 \m{\lg f(n) = \Theta(\lg n)},則函式 \m{f(n)} 多項式有界。

設 \m{f(n) = \lceil \lg{n} \rceil!}, \m{m = \lceil \lg{n} \rceil},由上個練習可得:
\startformula\startalign
 \NC \lg f(n) \NC = \lg \lceil\lg{n}\rceil ! \NR
 \NC          \NC = \lg m! \NR
 \NC          \NC = \Theta(m\lg m) \NR
 \NC          \NC = \Theta(\lceil\lg n\rceil \lg \lceil \lg n\rceil) \NR
 \NC          \NC \neq \Theta(\lg n) \NR
\stopalign\stopformula
所以 \m{\lceil \lg{n} \rceil!} {\EMP 不是}多項式有界的。

對於另外一個,
設 \m{f(n) = \lceil \lg{\lg{n}} \rceil!}, \m{p = \lceil \lg{\lg{n}} \rceil},
由上個練習可得:
\startformula\startalign
 \NC \lg f(n) \NC = \lg \lceil\lg{\lg{n}}\rceil ! \NR
 \NC          \NC = \lg p! \NR
 \NC          \NC = \Theta(p\lg p) \NR
 \NC          \NC = \Theta(\lceil\lg\lg n\rceil \lg \lceil \lg\lg n\rceil) \NR
 \NC          \NC = \Theta(\lg\lg n \lg\lg\lg n) \NR
 \NC          \NC = o(\lg\lg n \lg\lg n) \NR
 \NC          \NC = o(\lg^2\lg n) \NR
 \NC          \NC = o(\lg n) \NR
\stopalign\stopformula
所以 \m{\lceil \lg\lg{n} \rceil!} {\EMP 是}多項式有界的。
\stopANSWER

\startEXERCISE
\m{\lg(\lg^*n)} 和 \m{\lg^*(\lg n)} 那個更大?
\stopEXERCISE
\startANSWER
後者:
\startformula
\lg^*(\lg{n}) = \lg^*n - 1 > \lg(\lg^*(n))
\stopformula
\stopANSWER

\startEXERCISE
證明 \m{\phi} 及其共扼 \m{\hat \phi} 均滿足方程 \m{x^2 = x + 1}。
\stopEXERCISE
\startANSWER
\startformula\startalign
 \NC \phi^2 - \phi - 1 \NC = (\frac{1 + \sqrt5}{2})^2 - \frac{1 + \sqrt5}{2} - 1 \NR
 \NC                   \NC = \frac{1 + 2\sqrt{5} + 5 - 2 - 2\sqrt{5} - 4}{4} \NR
 \NC                   \NC = 0 \NR
\stopalign\stopformula

\startformula\startalign
 \NC \hat\phi^2 - \hat\phi - 1 \NC = (\frac{1 - \sqrt5}{2})^2 - \frac{1 - \sqrt5}{2} - 1 \NR
 \NC                           \NC = \frac{1 - 2\sqrt{5} + 5 - 2 + 2\sqrt{5} - 4}{4} \NR
 \NC                           \NC = 0 \NR
\stopalign\stopformula
\stopANSWER

\startEXERCISE
證明第 \m{i} 個 Fibonacci 數滿足方程:
\startformula
F_i = \frac{\phi^i - \hat{\phi^i}}{\sqrt5}
\stopformula
\stopEXERCISE
\startANSWER
初始:
\startformula\startalign
\frac{\phi^0 - \hat\phi^0}{\sqrt{5}} = \frac{1 - 1}{\sqrt{5}} = 0 = F_0 \NR
\frac{\phi - \hat{\phi}}{\sqrt{5}} = \frac{1 + \sqrt{5} - 1 + \sqrt{5}}{2\sqrt{5}} = 1 = F_1 \NR
\stopalign\stopformula
歸納:
\startformula\startalign
 \NC F_{n + 2} \NC = F_{n + 1} + F_n \NR
 \NC           \NC = \frac{\phi^{n+1} - {\hat\phi}^{n+1}}{\sqrt{5}} + \frac{\phi^n - {\hat\phi}^n}{\sqrt{5}} \NR
 \NC           \NC = \frac{\phi^n(\phi + 1) - {\hat\phi}^n(\hat\phi + 1)}{\sqrt{5}} \NR
 \NC           \NC = \frac{\phi^n\phi^2 - {\hat\phi}^n{\hat\phi}^2}{\sqrt{5}} \NR
 \NC           \NC = \frac{\phi^{n+2} + {\hat\phi}^{n+2}}{\sqrt{5}} \NR
\stopalign\stopformula
\stopANSWER

\startEXERCISE
證明 \m{k\ln k = \Theta(n)} 蘊含着 \m{k = \Theta(n/\ln n)}。
\stopEXERCISE
\startANSWER
由 \m{\Theta} 的對稱性可知:
\startformula
k\ln{k} = \Theta(n) \Rightarrow n = \Theta(k\ln{k})
\stopformula
則:
\startformula
\ln{n} = \Theta(\ln(k\ln{k})) = \Theta(\ln{k} + \ln\ln{k}) = \Theta(\ln{k})
\stopformula
兩式相除:
\startformula
\frac{n}{\ln{n}}
  = \frac{\Theta(k\ln{k})}{\Theta(\ln{k})}
  = \Theta(\frac{k\ln{k}}{\ln{k}})
  = \Theta(k)
\stopformula
\stopANSWER

\stopsection
