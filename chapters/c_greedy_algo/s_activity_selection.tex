\startsection[
  title={An activity-selection problem},
]

%e16.1-1
\startEXERCISE
根據遞歸式(16.2)爲活動選擇問題設計一個動態規劃算法。
算法應該按前文定義計算最大兼容活動集的大小 \m{c[i,j]} 並生成最大集本身。
假定輸入的活動已按公式(16.1)排好序。
比較你的算法和 \ALGO{GREEDY-ACTIVITY-SELECTOR} 的運行時間。
附公式(16.1):
\startformula
f_1 \le f_2 \le f_3 \le \ldots \le f_{n-1} \le f_n
\stopformula

附遞迴式(16.2):
\startformula
c[i,j] = \startcases
\NC 0 \NC 如果 \m{S_{ij} = \Phi} \NR
\NC \max_{a_k\in S_{ij}}\{c[i,k] + c[k,j] + 1\} \NC 如果 \m{S_{ij}\ne \Phi} \NR
\stopcases\stopformula
\stopEXERCISE

\startANSWER
令 \m{S_{ij}} 爲兩個不相重疊的活動 \m{a_i} 和 \m{a_j} 的間隔。
下列算法將爲活動選擇問題構建一張表,
其中 \m{s} 和 \m{f} 分別是活動的起始和結束時間,長度均爲 \m{n}。

\CLRSH{BUILD-ACTIVITY-TABLE(s, f)}
\startCLRS
n = s.length
for i = 0 upto n
	c[i][i] = 0

for m = 2 upto n
	for i = 1 upto (n-m+1)
		j = (i+m-1)
		c[i][j] = 0

		for k = i+1 upto j-1
			if f[i] <= s[k] && f[k] <= s[j]
				q = (c[i][k] + c[k][j] + 1)
				if q > c[i][j]
					c[i][j] = q
					c[j][i] = k
return c
\stopCLRS

此算法總運行時間爲 \m{O(n^3)},而貪婪算法運行時間爲 \m{O(n)}。
\stopANSWER

%e16.1-2
\startEXERCISE
假定我們不再一直選擇最早結束的活動,而是選擇最晚開始的活動,
前提仍然是與之前選出的所有活動兼容。
描述如何利用這一方法設計貪心算法,並證明算法會產生最優解。
\stopEXERCISE

\startANSWER
\CLRSH{GREEDY-ACTIVITY-SELECTOR-JMC(s,f)}
\startCLRS
n = s.length
A.push_front(a[n])
k = n

for m=(n-1) downto 1
	if f[m] <= s[k]
		A.push_front(a[m])
		k = m
\stopCLRS

建議的方式,選擇與開始最晚的活動,總是保持與之前選中的活動相兼容,
這確實是貪心算法,只是從後向前。

我們可以這樣來看:有活動集 \m{S=\{a_1,a_2,\ldots,a_n\}},
其中 \m{a_i=[s_i,f_i)},我們想通過選擇開始最晚的活動來得到最優方案,
當然要保證與之前選中的活動相兼容。
創建集合 \m{S'=\{a'_1,a'_2,\ldots,a'_n\}},
其中 \m{a'_i=[f_i,s_i)},即 \m{a'_i} 正好與 \m{a_i} 相反。
顯然,\m{\{a_{i_1},a_{i_2},\ldots,a_{i_k}\} \subseteq S} 的子集要想相互兼容,
當且僅當其對應的 \m{\{a'_{i_1},a'_{i_2},\ldots,a'_{i_k}\} \subseteq S'} 的子集也相互兼容。
因此, \m{S} 的最優解與 \m{S'} 的最優解可以相互直接映射。
\stopANSWER

%e16.1-3
\startEXERCISE
對於活動選擇問題,並不是所有貪心方法都能得到最大兼容活動子集。
請舉例說明,在剩餘兼容活動中選擇持續時間最短者不能得到最大集。
類似地,說明在剩餘兼容說動中選擇與其他剩餘活動重疊最少者,
以及選擇最早開始者均不能得到最優解。
\stopEXERCISE

\startANSWER
選擇持續時間最短活動:

\startxtable[
    option=max,
    align={middle,lohi},
    split=yes,
    header=repeat,
    footer=repeat,
    offset=.25em,
]

% head
\startxtablehead[frame=off,bottomframe=on]
\startxrow[foregroundstyle=bold,]
  \xcell[rightframe=on]{\m{i}}\processcommalist[1, 2, 3]\xcell
\stopxrow
\stopxtablehead

% body
\startxtablebody[frame=off]
\startxrow \xcell[rightframe=on]{\m{s_i}}   \processcommalist[0,2,3]\xcell \stopxrow
\startxrow \xcell[rightframe=on]{\m{f_i}}   \processcommalist[3,4,6]\xcell \stopxrow
\startxrow \xcell[rightframe=on]{\m{hold_i}}\processcommalist[3,2,3]\xcell \stopxrow
\stopxtablebody

\stopxtable

此方法會選擇 \m{\{a_2\}},但最優解應爲 \m{\{a_1,a_3\}}。

選擇與其他活動重疊最少的活動:

\startxtable[
    option=max,
    align={middle,lohi},
    split=yes,
    header=repeat,
    footer=repeat,
    offset=.25em,
]

\startxtablehead[frame=off,bottomframe=on]
\startxrow[foregroundstyle=bold,]
  \xcell[rightframe=on]{\m{i}}\processcommalist[1, 2, 3, 4, 5, 6, 7, 8, 9, 10, 11]\xcell
\stopxrow
\stopxtablehead

% body
\startxtablebody[frame=off]
\startxrow \xcell[rightframe=on]{\m{s_i}}   \processcommalist[0,1,1,1,2,3,4,5,5,5,6]\xcell \stopxrow
\startxrow \xcell[rightframe=on]{\m{f_i}}   \processcommalist[2,3,3,3,4,5,6,7,7,7,8]\xcell \stopxrow
\startxrow \xcell[rightframe=on]{\m{over_i}}\processcommalist[3,4,4,4,4,2,4,4,4,4,3]\xcell \stopxrow
\stopxtablebody

\stopxtable

此方法會先選擇 \m{a_6},然後只能選擇兩個: \m{a_1,a_2,a_3,a_4} 中的一個,
以及 \m{a_8,a_9,a_{10},a_{11}} 中的一個。而最優解爲 \m{a_1,a_5,a_7,a_{11}}。

選擇最早開始的活動:只要最早開始的活動持續時間足夠長,
就只能選擇這一個活動,其他的都無法與其兼容。
\stopANSWER

%e16.1-4
\startEXERCISE
假定右一組活動,我們需要將他們安排到一些教室,
任意活動都可以在任意教室進行。
我們希望使用最少的教室完成所有活動。
設計一個高效的貪心算法求每個活動應該在哪個教室進行。

(這個問題稱爲{\EMP 區間圖着色問題}(interval-graph color problem)。
我們可以構造一個區間圖,頂點表示給定的活動,
邊連接不兼容的活動。
要求用最少的顏色對頂點進行着色,
使得所有相鄰頂點顏色均不相同——
這與使用最少的教室完成所有活動的問題是對應的。)
\stopEXERCISE

\startANSWER
集合 \m{S} 包含 \m{n} 個活動。
一個顯而易見的方案:使用 \ALGO{GREEDY-ACTIVITY-SELECTOR},
先爲第一個教室找到最大集合 \m{S_1} (\m{S_1\subseteq S}),
再爲第二個教室找到最大集合 \m{S_1} (\m{S_1\subseteq S - S_1}),
……。最壞情況下需要時間 \m{\Theta(n^2)}。

還有一種更好的算法,然而,其漸進時間只是將活動按時間排序所需的時間 \m{O(n\lg n)}。
如果活動的時間是小整數,甚至有可能達到 \m{O(n)}。

通常我們會按起始時間遍歷所有活動,並將其指派給任何一個可用的教室。
要達到這個目的,我們需要對活動的起始時間和結束時間進行排序,並遍歷所有時間點。
維護兩個教室清單:一個是當前時間點 \m{t} 忙的教室,另一個是當前時間點 \m{t} 空閒的教室。
如果 \m{t} 是某個活動的結束時間點,,則將此活動所佔教室挪到空閒清單中。
如果 \m{t} 是另一個活動的起始時間點,則可以將這個活動安排到一個空閒的教室中。

爲了儘量減少教室佔用,活動結束時將教室放入空閒清單的頭部,爲活動安排教室也從空閒教室頭部選取。

運行時間包括兩部分: a)爲活動起始、結束時間排序。
(如果某活動的結束時間和某活動起始時間相同,則應將結束的活動放在前面。)
所需時間爲 \m{O(n\lg n)},而如果活動所用時間是受限的(比如,都是小整數),
則所需時間可能是 \m{O(n)}。
 b)處理所有起始、結束時間點所需時間爲 \m{O(n)}:掃描 \m{2n} 個時間點,
 每個時間點的時間爲 \m{O(1)}。
\stopANSWER

%e16.1-5
\startEXERCISE
考慮活動選擇問題的一個變形:
每個活動 \m{a_i} 除了開始和結束時間外,
還有一個值 \m{v_i}。
目標不再是求規模最大的兼容活動子集,
而是求值之和最大的兼容活動子集。
也就是說,
選擇一個兼容活動子集 \m{A},使得 \m{\sum_{a_k\in A} v_k} 最大化。
設計一個多項式時間的算法求解此問題。
\stopEXERCISE

\startANSWER
動態規劃。時間爲 \m{O(n^2)}。
\stopANSWER

\stopsection
