\startsection[
  title={Breadth-first search},
]

%e22.2-1
\startEXERCISE
請計算出在有向圖 22-2(a)上運行廣度優先搜索算法後的 \m{d} 值和 \m{\phi} 值。
這裏假定節點 3 爲算法所用的源節點。
\stopEXERCISE

\startANSWER
\TODO{略。}
\stopANSWER

%e22.2-2
\startEXERCISE
請計算出在有向圖 22-3 所示無向圖上運行廣度優先搜索算法後的 \m{d} 值和 \m{\phi} 值。
這裏假定節點 u 爲算法所用的源節點。
\stopEXERCISE

\startANSWER
\TODO{略。}
\stopANSWER

%e22.2-3
\startEXERCISE
證明:使用單個位來存放每個節點的顏色即可。
這個論點可以通過證明將算法第 18 行的僞碼刪除後, \ALGO{BFS} 過程生成的結果不變得到。
\stopEXERCISE

\startANSWER
算法只關心節點是不是白色,至於是黑還是灰則沒有區別。
\stopANSWER

%e22.2-4
\startEXERCISE
如果將輸入的圖用鄰接矩陣來表示,
並修改算法來應對此種形式的輸入,
請問 \ALGO{BFS} 的運行時間將是多少?
\stopEXERCISE

\startANSWER
\m{O(V^2)}
\stopANSWER

%e22.2-5
\startEXERCISE
證明:在廣度有限搜索算法裏,
賦給節點 \m{u} 的 \m{u.d} 值與節點在鄰接表裏出現的次序無關。
使用圖 22-3 作爲例子,
證明: \ALGO{BFS} 所計算出的廣度優先樹可以因鄰接鏈表中的次序不同而不同。
\stopEXERCISE

\startANSWER
最短路徑長度當然和次序無關。
而最短路徑可能有多條,就和次序有關了。
\stopANSWER

%e22.2-6
\startEXERCISE
舉出一個有向圖 \m{G=(V,E)} 的例子,
對於源節點 \m{s\in V} 和一組樹邊 \m{E_{\pi}\subseteq E},
使得對於每個節點 \m{v\in V},圖 \m{V, E_{\pi}} 中從源節點 \m{s} 到
節點 \m{v} 的唯一簡單路徑也是圖 \m{G} 中的一條最短路徑,
但是,不管鄰接表裏節點之間的次序如何,
邊集 \m{E_\pi} 都不能通過在圖 \m{G} 上運行 \ALGO{BFS} 來獲得。
\stopEXERCISE

\startANSWER
如下圖,實線屬於 \m{E_\pi}。

\externalfigure[output/e22_2_6-1]
\stopANSWER

%e22.2-7
\startEXERCISE
職業摔跤手可以分爲兩種類型:
“娃娃臉”(“好人”)型和“高跟鞋”(“壞人”)型。
在任意一對職業摔跤手之間都有可能存在競爭關係。
假定有 \m{n} 個職業摔跤手,
並且有一個給出競爭關係的 \m{r} 對摔跤手的鏈表。
請給出一個時間爲 \m{O(n+4)} 的算法來判斷是否可以將某些摔跤手劃分爲“娃娃臉”型,
而剩下的劃分爲”高跟鞋“型,
使得所有的競爭關係均只存在於娃娃臉型和高跟鞋型選手之間。
如果可以進行這種劃分,
則算法還應當生成一種這樣的劃分。
\stopEXERCISE

\startANSWER
遍歷邊,給頂點染色。
\stopANSWER

%e22.2-8
\startEXERCISE\DIFFICULT
我們將一棵樹 \m{T=(V, E)} 的{\EMP 直徑}定義爲 \m{\max_{u,v\in V}\delta(u,v)},
也就是說,樹中所有最短路徑距離的最大值即爲樹的直徑。
請給出一個有效算法來計算樹的直徑,
並分析算法的運行時間。
\stopEXERCISE

\startANSWER
\startigBase[n]
\item 選擇任意節點 \m{A},運行 \ALGO{BFS} 找到離 \m{A} 最遠的節點 \m{S};
\item 從 \m{S} 開始運行 \ALGO{BFS},找到最遠節點 \m{D};
\item 從 \m{S} 到 \m{D} 的路徑即爲所求。
\stopigBase

主要是證明 \m{S} 必爲直徑的一個端點。用反證法,參考如下鏈接:

\SIMPLEURL{http://stackoverflow.com/questions/12989578/prove-traversing-a-k-ary-tree-twice-yields-the-diameter}。
\stopANSWER

%e22.2-9
\startEXERCISE
設 \m{G=(V,E)} 爲一個連通無向圖。
請給出一個 \m{O(V+E)} 時間的算法來計算圖 \m{G} 中的一條這樣的路徑:
該路徑正反向通過 \m{E} 中每條邊恰好一次(該路徑通過每條邊兩次,但這兩次的方向相反)。
如果給你大量的分幣作爲獎勵,
請描述如何在迷宮中找出一條路。
\stopEXERCISE

\startANSWER
深度優先進行遍歷,優先訪問沒有訪問過的邊,如果所有相鄰點都訪問過,沿路徑返回上一個節點,繼續執行。
\stopANSWER

\stopsection
