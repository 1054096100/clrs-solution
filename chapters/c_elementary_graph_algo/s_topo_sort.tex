\startsection[
  title={Topological sort},
]

%e22.4-1
\startEXERCISE
給出算法 \ALGO{TOPOLOGICAL-SORT} 運行於圖 22-8 上時所生成的節點次序。
這裏的所有假設與\refexercise{22.3-2} 一樣。
附圖 22-8:

\externalfigure[output/e22_4_1-1]
\stopEXERCISE

\startANSWER
\m{pnosmryvwzxuqt}。
\stopANSWER

%e22.4-2
\startEXERCISE
請給出線性時間算法,其輸入爲一個有向無環圖 \m{G=(V,E)} 以及兩個節點 \m{s} 和 \m{t},
其輸出是從節點 \m{s} 到節點 \m{t} 之間的簡單路徑的數量。
例如,對於圖 22-8 所示的有向無環圖,
從節點 \m{p} 到節點 \m{v} 一共有 4 條簡單路徑,
分別是 \m{pov}、 \m{poryv}、 \m{posryv} 和 \m{psryv}。
(本題僅要求計數簡單路徑的條數,而不要求將簡單路徑本身列舉出來。)
\stopEXERCISE

\startANSWER
先進行拓撲排序,並取 \m{s} 到 \m{t} 間的子集,令序列爲 \m{v_1, v_2, \ldots, v_n},
其中 \m{v_1=s}, \m{v_n = t}。給序列中每一項加一個屬性用來計數,初始值全爲 0。
然後從左至右遍歷此序列。如果 \m{v_i=s},則 \m{c_i = 1};
否則, \m{c_i = \sum_{v_j:(v_j,v_i)\in E}p[j]}。

\startCLRS
for i = 1 upto n
	if v[i] == s
		c[i] = 1
	else
		c[i] = 0
for i = 1 upto n - 1
	for each (v[i],v[j]) in E and j <= n
		c[j] += c[i]

return c[n]
\stopCLRS
\stopANSWER

%e22.4-3
\startEXERCISE
給出一個算法來判斷給定無向圖 \m{G=(V,E)} 是否包含一個環路。
算法運行時間應該在 \m{O(V)} 數量級,且與 \m{|E|} 無關。
\stopEXERCISE

\startANSWER
運行 \ALGO{DFS},如果發現前向邊,則有環路。
如果沒有環路,則 \m{|E| < |V|}。
如果有環路,則掃描的邊數肯定小於 \m{|V|},因此算法時間複雜度爲 \m{O(V)}。
\stopANSWER

%e22.4-4
\startEXERCISE
證明或反證下述論斷:
如果有向圖 \m{G} 包含環路,
則在算法 \ALGO{TOPOLOGICAL-SORT(G)} 所生成的節點序列裏,
圖 \m{G} 中與所生成序列不一直的“壞”邊的條數最少。
\stopEXERCISE

\startANSWER
不成立。選擇不同的起點,會得到不同的序列,壞邊的數目也不盡相同,因此不能保證壞邊數目最少。
如下圖,若從 \m{a} 點開始,則得到序列 \m{a,b,c},有 1 條壞邊 \m{(c,a)}。
若從 \m{b} 點開始,則得到序列 \m{b,c,a},有 2 條壞邊 \m{(a,c)}、 \m{(a,b)}。

\externalfigure[output/e22_4_4-1]
\stopANSWER

%e22.4-5
\startEXERCISE
在有向無環圖 \m{G=(V,E)} 上執行拓撲排序還有一種辦法,
就是重複尋找入度爲 0 的節點,輸出該節點,
將該節點及從其發出的邊從圖中刪除。
請解釋如何在 \m{O(V+E)} 時間內實現這種思想。
如果圖 \m{G} 包含環路,將會發生什麼情況?
\stopEXERCISE

\startANSWER
首先運行 \ALGO{DFS} 或 \ALGO{BFS} 可以在 \m{O(V+E)} 時間內統計出每個點的入度和出度,
以後在刪除邊的時候要維護這些信息。
每次輸出入度爲 0 的那個點並刪除其出邊,並更新相應節點的入度。
要執行 \m{O(V)} 次輸出和 \m{O(E)} 次的刪除。
總時間爲 \m{O(V+E)}。

如果有環路,那可能沒有入度爲 0 的點。
\stopANSWER

\stopsection

\startsection[
  title={Strongly connected components},
]

%e22.5-1
\startEXERCISE
如果在圖 \m{G} 中加入一條新的邊, \m{G} 中的強連通分量的數量會發生怎樣的變化?
\stopEXERCISE

\startANSWER
如果增加的邊位於某一強連通分量內部,則不會發生變化。
如果增加的邊跨越兩個分量,則分量的數量可能不變,也可能減少,甚至減少到只有一個強連通分量。
\stopANSWER

%e22.5-2
\startEXERCISE
給出算法 \ALGO{STRONGLY-CONNECTED-COMPONENTS} 在圖 22-6 上的運行過程。
具體要求是,給出算法第 1 行所計算出的完成時間和第 3 行所生成的森林。
假定 \ALGO{DFS} 的第 5~7 行的循環是以字母表順序來對節點進行處理,
並且連接鏈表中的節點也是以字母表順序排列好的。
附圖 22-6:

\externalfigure[output/e22_3_2-1]
\stopEXERCISE

\startANSWER
下圖爲 \m{G^T},其中的數字爲第一行所計算出的時間。
黑色邊連接兩個強連通分量,其他顏色的邊標記不同的顏色分量。

\externalfigure[output/e22_5_2-2]
\stopANSWER

%e22.5-3
\startEXERCISE
Bacon 教授聲稱,如果在第二次深度優先搜索時使用原始圖 \m{G} 而不是圖 \m{G} 的轉置圖 \m{G^T},
並且以完成時間的遞增次序來掃描節點,則計算強連通分量的算法將更加簡單。
這個更加簡單的算法總是能計算出正確的結果嗎?
\stopEXERCISE

\startANSWER
不正確。
\stopANSWER

%e22.5-4
\startEXERCISE
證明:對於任意有向圖 \m{G} 來說, \m{((G^T)^{SCC})^T=G^{SCC}}。
也就是說,轉置圖 \m{G^T} 的分量圖的轉置與圖 \m{G} 的分量圖相同。
\stopEXERCISE

\startANSWER
\m{G} 和 \m{G^T} 有相同的強連通分量,但分量間的連接關係正好相反。
\stopANSWER

%e22.5-5
\startEXERCISE
給出一個時間複雜度爲 \m{O(V+E)} 的算法來計算有向圖 \m{G=(V,E)} 的分量圖。
請確保在算法所生成的分量圖中,任意兩個節點之間至多存在一條邊。
\stopEXERCISE

\startANSWER
給各分量編號,記爲 \m{1,2,\ldots,k}, \m{k \le V},
並記錄每一個節點 \m{u} 所屬分量 \m{s[u]}。
遍歷所有邊 \m{(u,v)},如果 \m{s[u] \ne s[v]},則將 \m{(s[u],s[v])} 加入到新集合 \m{T} 中。
對 \m{T} 中元素進行基排序。
將 \m{T} 中所有與前一個元素不同的元素加入到新集合 \m{S} 中,
則 \m{S} 即爲所求。
\stopANSWER

%e22.5-6
\startEXERCISE
給定有向圖 \m{G=(V,E)},
請說明如何創建另一個圖 \m{G'=(V,E')},
使得:
(a) \m{G'} 的強連通分量與 \m{G} 的相同,
(b) \m{G'} 的分量圖與 \m{G} 的相同,
以及(c) \m{E'} 所包含的邊儘可能少。
請給出一個計算圖 \m{G'} 的快速算法。
\stopEXERCISE

\startANSWER
利用上一題的結果,另外每個分量內部進行 \ALGO{DFS},保留一個環路即可。
\stopANSWER

%e22.5-7
\startEXERCISE
給定有向圖 \m{G=(V,E)},
如果對於所有節點對 \m{u,v\in V},
我們有 \m{u\leadsto v} 或 \m{v\leadsto u},
則 \m{G} 是{\EMP 半連通}的。
請給出一個有效的算法來判斷圖 \m{G} 是否是半連通的。
證明算法的正確性並分析其運行時間。
\stopEXERCISE

\startANSWER
先生成強連通分量圖,若所有分量間存在一個線性鏈(開環),則此圖是半連通的。
\stopANSWER

\stopsection
