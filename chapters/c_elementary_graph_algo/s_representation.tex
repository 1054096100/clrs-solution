\startsection[
  title={Representations of graphs},
]

%e22.1-1
\startEXERCISE
給定有向圖的鄰接鏈表,需要多長時間才能計算出每個節點的出度(發出的邊的條數)?
多長時間才能計算出每個節點的入度(進入的邊的條數)?
\stopEXERCISE

\startANSWER
\m{\Theta(|V|+|E|)}。

\m{\Theta(|V||E|)}。而如果使用額外 \m{\Theta(|V|)} 空間,可以在 \m{\Theta(|V| + |E|)} 時間內完成。
\stopANSWER

%e22.1-2
\startEXERCISE
給定一棵有 7 個節點的完全二叉樹的鄰接鏈表,
請給出等價的鄰接矩陣表示。
這裏假設節點的編號爲從 1~7。
\stopEXERCISE

\startANSWER
\startformulas
\startformula\startmathalignment
\NC 1 \NC \qquad 2,3 \NR
\NC 2 \NC \qquad  1,4,5 \NR
\NC 3 \NC \qquad  1,6,7 \NR
\NC 4 \NC \qquad  2 \NR
\NC 5 \NC \qquad  2 \NR
\NC 6 \NC \qquad  3 \NR
\NC 7 \NC \qquad  3 \NR
\stopmathalignment\stopformula

\startformula\startmatrix
\NC 0 \NC 1 \NC 1 \NC 0 \NC 0 \NC 0 \NC 0 \NR
\NC 1 \NC 0 \NC 0 \NC 1 \NC 1 \NC 0 \NC 0 \NR
\NC 1 \NC 0 \NC 0 \NC 0 \NC 0 \NC 1 \NC 1 \NR
\NC 0 \NC 1 \NC 0 \NC 0 \NC 0 \NC 0 \NC 0 \NR
\NC 0 \NC 1 \NC 0 \NC 0 \NC 0 \NC 0 \NC 0 \NR
\NC 0 \NC 0 \NC 1 \NC 0 \NC 0 \NC 0 \NC 0 \NR
\NC 0 \NC 0 \NC 1 \NC 0 \NC 0 \NC 0 \NC 0 \NR
\stopmatrix\stopformula
\stopformulas
\stopANSWER

%e22.1-3
\startEXERCISE
有向圖 \m{G=(V,E)} 的{\EMP 轉置}是圖 \m{G^T=(V,E^T)},
這裏 \m{E^T=\{(v,u)\in V \times V:(u,v)\in E\}}。
\stopEXERCISE

\startANSWER
\m{O(V+E)},需要額外大小爲 \m{O(V)} 的指針數組。

轉置矩陣即可, \m{O(|V|^2)}。
\stopANSWER

%e22.1-4
\startEXERCISE
給定多圖 \m{G=(V,E)} 的鄰接鏈表(多圖是允許重複邊和自循環邊的圖),
請給出一個時間爲 \m{O(V+E)} 的算法,
用來計算該圖的“等價”無向圖 \m{G'=(V,E')} 的鄰接鏈表表示。
這裏 \m{E'} 是將 \m{E} 中的冗餘邊和自循環邊刪除後餘下的邊。
刪除冗餘邊指的是將兩個節點之間的多條邊替換爲一條邊。
\stopEXERCISE

\startANSWER
遍歷所有頂點 \m{V_i},新建一個空數組,用來記錄是否有 \m{V_i} 到其他節點的邊,
遍歷 \m{V_i} 的鄰接鏈表中所有元素 \m{V_j},如果 \m{V_j} 在新數組中沒有被標記過,則標記,
如果已被標記過,則刪除當前節點。然後清空新數組,遍歷下一個頂點的鄰接鏈表。
\stopANSWER

%e22.1-5
\startEXERCISE
有向圖 \m{G=(V,E)} 的{\EMP 平方圖}是圖 \m{G^2=(V,E^2)},
這裏,邊 \m{(u,v)\in E^2} 當且僅當圖 \m{G} 包含一條最多由兩條邊構成的從 \m{u} 到 \m{v} 的路徑。
請給出一個有效算法來計算圖 \m{G} 的平方圖 \m{G^2}。
這裏圖 \m{G} 既可以以鄰接表表示,
也可以以鄰接矩陣表示。
請分析算法的運行時間。
\stopEXERCISE

\startANSWER
鄰接矩陣: \m{M^2[u,w] = \sum_{v}M[u,v]\cdot M[v,w]},即矩陣相乘;時間複雜度爲 \m{O(V^3)}。
如果採用 Strassen 算法則爲 \m{O(V^{2.376})}。

若採用鄰接鏈表:遍歷所有邊 \m{(u,v)},然後在遍歷 \m{v} 的所有鄰接頂點 \m{w},將 \m{w} 加入到新的鄰接鏈表中(利用額外數組記錄是否已經添加過,以解決衝突)。時間複雜度爲 \m{O(VE)}。
\stopANSWER

%e22.1-6
\startEXERCISE
多數以鄰接矩陣作爲輸入的圖算法的運行時間爲 \m{\Omega(V^2)},但也有例外。
給定圖 \m{G} 的鄰接矩陣表示,
請給出一個 \m{O(V)} 時間的算法來判斷有向圖 \m{G} 是否存在一個 {\EMP 通用匯點}(universal sink)。
通用匯點指的是入度爲 \m{|V|-1} 但出度爲 0 的節點。
\stopEXERCISE

\startANSWER
只需判斷一行加一列元素。
\stopANSWER

%e22.1-7
\startEXERCISE
有向無環圖 \m{G=(V,E)} 的{\EMP 關聯矩陣}(incidence matrix)是一個滿足下述條件的 \m{|V|\times|E|} 矩陣 \m{B=(b_{ij})}:
\startformula
b_{ij} = \startcases
\NC -1 \NC 如果邊 \m{j} 從節點 \m{i} 出發 \NR
\NC 1 \NC 如果邊 \m{j} 進入節點 \m{i} \NR
\NC 0 \NC 其他 \NR
\stopcases
\stopformula
請說明矩陣乘積 \m{BB^T} 裏的每一個元素代表什麼意思。
這裏 \m{B^T} 是矩陣 \m{B} 的轉置。
\stopEXERCISE

\startANSWER
與頂點 \m{i} 相連的邊的數目。
\stopANSWER

%e22.1-8
\startEXERCISE
假定數組 \m{Adj[u]} 的每個記錄項不是鏈表,而是一個散列表,
裏面包含的是 \m{(u,v)\in E} 的節點 \m{v}。
如果每條邊被查詢的概率相同,
則判斷一條邊是否在圖中的期望時間值是多少?
這種表示方法的缺陷是什麼?
請爲每條邊鏈表給出一個不同的數據結構來解決這個問題。
與散列表相比較,你給出的新方法存在什麼缺陷嗎?
\stopEXERCISE

\startANSWER
每個散列表越大,則期望時間越少,當散列表跟頂點個數一樣大時,期望的查詢時間就是 \m{O(1)} 了。

缺陷是空間佔用大。

用鄰接矩陣。
\stopANSWER

\stopsection
