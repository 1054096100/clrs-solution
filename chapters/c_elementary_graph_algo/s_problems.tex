\startsubject[
  title={Problems},
]

%e22-1
\startPROBLEM
(Classifying edges by breadth-first search)
深度優先搜索將圖中的邊分類爲樹邊、後向邊、前向邊和橫向邊。
廣度優先搜索也可以用來進行這種分類。
具體來說,廣度優先搜索將從源節點可以到達的邊劃分爲同樣的 4 種類型。
\startigBase[a]
\startitem
證明在對無向圖進行的廣度優先搜索中,下面的性質成立:
\startigBase[n]
\item 不存在後向邊,也不存在前向邊。
\item 對於每條樹邊 \m{(u,v)},我們有 \m{v.d=u.d+1}。
\item 對於每條橫向邊 \m{(u,v)},我們有 \m{v.d=u.d} 或 \m{v.d=u.d+1}。
\stopigBase
\stopitem

\startANSWER
\TODO{略。}
\stopANSWER

\startitem
證明在對有向圖進行廣度優先搜索時,下面的性質成立:
\startigBase[n]
\item 不存在前向邊。
\item 對於每條樹邊 \m{(u,v)},我們有 \m{v.d=u.d+1}。
\item 對於每條橫向邊 \m{(u,v)},我們有 \m{v.d\le u.d+1}。
\item 對於每條後向邊 \m{(u,v)},我們有 \m{0\le v.d\le u.d}。
\stopigBase
\stopitem

\startANSWER
\TODO{略。}
\stopANSWER

\stopigBase
\stopPROBLEM

%e22-2
\startPROBLEM
(Articulation points, bridges, and biconnected components)
設 \m{G=(V,E)} 爲一個連通無向圖。
圖 \m{G} 的{\EMP 銜接點}是指圖 \m{G} 中的一個節點,
刪除該節點將導致圖不連通。
圖 \m{G} 的{\EMP 橋}是指圖中的一條邊,刪除該邊,圖就不再連通。
圖 \m{G} 的{\EMP 雙連通分量}是指一個最大的邊集合,
裏面的任意兩條邊都處於同一條簡單環路中。
圖 22-10 描述的就是這些概念的定義。
我們可以使用深度優先搜索算法來判斷 \m{G} 的銜接點、橋和雙連通分量。
設 \m{G_\pi = (V, E_\pi)} 來爲圖 \m{G} 的深度優先樹。
附圖 22-10:

\externalfigure[output/p22-2-1]

\startigBase[a]\startitem
證明: \m{G_\pi} 的根節點是圖 \m{G} 的銜接點當且僅當他在 \m{G_\pi} 中至少有兩個子節點。
\stopitem\stopigBase

\startANSWER
如果根節點是銜接點,如果只有一個子節點,那麼刪除根節點,不會影響圖的連通性,則不是銜接點,與假設不符。

如果根節點有至少兩個子節點,由於沒有連接各子樹的橫向邊,那麼刪除根節點會影響圖的連通性,符合銜接點的定義。
\stopANSWER

\startigBase[continue]\startitem
設節點 \m{v} 爲 \m{G_\pi} 的一個非根節點。證明:
 \m{v} 是 \m{G} 的銜接點當且僅當節點 \m{v} 有一個子節點 \m{s},
且沒有任何從節點 \m{s} 或任何 \m{s} 的後代節點指向 \m{v} 的真祖先的後向邊。
\stopitem\stopigBase

\startANSWER
有後向邊會導致成環。
\stopANSWER

\startigBase[continue]\startitem
定義:
\startformula
v.low = \min \startcases
\NC v.d \NC \NR
\NC w.d \NC :\m{(u,w)} 是節點 \m{v} 的某個後代節點 \m{u} 的一條後向邊\NR
\stopcases
\stopformula
請說明如何在 \m{O(E)} 的時間內爲所有節點 \m{v} 計算出 \m{v.low} 的值。
\stopitem\stopigBase

\startANSWER
\CLRSH{DFS-VISIT(u)}
\startCLRS
color[u] = GRAY
time = time + 1
d[u] = time
low[u] = d[u]
for each v Adj[u]
	if color[v] = WHITE
		pi[v] = u
		DFS-VISIT(v)
		if low[v] < low[u]
			low[u] = low[v]
	elseif color[v] = GRAY		// back edge
		if d[v] < low[u]
			low[u] = d[v]
color[u] = BLACK
time = time + 1
f[u] = time
\stopCLRS
\stopANSWER

\startigBase[continue]\startitem
說明如何在 \m{O(E)} 時間內計算出圖 \m{G} 的所有銜接點。
\stopitem\stopigBase

\startANSWER
對於根節點,檢查有幾個子節點,如果多於一個,則爲銜接點,在(a)中已經證明。
對於其他節點 \m{v},如果 \m{v} 有一個孩子 \m{b},滿足 \m{low[b]\ge d[v]},
意味着以 \m{b} 爲根的子樹上沒有指向 \m{v} 的祖先的後向邊,
即 \m{v} 是銜接點,因爲刪除 \m{v} 會破壞圖的連通性。
\stopANSWER

\startigBase[continue]\startitem
證明:圖 \m{G} 的一條邊是橋當且僅當該邊不屬於 \m{G} 中的任何簡單環路。
\stopitem\stopigBase

\startANSWER
根據環路的定義,從環路中的任意節點開始,都可以返回到該節點的路。
因此,移除環路中的任意一條邊都不會孤立這個節點。
如果邊 \m{(u,v)} 沒有在環路中,
則只有一條從 \m{u} 到 \m{v} 的路。
如果可以從 \m{v} 沿其他路徑返回到 \m{u},則會形成一個環路。
因此,移除 \m{(u,v)} 會斷開 \m{u} 和 \m{v} 的連接,因此 \m{(u,v)} 是橋。
\stopANSWER

\startigBase[continue]\startitem
說明如何在 \m{O(E)} 時間內計算出圖 \m{G} 的所有橋。
\stopitem\stopigBase

\startANSWER
在 \ALGO{DFS-VISIT} 中,計算 \m{low[v]} 時,如果發現 \m{u} 的子節點 \m{v} 滿足 \m{low[v]>d[u]},
則我們可以任務此邊是橋,因爲移除他會導致 \m{u} 和 \m{v} 失連,這在(d)中已經證明。
\stopANSWER

\startigBase[continue]\startitem
證明: \m{G} 的雙連通分量是 \m{G} 的非橋邊的一個劃分。
\stopitem\stopigBase

\startANSWER
根據雙連通分量的定義,其內所有邊都不是橋,由由於其“最大”性,外部的邊都是橋。
因此他是 \m{G} 的非橋邊的一個劃分。
\stopANSWER

\startigBase[continue]\startitem
給出一個 \m{O(E)} 時間複雜度的算法來給圖 \m{G} 的每條邊 \m{e} 做出標記。
這個標記是一個正整數 \m{e.bcc} 且滿足 \m{e.bcc = e'.bcc} 當且僅當邊 \m{e} 和邊 \m{e'} 在同一個雙連通分量中。
\stopitem\stopigBase

\startANSWER
移除橋邊,遍歷銜接點。
\stopANSWER

\stopPROBLEM

%p22-3
\startPROBLEM
(Euler tour)
強連通有向圖 \m{G=(V,E)} 中的一個歐拉回路是指一條遍歷圖 \m{G} 中每條邊恰好一次的環路。
不過,這條環路可以多次訪問同一個節點。

\startigBase[a]\startitem
證明:圖 \m{G} 有一條歐拉回路當且僅當對於圖中的每個節點 \m{v},
有 \m{in-degree(v) = out-degree(v)}。
\stopitem\stopigBase

\startANSWER
如果圖 \m{G} 有歐拉回路 \m{C},所有節點的出度和入度顯然是相等的。

反過來,如果所有節點的入度和出度均相等,任取節點 \m{v},
則肯定存在一條路徑從 \m{v} 出發最後又返回到 \m{v}。
不停的刪除簡單環路,並刪除孤立的節點,然後再從被刪除了部分邊但還未被孤立的節點開始,
重新尋找簡單環路……
最終將所有簡單環路組合起來就是一條歐拉回路。
\stopANSWER

\startigBase[continue]\startitem
給出一個複雜度爲 \m{O(E)} 的算法來找出圖 \m{G} 的一條歐拉回路。
(\hint 對邊不相交環路進行歸併。)
\stopitem\stopigBase

\startANSWER
先以 \m{O(|E|)} 時間檢查是不是所有節點的入度都等於其出度。
如果是,然後再以不停搜索、刪除簡單環路的方式給出歐拉回路。
\stopANSWER

\stopPROBLEM

%p22-4
\startPROBLEM
(Reachability)
設 \m{G=(V,E)} 爲一個有向圖,
且每個節點 \m{u\in V} 都標有一個唯一的整數值標記 \m{L(u)},
 \m{L(u)} 的取值爲集合 \m{\{1,2,\ldots,|V|\}}。
對於每個節點 \m{u\in V},
設 \m{R(u)=\{v\in V: u\leadsto v\}} 爲從節點 \m{u} 可以到達的所有節點的集合。
定義 \m{\min(u)} 爲 \m{R(u)} 中標記爲最小的節點,
即 \m{\min(u)} 爲節點 \m{v},滿足 \m{L(v)=\min\{L(w): w\in R(u)\}}。
請給出一個時間複雜度爲 \m{O(V+E)} 的算法來計算所有節點 \m{u\in V} 的 \m{\min(u)}。
\stopPROBLEM

\startANSWER
從標記最小的節點開始反向深度優先搜索,路徑上的所有點的結果均爲起始點的標記值。
完成後,若還有僞確定的點,繼續執行上述過程。
\stopANSWER

\stopsubject%Problems
