\startsection[
  title={Depth-first search},
]

%e22.3-1
\startEXERCISE
劃一個 \m{3\times 3} 的網格,行和列的擡頭分別標記爲白色、灰色和黑色。
對於每個表單元 \m{(i,j)},
請指出在對有向圖進行深度有限搜索的過程中,是否可能存在一條邊,
鏈接一個顏色爲 \m{i} 的節點和一個顏色爲 \m{j} 的節點。
對於每種可能的邊,指明該種邊的類型。
另外,請針對無向圖的深度優先搜索再製作一張這樣的網格。
\stopEXERCISE

\startANSWER
\startcolumns[n=3, rule=on]
\startigBase[2]
\item T:樹邊
\item B:後向邊
\item F:前向邊
\item C:橫向邊
\stopigBase
\column
有向圖:

\startxtable[
    option=max,
    align={middle,lohi},
    split=yes,
    header=repeat,
    footer=repeat,
    offset=.25em,
]

% head
\startxtablehead[frame=off,bottomframe=on]
\startxrow[foregroundstyle=bold,]
  \xcell{\m{i,j}}\xcell{White}\xcell{Gray}\xcell{Black}
\stopxrow
\stopxtablehead

% body
\startxtablebody[frame=off]
\startxrow \xcell{White}\xcell{TBFC}\xcell{BC}  \xcell{C}   \stopxrow
\startxrow \xcell{Gray} \xcell{TF}  \xcell{TBF} \xcell{TFC} \stopxrow
\startxrow \xcell{Black}\xcell{}    \xcell{B}   \xcell{TBFC}\stopxrow
\stopxtablebody

\stopxtable

\column
無向圖:

\startxtable[
    option=max,
    align={middle,lohi},
    split=yes,
    header=repeat,
    footer=repeat,
    offset=.25em,
]

% head
\startxtablehead[frame=off,bottomframe=on]
\startxrow[foregroundstyle=bold,]
  \xcell{\m{i,j}}\xcell{White}\xcell{Gray}\xcell{Black}
\stopxrow
\stopxtablehead

% body
\startxtablebody[frame=off]
\startxrow \xcell{White}\xcell{TB}\xcell{TB}\xcell{}  \stopxrow
\startxrow \xcell{Gray} \xcell{TB}\xcell{TB}\xcell{TB}\stopxrow
\startxrow \xcell{Black}\xcell{}  \xcell{TB}\xcell{TB}\stopxrow
\stopxtablebody

\stopxtable

\stopcolumns
\stopANSWER

%e22.3-2
\startEXERCISE[exercise:22.3-2]
給出深度有限搜索算法在圖 22-6 上的運行過程。
假定深度有限搜索算法的第 5~7 行的 {\EMP for} 循環是以字母表順序依次處理每個節點,
並假定每條鄰接鏈表皆以字母表順序對裏面的節點進行了排序。
請給出每個節點的發現時間和完成時間,並給出每條邊的分類。
附圖 22-6:

\externalfigure[output/e22_3_2-1]
\stopEXERCISE

\startANSWER
如下圖,其中紅色的是樹邊,藍色的是後向邊,黑色的是前向邊,虛線是橫向邊。

\externalfigure[output/e22_3_2-2]
\stopANSWER

%e22.3-3
\startEXERCISE
給出圖 22-4 的深度優先搜索的括號化結構。
\stopEXERCISE

\startANSWER
\externalfigure[output/e22_3_3-1]
\stopANSWER

%e22.3-4
\startEXERCISE
證明:使用單個位來存放每個節點的顏色已經足夠。
這一點可以通過證明如下事實來得到:
如果將 \ALGO{DFS-VISIT} 的第 2 行刪除, \ALGO{DFS} 給出的結果相同。
\stopEXERCISE

\startANSWER
\TODO{略。}
\stopANSWER

%e22.3-5
\startEXERCISE
證明邊 \m{(u,v)} 是:
\startigBase[a]
\item 樹邊或前向邊當且僅當 \m{u.d < v.d < v.f < u.f};
\item 後向邊當且僅當 \m{v.d\le u.d < u.f \le v.f};
\item 橫向邊當且僅當 \m{v.d<v.f<u.d<u.f}。
\stopigBase
\stopEXERCISE

\startANSWER
\TODO{略。}
\stopANSWER

%e22.3-6
\startEXERCISE
證明:在無向圖中,根據深度有限搜索算法是先探索 \m{(u,v)} 還是先探索 \m{(v,u)} 来将邊 \m{(u,v)} 分類爲樹邊或者後向邊,
與根據分類列表中的 4 種類型的次序進行分類是等價的。
\stopEXERCISE

\startANSWER
\TODO{略。}
\stopANSWER

%e22.3-7
\startEXERCISE
請重寫 \ALGO{DFS} 算法的僞碼,以便使用棧來消除遞迴調用。
\stopEXERCISE

\startANSWER
\CLRSH{DFS-STACK(G, s)}
\startCLRS
for each vertex u in V(G) - {s}
	u.color = WHITE
	u.pred = NIL
time = 0
S = EMPTY

time = time + 1
s.td = time
s.color = GRAY
PUSH(S, s)
while S != EMPTY
	t = TOP(S)
	if exist v in V(G).Adj[t], v.color == WHITE	// v's adjacency list hasn't been fully examined
		v.pred = t
		time = time + 1
		v.td = time
		v.color = GRAY
		PUSH(S, v)
	else
		t = POP(S)
		time = time + 1
		v.tf = time
		v.color = BLACK
\stopCLRS
\stopANSWER

%e22.3-8
\startEXERCISE
請給出如下猜想的一個反例:
如果有向圖 \m{G} 包含一條從節點 \m{u} 到節點 \m{v} 的路徑,
並且在對圖 \m{G} 進行深度優先搜索時有 \m{u.d < v.d},
則節點 \m{v} 是節點 \m{u} 在深度優先森林中的一個後代。
\stopEXERCISE

\startANSWER
\externalfigure[output/e22_3_8-1]
\stopANSWER

%e22.3-9
\startEXERCISE
請給出如下猜想的一個反例:
如果有向圖 \m{G} 包含一條從節點 \m{u} 到節點 \m{v} 的路徑,
則任何對圖 \m{G} 的深度優先搜索都將導致 \m{v.d\le u.f}。
\stopEXERCISE

\startANSWER
參考上一題。
\stopANSWER

%e22.3-10
\startEXERCISE
修改深度優先搜索的僞碼,
讓其打印出有向圖 \m{G} 的每條邊及其分類。
並指出,如果圖 \m{G} 是無向圖,
要進行何種修改才能達到相同的效果。
\stopEXERCISE

\startANSWER
\TODO{略。}
\stopANSWER

%e22.3-11
\startEXERCISE
請解釋有向圖的一個節點 \m{u} 怎樣才能成爲深度優先樹中的唯一節點,
即使節點 \m{u} 同時有入邊和出邊。
\stopEXERCISE

\startANSWER
有頂點 \m{v_1,v_2,\ldots,v_k},存在邊 \m{(v_i,u)};
有頂點 \m{w_1,w_2,\ldots,w_l},存在邊 \m{(u,w_j)};
在執行 \ALGO{DFS} 時,可能先發現所有 \m{w_j},
然後再以 \m{u} 爲根進行搜索時, \m{u} 就會成爲一個孤立的節點。
\stopANSWER

%e22.3-12
\startEXERCISE
證明:我們可以在無向圖 \m{G} 上使用深度優先搜索來獲得圖 \m{G} 的連通分量,
並且深度優先森林所包含的樹的棵樹與 \m{G} 的連通分量數量相同。
更準確地說,請給出如何修改深度優先搜索來讓其給每個節點賦予一個介於 1 和 \m{k} 之間的整數值 \m{v.cc},
這裏 \m{k} 是 \m{G} 的連通分量數,
使得 \m{u.cc = v.cc} 當且僅當節點 \m{u} 和節點 \m{v} 處於同一個連通分量中。
\stopEXERCISE

\startANSWER
\TODO{略。}
\stopANSWER

%e22.3-13
\startEXERCISE\DIFFICULT
對於有向圖 \m{G=(V,E)} 來說,
如果 \m{u\leadsto v} 意味着圖 \m{G} 至多包含一條從 \m{u} 到 \m{v} 的簡單路徑,
則圖 \m{G} 是{\EMP 單連通圖}(singly connected)。
請給出一個有效算法來判斷一個有向圖是否是單連通圖。
\stopEXERCISE

\startANSWER
\TODO{略。}
\stopANSWER

\stopsection
