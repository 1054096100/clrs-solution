\startsection[
  reference={section:geometric_binom},
  title={The geometric and binomial distributions},
]

%eC.4-1
\startEXERCISE
對集合分佈驗證概率公理 2:
\startformula
\Pr\{S\} = 1
\stopformula
\stopEXERCISE

\startANSWER
\startformula\startmathalignment
\NC \sum_{k=1}^{\infty} \Pr\{X = k\}
    \NC = \sum_{k=1}^{\infty} q^{k-1}p \NR
    \NC = p \sum_{k=0}^{\infty} q^k \NR
    \NC = p \frac{1}{1-q} \NR
    \NC = \frac{p}{p} \NR
    \NC = 1 \NR
\stopmathalignment\stopformula
\stopANSWER

%eC.4-2
\startEXERCISE
現有 6 個均勻硬幣,平均要擲多少次才能得到 3 個正面和 3 個反面的結果?
\stopEXERCISE

\startANSWER
有 \m{\binom{6}{3}} 種方式得到想要的結果:
\startformula
\Pr\{\text{3H3T}\}
     = \frac{\binom{6}{3}}{2^6}
     = \frac{5}{16}
\stopformula
期望次數爲:
\startformula
\E[\text{3H3T}] = \frac{1}{\Pr\{\text{3H3T}\}} = \frac{16}{5} = 3.2
\stopformula
\stopANSWER

%eC.4-3
\startEXERCISE
證明: \m{b(k;n,p) = b(n-k;n,q)},其中 \m{q=1-p}。
\stopEXERCISE

\startANSWER
\startformula
b(k;n,p) = \binom{n}{k} p^k q^{n-k}
            = \binom{n}{n-k} p^k q^{n-k}
            = \binom{n}{n-k} q^{n-k} p^k
            = b(n-k;n,q)
\stopformula
\stopANSWER

%eC.4-4
\startEXERCISE
證明:二項分佈 \m{b(k;n,p)} 的最大值近似等於 \m{1/\sqrt{2\pi npq}},其中 \m{q=1-p}。
\stopEXERCISE

\startANSWER
\simpleurl{https://en.wikipedia.org/wiki/De_Moivre-Laplace_theorem}

令 \m{k=np + x\sqrt{npq}},其中 \m{x} 是一個固定值。
當 \m{n\rightarrow \infty} 時,
\startformula
\frac{
  \frac{1}{\sqrt{2\pi npq}} e^{\frac{-(k+1-np)^2}{2npq}}
}{
  \frac{1}{\sqrt{2\pi npq}} e^{\frac{-(k-np)^2}{2npq}}
}
  = e^{\frac{2np-2k-1}{2npq}}
  \simeq e^{\frac{-x}{\sqrt{npq}}}
  \simeq e^0
  = 1
\stopformula

首先,根據 Stirling 公式,當 \m{n\rightarrow \infty} 時:
\startformula
n! \simeq n^n e^{-n} \sqrt{2\pi n}
\stopformula
因此:
\startformula\startmathalignment
\NC \binom{n}{k}p^k q^{n-k}
    \NC = \frac{n!}{k!(n-k)!} p^k q^{n-k} \NR
\NC \NC \simeq \frac{
                  n^n e^{-n} \sqrt{2\pi n}
               }{
                  k^k e^{-k} \sqrt{2\pi k} (n-k)^{n-k} e^{-(n-k)} \sqrt{2\pi (n-k)}
               } p^k q^{n-k} \NR
\NC \NC = \sqrt{\frac{n}{2\pi k (n-k)}}
          \frac{n^n}{k^k (n-k)^{n-k}}
          p^k q^{n-k} \NR
\NC \NC = \sqrt{\frac{n}{2\pi k (n-k)}}
          \left(\frac{np}{k}\right)^k
          \left(\frac{nq}{n-k}\right)^{n-k} \NR
\stopmathalignment\stopformula

然後,根據 \m{\frac{k}{n}\rightarrow p}:
\startformula\startmathalignment
\NC \binom{n}{k}p^k q^{n-k}
    \NC \simeq \sqrt{\frac{1}{2\pi n\frac{k}{n}(1-\frac{k}{n})}}
               \left(\frac{np}{k}\right)^k \left(\frac{nq}{n-k}\right)^{n-k} \NR
\NC \NC \simeq \frac{1}{\sqrt{2\pi npq}}
               \left(\frac{np}{k}\right)^k \left(\frac{nq}{n-k}\right)^{n-k} \qquad p+q=1 \NR
\stopmathalignment\stopformula

最後根據 Taylor Series 可知:
\startformula
\ln(1+x)\simeq x - \frac{x^2}{2} + \frac{x^3}{3} - \ldots
\stopformula
所以:
\startformula\startmathalignment
\NC \binom{n}{k}p^k q^{n-k}
    \NC \simeq \frac{1}{\sqrt{2\pi npq}}
        \exp\left\{ \ln\left(\frac{np}{k}\right)^k
                    + \ln\left(\frac{nq}{n-k}\right)^{n-k} \right\} \NR
\NC \NC = \frac{1}{\sqrt{2\pi npq}}
        \exp\left\{ -k\ln\left(\frac{k}{np}\right)
                    + (k-n)\ln\left(\frac{n-k}{nq}\right) \right\} \NR
\NC \NC = \frac{1}{\sqrt{2\pi npq}}
        \exp\left\{ -k\ln\left(\frac{np + x\sqrt{npq}}{np}\right)
                    + (k-n)\ln\left(\frac{n-np-x\sqrt{npq}}{nq}\right) \right\} \NR
\NC \NC = \frac{1}{\sqrt{2\pi npq}}
        \exp\left\{ -k\ln\left(1 + x\sqrt{\frac{q}{np}}\right)
                    + (k-n)\ln\left(1-x\sqrt{\frac{p}{nq}}\right) \right\} \qquad p+q=1 \NR
\NC \NC = \frac{1}{\sqrt{2\pi npq}}
        \exp\left\{ -k\left(x\sqrt{\frac{q}{np}} - \frac{x^2q}{2np} + \ldots \right)
                    + (k-n)\left(-x\sqrt{\frac{p}{nq}} - \frac{x^2p}{2nq} - \ldots \right) \right\} \NR
\NC \NC = \frac{1}{\sqrt{2\pi npq}}
        \exp\left\{ (-np - x\sqrt{npq})\left(x\sqrt{\frac{q}{np}} - \frac{x^2q}{2np} + \ldots \right) \right. \NR
\NC \NC \hskip 10em \left. + (np + x\sqrt{npq} - n)\left(-x\sqrt{\frac{p}{nq}} - \frac{x^2p}{2nq} - \ldots \right) \right\} \NR
\NC \NC = \frac{1}{\sqrt{2\pi npq}}
        \exp\left\{ (-np - x\sqrt{npq})\left(x\sqrt{\frac{q}{np}} - \frac{x^2q}{2np} + \ldots \right) \right. \NR
\NC \NC \hskip 10em \left. - (nq - x\sqrt{npq})\left(-x\sqrt{\frac{p}{nq}} - \frac{x^2p}{2nq} - \ldots \right) \right\} \NR
\NC \NC = \frac{1}{\sqrt{2\pi npq}}
        \exp\left\{ \left( -x\sqrt{npq} + \frac{1}{2}x^2q - x^q + \ldots \right)
                    + \left( x\sqrt{npq} + \frac{1}{2}x^2p - x^p - \ldots \right) \right\} \NR
\NC \NC = \frac{1}{\sqrt{2\pi npq}}
        \exp\left\{ -\frac{1}{2}x^2q - \frac{1}{2}x^2p - \ldots \right\} \NR
\NC \NC = \frac{1}{\sqrt{2\pi npq}}
        \exp\left\{ -\frac{1}{2}x^2(p+q) - \ldots \right\} \NR
\NC \NC \simeq \frac{1}{\sqrt{2\pi npq}}
        \exp\left\{ -\frac{1}{2}x^2 \right\} \NR
\NC \NC = \frac{1}{\sqrt{2\pi npq}}
        e^{ \frac{-(k-np)^2}{2npq} } \NR
\stopmathalignment\stopformula

\stopANSWER

%eC.4-5
\startEXERCISE\DIFFICULT
證明: \m{n} 次 Bernoulli 試驗(成功概率爲 \m{p=1/n})
一次也未成功的概率近似等於 \m{1/e}。
證明:只有一次成功的概率也近似爲 \m{1/e}。
\stopEXERCISE

\startANSWER
\m{n} 次均未成功的概率爲:
\startformula
b(0;n,p)
  = \left.\binom{n}{k} p^k q^{n-k}\right|_{k=0,p=1/n}
  = \binom{n}{0} q^n
  = (1-p)^n
  = \left(1 - \frac{1}{n}\right)^n
  = \frac{1}{\left(1-\frac{1}{n}\right)^{-n}}
\stopformula

\m{n} 次中只有一次成功的概率爲:
\startformula
b(1;n,p)
  = \binom{n}{1} \frac{1}{n} \left(1-\frac{1}{n}\right)^{n-1}
  = \frac{1}{\left(1-\frac{1}{n}\right)^{-n-1}}
\stopformula

根據下面公式可知上面兩個值近似均爲 \m{1/e}。
\startformula
e = \lim_{n \to \infty}\left(1 + \frac{1}{n}\right)^n
     = \lim_{n \to \infty}\left(1 - \frac{1}{n}\right)^{-n}
\stopformula
\stopANSWER

%eC.4-6
\startEXERCISE\DIFFICULT
Rosencrantz 教授與 Guildenstern 教授各扔一枚均勻硬幣 \m{n} 次。
證明:他們得到正面朝上次數相同的概率爲 \m{\binom{2n}{n}/4^n}。
(\hint 對於 Rosencrantz 教授,稱正面爲成功;
對於 Guildenstern 教授,稱背面爲成功。)
並用你的結論在驗證恆等式:
\startformula
\sum_{k=0}^{n}\binom{n}{k}^2 = \binom{2n}{n}
\stopformula
\stopEXERCISE

\startANSWER
根據提示,總共 \m{2n} 次,要想兩個教授正面朝上次數相同,
如果 Rosencrantz 教授成功了 \m{k} 次,
則 Guildenstern 教授成功了 \m{n-k} 次,
共成功 \m{n} 次。
\startformula
\Pr\{R=G\} = b(n;2n,1/2)
           = \binom{2n}{n}\frac{1}{2^n}\frac{1}{2n}
           = \binom{2n}{n}/4^n
\stopformula
也可以通過概率來計算:
\startformula\startmathalignment
\NC \Pr\{R=G\}
    \NC = \sum_{k=0}^n \Pr\{R=k\}\Pr\{G=n-k\} \NR
\NC \NC = \sum_{k=0}^n \binom{n}{k} \cdot \frac{1}{2^n} \cdot
                               \binom{n}{n-k} \cdot \frac{1}{2^n} \NR
\NC \NC = \frac{1}{4^n} \sum_{k=0}^n \binom{n}{k} \binom{n}{n-k} \NR
\NC \NC = \frac{1}{4^n} \sum_{k=0}^n \binom{n}{k}^2 \NR
\stopmathalignment\stopformula
\stopANSWER

%eC.4-7
\startEXERCISE\DIFFICULT
證明:對於 \m{0\le k\le n},
\startformula
b(k;n,1/2) \le 2^{n H(k/n) - n}
\stopformula
其中 \m{H(x)} 是熵函數:
\startformula
H(\lambda) = -\lambda\lg{\lambda} - (1 - \lambda)\lg(1 - \lambda) \eqno{(C.7)}
\stopformula
\stopEXERCISE

\startANSWER
\startformula
b(k;n,1/2) = \binom{n}{k}\frac{1}{2^n}
           = \binom{n}{n\frac{k}{n}}\frac{1}{2^n}
           \le \frac{2^{nH(k/n)}}{2^n}
           = 2^{nH(k/n) - n}
\stopformula
\stopANSWER

%eC.4-8
\startEXERCISE
考慮 \m{n} 次 Bernoulli 試驗,
其中 \m{i=1,2,\ldots,n},
第 \m{i} 次試驗成功的概率爲 \m{p_i},
令 \m{X} 爲表示總成功次數的隨機變量。
令對所有 \m{i=1,2,\ldots,n} 有 \m{p\ge p_i}。
證明:對於 \m{1\le k\le n},
\startformula
\Pr\{X<k\} \ge \sum_{i=0}^{k-1} b(i;n,p)
\stopformula
\stopEXERCISE

\startANSWER
另外一組 Bernoulli 試驗 \m{Y},每次成功的概率均爲 \m{p},
利用\refexercise{C.4-9} 的結論有:
\startformula
\Pr\{Y < k\} = 1 - \Pr\{Y \ge k\} \le 1 - \Pr\{X \ge k\} = \Pr\{X < k\}
\stopformula
\stopANSWER

%eC.4-9
\startEXERCISE[exercise:C.4-9]\DIFFICULT
令 \m{X} 爲表示 \m{n} 次 Bernoulli 試驗構成的集合 \m{A} 中總成功次數的隨機變量,
其中第 \m{i} 次試驗成功的概率爲 \m{p_i},
並令 \m{X'} 爲表示另一個由 \m{n} 個 Bernoulli 試驗構成的集合 \m{A'} 中
總成功次數的隨機變量,
其中第 \m{i} 次試驗成功的概率 \m{p'_i \ge p_i}。
證明:對於 \m{0\le k\le n},
\startformula
\Pr\{X'\ge k\}\ge \Pr\{X\ge k\}
\stopformula
(\hint 說明如何通過包含 \m{A} 中試驗的實驗
來獲得 \m{A'} 中的 Bernoulli 試驗,
並使用\refexercise{C.3-7} 中的結論。)
\stopEXERCISE

\startANSWER
令 \m{Y_1,Y_2,\ldots,Y_n} 爲 \m{A} 中事件的指示隨機變量。
令 \m{Z_1,Z_2,\ldots,Z_n} 爲新的隨機變量:
\startigBase[a]
\item 如果 \m{Y_i = 1},則 \m{Z_i = 1};
\item 如果 \m{Y_i = 0},則 \m{Z_i = 1} 的概率爲 \m{\frac{p'_i - p_i}{1-p_i}}。
\stopigBase
下面計算 \m{\Pr\{Z_i\}}:
\startformula
\Pr\{Z_i\} = p_i \Pr\{Y_i = 1\} + (1 - p_i) \cdot \frac{p_i' - p_i}{1-p_i}
                = p_i + p_i' - p_i
                = p_i'
\stopformula

由於 \m{p_i'\ge p_i},則:
\startformula
\Pr\{X' \ge k\} = \Pr\{Z_1 + Z_2 + \ldots + Z_n\}
                \ge \Pr\{Y_1 + Y_2 + \ldots + Y_n\}
                = \Pr\{X \ge k\}
\stopformula
\stopANSWER

\stopsection
