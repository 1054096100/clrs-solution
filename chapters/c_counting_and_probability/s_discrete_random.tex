\startsection[
  title={Discrete random variables},
]

%eC.3-1
\startEXERCISE
投擲兩個普通的 6 面體骰子。
兩個骰子值相加,所得和的期望值是多少?
兩個值中的最大值的期望值又是多少?
\stopEXERCISE

\startANSWER
和的期望值:
\startformula\startmathalignment
\NC \E[X] \NC = \sum_{x=2}^{12}x\Pr\{X = x\} \NR
\NC \NC =  2 \cdot \frac{1}{36} +
           3 \cdot \frac{2}{36} +
           4 \cdot \frac{3}{36} +
           5 \cdot \frac{4}{36} +
           6 \cdot \frac{5}{36} +
           7 \cdot \frac{6}{36} +
           8 \cdot \frac{5}{36} +
           9 \cdot \frac{4}{36} +
          10 \cdot \frac{3}{36} +
          11 \cdot \frac{2}{36} +
          12 \cdot \frac{1}{36} \NR
\NC \NC = 7 \NR
\stopmathalignment\stopformula

較大值的期望值:
\startformula\startmathalignment
\NC \E[Y]
    \NC = \sum_{i=1}^{6}i\Pr\{Y = i\} \NR
\NC \NC = \sum_{i=1}^{6}\left(i \cdot \frac{2i - 1}{36}\right) \NR
\NC \NC = \frac{2\sum{i^2} - \sum{i}}{36} \NR
\NC \NC = \frac{(6 \cdot 7 \cdot 13)/6 - (6 \cdot 7)/2}{36} \NR
\NC \NC = \frac{161}{36} = 4.47\ldots \NR
\stopmathalignment\stopformula
\stopANSWER

%eC.3-2
\startEXERCISE
數組 \m{A[1..n]} 包含 \m{n} 個不同數字,且順序隨機,
每種排列均爲等可能的。
該數組中最大元素下標的期望值是多少?
最小元素下標值的期望是多少?
\stopEXERCISE

\startANSWER
最大值下標爲 \m{i} 的概率爲 \m{\Pr\{X=i\}=\frac{1}{n}}。
\startformula\startmathalignment
\NC \E[X]
    \NC = \sum_{i=1}^n i \cdot \Pr\{X = i\} \NR
\NC \NC = \sum_{i=1}^n i \cdot \frac 1 n \NR
\NC \NC = \frac 1 n \sum_{i=1}^n i \NR
\NC \NC = \frac 1 n \frac{n(n+1)}{2} \NR
\NC \NC = \frac{n+1}{2} \NR
\stopmathalignment\stopformula

最小值下標的期望與最大值相同,也是 \m{\frac{n+1}{2}}。
\stopANSWER

%eC.3-3
\startEXERCISE
在一場狂歡節遊戲中,將 3 個骰子放在一個罩子中。
一位遊戲者可以在 1 到 6 中的任意數字上賭 1 美元。
主持人搖罩子,並按如下方案確定遊戲者的回報。
如果遊戲者賭的數字沒有出現在任何一個骰子上,
則他輸掉 1 美元。
如果他賭的數字恰好出現在 \m{k} 個骰子上,
 \m{k=1,2,3},則他可以保留他的 1 美元,並贏得 \m{k} 美元。
請計算玩一次這個遊戲的期望收入。
\stopEXERCISE

\startANSWER
概率如下:
\startformula\startmathalignment
\NC \Pr\{X = 3\} \NC = \frac{1}{6} \cdot \frac{1}{6} \cdot \frac{1}{6}
                     = \frac{1}{216} \NR
\NC \Pr\{X = 2\} \NC = \binom{3}{2}
                       \cdot \frac{1}{6} \cdot \frac{1}{6} \cdot \frac{5}{6}
		     = \frac{15}{216} \NR
\NC \Pr\{X = 1\} \NC = \binom{3}{1}
                       \cdot \frac{1}{6} \cdot \frac{5}{6} \cdot \frac{5}{6}
		     = \frac{75}{216} \NR
\NC \Pr\{X = -1\} \NC = \frac{5}{6} \cdot \frac{5}{6} \cdot \frac{5}{6}
		     = \frac{125}{216} \NR
\stopmathalignment\stopformula

\startformula\startmathalignment
\NC \E[X] \NC = -1 \cdot \Pr\{X = 0\} +
             1 \cdot \Pr\{X = 1\} +
             2 \cdot \Pr\{X = 2\} +
             3 \cdot \Pr\{X = 3\} \NR
\NC \NC = - 1 \cdot \frac{125}{216}
            + 1 \cdot \frac{75}{216}
            + 2 \cdot \frac{15}{216}
            + 3 \cdot \frac{1}{216} \NR
\NC \NC = - \frac{17}{216} \NR
\NC \NC \approx -0.07 \NR
\stopmathalignment\stopformula
\stopANSWER

%eC.3-4
\startEXERCISE
證明:
若 \m{X} 和 \m{Y} 是非負隨機變量,則:
\startformula
\E[\max(X,Y)] \le \E[X] + \E[Y]
\stopformula
\stopEXERCISE

\startANSWER
\startformula\startmathalignment
\NC \E[\max(X, Y)] \NC = \sum n \cdot \max(\Pr\{X = n\}, \Pr\{Y = n\}) \NR
\NC \NC \le \sum n \cdot (\Pr\{X=n\} + \Pr\{Y=n\}) \NR
\NC \NC = \sum n\cdot \Pr\{X=n\} + \sum n\cdot \Pr\{Y=n\} \NR
\NC \NC = \E[X] + \E[Y] \NR
\stopmathalignment\stopformula
\stopANSWER

%eC.3-5
\startEXERCISE\DIFFICULT
令 \m{X} 和 \m{Y} 是獨立隨機變量。
證明對於任何函數 \m{f} 和 \m{g},
 \m{f(x)} 和 \m{g(Y)} 是獨立的。
\stopEXERCISE

\startANSWER
\TODO{略。}
\stopANSWER

%eC.3-6
\startEXERCISE\DIFFICULT
令 \m{X} 是非負隨機變量,
並假定 \m{\E[X]} 是有定義的。
證明 {\EMP Markov 不等式}:對於所有 \m{t>0},
\startformula
\Pr\{X\ge t\}\le \frac{\E[X]}{t} \eqno{(C.30)}
\stopformula
\stopEXERCISE

\startANSWER
\startformula\startmathalignment
\NC \E[X] \NC = \sum_{x}x \cdot \Pr\{X=x\} \NR
\NC \NC = \sum_{x < t}x \cdot \Pr\{X = x\} + \sum_{x \ge t} x \cdot \Pr\{X = x\} \NR
\NC \NC \ge \sum_{x < t}x \cdot \Pr\{X = x\} + \sum_{x \ge t} t \cdot \Pr\{X = x\} \NR
\NC \NC \ge t \sum_{x \ge t} \Pr\{X = x\} \NR
\NC \NC = t \cdot \Pr\{X \ge t\} \NR
\stopmathalignment\stopformula

\startformula\startmathalignment[n=1]
\NC \E[X] \ge t \cdot \Pr\{X \ge t\} \NR
\NC \Downarrow \NR
\NC \Pr\{X \ge t\} \le \E[X]/t \NR
\stopmathalignment\stopformula
\stopANSWER

%eC.3-7
\startEXERCISE[exercise:C.3-7]\DIFFICULT
令 \m{S} 爲樣本空間, \m{X} 和 \m{X'} 是隨機變量,
滿足對於所有 \m{s\in S},有 \m{X(s)\ge X'(s)}。
證明:對於任意實常數 \m{t},
\startformula
\Pr\{X\ge t\} \ge \Pr\{X'\ge t\}
\stopformula
\stopEXERCISE

\startANSWER
\startformula\startmathalignment
\NC \Pr\{X \ge t\} \NC = \sum_{s \in S:X(s) \ge t}\Pr\{s\} \NR
\NC \Pr\{X' \ge t\} \NC = \sum_{s \in S:X'(s) \ge t}\Pr\{s\} \NR
\stopmathalignment\stopformula
由於 \m{X(s)\ge X'(s)},第二個和式的任何一項都包含在第一個和式中。
也就是說第一個和式的值不會比第二個和式小。
\stopANSWER

%eC.3-8
\startEXERCISE
一個隨機變量的平方的期望與其期望的平方哪個大?
\stopEXERCISE

\startANSWER
由於 \m{\E[f(X)]\ge f(\E[X])},當 \m{f(x)=x^2} 時,有 \m{\E[X^2]\ge E^2[X]}。
\stopANSWER

%eC.3-9
\startEXERCISE
證明:對於任意取值僅爲 0 或 1 的隨機變量 \m{X},
有 \m{\Var[X] = \E[X] \E[1-X]}。
\stopEXERCISE

\startANSWER
\startformula\startmathalignment
\NC \E[X] \NC = 0 \cdot \Pr\{X = 0\} + 1 \cdot \Pr\{X = 1\} = \Pr\{X = 1\} \NR
\NC \E[1-X] \NC = \Pr\{X = 0\} \NR
\NC \E[X]\E[1-X] \NC = \Pr\{X = 0\} \cdot \Pr\{X = 1\} \NR
\stopmathalignment\stopformula
據此有:
\startformula\startmathalignment
\NC \Var[X]
    \NC = \E[X^2] - E^2[X] = \Pr\{X = 1\} - (\Pr\{X = 1\})^2 \NR
\NC \NC = \Pr\{X = 1\} (1 - \Pr\{X = 1\}) \NR
\NC \NC = \Pr\{X = 0\} \cdot Pr\{X = 1\} \NR
\stopmathalignment\stopformula
\stopANSWER

%eC.3-10
\startEXERCISE
根據方差定義(公式(C.27))證明: \m{\Var[\alpha X] = \alpha^2\Var[X]}。
附公式(C.27):
\startformula\startmathalignment
\NC \Var[X]
    \NC = E[(X-E[X])^2] \NR
\NC \NC = E[X^2 - 2X E[X] + E^[X]] \NR
\NC \NC = E[X^2] - 2E[ X E[X] ] + E^2[X] \NR
\NC \NC = E[X^2] - 2E^2[X] + E^2[X] \NR
\NC \NC = E[X^2] - E^2[X] \NR
\stopmathalignment\stopformula
\stopEXERCISE

\startANSWER
\startformula\startmathalignment
\NC \Var[\alpha X]
    \NC = \E[\alpha^2 X^2] - \E^2[\alpha X] \NR
\NC \NC = \alpha^2 \E[X^2] - \alpha^2\E[X] \NR
\NC \NC = \alpha^2 (\E[X^2] - \E^2[X]) \NR
\NC \NC = \alpha^2 \Var[X] \NR
\stopmathalignment\stopformula
\stopANSWER

\stopsection
