\startsubject[
  title={Problems},
]

%pC-1
\startPROBLEM
(Balls and bins)
本題中,我們會研究在幾種假設條件下,
有多少種方法可以將 \m{n} 個球放入 \m{b} 個箱子裏。
\startigBase[a]\startitem
假定 \m{n} 個球是不同的,
且不考慮他們在盒子中的順序。
證明:有 \m{b^n} 種放置方法。
\stopitem\stopigBase

\startANSWER
每個球都有 \m{b} 種放法,一共 \m{b^n} 種放法。
\stopANSWER

\startigBase[continue]\startitem
假定 \m{n} 個球是不同的,且在盒子中有序。
證明:有 \m{(b+n-1)!/(b-1)!} 種放置方法。
(\hint 考慮有多少種方法可以將 \m{n} 個不同的球
和 \m{b-1} 根相同的棍子排成一排。)
\stopitem\stopigBase

\startANSWER
將 \m{n} 個球和 \m{b-1} 根棍子排成一排,有 \m{(n+b-1)!} 種方法。
所有棍子都是相同的,棍子排列方法爲 \m{(b-1)!} 種。
兩個棍子中間的球放入對應的箱子裏。
\stopANSWER

\startigBase[continue]\startitem
假定 \m{n} 個球是相同的,從而無需考慮他們在盒子中的順序。
證明:有 \m{\binom{b+n-1}{n}} 種放置方法。
(\hint 若球相同,則(b)中的排列有多少是重複的?)
\stopitem\stopigBase

\startANSWER
球的排列有 \m{n!} 種是重複的。
因此:
\startformula
\frac{(b+n-1)!}{(b-1)!n!} = \binom{b+n-1}{n}
\stopformula
\stopANSWER

\startigBase[continue]\startitem
假定球是相同的,每個盒子只能放一個球,從而有 \m{n\le b}。
證明:將球放入盒子中的方法數是 \m{\binom{b}{n}}。
\stopitem\stopigBase

\startANSWER
在 \m{b} 個箱子中選擇 \m{n} 個用來放球,即 \m{\binom{b}{n}}。
\stopANSWER

\startigBase[continue]\startitem
假定球是相同的,且盒子不能爲空,即 \m{n\ge b},
證明:有 \m{\binom{n-1}{b-1}} 種放置方法。
\stopitem\stopigBase

\startANSWER
先將球排成一列,將 \m{b-1} 根棍子放入 \m{n-1} 個空隙。
即 \m{\binom{n-1}{b-1}}。
\stopANSWER
\stopPROBLEM

\stopsubject%Problems
