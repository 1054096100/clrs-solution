\startsection[
  title={Probability},
]

%eC.2-1
\startEXERCISE
Rosencrantz 教授拋擲一枚均勻硬幣一次。
 Guildenstern 教授拋擲一枚均勻硬幣兩次。
 Rosencrantz 教授得到的正面朝上次數多於 Guildenstern 教授的概率是多少?
\stopEXERCISE

\startANSWER
令 \m{G_n} 和 \m{R_n} 分別爲兩個教授證明朝上的次數。則:
\startformula
\Pr\{R_n = 1\} \cdot \Pr\{G_n = 0\} = \frac{1}{2} \cdot \frac{1}{4} = \frac{1}{8}
\stopformula
\stopANSWER

%eC.2-2
\startEXERCISE
證明 {\EMP Boole 不等式}:對於有限或可數無限事件序列 \m{A_1,A_2,\ldots},
\startformula
\Pr\{A_1 \cup A_2 \cup \ldots \} \le \Pr\{A_1\} + \Pr\{A_2\} + \ldots
\stopformula
\stopEXERCISE

\startANSWER
\startformula\startmathalignment
\NC \Pr\{A_1 \cup A_2 \cup \cdots\}
    \NC = \Pr\{A_1\}
        + \Pr\{A_2 \cup A_3 \cup \cdots\}
	- \Pr\{A_1 \cap (A_2 \cup A_3 \cup \cdots)\} \NR
\NC \NC \le \Pr\{A_1\} + \Pr\{A_2 \cup A_3 \cup \cdots\} \NR
\NC \NC \le \Pr\{A_1\} + \Pr\{A_2\} + \Pr\{A_3 \cup A_4 \cup \cdots\} \NR
\NC \NC \qquad \vdots \NR
\NC \NC \le \Pr\{A_1\} + \Pr\{A_2\} + \ldots \NR
\stopmathalignment\stopformula
\stopANSWER

%eC.2-3
\startEXERCISE
假設有 10 張牌,每張牌上分別標有從 1 到 10 的數字,
且每張牌上的數字均不同,將牌充分混合。
然後從牌堆頂部一次一張地移除三張牌。
那麼我們選出的三張牌按照(遞增)順序排列的概率是多少?
\stopEXERCISE

\startANSWER
有 \m{n!/(n-k)!} 種方式拾取 \m{k} 張牌。其中只有 \m{\binom{n}{k}} 種按遞增排列。
因此概率爲:
\startformula
\binom{n}{k} \div \frac{n!}{(n-k)!} = \frac{n!(n-k)!}{k!(n-k)!n!} = \frac{1}{k!}
\stopformula

也可以這樣計算:共有 \m{n!} 種排列方式。
有 \m{\binom{n}{k}} 種方式選取 \m{k} 張牌,剩下的牌有 \m{(n-k)!} 種排列方式。
因此總的概率爲 \m{\binom{n}{k}\cdot (n-k)! / n! = \frac{1}{k!}}。
\stopANSWER

%eC.2-4
\startEXERCISE
證明:
\startformula
\Pr\{A|B\} + \Pr\{\overbar{A}|B\} = 1
\stopformula
\stopEXERCISE

\startANSWER
\startformula\startmathalignment
\NC \Pr\{A|B\} + \Pr\{\overline A | B\}
    \NC = \frac{\Pr\{A \cap B\}}{\Pr\{B\}} +
          \frac{\Pr\{\overline A \cap B\}}{\Pr\{B\}} \NR
\NC \NC = \frac{\Pr\{A \cap B\} + \Pr\{\overline A \cap B\}}{\Pr\{B\}} \NR
\NC \NC = \frac{\Pr\{(\overline A \cap B) \cup (A \cap B)\}}{\Pr\{B\}} \NR
\NC \NC = \frac{\Pr\{(A \cup \overline A) \cap B)}{\Pr\{B\}} \NR
\NC \NC = \frac{\Pr\{B\}}{\Pr\{B\}} \NR
\NC \NC = 1 \NR
\stopmathalignment\stopformula
\stopANSWER

%eC.2-5
\startEXERCISE
證明:對於任意事件集 \m{A_1,A_2,\ldots,A_n},
\startformula
  \Pr\{A_1 \cap A_2 \cap \cdots \cap A_n\}
= \Pr\{A_1\} \cdot \Pr\{A_2 | A_1\} \cdot
  \Pr\{A_3 | A_1 \cap A_2\} \cdots
  \Pr\{A_n | A_1 \cap A_2 \cap \cdots \cap A_{n-1}\}
\stopformula
\stopEXERCISE

\startANSWER
\startformula\startmathalignment
\NC \Pr\{A_1 \cap A_2 \cap \cdots\}
    \NC = \Pr\{A_n | A_1 \cap \cdots \cap A_{n-1}\} \cdot
          \Pr\{A_1 \cap \cdots \cap A_{n-1}\} \NR
\NC \NC = \Pr\{A_n | A_1 \cap \cdots \cap A_{n-1}\} \cdot
        \Pr\{A_{n-1} | A_1 \cap \cdots \cap A_{n-2}\} \cdot
        \Pr\{A_1 \cap \cdots \cap A_{n-2}\} \NR
\NC \NC \qquad \vdots \NR
\NC \NC = \Pr\{A_n | A_1 \cap \cdots \cap A_{n-1}\} \cdots
        \Pr\{A_3 | A_1 \cap A_2\} \cdot
        \Pr\{A_2 | A_1\} \cdot
        \Pr\{A_1\} \NR
\stopmathalignment\stopformula
\stopANSWER

%eC.2-6
\startEXERCISE\DIFFICULT
描述一個以整數 \m{a} 和 \m{b} 爲輸入的過程,其中 \m{0<a<b}。
在這個過程中,拋擲均勻硬幣,
結果正面朝上的概率爲 \m{a/b},
反面朝上的概率爲 \m{(b-a)/b}。
請給出拋擲硬幣次數期望的界,應爲 \m{O(1)}。
(\hint 將 \m{a/b} 表示爲二進制。)
\stopEXERCISE

\startANSWER
將 \m{a/b} 表示爲二進制 \m{0.b_1b_2b_3b_4\ldots}。
其中 \m{b_i} 中的 \m{i} 表示第幾次拋硬幣。
如果 \m{b_i = 0},但是拋硬幣得到的是反面,則返回反面;
如果 \m{b_i = 1},但是拋硬幣得到的是正面,則返回正面;
其他情況都繼續拋硬幣。
\stopANSWER

%eC.2-7
\startEXERCISE\DIFFICULT
請給出一種方法,用於構造滿足下列條件的集合:
集合中元素爲兩兩獨立的 \m{n} 個事件,
但不存在包含 \m{k>2} 個相互獨立元素的事件子集。
\stopEXERCISE

\startANSWER
構造一個有 \m{n^2} 個面的骰子。
選取 \m{n} 種顏色,將每種顏色塗到 \m{n-1} 個面上。
在剩下的 \m{n} 個面中選取一個塗上所有顏色,
最後剩下的 \m{n-1} 個面不塗任何顏色。

\startxtable[
    option=max,
    align={middle,lohi},
    split=yes,
    header=repeat,
    footer=repeat,
    offset=.25em,
]

% head
\startxtablehead[frame=off,bottomframe=on]
\startxrow[foregroundstyle=bold,]
  \xmcell[rightframe=on]{}\processcommalist[ 1,2,3,\cdots,n]\xmcell
\stopxrow
\stopxtablehead

% body
\startxtablebody[frame=off]
\startxrow \xmcell[rightframe=on]{1}\processcommalist[ 1,2,3,\cdots,n]\xmcell \stopxrow
\startxrow \xmcell[rightframe=on]{2}\processcommalist[ 1,2,3,\cdots,n]\xmcell \stopxrow
\startxrow \xmcell[rightframe=on]{3}\processcommalist[ 1,2,3,\cdots,n]\xmcell \stopxrow
\startxrow \xmcell[rightframe=on]{\vdots}\processcommalist[\vdots,\vdots,\vdots,\vdots,\vdots]\xmcell \stopxrow
\startxrow \xmcell[rightframe=on]{n-1}\processcommalist[1,2,3,\cdots,n]\xmcell \stopxrow
\startxrow \xmcell[rightframe=on]{n}\processcommalist[all,-,-,\cdots,-]\xmcell \stopxrow
\stopxtablebody

\stopxtable


事件 \m{i}:擲骰子,所得面包含第 \m{i} 種顏色。
則任一事件 \m{i} 的概率都是 \m{(n-1 + 1)/n^2 = 1/n}。
任兩個事件同時發生的概率都是 \m{1/n^2}。
所以這些事件兩兩相互獨立。
而任意 \m{k} 個事件同時發生的概率仍然是 \m{1/n^2},
所以當 \m{k>2} 時,這 \m{k} 個事件並不互相獨立。
\stopANSWER

%eC.2-8
\startEXERCISE\DIFFICULT
已知事件 \m{C} 已發生,如果
\startformula
\Pr\{A \cap B | C\} = \Pr\{A | C\} \cdot \Pr\{B | C\}
\stopformula
則稱事件 \m{A} 和事件 \m{B} 是 {\EMP conditionally independent} 的。
請給一個簡單但非平凡的例子,
其中兩個事件不獨立,但是在已知第三個事件發生時卻獨立。
\stopEXERCISE

\startANSWER
\startigBase[A]
\item 第一個人擲得正面;
\item 第二個人擲得正面;
\item 硬幣是非均勻的(每次擲都會得到正面)——概率是 \m{1/3}。
\stopigBase
因此:
\startformula\startmathalignment[n=1]
\NC \Pr\{A\} = \frac 1 3 \cdot 1 + \frac 2 3 \cdot \frac 1 2 = \frac 2 3 \NR
\NC \Pr\{B\} = \frac 1 3 \cdot 1 + \frac 2 3 \cdot \frac 1 2 = \frac 2 3 \NR
\NC \Pr\{A \cap B\} = \frac 1 3 \cdot 1 + \frac 2 3 \cdot \frac 1 4 = \frac 1 2 \NR
\stopmathalignment\stopformula
\m{A} 和 \m{B} 不獨立,因爲:
\startformula
\frac 1 2 = \Pr\{A \cap B\} \neq \Pr\{A\} \cdot \Pr\{B\} = \frac 4 9
\stopformula
但如果 \m{C} 已發生,則 \m{A} 和 \m{B} 相互獨立:
\startformula
1 = \Pr\{A \cap B | C \} = \Pr\{A | C\} \cdot \Pr\{B | C\} = 1
\stopformula
\stopANSWER

%eC.2-9
\startEXERCISE\DIFFICULT
你參加了一個遊戲。
該遊戲將獎品藏在了三個幕布之後。
如果你選對了幕布,則可以贏得獎品。
在你選擇了一個幕布,但還未揭開時,
主持人會揭開另兩個幕布中的一個空幕布
(主持人直到哪個幕布後是空的),
之後會詢問你要不要改變你的選擇。
請問如果你改變了選擇,
那麼你贏得獎品的機率將如何改變?
(這一問題是著名的 {\EMP Monty hall 問題},
是以一個主持人經常讓參賽者陷入這種困境的節目命名的。)
\stopEXERCISE

\startANSWER
如果你已經選擇了有獎品的幕布(概率爲 1/3),則改變選擇會導致得不到獎品。

如果你的選擇沒有獎品(概率爲 2/3),則改變選擇必然會得到獎品。

也就是說改變選擇能得到獎品的概率爲 2/3。

\SIMPLEURL{https://en.wikipedia.org/wiki/Monty_Hall_problem}
\stopANSWER

%eC.2-10
\startEXERCISE
一個監獄看守從三個罪犯中隨機挑選一個釋放,並處理另外兩個。
這個看守知道每個人會被釋放還是處死,但禁止將此信息透漏給囚犯。
令三個罪犯分別爲 \m{X}、 \m{Y} 和 \m{Z}。
罪犯 \m{X} 已經知道了 \m{Y} 和 \m{Z} 中至少有一人會死,
並以此爲由私下問警衛兩人中哪個會被處死。
警衛不能透漏給 \m{X} 關於他自身的信息,
但他告訴 \m{X}, \m{Y} 將被處死。
 \m{X} 感到很開心,因爲他認爲他或者 \m{Z} 將被釋放,
這意味着他被釋放的概率現在是 \m{1/2} 了。
請問他的想法正確嗎?或者他被釋放的概率仍爲 \m{1/3}?
請解釋。
\stopEXERCISE

\startANSWER
顯然在 \m{X} 問警衛之前,所有事件都已確定。
 \m{X} 是否得到額外信息對結果沒有任何影響。

\startigBase[a]
\item 事件 y:警衛說 \m{Y} 會被處死;
\item \m{X}、 \m{Y} 和 \m{Z} 代表各自被釋放的事件;
\item 如果 \m{X} 被釋放,則警衛有 50\% 的概率說 \m{Y} 會被處死。
\stopigBase
\startformula\startmathalignment[n=1]
\NC \Pr\{y|X\} = \frac 1 2 \NR
\NC \Pr\{y\} = \Pr\{y \cap X\} + \Pr\{y \cap Y\} + \Pr\{y \cap Z\}
             = \frac 1 3 \cdot \frac 1 2 + \frac 1 3 \cdot 0 + \frac 1 3 \cdot 1
             = \frac 1 6 + \frac 1 3
             = \frac 1 2 \NR
\NC \Pr\{X|y\} = \frac{\Pr\{y|X\} \cdot \Pr\{X\}}{\Pr\{y\}}
               = \frac{\frac 1 2 \cdot \frac 1 3}{\frac 1 2}
               = \frac 1 3 \NR
\stopmathalignment\stopformula

\SIMPLEURL{https://en.wikipedia.org/wiki/Three_Prisoners_problem}
\stopANSWER

\stopsection
