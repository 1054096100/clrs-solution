\startsection[
  title={Multithreadedmatrix multiplication},
]

%e27.2-1
\startEXERCISE[exercise:27.2-1]
請畫出在 \m{2\times 2} 矩陣上計算 \ALGO{P-SQUARE-MATRIX-MULTIPLY} 的計算有向無環圖,
並在圖中標出與算法執行中的股對應的所有頂點。
根據慣例,派生調用和普通調用的邊指向下,連接邊水平指向右,返回邊指向上。
假設每股消耗單位時間,
試分析該計算的工作量、持續時間和並行度。
\stopEXERCISE

\startANSWER
\externalfigure[output/e27_2_1-1]
\stopANSWER

%e27.2-2
\startEXERCISE
對 \ALGO{P-MATRIX-MULTIPLY-RECURSIVE} 過程,重做\refexercise{27.2-1}。
\stopEXERCISE

\startANSWER
\TODO{略。}
\stopANSWER

%e27.2-3
\startEXERCISE
請給出工作量爲 \m{\Theta(n^3)},
而持續時間僅爲 \m{\Theta(\lg n)} 的兩個 \m{n\times n} 矩陣相乘的多線程算法僞碼,
並分析該算法。
\stopEXERCISE

\startANSWER
根據\refexercise{27.1-6},矢量相乘用時 \m{\Theta(\lg n)},
因此矩陣相乘爲:
\startformula
T_{\infty} = \Theta(\lg n) + \Theta(\lg n) + \Theta(\lg n) = \Theta(\lg n)
\stopformula
\stopANSWER

%e27.2-4
\startEXERCISE
請給出 \m{p\times q} 矩陣和 \m{q\times r} 矩陣相乘的一個有效多線程算法的僞碼。
即使 \m{p}、 \m{q} 或 \m{r} 中任意一個爲 1,
你的算法也要有好的並行性能。分析該算法。
\stopEXERCISE

\startANSWER
\startformula
T_{\infty} = \Theta(\min\{\lg p, \lg q, \lg r\})
\stopformula
\stopANSWER

%e27.2-5
\startEXERCISE
請給出原地轉置 \m{n\times n} 矩陣的一個有效的多線程算法僞碼,
使用分治法原地將 \m{n\times n} 矩陣遞歸地劃分爲 4 個 \m{n/2\times n/2} 子矩陣。
分析該算法。
\stopEXERCISE

\startANSWER
\startformula
M = \left(\startmatrix
\NC A \NC B \NR
\NC C \NC D \NR
\stopmatrix\right)
\stopformula

\startformula
M^T = \left(\startmatrix
\NC A^T \NC C^T \NR
\NC B^T \NC D^T \NR
\stopmatrix\right)
\stopformula

\startformula\startmathalignment
\NC T_{\infty}(n) \NC = T_{\infty}(n/2) + \Theta(1) \NR
\NC T_{\infty}(n) \NC = \Theta(\lg n) \NR
\stopmathalignment\stopformula
\stopANSWER

%e27.2-6
\startEXERCISE
請給出 \ALGO{FLOYD-WARSHALL} (參見\refsection{floyd_warshall})的一個有效多線程實現的僞碼,
該算法在帶權圖上計算所有點對間的最短路徑。
分析該算法。
\stopEXERCISE

\startANSWER
兩個循環均可平行化, \m{T_{\infty} = \Theta(n\lg n)}。
\stopANSWER

\stopsection
