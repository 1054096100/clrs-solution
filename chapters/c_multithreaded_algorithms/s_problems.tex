\startsubject[
  title={Problems},
]

%p27-1
\startPROBLEM
(Implementing parallel loops using nested parallelism)
考慮下面多線程算法,將兩個 \m{n} 元素數列 \m{A[1..n]} 和 \m{B[1..n]} 相加,
並將結果存放在 \m{C[1..n]} 中:

\CLRSH{SUM-ARRAYS(A,B,C)}
\startCLRS
parallel for i = 1 to A.length
	C[i] = A[i] + B[i]
\stopCLRS

\startigBase[a]\startitem
按照 \ALGO{MAT-VEC-MAIN-LOOP} 的樣式,
使用嵌套並行(spawn 和 sync)改寫 \ALGO{SUM-ARRAYS} 中的並行循環。
分析你的實現的並行度。
\stopitem\stopigBase

\startANSWER
\CLRSH{MAT-VEC-MAIN-LOOP(A,B,C,l,r)}
\startCLRS
if l == r
	c[l] = A[l] + B[l]
mid = (l + r) / 2
spawn MAT-VEC-MAIN-LOOP(A,B,C, l, mid)
MAT-VEC-MAIN-LOOP(A,B,C, mid+1, r)
sync
\stopCLRS

\CLRSH{SUM-ARRAYS(A,B,C)}
\startCLRS
len = A.length
MAT-VEC-MAIN-LOOP(A,B,C, 1, len)
\stopCLRS
\stopANSWER

考慮並行循環的另一種實現方式,其中包含了一個指定的 grainsize:

\CLRSH{SUM-ARRAYS'(A,B,C)}
\startCLRS
n = A.length
grainsize = ?		// to be determined
r = /BTEX \m{\left\lceil n / grainsize \right\rceil} /ETEX
for k = 0 to r-1
	spawn /BTEX \CLRSH{ADD-SUBARRAY(A,B,C,k\cdot grainsize + 1, \min((k+1)\cdot grainsize, n))} /ETEX
sync
\stopCLRS

\CLRSH{ADD-SUBARRAY(A,B,C,i,j)}
\startCLRS
for k = i to j
	C[k] = A[k] + B[k]
\stopCLRS

\startigBase[continue]\startitem
假定置 \m{grainsize = 1},上述實現的並行度是多少?
\stopitem\stopigBase

\startANSWER
\m{T_1 = \Theta(n), T_{\infty} = \Theta(n), T_1/T_{\infty} = 1}。
\stopANSWER

\startigBase[continue]\startitem
給出 \ALGO{SUM-ARRAYS'} 的持續時間公式,用 \m{n} 和 \m{grainsize} 表示,
求解 \m{grainsize},使得並行度最大。
\stopitem\stopigBase

\startANSWER
\m{grainsize = \sqrt{n}, T_{\infty} = n/grainsize + grainsize = 2\sqrt{n}}。
\stopANSWER

\stopPROBLEM

%p27-2
\startPROBLEM
(Saving temporary space in matrix multiplication)
 \ALGO{P-MATRIX-MULTIPLY-RECURSIVE} 的缺點是需要分配一個 \m{n\times n} 的臨時矩陣 \m{T},
不利於 \m{\Theta} 記號中的常數因子。
然而 \ALGO{P-MATRIX-MULTIPLY-RECURSIVE} 有很高的並行度。
例如,如果胡略符號 \m{\Theta} 中的常數因子,
對於 \m{1000\times 1000} 的矩陣相乘,
其並行度接近 \m{1000^3 / 10^2 = 10^7},
因爲 \m{\lg 1000 \approx 10}。
絕大多數並行計算機的處理器數目都遠小於 1000 萬。

\startigBase[a]\startitem
描述一個多線程算法,該算法不需要臨時矩陣 \m{T} 且持續時間以 \m{\Theta(n)} 增長。
(\hint 根據 \ALGO{P-MATRIX-MULTIPLY-RECURSIVE} 中的方法,
計算 \m{C=C+AB},但可以並行初始化 \m{C} 並且,
注意在合適的地方插入 sync 語句。)
\stopitem\stopigBase

\startANSWER
並行初始化 \m{C=0},用時 \m{\Theta(\lg n)},
在第 4 個 spawn \m{c_{11} = c_{11} + a_{11}\cdot b_{11}} 後插入 sync,
 \m{T_{\infty}(n) = 2T_{\infty}(n/2) + \Theta(\lg n) = \Theta(n)}。
\stopANSWER

\startigBase[continue]\startitem
給出該算法的工作量和持續時間的遞歸式,並求解。
\stopitem\stopigBase

\startANSWER
\m{T_1 = \Theta(n^3), T_{\infty} = \Theta(n)}。
\stopANSWER

\startigBase[continue]\startitem
分析該算法的並行度。
忽略符號 \m{\Theta} 中的常數因子,
估算 \m{1000\times 1000} 矩陣上的並行度。
並與 \ALGO{P-MATRIX-MULTIPLY-RECURSIVE} 的並行度進行比較。
\stopitem\stopigBase

\startANSWER
並行度: \m{T_1/T_{\infty} = \Theta(n^2) = 1000^2 = 10^6}。

絕大多數計算機的處理器個數遠小於 100 萬。
\stopANSWER
\stopPROBLEM

%p27-3
\startPROBLEM
(Multithreaded matrix algorithms)
\startigBase[a]\startitem
參考\refsection{solve_system_of_linear_equation} 中的 \ALGO{LU-DECOMPOSITION},
給出其多線程實現的僞碼。使其儘可能並行,分析其工作量、持續時間和並行度。
\stopitem\stopigBase

\startANSWER
\TODO{略。}
\stopANSWER

\startigBase[continue]\startitem
對\refsection{solve_system_of_linear_equation} 中的 \ALGO{LUP-DECOMPOSITION} 做同樣處理。
\stopitem\stopigBase
\stopPROBLEM

\startANSWER
\TODO{略。}
\stopANSWER

\startigBase[continue]\startitem
對\refsection{solve_system_of_linear_equation} 中的 \ALGO{LUP-SOLVE} 做同樣處理。
\stopitem\stopigBase

\startANSWER
\TODO{略。}
\stopANSWER

\startigBase[continue]\startitem
參考等式 28.13 對稱正定矩陣求逆,對其做同樣處理。
\stopitem\stopigBase

\startANSWER
\TODO{略。}
\stopANSWER
\stopPROBLEM

%p27-4
\startPROBLEM
(Multithreading reductions and prefix computations eduction)
數列 \m{x[1..n]} 的 \m{\otimes} {\EMP 歸約}(\m{\otimes}-reduction)
就是 \m{y=x[1]\otimes x[2]\otimes\cdots x[n]} 的值,
其中 \m{\otimes} 滿足結合律。

下面程序以串行方式計算子數列 \m{x[i..j]} 的 \m{\otimes} 歸約:

\CLRSH{REDUCE(x,i,j)}
\startCLRS
y=x[i]
for k = i+1 to j
	y = y /BTEX \m{\otimes}/ETEX x[k]
return y
\stopCLRS

\startigBase[a]\startitem
應用嵌套並行實現一個多線程算法 \ALGO{P-REDUCE},
工作量爲 \m{\Theta(n)},要求持續時間爲 \m{\Theta(\lg n)}。
分析該算法。
\stopitem\stopigBase

\startANSWER
\CLRSH{P-REDUCE(x,i,j)}
\startCLRS
if i == j
	return x[i]
else if i + 1 == j
	return x[i] /BTEX \m{\otimes}/ETEX x[j]
mid = (i + j) / 2
spawn y1 = P-REDUCE(x, i, mid)
y2 = P-REDUCE(x, mid + 1, j)
sync
return y1 /BTEX \m{\otimes}/ETEX y2
\stopCLRS
\stopANSWER

另一個相關問題,在數列 \m{x[1..n]} 上求解 \m{\otimes} {\EMP 前綴計算}
(\m{\otimes}-prefix computation),
有時也稱 \m{\otimes} {\EMP 掃描}(\m{\otimes}-scan),
其中 \m{\otimes} 滿足結合律。
 \m{\otimes} 掃描產生數列 \m{y[1..n]}:
\startformula\startmathalignment
\NC y[1] \NC = x[1] \NR
\NC y[2] \NC = x[1] \otimes x[2] \NR
\NC y[3] \NC = x[1] \otimes x[2] \otimes x[3] \NR
\NC      \NC \vdots \NR
\NC y[n] \NC = x[1] \otimes x[2] \otimes x[3] \otimes \cdots \otimes x[n] \NR
\stopmathalignment\stopformula
也就是說,通過運算法 \m{\otimes} 計算出 \m{x} 所有前綴的”和“。
下面串行 SCAN 過程會計算出一組 \m{\otimes} 前綴:

\CLRSH{SCAN(x)}
\startCLRS
n = x.length
let y[1..n] be a new array
y[1] = x[1]
for i = 2 to n
	y[i] = y[i-1] /BTEX \m{\otimes} /ETEX x[i]
return y
\stopCLRS

遺憾地是,無法直接進行多線程 SCAN。
例如,將 for 循環改爲 parallel for 循環會產生競爭,
因爲循環體的每一步迭代都依賴前一個迭代。
下面的 \ALGO{P-SCAN-1} 可以並行計算 \m{\otimes} 前綴,
儘管十分低效:

\CLRSH{P-SCAN-1(x)}
\startCLRS
n = x.length
let y[1..n] be a new array
P-SCAN-1-AUX(x, y, 1, n)
return y
\stopCLRS

\CLRSH{P-SCAN-1-AUX(x, y, i, j)}
\startCLRS
parallel for l = i to j
	y[l] = P-REDUCE(x, 1, l)
\stopCLRS

\startigBase[continue]\startitem
分析 \ALGO{P-SCAN-1} 的工作量、持續時間和並行度。
\stopitem\stopigBase

\startANSWER
\startformula\startmathalignment
\NC T_1 \NC = \Theta(n^2) \NR
\NC T_{\infty} \NC = \Theta(\lg n) + \Theta(\lg n) = \Theta(\lg n) \NR
\NC T_1 / T_{\infty} \NC = \Theta(n^2/\lg n) \NR
\stopmathalignment\stopformula
\stopANSWER

使用嵌套並行能更有效的計算 \m{\otimes} 前綴,過程如下:

\CLRSH{P-SCAN-2(x)}
\startCLRS
n = x.length
let y[1..n] be a new array
P-SCAN-2-AUX(x,y,1,n)
return y
\stopCLRS

\CLRSH{P-SCAN-2-AUX(x,y,i,j)}
\startCLRS
if i == j
	y[i] = x[i]
else
	k = /BTEX \m{\left\lfloor (i+j)/2 \right\rfloor} /ETEX
	spawn P-SCAN-2-AUX(x, y, i, k)
	P-SCAN-2-AUX(x, y, k+1, j)
	sync
	parallel for l = k+1 to j
		/BTEX \m{y[l] = y[k] \otimes y[l]} /ETEX
\stopCLRS

\startigBase[continue]\startitem
論證 \ALGO{P-SCAN-2} 是正確的,並分析其工作量、持續時間和並行度。
\stopitem\stopigBase

\startANSWER
\startformula\startmathalignment
\NC T_1 \NC = 2T_1(n/2) + \Theta(n) = \Theta(n\lg n) \NR
\NC T_{\infty} \NC = T_{\infty}(n/2) + \Theta(\lg n) = \Theta(\lg^2 n) \NR
\NC T_1 / T_{\infty} \NC = \Theta(n / \lg n) \NR
\stopmathalignment\stopformula
\stopANSWER

我們可以遍歷兩遍數據來計算 \m{\otimes} 前綴,
從而改進 \ALGO{P-SCAN-1} 和 \ALGO{P-SCAN-2}。
第一趟,收集不同的連續子數列 \m{x} 的“和”,
存入臨時數列 \m{t} 中;
第二趟,使用 \m{t} 中的“和”計算最終結果 \m{y}。
下面的僞碼實現了這種策略,但省去了一些算式:

\CLRSH{P-SCAN-3(x)}
\startCLRS
n = x.length
let y[1..n] and t[1..n] be new arrays
y[1] = x[1]
if n > 1
	P-SCAN-UP(x, t, 2, n)
	P-SCAN-DOWN(x[1], x, t, y, 2, n)
return y
\stopCLRS

\CLRSH{P-SCAN-UP(x, t, i, j)}
\startCLRS
if i == j
	return x[i]
else
	k = /BTEX \m{\left\lfloor (i+j)/2 \right\rfloor} /ETEX
	t[k] = spawn P-SCAN-UP(x, t, i, k)
	right = P-SCAN-UP(x, t, k+1, j)
	sync
	return ____		// fill in the blank
\stopCLRS

\CLRSH{P-SCAN-DOWN(v, x, t, y, i, j)}
\startCLRS
if i == j
	/BTEX \m{y[i] = v\otimes x[i]} /ETEX
else
	k = /BTEX \m{\left\lfloor (i+j)/2 \right\rfloor}/ETEX
	spawn P-SCAN-DOWN(____, x, t, y, i, k)		// fill in the blank
	P-SCAN-DOWN(____, x, t, y, k+1, j)		// fill in the blank
	sync
\stopCLRS

\startigBase[continue]\startitem
對 \ALGO{P-SCAN-UP} 第 8 行、 \ALGO{P-SCAN-DOWN} 第 5 行和第 6 行中的三個缺省算式填空。
然後,論證 \ALGO{P-SCAN-3} 的正確性。
(\hint 證明傳給 \ALGO{P-SCAN-DOWN(v,x,t,y,i,j)} 的 \m{v} 滿足 \m{v=x[1]\otimes x[2]\otimes\cdots\otimes x[i-1]})
\stopitem\stopigBase

\startANSWER
\m{t[k] \times right}

\m{v}

\m{t[k]}
\stopANSWER

\startigBase[continue]\startitem
分析 \ALGO{P-SCAN-3} 的工作量、持續時間和並行度。
\stopitem\stopigBase

\startANSWER
\startformula\startmathalignment
\NC T_1 \NC = \Theta(n) \NR
\NC T_{\infty} \NC = \Theta(\lg n) \NR
\NC T_1 / T_{\infty} \NC = \Theta(n / \lg n) \NR
\stopmathalignment\stopformula
\stopANSWER

\stopPROBLEM

%p27-5
\startPROBLEM
(Multithreading a simple stencil calculation)
計算科學中有很多這樣一類算法,
填充數列中的元素,所填值取決於已計算出的相鄰元素以及其他一些信息
(計算過程中這些信息保持不變)。
這種計算過程中臨近元素保持不變的模式稱爲{\EMP 模板(stencil)}計算。
例如,\refsection{lcs} 提供了一個計算最長公共子序列的模板算法,
其中 \m{c[i,j]} 的值僅取決於 \m{c[i-1,j]}、 \m{c[i,j-1]} 和 \m{c[i-1,j-1]} 以及
兩個給定輸入串中的元素 \m{x_i} 和 \m{y_j}。
輸入的序列是固定不變的,
但算法需要填充二維數列 \m{c},
先填充 \m{c[i-1,j]}、 \m{c[i,j-1]} 和 \m{c[i-1,j-1]},然後再填充 \m{c[i,j]}。

本題中,我們探討如何在一個 \m{n\times n} 數列 \m{A} 上使用嵌套並行實現一個簡單的模板計算,
其中存入 \m{A[i,j]} 的值僅取決於 \m{A[i',j']},
其中 \m{i'\le i}, \m{j'\le j} (當然 \m{i'\neq i} 或者 \m{j'\neq j})。
換句話說,一個元素的值僅取決於上面和左面臨近元素的值,
以及數列外一些靜態信息。
此外,整個過程中假設,
一旦計算 \m{A[i,j]} 時所需的元素都已填寫完,
就可以在 \m{\Theta(1)} 的時間內填入 \m{A[i,j]}。
(與\refsection{lcs} 中的 \ALGO{LCS-LENGTH} 一樣)

將 \m{n\times n} 數列 \m{A} 劃分爲 \m{n/2\times n/2} 的子數列:
\startformula
A = \left[\startmatrix
\NC A_{11} \NC A_{12} \NR
\NC A_{21} \NC A_{22} \NR
\stopmatrix\right]
\stopformula

現在看到,可以遞歸地填入子數列 \m{A_{11}} 中的元素,
因爲他並不依賴其他三個子數列。
一旦填完 \m{A_{11}},就可以遞歸地並行填入 \m{A_{12}} 和 \m{A_{21}} 中的元素,
這是因爲他們都依賴於 \m{A_{11}},但彼此無依賴關係。
最後遞歸地填入 \m{A_{22}}。

\startigBase[a]\startitem
基於上面討論,給出分治算法 \ALGO{SIMPLE-STENCIL} 的多線程僞碼,
來執行這個簡單模板計算。
(不用擔心基礎情形的處理,這取決於特定的模板。)
給出並求解對應規模 \m{n} 的工作量和持續時間的遞歸式。
並行度是多少?
\stopitem\stopigBase

\startANSWER
\CLRSH{SIMPLE-STENCIL(A)}
\startCLRS
SIMPLE-STENCIL(A11)
spawn SIMPLE-STENCIL(A12)
SIMPLE-STENCIL(A21)
sync
SIMPLE-STENCIL(A22)
\stopCLRS

\startformula\startmathalignment
\NC T_1 \NC = \Theta(n^2) \NR
\NC T_{\infty} \NC = 3T_{\infty}(n/2) + \Theta(1) = \Theta(n^{\lg 3}) \approx \Theta(n^{1.58}) \NR
\NC T_1 / T_{\infty} \NC = \Theta(n^{2/\lg 3}) \approx \Theta(n^{1.26}) \NR
\stopmathalignment\stopformula
\stopANSWER

\startigBase[continue]\startitem
修改上面 a)的答案,將 \m{n\times n} 的數列劃分成 9 個 \m{n/3 \times n/3} 子數列,
遞歸下去使其儘可能地並行。
分析該算法,與 a)相比,並行度如何?
\stopitem\stopigBase

\startANSWER
\startCLRS
11
spawn 12
21
sync
spawn 13
spawn 22
31
sync
spawn 23
32
sync
33
\stopCLRS

\startformula\startmathalignment
\NC T_1 \NC = \Theta(n^2) \NR
\NC T_{\infty} \NC = 5T_{\infty}(n/3) + \Theta(1) = \Theta(n^{\lg_{3}5}) \approx \Theta(n^{1.46}) \NR
\NC T_1 / T_{\infty} \NC = \Theta(n^{2/\lg_{3}5}) \approx \Theta(n^{1.37}) \NR
\stopmathalignment\stopformula
\stopANSWER

\startigBase[continue]\startitem
對照 a)和 b),進行如下推廣。
選擇一個整數 \m{b\ge 2}。
將 \m{n\times n} 數列分爲 \m{b^2} 個子數列,每個大小都是 \m{n/b\times n/b},
遞歸下去使其儘可能並行。
關於 \m{n} 和 \m{b},
該算法的工作量、持續時間和並行度各爲多少?
用這種方法證明:對於任何 \m{b\ge 2},並行度一定是 \m{o(n)}。
(\hint 最後一個問題,證明對於任何 \m{b\ge 2},
並行度是 \m{n} 的指數,其指數嚴格小於 1。)
\stopitem\stopigBase

\startANSWER
\startformula\startmathalignment
\NC T_1 \NC = \Theta(n^2) \NR
\NC T_{\infty} \NC = (2b-1) T_{\infty}(n/b) + \Theta(1) = \Theta(n^{\log_{b}(2b-1)}) \NR
\NC T_1 / T_{\infty} \NC = \Theta(n^{2/\log_{b}(2b-1)}) = \Theta(n^{\log_{2b-1}{b^2}}) \NR
\stopmathalignment\stopformula

由於 \m{b\ge 2},\m{b^2 \ge 2b-1}, \m{(b-1)^2\ge 0},並行度一定是 \m{o(n)}。
\stopANSWER

\startigBase[continue]\startitem
給出一個求解這個簡單模板問題的多線程算法的僞碼,
使其並行度達到 \m{\Theta(n/\lg n)}。
使用工作量和持續時間的概念,
論證該問題事實上有 \m{\Theta(n)} 的固有並行度。
然而,我們使用分治法的多線程僞碼,實際上達不到這個最大的並行度。
\stopitem\stopigBase

\startANSWER
\TODO{略。}
\stopANSWER

\stopPROBLEM

%p27-6
\startPROBLEM
(Randomized multithreaded algorithms)
正如使用普通的串行算法一樣,有時想要實現隨機多線程算法。
本題探討如何修改各種性能度量以處理這些算法的期望行爲。
另外,要求設計並分析一個隨機快速排序的多線程算法。

\startigBase[a]\startitem
修改工作量定律(27.2)、持續時間定律(27.3)和貪心調度界(27.4),
用期望值來表示 \m{T_P}、 \m{T_1} 和 \m{T_{\infty}} 都是隨機變量的情形。
\stopitem\stopigBase

\startANSWER
\startformula\startmathalignment
\NC E[T_P] \NC \ge E[T_1] / P \NR
\NC E[T_P] \NC \ge E[T_{\infty}] \NR
\NC E[T_P] \NC \ge E[T_1] / P + E[T_{\infty}] \NR
\stopmathalignment\stopformula
\stopANSWER

\startigBase[continue]\startitem
考慮一個隨機多線程算法,他在 1\% 的時間裏有 \m{T_1 = 10^4} 和 \m{T_{10000} = 1},
但在 99\% 的時間裏有 \m{T_1 = T_{10000} = 10^9}。
說明一個隨機多線程算法的{\EMP 加速比}應該被定義爲 \m{E[T_1]/E[T_P]},
而不是 \m{E[T_1/T_P]}。
\stopitem\stopigBase

\startANSWER
\startformula\startmathalignment
\NC E[T_1] \NC \approx E[T_{10000}] \approx 9.9 \times 10^8 \NR
\NC E[T_1] / E[T_P] \NC = 1 \NR
\NC E[T_1/T_{10000}] \NC = 10^4 \times 0.01 + 0.99 = 100.99 \NR
\stopmathalignment\stopformula
\stopANSWER

\startigBase[continue]\startitem
說明一個隨機多線程算法的{\EMP 並行度}應該定義爲 \m{E[T_1]/E[T_{\infty}]}。
\stopitem\stopigBase

\startANSWER
同上題。
\stopANSWER

\startigBase[continue]\startitem
使用嵌套並行,多線程化\refsection{rand_quicksort} 中的 \ALGO{RANDOMIZED-QUICKSORT}
(注意不是並行化 \ALGO{RANDOMIZED-PARTITION})。
給出 \ALGO{P-RANDOMIZED-QUICKSORT} 的僞碼。
\stopitem\stopigBase

\startANSWER
\CLRSH{RANDOMIZED-QUICKSORT(A, p, r)}
\startCLRS
if p < r
	q = RANDOMIZED-PARTITION(A, p, r)
	spawn RANDOMIZED-QUICKSORT(A, p, q-1)
	RANDOMIZED-QUICKSORT(A, q+1, r)
	sync
\stopCLRS
\stopANSWER

\startigBase[continue]\startitem
分析給出的隨機快速排序的多線程算法。
(\hint 回顧\refsection{selection_in_expected_linear_time} 中
關於 \ALGO{RANDOMIZED-SELECT} 的分析。)
\stopitem\stopigBase

\startANSWER
\startformula\startmathalignment
\NC E[T_1] \NC = O(n\lg n) \NR
\NC E[T_{\infty}] \NC = O(\lg n) \NR
\NC E[T_1] / E[T_{\infty}] \NC = O(n) \NR
\stopmathalignment\stopformula
\stopANSWER
\stopPROBLEM

\stopsubject%Problems
