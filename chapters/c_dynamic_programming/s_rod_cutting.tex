\startsection[
  title={Rod cutting},
]

%e15.1-1
\startEXERCISE
由公式 15.3 和初始條件 \m{T(0)=1},證明公式 15.4 成立。

附公式 15.3:
\startformula
T(n) = 1 + \sum_{j=0}^{n-1} T(j)
\stopformula

附公式 15.4:
\startformula
T(n) = 2^n
\stopformula
\stopEXERCISE

\startANSWER
\m{n=0} 時顯然成立。

如果 \m{n\le i} 時成立,則 \m{n=i+1} 時有:
\startformula\startmathalignment
\NC T(i+1) \NC = 1 + \sum_{j=0}^{i} T(j) \NR
\NC        \NC = 1 + \sum_{j=0}^{i} 2^j \NR
\NC        \NC = 1 + 2^{i+1} - 1 \NR
\NC        \NC = 2^{i+1} \NR
\stopmathalignment\stopformula
\stopANSWER

%e15.1-2
\startEXERCISE
下面描述的“貪心”策略,可以用來解決鋼條切割問題,請舉出反例證明此策略並不能保證得到最優方案。

定義鋼條的長度為 \m{i},{\EMP 密度}為 \m{p_i / i},即每英寸的價值。
貪心策略將長度為 \m{n} 的鋼條切下長度為 \m{i} (\m{1\le i \le n})的壹段,其密度最高。
接下來繼續使用相同的策略切割長度為 \m{n-i} 的剩余部分。
\stopEXERCISE

\startANSWER
假設密度有三種,\m{d_a, d_b, d_c},其中 \m{d_a > d_b > d_c}。
對應的鋼條長度為 \m{l_a, l_b, l_c},其中 \m{l_a + l_c = 2 l_b}。
則對於長度為 \m{2 l_b} 的鋼條,按“貪心”策略,應切割成長度為 \m{l_a} 和 \m{l_c} 的兩段。
其價值為 \m{l_a d_a + l_c d_c};而如果切割成長度均為 \m{l_b} 的兩段,價值為 \m{2 l_b d_b}。
若“貪心”策略的結果不是最優解,需要滿足 \m{l_a d_a + l_c d_c < 2 l_b d_b}。

\startformula\startmathalignment
\NC l_a d_a + l_c d_c \NC < 2 l_b d_b \NR
\NC l_a d_a + l_c d_c \NC < (l_a + l_c) d_b \NR
\NC l_a d_a + l_c d_c \NC < l_a d_b + l_c d_b \NR
\NC l_a (d_a - d_b)   \NC < l_c (d_b - d_c) \NR
\stopmathalignment\stopformula

令 \m{l_a = 3, l_c = 2, d_a = 5, d_b = 4, d_c = 2},滿足上式,鋼條長度為 5,即為反例。
\stopANSWER

%e15.1-3
\startEXERCISE
我們修改壹下鋼條切割問題,除了切下的鋼條段具有不同價格 \m{p_i} 外,
每次切割還要付出固定成本 \m{c}。
這樣,切割方案的收益就等於鋼條段價格之和減去切割的成本。
設計壹個動態規劃算法解決修改後的鋼條切割問題。
\stopEXERCISE

\startANSWER
\CLRSH{EXTENDED2-BOTTOM-UP-CUT-ROD(p, n)}
\startCLRS
let r[0..n] and s[0..n] be new arrays
r[0] = 0
for j = 1 to n
	q = -∞
	for i = 1 to j
		if q < p[i] + r[j - i] - c
			q = p[i] + r[j - i] - c
			s[j] = i
	r[j] = q
return r and s
\stopCLRS
\stopANSWER

%e15.1-4
\startEXERCISE
修改\ALGO{MEMOIZED-CUR-ROD},使之不僅返回最優收益值,還返回切割方案。
\stopEXERCISE

\startANSWER
\CLRSH{MEMOIZED-CUT-ROD'(p, n)}
\startCLRS
let r[0..n] and s[0..n] be a new arrays
for i = 0 to n
	r[i] = -∞
return MEMOIZED-CUT-ROD-AUX'(p, n, r, s) and s
\stopCLRS

\CLRSH{MEMOIZED-CUT-ROD-AUX'(p, n, r, s)}
\startCLRS
if r[n] >= 0
	return r[n]
if n == 0
	q = 0
else
	q = -∞
	for i = 1 to n
		tmp = p[i] + MEMOIZED-CUT-ROD-AUX'(p, n-i, r, s)
		if q < tmp
			q = tmp
			s[n] = i
r[n] = q
return q
\stopCLRS
\stopANSWER

%e15.1-5
\startEXERCISE
Fibonacci 數列可以用遞迴式(3.22)定義。
設計一個 \m{O(n)} 時間的動態規劃算法計算第 n 個 Fibonacci 數。
畫出子問題圖。圖中有多少頂點和邊?
\stopEXERCISE

\startANSWER
\CLRSH{Fibonacci(n)}
\startCLRS
let F[0..n] be a new array
F[0] = 1
F[1] = 1
for i = 2 to n
	F[i] = F[i - 1] + F[i - 2]
return F[n]
\stopCLRS

圖中有 n 個節點,數列中的每個元素是一個節點。
所有邊都是由稍大的子問題指向稍小的子問題:
如由 F[2] 起始的兩條邊分別指向 F[1] 和 F[0]。
邊的總數爲 \m{
  \left\lfloor \frac{n}{2} \right\rfloor
+ \left\lfloor \frac{n-1}{2} \right\rfloor
+ n}。
\stopANSWER

\stopsection
