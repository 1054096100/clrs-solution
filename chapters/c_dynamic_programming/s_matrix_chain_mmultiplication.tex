\startsection[
  title={Matrix-chain multiplication},
  reference=section:15.2,
]

%e15.2-1
\startEXERCISE
對於矩陣規模序列 \m{\langle 5,10,3,12,5,50,6\rangle},求其矩陣鏈最優括號化方案。
\stopEXERCISE

\startANSWER
解決此問題,可以直接用 \ALGO{MATRIX-CHAIN-ORDER(p)},
其中 \m{p = \langle 5,10,3,12,5,50,6\rangle},
或者直接使用下列方程:
\startformula
m[i,j]=\startmathcases
\NC 0 \NC 如果 \m{i = j} \NR
\NC \min_{i\le k < j}\{m[i,k]+m[k+1,j] + p_{i-1} p_k p_j\} \NC 如果 \m{i<j} \NR
\stopmathcases
\stopformula
其中對於所有 \m{i},都有 \m{m[i,i] = 0},可用以計算 \m{m[i,i+1]}。

\externalfigure[output/e15_2_1-1]
\externalfigure[output/e15_2_1-2]

最終,結果爲 \m{(A_1 A_2)((A_3 A_4)(A_5 A_6))}。
\stopANSWER

%e15.2-2
\startEXERCISE
設計遞迴算法 \ALGO{MATRIX-CHAIN-MULTIPLY(A, s, i, j)},
實現矩陣鏈最優代價乘法計算的真正計算過程,
其輸入參數爲矩陣序列 \m{\langle A_1, A_2, \cdots, A_n \rangle},
 \ALGO{MATRIX-CHAIN-ORDER} 得到的表 \m{s},以及下標 \m{i} 和 \m{j}。
(初始調用應爲 \ALGO{MATRIX-CHAIN-MULTIPLY(A, s, 1, n)}。)
\stopEXERCISE

\startANSWER
\CLRSH{MATRIX-CHAIN-MULTIPLY(A, s, i, j)}
\startCLRS
if i == j
	return A[i]
else
	X = MATRIX-CHAIN-MULTIPLY(A, s, i, s[i,j])
	Y = MATRIX-CHAIN-MULTIPLY(A, s, s[i,j] + 1, j)
	return MATRIX-MULTIPLY(X, Y)
\stopCLRS
\stopANSWER

%e15.2-3
\startEXERCISE
用代入法證明遞迴公式(15.6)的結果爲 \m{\Omega(2^n)}。附公式(15.6):
\startformula
P(n) = \startmathcases
\NC 1 \NC 如果 \m{n = 1} \NR
\NC \sum_{k=1}^{n-1} P(k) P(n-k) \NC 如果 \m{n\ge 2} \NR
\stopmathcases
\stopformula
\stopEXERCISE

\startANSWER
如果對於所有 \m{1\le k \le n},都有 \m{P(k) \ge 2^k},則:
\startformula
P(n+1) \ge \sum_{k=1}^{n} 2^k 2^{n+1-k} = n 2^{n+1} \ge 2^{n+1}
\stopformula
\stopANSWER

%e15.2-4
\startEXERCISE
對輸入鏈長度爲 \m{n} 的矩陣鏈乘法問題,描述子問題圖:
他包含多少個頂點?包含多少條邊?這些邊分別鏈接哪些頂點?
\stopEXERCISE

\startANSWER
\TODO{}
\stopANSWER

%e15.2-5
\startEXERCISE
令 \m{R(i,j)} 表示在一次調用 \ALGO{MATRIX-CHAIN-ORDER} 過程中,
計算其他表項時訪問表項 \m{m[i,j]} 的次數。證明:
\startformula
\sum_{i=1}^n\sum_{j=i}^{n} R(i,j) = \frac{n^3 - n}{3}
\stopformula
({\EMP 提示:}證明中可用到公式(A.3)。)
\stopEXERCISE

\startANSWER
\startformula\startmathalignment
\NC \sum_{i=1}^n\sum_{j=i}^{n} R(i,j)
    \NC = \sum_{l=2}^{n} 2(n-l+1)(l-1) \NR
\NC \NC = 2\sum_{l=1}^{n-1} (n-l) l \NR
\NC \NC = 2\sum_{l=1}^{n-1} nl - 2\sum_{l=1}^{n-1}l^2 \NR
\NC \NC = \frac{2n(n-1)n}{2} - \frac{2(n-1)n(2n-1)}{6} \NR
\NC \NC = \frac{n^3 - n}{3} \NR
\stopmathalignment\stopformula
\stopANSWER

%e15.2-6
\startEXERCISE
證明:對 \m{n} 個元素的表達式進行完全括號化,恰好需要 \m{n-1} 對括號。
\stopEXERCISE

\startANSWER
最內層括號括住兩個矩陣,之後每加一對括號會括住一個新的矩陣。
因此當 \m{n\ge 2} 時,共需要 \m{n-1} 對括號。
\stopANSWER

\stopsection
