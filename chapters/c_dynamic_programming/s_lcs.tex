\startsection[
  title={Longest common subsequence},
  reference=section:lcs,
]

%e15.4-1
\startEXERCISE
求 \m{\langle 1,0,0,1,0,1,0,1\rangle} 和 \m{\langle 0,1,0,1,1,0,1,1,0\rangle} 的一個 LCS。
\stopEXERCISE

\startANSWER
LCS 爲 \m{\langle 1,0,0,1,1,0\rangle}。

\externalfigure[output/e15_4_1-1]
\stopANSWER

%e15.4-2
\startEXERCISE
設計僞碼,利用完整的表 \m{c} 及
原始序列 \m{X=\langle x_1, x_2, \ldots, x_m \rangle}
和 \m{Y=\langle y_1, y_2, \ldots, y_n\rangle} 來重構 LCS,
要求運行時間爲 \m{O(m+n)},不能使用表 \m{b}。
\stopEXERCISE

\startANSWER
\CLRSH{PRINT-LCS(c, X, Y, i, j)}
\startCLRS
if i == 0 or j == 0
	return
if X[i] == Y[j]
	PRINT-LCS(c, X, Y, i-1, j-1)
	printf X[i]
elseif c[i-1, j] >= c[i, j-1]
	PRINT-LCS(c, X, Y, i-1, j)
else
	PRINT-LCS(c, X, Y, i, j-1)
\stopCLRS
\stopANSWER

%e15.4-3
\startEXERCISE
設計 \ALGO{LCS-LENGTH} 帶備忘的版本,運行時間爲 \m{O(mn)}。
\stopEXERCISE

\startANSWER
\CLRSH{LCS-LENGTH(X, Y, c, i, j)}
\startCLRS
if c[i, j] >= 0
	return c[i, j]
if i == 0 or j == 0
	c[i, j] = 0
	return 0
if X[i] == Y[j]
	c[i, j] = LCS-LENGTH(X, Y, c, i-1, j-1) + 1
else
	c[i, j] = max(LCS-LENGTH(X, Y, c, i-1, j), LCS-LENGTH(X, Y, c, i, j-1))
return c[i, j]
\stopCLRS
\stopANSWER

%e15.4-4
\startEXERCISE
說明如何只使用表 c 中 \m{2 \times \min(m,n)} 個表項及 \m{O(1)} 的額外空間來計算 LCS 的長度。
然後說明如何只用 \m{\min(m,n)} 個表項及 \m{O(1)} 的額外空間完成相同的工作。
\stopEXERCISE

\startANSWER
選取元素個數少的序列作爲 Y。
計算 c 中某一行的值時,僅需要上一行的值。

進一步優化後,計算 \m{c[i,j]} 時,僅需保留
\m{c[i, k], 1\le k < j-1}、
\m{c[i-1, k], k \ge j-1} 即可。
\stopANSWER

%e15.4-5
\startEXERCISE
給定一個含 \m{n} 個數的序列,
設計一個時間復雜度爲 \m{O(n^2)} 的算法,
求其最長單調遞增子序列。
\stopEXERCISE

\startANSWER
排序後的序列與原序列的 LCS 即爲所求。
排序時間爲 \m{n\lg n},求 LCS 需時間 \m{O(n^2)},總共爲 \m{O(n^2)}。
\stopANSWER

%e15.4-6
\startEXERCISE\DIFFICULT
給定一個含 \m{n} 個數的序列,
設計一個時間復雜度爲 \m{O(n\lg n)} 的算法,
求其最長單調遞增子序列。
({\EMP 提示:}一個長度爲 \m{i} 的候選子序列的尾元素
至少不比長度爲 \m{i-1} 候選子序列的尾元素小。
因此,可以在輸入序列中將候選子序列鏈接起來。)
\stopEXERCISE

\startANSWER
參考 \goto{Dynamic Programming – a Quick Review}
[url(http://www.csie.ntu.edu.tw/~kmchao/seq06fall/dp.pdf)] 中 2.2 節。

\CLRSH{LIS_LENGTH(X)}
\startCLRS
BESTEND[]	// 所有長度爲 k 的子序列中最小的尾元素
		// 單調遞增
len = 1
BESTEND[1] = X[1]
for i = 2 to X.len
	if X[i] > BESTEND[len]
		len = len + 1
		BESTEND[len] = X[i]
	else
		// 查找大於 X[i] 的最小元素的位置
		pos = BI-SEARCH(BESTEND, len, X[i])
		BESTEND[pos] = X[i]
return len
\stopCLRS
\stopANSWER

\stopsection
