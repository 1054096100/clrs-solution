\startsection[
  title={Elements of dynamic programming},
]

%e15.3-1
\startEXERCISE
對於矩陣鏈乘法問題,下面兩種確定最優代價的方法哪種效率更高?
第一種是窮舉所有可能的括號化方案,
對每種方案計算乘法運算次數,
第二種是運行 \ALGO{RECURSIVE-MATRIX-CHAIN}。
證明你的結論。
\stopEXERCISE

\startANSWER
第二種方法效率更高。

考慮兩種方法都是如何對待子問題的。

對於用於分割矩陣鏈的每個位置,窮舉法會找出左、右兩部分各自的所有括號化方案,
然後觀察左右兩側所有可能的組合。組合的數目是左右各自括號化方案數目的乘積。

而第二種方法會找到左右兩側各自的最優括號化方案,然後將這兩種方案組合起來。
組合的數目是 \m{O(1)}。

實際運行時間:第一種方法是 \m{O(4^n / n^{3/2})},第二種方法是 \m{O(n3^{n-1})}。
\stopANSWER

%e15.3-2
\startEXERCISE
對於一個有 16 個元素的數列,畫出節 2.3.1 中 \ALGO{MERGE-SORT} 運行過程的遞迴調用樹。
解釋備忘技術爲什麼對 \ALGO{MERGE-SORT} 這種分治算法無效。
\stopEXERCISE

\startANSWER
子問題的屬性不滿足要求。
\stopANSWER

%e15.3-3
\startEXERCISE
考慮矩陣鏈乘法問題的一個變形:
目標改爲最大化矩陣序列括號化方案的標量乘法運算次數,而非最小化。
此問題具有最優子結構性質嗎?
\stopEXERCISE

\startANSWER
是的。
\stopANSWER

%e15.3-4
\startEXERCISE
如前所述,使用動態規劃方法,我們首先求解子問題,
然後選擇哪些子問題用來構造原問題的最優解。
 Capulet 教授認爲,我們不必爲了求原問題的最優解而總是求解出所有子問題。
他建議,在求矩陣鏈乘法問題的最優解時,
我們總可以在求解子問題{\EMP 之前}選定 \m{A_i A_{i+1}\cdots A_j} 的劃分位置 \m{A_k}
(選定的 \m{k} 使得 \m{p_{i-1} p_k p_j} 最小)。
請找出一個反例,證明這個貪心方法可能生成次優解。
\stopEXERCISE

\startANSWER
錯誤在於最小 cost 是指最小左半 cost 加最小有半 cost 加 \m{p_{i-1} p_k p_j} 的和,
而不單單是 \m{p_{i-1} p_k p_j} 最小。
例如 \m{[1x1][1x2][2x3]}。
\stopANSWER

%e15.3-5
\startEXERCISE
對於節 15.1 的鋼條切割問題加入限制條件:
假定對於每種鋼條長度 \m{i}(\m{i=1,2,\cdots,n-1}),
最多允許切割出 \m{l_i} 段長度爲 \m{i} 的鋼條。
證明:節 15.1 所描述的最優子結構性質不再成立。
\stopEXERCISE

\startANSWER
由於限制了總數,所以子結構之間實際是相互影響的,沒有最優子結構的性質。
\stopANSWER

%e15.3-6
\startEXERCISE
假定你希望兌換外匯,你意識到與其直接兌換,不如進行多種外幣的一系列兌換,
最後兌換到你想要的那種外幣,可能會獲得更大收益。
假定你可以交易 \m{n} 種不同的貨幣,
編號爲 \m{1,2,\cdots,n},兌換從 1 號貨幣開始,最終兌換爲 \m{n} 號貨幣。
對於每兩種貨幣 \m{i} 和 \m{j},給定匯率 \m{r_{ij}},
意味着你如果有 \m{d} 個單位的貨幣 \m{i},可以兌換 \m{d r_{ij}} 個單位的貨幣 \m{j}。
進行一系列的交易需要支付一定的佣金,金額取決於交易的次數。
令 \m{c_k} 表示 \m{k} 次交易需要支付的佣金。
證明:如果對於所有 \m{k=1,2,\cdots,n}, \m{c_k = 0},
那麼尋找最優兌換序列的問題具有最優子結構性質。
然後請證明:如果佣金 \m{c_k} 爲任意值,那麼問題不一定具有最優子結構性質。
\stopEXERCISE

\startANSWER
\TODO{略。}
\stopANSWER

\stopsection
