\startsubject[
  title={Problems},
]

%p15-1
\startPROBLEM
(Longest simple path in a directed acyclic graph)
給定一個有向無環圖 \m{G=(V,E)},
邊權重爲實數,給定途中兩個頂點 \m{s} 和 \m{t}。
設計動態規劃算法,求從 \m{s} 到 \m{t} 的最長加權簡單路徑。
子問題圖是怎樣的?
算法效率如何?
\stopPROBLEM

\startANSWER
\CLRSH{LSP-DAG(G, s, t)}
\startCLRS
for each vertex v in V(G)
	L[v] = 0

for each vertex v in topOrder(G)
	for each edge (v, w) in E(G)
		L[w] = max(L[w], L[v] + weight(G, (v, w)))

return L[t]
\stopCLRS

其中 toporder 爲關鍵所在,爲所有節點排序,保證任何一個有向邊的起始點在結束點的前面。
\stopANSWER

%p15-2
\startPROBLEM
(Longest palindrome subsequence)
{\EMP 回文}(palindrome)是正序與逆序相同的非空字串。
例如,所有長度爲 1 的字串、 civic、 racecar、 aibohphobia(害怕回文之意)都是回文。

設計高效算法,求給定輸入字串的最長回文子序列。
例如,給定輸入 character,算法應該返回 carac。
算法的運行時間如何?
\stopPROBLEM

\startANSWER
設 \m{m} 爲最長回文子序列的長度,則:
\startformula
m[i,j] = \startcases
\NC m[i+1,j-1] + 1 \MC \text{如果 \m{A[i] = A[j]};} \NR
\NC \max\{m[i,j-1], m[i+1,j]\} \MC \text{如果 \m{A[i] \ne A[j]}。}\NR
\stopcases
\stopformula

\CLRSH{LPS-LENGTH(A)}
\startCLRS
n = A.length
for i = 1 to n
	m[i,i] = 1
for l = 2 to n
	for i = 1 to n - l + 1
		j = i + l - 1
		if A[i] == A[j]
			m[i,j] = m[i+1, j-1] + 1
			s[i,j] = "↙"
		else if m[i+1, j] >= m[i, j-1]
			m[i, j] = m[i+1, j]
			s[i, j] = "↓"
		else
			m[i ,j] = m[i, j-1]
			s[i, j] = "←"
return m and s
\stopCLRS
\stopANSWER

%p15-3
\startPROBLEM[problem:15-3]
(Bitonic euclidean traveling-salesman problem)
在{\EMP 歐幾裏德旅行商}問題中,給定平面商 \m{n} 個點作爲輸入,
希望求出連接所有 \m{n} 個點的最短巡遊路線。
圖 15-11(a) 給出了一個 7 點問題的解。
此問題是 NP 難問題,因此大家相信他並不存在多項式時間的求解算法(參見第 34 章)。

J.L. Bentley 建議將問題簡化,
限制巡遊路線爲{\EMP 雙調巡遊}(bitonic tours),
即從最左邊的點開始,嚴格向右前進,直至最右邊的點,
然後掉頭嚴格向左前進,直至回到起始點。
圖 15-11(b) 給出了相同 7 個點的最短雙調巡遊路線。
問題簡化後,存在一個多項式時間的算法。

設計一個 \m{O(n^2)} 時間的最優雙調巡遊路線算法。
你可以人外任何兩個點的 \m{x} 坐標均不同,
且所有實數運算都花費單位時間。
(\hint 由左至右掃描,對巡遊路線的兩個部分分別維護可能的最優解。)
\stopPROBLEM

\startANSWER
\startcombination[2*1]
{\externalfigure[output/p15_3-1][width=.3\textwidth]}{a}
{\externalfigure[output/p15_3-2][width=.3\textwidth]}{b}
\stopcombination

先對所有點按橫坐標進行排序,得到序列 \m{\langle p_1, p_2, \ldots, p_n\rangle}。
最左端的點爲 \m{p_1},最右端的點爲 \m{p_n}。
定義最短雙調路徑爲 \m{B[i,j]},包括 \m{p_1} 到 \m{p_j} 中所有點,其中 \m{1\le i \le j\le n}。
此路徑包含兩部分:從 \m{p_i} 到 \m{p_1} 的向左子圖,以及從 \m{p_1} 到 \m{p_j} 的向右子圖。
根據題意,需要計算 \m{b[n,n]}。

\startformula\startmathalignment
\NC B[1,2] \NC = |p_1 p_2| \NR
\NC B[i,j] \NC = B[i, j-1] + |p_{j-1} p_j| \qquad \text{如果 \m{i < j-1};} \NR
\NC B[j-1, j] \NC = \min_{k=1}^{j-1}\{B[k,j-1] + |p_k p_j|\} \NR
\stopmathalignment\stopformula
\stopANSWER

%p15-4
\startPROBLEM
(Printing neatly)
考慮整齊打印問題,
即在打印機商用等寬字符打印一段文本。
輸入文本爲 \m{n} 個單詞的序列,單詞長度分別爲 \m{l_1,l_2,\ldots,l_n} 個字符。
我們希望將此段文本整齊打印在若幹行商,每行最多 \m{M} 個字符。
“整齊”的標準是這樣的:
如果某行包含第 \m{i} 到第 \m{j} (\m{i\le j})個單詞,且單詞間隔爲一個空格符,
則行尾的額外空格符數量爲 \m{M-j+i-\sum_{k=i}{j}l_k},
此值必須爲非負的,否則一行內無法容納這些單詞。
我們希望能最小化所有行(最後一行除外)的額外空格數的立方之和。
設計一個動態規劃算法,在打印機商整齊打印一段 \m{n} 個單詞的文本。
分析算法的時間、空間復雜性。
\stopPROBLEM

\startANSWER
如果某一行包含單詞 \m{i} 到 \m{j},
令 \m{\EXTRAS[i,j] = M - j + i - \sum_{k=1}^{j}l_k} 爲此行行尾空格數目。
需要注意的是, \m{\EXTRAS} 可能是負值。

定義此行的代價爲 \m{\LINECOST[i,j]},則:
\startformula
\LINECOST[i,j] = \startcases
\NC \infty \MC \text{如果 \m{\EXTRAS[i,j] < 0} (即單詞 \m{i,\ldots,j} 不滿足要求); }\NR
\NC 0 \MC \text{如果 \m{j=n} 且 \m{\EXTRAS\ge 0} (最後一行代價爲 0);} \NR
\NC (\EXTRAS[i,j])^3 \MC \text{否則。} \NR
\stopcases
\stopformula

目標就是使得所有 \m{\LINECOST} 的和最小。

令 \m{C(j)} 爲前 \m{j} 個單詞的最小代價,其中第 \m{j} 個單詞位於一行的末尾,則:
\startformula\startmathalignment
\NC C(0) \NC = 0 \NR
\NC C(j) \NC = \min_{i = 1}^{j}\{C(i-1) + \LINECOST(i, j)\} \NR
\stopmathalignment\stopformula
\m{C(n)} 即爲最終結果。

\CLRSH{PRINT_NEATLY(M, l, n)}
\startCLRS
C[0] = 0
for j = 1 to n
	C[j] = ∞
	for i = 1 to j
		lc = linecost(i, j)
		if lc == ∞
			break;
		tmp = C[i-1] + lc
		if tmp < C[j]
			C[j] = tmp
			L[j] = i
return L
\stopCLRS

\m{L[n]} 的值爲最後一行第一個單詞的序號,
即,令 \m{k=L[n]},則最後一行的單詞爲 \m{l_k,\ldots, l_n}。
而 \m{L[k]} 的值爲倒數第二行第一個單詞的序號。
以此類推,我們可以找到每一行的第一個單詞,即可打印出所有單詞。

\m{C} 和 \m{L} 的長度均爲 \m{n+1},因此空間複雜度爲 \m{O(n)}。

運行時間爲填充 \m{C} 和 \m{L} 的時間,加上遍歷 \m{L} 打印所有單詞的時間。
其中遍歷 \m{L} 的時間爲 \m{O(n)}。
填充 \m{C[j]} 時,我們做了 \m{j} 次比較,因此循環所需時間爲 \m{O(n^2)}。
但是考慮到每一行的單詞數目肯定小於 \m{M/2} (一旦打印長度超過了 \m{M},
即可提前退出循環),因此比較所用時間爲 \m{O(M)}。
從而整體時間爲 \m{O(nM)}。
\stopANSWER

%p15-5
\startPROBLEM
(Edit distance)
爲將一文本串 \m{x[1..m]} 轉換爲目標串 \m{y[1..n]},
我們可以使用多種變換操作。
我們的目標是:給定 \m{x} 和 \m{y},
求將 \m{x} 轉換爲 \m{y} 的一個變換操作序列。
使用 \m{z} 保存中間結果,假定他足夠大,可寸下中間結果的所有字符。
初始時, \m{z} 是空的,
結束時,應有 \m{z[j]=y[j]},其中 \m{j=1,2,\ldots,n}。
我們維護兩個下標 \m{i} 和 \m{j},分別用來索引 \m{x} 和 \m{z},
變換操作允許改變 \m{z} 的內容和這兩個下標。
初始時, \m{i=j=1}。
在轉換過程中應處理 \m{x} 的所有字符,
這意味着在變換操作結束時,應有 \m{i=m+1}。

我們可以使用如下 6 種變換操作:
\startigBase
\item {\EMP 複製}(copy)——從 \m{x} 複製一個字符到 \m{z},
即賦值操作 \m{z[j] = x[i]},並將兩個下標 \m{i} 和 \m{j} 都增 1。
此操作處理了 \m{x[i]}。

\item {\EMP 替換}(replace)——將 \m{x} 中的一個字符替換爲另一個字符 \m{c}, \m{z[j]=c},
並將兩個下標 \m{i} 和 \m{j} 都增 1。
此操作處理了 \m{x[i]}。

\item {\EMP 刪除}(delete)——刪除 \m{x} 中一個字符,即將 \m{i} 增 1, \m{j} 不變。
此操作處理了 \m{x[i]}。

\item {\EMP 插入}(insert)——將字符 \m{c} 插入 \m{z} 中, \m{z[j]=c},
將 \m{j} 增 1, \m{i} 不變。此操作未處理 \m{x} 中字符。

\item {\EMP 旋轉}(twiddle,即交換)——將 \m{x} 中下兩個字符複製到 \m{z} 中,
但交換順序, \m{z[j]=x[i+1]}、 \m{z[j+1]=x[i]},
將 \m{i} 和 \m{j} 都增 2。此操作處理了 \m{x[i]} 和 \m{x[i+1]}。

\item {\EMP 終止}(kill)——刪除 \m{x} 中剩餘字符,令 \m{i=m+1}。
此操作處理了 \m{x} 中所有尚未處理的字符。
如果執行此操作,則轉換過程結束。
\stopigBase

下面給出了將源字串 {\tt algorithm} 轉換爲目標字串 {\tt altruistic} 的一種變換操作序列,
下劃線指出執行一個變換操作後兩個下標的位置,
注意,還有其他方法完成這個轉換。

\startxtable[
    option=max,
    align={flushleft,lohi},
    split=yes,
    header=repeat,
    footer=repeat,
    offset=.25em,
]

% head
\startxtablehead[frame=off,topframe=on,bottomframe=on,align={middle}]
\startxrow[foregroundstyle=bold,]
  \xcell{操作}\xcell[frame=on]{x}\xcell{z}
\stopxrow
\stopxtablehead

% body
\startxtablebody[frame=off,topframe=on,bottomframe=on,foregroundstyle={\tt}]
\startxrow \xcell{初始字符串}\xcell{\underbar{a}lgorithm}\xcell{\uerbarspace} \stopxrow
\startxrow \xcell{複製}\xcell{a\underbar{l}gorithm}\xcell{a\uerbarspace} \stopxrow
\startxrow \xcell{複製}\xcell{al\underbar{g}orithm}\xcell{al\uerbarspace} \stopxrow
\startxrow \xcell{替換爲 t}\xcell{alg\underbar{o}rithm}\xcell{alt\uerbarspace} \stopxrow
\startxrow \xcell{刪除}\xcell{algo\underbar{r}ithm}\xcell{alt\uerbarspace} \stopxrow
\startxrow \xcell{複製}\xcell{algor\underbar{i}thm}\xcell{altr\uerbarspace} \stopxrow
\startxrow \xcell{插入 u}\xcell{algor\underbar{i}thm}\xcell{altru\uerbarspace} \stopxrow
\startxrow \xcell{插入 i}\xcell{algor\underbar{i}thm}\xcell{altrui\uerbarspace} \stopxrow
\startxrow \xcell{插入 s}\xcell{algor\underbar{i}thm}\xcell{altruis\uerbarspace} \stopxrow
\startxrow \xcell{旋轉}\xcell{algorit\underbar{h}m}\xcell{altruisti\uerbarspace} \stopxrow
\startxrow \xcell{插入 c}\xcell{algorit\underbar{h}m}\xcell{altruistic\uerbarspace} \stopxrow
\startxrow \xcell{終止}\xcell{algorithm\uerbarspace}\xcell{altruistic\uerbarspace} \stopxrow
\stopxtablebody

\stopxtable



每個變換操作都有相應的代價。
具體代價依賴於特定應用,但我們假定每個操作的代價均爲已知常量。
我們還假定複製和替換的代價小於刪除和插入的組合代價,否則複製和替換操作就沒有意義了。
一個給定的變換操作序列的代價爲其中所有變換操作的代價總和。
在上例中,將 {\tt algorithm} 轉換爲 {\tt altruistic} 的代價爲:
\m{3 \times \COST(\text{複製})
+ \COST (\text{替換})
+ \COST (\text{刪除})
+ 4 \times \COST (\text{插入})
+ \COST (\text{交換})
+ \COST (\text{終止})}。

\startigNum
\startitem
給定兩個字串 \m{x[1..m]} 和 \m{y[1..n]} 以及變換操作的代價,
 \m{x} 到 \m{y} 的{\EMP 編輯距離}(edit distance)是
將 \m{x} 轉換爲 \m{y} 所需變換序列的最小代價。
設計動態規劃算法,求 \m{x[1..m]} 到 \m{y[1..n]} 的編輯距離並打印最優變換序列。
分析算法的時間、空間複雜度。
\stopitem
\stopigNum

\startANSWER
子問題:將 \m{x[1..i]} 轉換成 \m{y[1..j]},
其中 \m{1\le i\le m}, \m{1\le j\le n},
其代價爲 \m{D(i,j)}。則:
\startformula\startmathalignment
\NC D(0,0) \NC = 0 \NR
\NC D(i,j) \NC = \min\startcases
\NC D(i-1,j-1) + \COST(\text{複製}) \MC \text{如果\m{x_i=y_j};}\NR
\NC D(i-1,j-1) + \COST(\text{替換}) \MC \text{如果\m{x_i\ne y_j};}\NR
\NC D(i-1,j) + \COST(\text{刪除}) \MC \NR
\NC D(i,j-1) + \COST(\text{插入}) \MC \NR
\NC D(i-2,j-2) + \COST(\text{交換}) \MC \text{如果\m{[x_{i-1}x_i]=[y_i y_{i-1}]};}\NR
\stopcases \NR
\stopmathalignment\stopformula
除 \m{D} 外,我們還可以維護一張表用來記錄執行了哪些操作。
最小代價爲
\startformula
\min(D[m,n], \min_{i<m}D[i,n] + \COST[\text{終止}])
\stopformula

時間、空間複雜度均爲 \m{\Theta(mn)}。
\stopANSWER

編輯距離問題是 DNA 序列對齊問題的推廣
(參考其他文獻,如 Setubal 和 Meidanis [310, 3.2 節])。
已有多種方法可以通過對齊兩個 DNA 序列來衡量他們的相似度。
有一種對齊方法是將空格符插入到兩個序列 \m{x} 和 \m{y} 中,
可以插入到任何位置(包括兩端),
使得結果序列 \m{x'} 和 \m{y'} 的長度相同,
但不會在相同的位置出現空格符
(即不存在位置 \m{j} 使得 \m{x'[j]} 和 \m{y'[j]} 都是空格符)。
然後爲每個位置“打分”,位置 \m{j} 的分數爲:
\startigBase
\item +1,如果 \m{x'[j]=y'[j]} 且不是空格符;
\item -1,如果 \m{x'[j]\ne y'[j]} 且都不是空格符;
\item -2, \m{x'[j]} 或 \m{y'[j]} 是空格符;
\stopigBase

對齊方案的分數爲每個位置的分數之和。
例如,給定序列 \m{x=\text{\tt GATCGGCAT}} 和 \m{y=\text{\tt CAATGTGAATC}},
一種對齊方案爲:

{\tt G ATCG GCAT }

{\tt CAAT GTGAATC}

{\tt -*++*+*+-++*}

其中 + 表示該位置分數爲 +1, - 表示分數爲 -1, * 表示分數爲 -2,
因此此方案的總分數爲 \m{6\times 1 - 2 \times 1 - 4 \times 2 = -4}。

\startigNum[continue]
\startitem
解釋如何將最優對齊問題轉換爲編輯距離問題,
使用的操作爲變換操作復制、替換、刪除、插入、旋轉和終止的子集。
\stopitem
\stopigNum

\startANSWER
如果 \m{x'[j]=y'[j]} 且不是空格符,則等效爲複製;
如果 \m{x'[j]} 爲空格, \m{y'[j]} 不是空格,則等效爲插入;
如果 \m{x'[j]} 不是空格, \m{y'[j]} 是空格,則等效爲刪除;
如果 \m{x'[j]\ne y'[j]} 且都不是空格符,則等效爲替換。
\stopANSWER

\stopPROBLEM

%p15-6
\startPROBLEM
(Planning a company party)
一位 CEO 正在向 Stewart 教授諮詢公司聚會的計劃。
公司的內部結構關係是層次化的,
即員工按主管——下屬關係構成一棵樹,根節點爲 CEO。
人事部按“宴會交際能力”爲每個員工打分,分值爲實數。
爲了使所有參加聚會的員工都感到愉快, CEO 不希望員工及其主管同時出席。

公司主席向 Stewart 教授提供公司結構樹,
採用 10.4 節介紹的左孩子右兄弟表示法描述。
樹中每個節點除保存指針外,
還保存員工的名字和宴會交際評分。
設計算法,求宴會交際評分之和最大的賓客名單。
分析算法的時間複雜度。
\stopPROBLEM

\startANSWER
令 \m{D[i]} 爲僅邀請 \m{i} 及其下屬的最大得分。
令 \m{E[i]} 爲僅邀請 \m{i} 的下屬的最大得分。
則:
\startformula\startmathalignment
\NC E[i] \NC = \sum_{j\in C_i}D[j] \NR
\NC D[i] \NC = \max(E[i], a_i + \sum_{j\in C_i}E[j]) \NR
\stopmathalignment\stopformula

\m{D[1]} 即爲所求。
由於 \m{D[i]} 和 \m{E[i]} 僅依賴於 \m{D[j]} 和 \m{E[j]},
其中 \m{j} 爲 \m{i} 的下屬,我們可以按樹中自下而上的順序進行計算。
如果我們知道 \m{i} 的下屬序號均大於 \m{i},
則可以按從 \m{n} 到 \m{1} 的順序計算 \m{E[i]}、 \m{D[i]}。

\CLRSH{PLAN-PARTY(A, C)}
\startCLRS
for i = n downto 1
	E[i] = 0
	D[i] = 0
	foreach j ∈ C[i]
		E[i] = E[i] + D[j]
		D[i] = D[i] + E[j]
	D[i] = D[i] + A[i]
	if E[i] > D[i]
		D[i] = E[i]
\stopCLRS

第 \m{i} 次迭代中的循環需要 \m{2|C_i|+4} 步,
總共需要時間 \m{4n + 2\sum_i|C_i|}。
除 CEO 外,其他員工都有一個直接領導,因此 \m{\sum_i|C_i|=n-1}。
因此總時間爲 \m{\Theta(n)}。
\stopANSWER

%p15-7
\startPROBLEM
(Viterbi algorithm)
我們可以通過在有向圖 \m{G=(V,E)} 上使用動態規劃方法實現語音識別。
對每條邊 \m{(u,v)\in E} 打上標籤 \m{\sigma(u, v)},
該聲音來自於有限聲音集 \m{\Sigma}。
這樣的標籤圖就成爲一個特定人說限定語言的形式化模型。
圖中從特定頂點 \m{v_0\in V} 開始的每條路徑都對應模型產生的一個可能的語音序列。
對於一條有向路徑,我們定義其標籤爲路徑中邊上標籤的簡單連接。
\startigNum
\startitem
設計高效算法,對給定的圖 G (邊上帶標籤)、
特定頂點 \m{v_0} 及 \m{\Sigma} 上的聲音序列
 \m{s=\langle \sigma_1,\sigma_2,\ldots,\sigma_k\rangle},
返回 \m{G} 中從 \m{v_0} 開始的一條路徑, \m{s} 爲該路徑的標籤(如果存在這樣的路徑)。
否則,算法返回 \ALGO{NO-SUCH-PATH}。
分析算法的時間複雜度(\hint 你可能發現第 22 章中的概念可用於此題)。
\stopitem
\stopigNum

\startANSWER
\startformula\startmathalignment
\NC S(i,j) \NC = \bigvee_{k:(v_k,v_i)\in E \land \sigma(v_k,v_i)=\sigma_j} S(k,j-1) \NR
\NC S(0,0) \NC = 1 \NR
\NC S(i,0) \NC = 0 \qquad 1\le i \le n-1 \NR
\stopmathalignment\stopformula
\stopANSWER

現在,假定每條邊 \m{(u,v)\in E} 都關聯一個非負概率 \m{p(u,v)},
他表示從頂點 \m{u} 開始,經過邊 \m{(u,v)},產生對應的聲音的概率。
任何頂點的出射邊的概率之和均爲 1。
一條路徑的概率定義爲路徑上邊的概率之積。
對於從 \m{v_0} 開始的一條路徑,
我們可以將其概率看作從 \m{v_0} 開始進行“隨機遊走”(random walk),
最後恰巧經過這條路徑的概率。
所謂“隨機遊走”,是指當位於頂點 \m{u} 時,隨機選擇一條出射邊前進,
每條邊被選中的概率jiujitsu他所關聯的概率。

\startigNum[continue]
\startitem
擴展(a)中的算法,使得返回的路徑從 \m{v_0} 開始、標籤爲 \m{s},且概率最大。
分析算法的時間複雜度。
\stopitem
\stopigNum

\startANSWER
\startformula\startmathalignment
\NC S(i,j) \NC = \max_{k:(v_k,v_i)\in E \land \sigma(v_k,v_i)=\sigma_j}
  \left{p(v_k,v_i) \times S(k,j-1)\right} \NR
\NC S(0,0) \NC = 1 \NR
\NC S(i,0) \NC = 0 \qquad 1\le i \le n-1 \NR
\NC R(i,j) \NC = \argmax_{k:(v_k,v_i)\in E \land \sigma(v_k,v_i)=\sigma_j}
  \left{p(v_k,v_i) \times S(k,j-1)\right} \NR
\stopmathalignment\stopformula

計算每一列時,僅需要考慮左側相鄰列。
計算一列的時間爲 \m{O(n+e)},共 \m{k} 列,
因此總時間爲 \m{O((n+e)\times k)}。
其中 \m{e} 爲邊的數目, \m{k} 爲輸入序列的長度。
\stopANSWER
\stopPROBLEM

%p15-8
\startPROBLEM
(Image compression by seam carving)
給定一副彩色圖像,他由數組 \m{A[1..m,1..n]} 構成,
每個元素都是包含紅綠藍(RGB)亮度的三元組。
假如我們希望輕度壓縮這幅圖像。具體而言,
我們希望從每一行中刪除一個像素,使得圖像變窄一個像素。
但爲了避免影響視覺效果,我們要求相鄰兩行中刪除的像素必須位於統一列或相鄰列。
也就是說,刪除的像素構成從頂端行到底端行的一條“接縫”(seam),
相鄰像素均在垂直或對角線方向上相鄰。
\startigNum
\startitem
證明:可能的接縫數量是 \m{m} 的指數函數,假定 \m{n>1}。
\stopitem
\stopigNum

\startANSWER
窮舉法,第一行有 \m{n} 種選擇,以後每一行都有 3 種選擇(不考慮邊界的特殊性)。
共 \m{n\times (m-1)^3} 種選擇,命題得證。
\stopANSWER

\startigNum[continue]
\startitem
假定現在對每個像素 \m{A[i,j]},我們都已計算出其“破壞度” \m{d[i,j]} (實數),
表示刪除像素 \m{A[i,j]} 對圖像可視效果的破壞程度。
直觀地,一個像素的破壞度越低,他與相鄰像素的相似度越高。
再假定一條接縫的破壞度定義爲他包含的像素的破壞度之和。
設計算法,尋找破壞度最低的接縫。
分析算法的時間複雜度。
\stopitem
\stopigNum

\startANSWER
定義 \m{a[i,j]}:如果接縫只包含第 1 行到第 \m{i} 行,選擇了 \m{A[i,j]} 作爲接縫時的最小破壞度。
則:
\startformula
a[i,j] = d[i,j] + \min(a[i-1,j-1], a[i-1,j], a[i-1,j+1])
\stopformula
注意,對於第一行元素 \m{a[i,j] = d[i,j]},其他行邊緣像素只有兩種選擇。

算法複雜度爲 \m{O(nm)}。
\stopANSWER

\stopPROBLEM

%p15-9
\startPROBLEM
(Breaking a string)
某種字串處理語言允許程序員將一個字符串拆分爲兩段。
由於此操作需要複製字串,因此要話費 \m{n} 個時間單位將長度爲 \m{n} 的字符串拆分爲兩段。
假定一個程序員希望將一個字串拆分爲多段,
拆分的順序會影響所花費的總時間。
例如,假定這個程序員希望將長度爲 20 的字串在第 2 個、第 8 個以及第 10 個字符後進行拆分
(字符從左至右,從 1 開始編號)。
如果他按由左至右的順序進行拆分,則第一次拆分花費 20 個時間單位,
第二次拆分花費 18 個時間單位(在第 8 個字符處拆分 3~20 間的字串),
第三次拆分花費 12 個時間單位,共花費 50 個時間單位。
但如果從右至左進行拆分,所花時間爲 20、 10、 8 共 38 個時間單位。
還可以其他順序進行拆分,如,可以線在第 8 個字符處拆分(時間 20),
接着在左邊第 2 個字符處拆分(時間 8),
最後在右邊第 10 個字符處拆分(時間 12),總時間爲 40。

設計算法,對給定的拆分位置,確定拆分順序,使其代價最小。
更形式化地,給定長度爲 \m{n} 的字串 \m{S} 和拆分點序列 \m{L[1..m]},
計算拆分的最小代價,以及最優拆分序列。

\startANSWER
\startformula
c[i,j] = L[i,j] + \min_{i<k<j}(c[i,k] + c[k,j]) \qquad \text{\m{k} 爲拆分位置}
\stopformula
\stopANSWER
\stopPROBLEM

%p15-10
\startPROBLEM
(Planning an investment strategy)
你所掌握的算法知識從 Acme 計算機公司獲得了一份令人興奮的工作,
簽約獎金 1 萬美元。
你決定利用這筆錢進行投資,目標是 10 年後獲得最大回報。
你決定請 Amalgamated 投資公司管理你的投資,
該公司的投資回報規則如下。
該公司提供 \m{n} 種不同的投資,從 1~\m{n} 編號。
在第 \m{j} 年,第 \m{i} 種投資的回報率爲 \m{r_{ij}}。
換句話說,如果你在第 \m{j} 年第 \m{i} 種投資投入 \m{d} 美元,
那麼在第 \m{j} 年底,你會得到 \m{d r_{ij}} 美元。
回報率是有保證的,即未來 10 年每種投資的回報率均已知。
你每年只能做出一次投資決定。
在每年年底,你既可以將錢繼續投入到上一年選擇的投資種類中,
也可以轉移到其他投資中(轉移到已有的投資種類,或者新的投資種類)。
如果跨年時不做投資轉移,
需要支付 \m{f_1} 美元的費用,
否則,需要支付 \m{f_2} 美元的費用,其中 \m{f_2 > f_1}。

\startigNum
\startitem
如上所述,本題允許你每年將錢投入到多種投資中。
證明:如果每年都將所有錢投入到單一投資中,存在最優投資策略
(記住最優投資策略只需最大化 10 年的回報,
無需關心任何其他目標,如最小化風險)。
\stopitem
\stopigNum

\startigNum[continue]
\startitem
證明:規劃最優投資策略問題具有最優子結構性質。
\stopitem
\stopigNum

\startigNum[continue]
\startitem
設計最優投資策略規劃算法,分析算法時間複雜度。
\stopitem
\stopigNum

\startANSWER
令 \m{P_i} 爲第 \m{i} 年拿到的利潤。
令 \m{S_i} 爲第 \m{i} 年選擇的項目。
令 \m{r[i,j]} 爲第 \m{i} 項目在第 \m{j} 年的利潤率。
令 \m{R_j} 爲第 \m{j} 年利潤率最高的項目編號。
\startformula
P_i = \max(
(P_{i-1} - f_1) \times r_{S_{i-1} i},
(P_{i-1} - f_2) \times r_{R_i i}
)
\stopformula

時間複雜度爲 \m{\lg n}。
\stopANSWER

\startigNum[continue]
\startitem
假定 Amalgamated 投資公司在上述規則上又加入了新的限制條款,
在任何時刻你都不能在任何單一投資種類中投入 15000 美元以上。
證明:最大化 10 年回報問題不再具有最優子結構性質。
\stopitem
\stopigNum
\stopPROBLEM

%p15-11
\startPROBLEM
(Inventory planning)
Rinky Dink 公司是一家製造溜冰場冰面整修設備的公司。
這種設備每個月的需求量都在變化,因此公司希望設計一種策略來規劃生產,
需求是給定的,即他雖然是波動的,但是可預測的。
公司希望設計接下來 \m{n} 個月的生產計劃。
對第 \m{i} 個月,公司知道需求 \m{d_i},
即該月能夠銷售出去的設備數量。
令 \m{D=\sum_{i=1}{n} d_i} 爲後 \m{n} 個月的總需求。
公司僱傭的全職員工,可以提供一個月製造 \m{m} 臺設備的勞動力。
如果公司希望一個月內製造多於 \m{m} 臺設備,
可以僱傭額外的兼職勞動力,
僱傭稱霸爲每製造一臺機器付出 \m{c} 美元。
而且,如果在月末有設備尚未售出,
公司還要付出庫存成本。
保存 \m{j} 臺設備的成本可描述爲一個函數 \m{h(j)},
其中 \m{1\le j \le D}、 \m{h(j)\ge 0},
且對於所有 \m{1\le j\le D-1},滿足 \m{h(j)\le h(j+1)}。

設計庫存規劃算法,在滿足所有需求的前提下最小化成本。
算法運行時間應爲 \m{n} 和 \m{D} 的多項式函數。
\stopPROBLEM

\startANSWER
令 \m{L_i} 爲第 \m{i} 月的產品剩餘; \m{C_i} 爲第 \m{i} 月的成本。
\startformula
C_i = \startcases
\NC h(L_{i-1}-d_i) \MC \text{如果 \m{L_{i-1}\ge d_i};(\m{L_i=L_{i-1}-d_i})} \NR
\NC 0 \MC \text{如果 \m{d_i - m\le L_{i-1} \le d_i};(\m{L_i=0})} \NR
\NC c\times (L_{i-1}+m-d_i) \MC \text{如果 \m{L_{i-1} < di - m};(\m{L_i=0})} \NR
\stopcases
\stopformula

\m{C} 依賴於 \m{L}, \m{L_i} 依賴於 \m{L_{i-1}}。
\stopANSWER

%p15-12
\startPROBLEM
(Signing free-agent baseball players)
假設你是一支棒球大聯盟球隊的總經理。
在賽季休季期間,你需要簽入一些自由球員。
球隊老闆給你的預算爲 \m{X} 美元,
你可以使用少於 \m{X} 美元來簽入球員,
但如果超支,球隊老闆就會解僱你。

你正在考慮在 \m{N} 個不同位置簽入球員,
在每個位置上,有 \m{P} 個該位置的自由球員供你選擇。
由於你不希望任何位置過於臃腫,
每個位置最多簽入一名球員,
如果在某個位置上你沒有簽入新球員,則意味着計劃繼續使用現有球員。

爲了確定一名球員的價值,你決定使用一種稱爲“VORP”,
或“球員替換價值”(value over replacement player)的統計評價指標(sabermetric)。
球員的 VORP 值越高,其價值越高。
但 VORP 值高的球員的簽約費用並不一定比 VORP 值低的球員高,
因爲還有球員價值之外的因素影響簽約費用。

對於每個可選擇的自由球員,你知道他三方面信息:
\startigBase
\item 他打哪個位置;
\item 他的簽約費用;
\item 他的 VORP。
\stopigBase

設計一個球員選擇算法,使得簽約總費用不超過 \m{X} 美元,
而球員的 VORP 總值最大。
你可以假定每位球員的簽約費用是 10 萬美元的整數倍。
算法應輸出簽約球員的 VORP 總值、簽約總費用,以及球員名單。
分析算法的時間、空間複雜度。
\stopPROBLEM

\startANSWER
令 \m{c_{ij}} 爲第 \m{i} 個位置、第 \m{j} 個球員的價碼;
令 \m{v_{ij}} 爲第 \m{i} 個位置、第 \m{j} 個球員的價值;
共有 \m{n} 個位置。
令 \m{f(m,n)} 爲用錢 \m{m} 購買 \m{1..n} 位置球員的最大價值總和。

如果在第 \m{n} 個位置購買球員,則 \m{f(m,n)=f(m,n-1)};
否則 \m{f(m,n) = \max_{1\le i\le p \land m > c_{ni}}(v_{ni} + f(m-c_{ni},n-1))}。
最終 \m{f(m,n)} 是兩種情況中值較大的那個。
\stopANSWER

\stopsubject%Problems
