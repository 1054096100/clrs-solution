\startsection[
  title={Polynomial-time verification},
]

%e34.2-1
\startEXERCISE
考慮語言 \m{\text{\ALGO{GRAPH-ISOMORPHISM}} =
 \{\langle G_1,G_2\rangle: \text{\m{G_1} 和 \m{G_2} 是同構圖}\}}。
通過描述一個可以在多項式時間內驗證該語言的算法,
來證明 \m{\text{\ALGO{GRAPH-ISOMORPHISM}} \in NP}。
\stopEXERCISE

\startANSWER
\TODO{略。}
\stopANSWER

%e34.2-2
\startEXERCISE
證明:如果 \m{G} 是一個有奇數個頂點的無向二分圖,
則 \m{G} 是非哈密頓圖。
\stopEXERCISE

\startANSWER
\TODO{略。}
\stopANSWER

%e34.2-3
\startEXERCISE
證明:如果 \m{\text{\ALGO{HAM-CYCLE}} \in P},
則可在多項式時間內按序列出一條哈密頓迴路中各個頂點。
\stopEXERCISE

\startANSWER
\TODO{略。}
\stopANSWER

%e34.2-4
\startEXERCISE
證明:由語言構成 NP 類在並集、空集、連接和 Kleene 星運算下是封閉的,
討論一下 NP 在補集運算下的封閉性。
\stopEXERCISE

\startANSWER
\TODO{略。}
\stopANSWER

%e34.2-5
\startEXERCISE
證明:對某個常數 \m{k}, NP 中的任何語言都可以
用一個運行時間爲 \m{2^{O(n^k)}} 的算法加以判定。
\stopEXERCISE

\startANSWER
\TODO{略。}
\stopANSWER

%e34.2-6
\startEXERCISE
圖中的哈密頓路徑是一種簡單路徑,
他遍歷圖中所有頂點,且每個頂點僅一次。
證明:語言 \m{\text{\ALGO{HAM-PATH}} = \{\langle G,u,v\rangle:
\text{圖 \m{G} 中存在一條從 \m{u} 到 \m{v} 的哈密頓路徑}} 屬於 NP。
\stopEXERCISE

\startANSWER
\TODO{略。}
\stopANSWER

%e34.2-7
\startEXERCISE
\refexercise{34.2-6} 中提出了哈密頓路徑問題,
證明:
在有向無環圖中,哈密頓路徑問題可以在多項式時間內求解。
給出一個有效算法解決此問題。
\stopEXERCISE

\startANSWER
\TODO{略。}
\stopANSWER

%e34.2-8
\startEXERCISE
設 \m{\phi} 爲一個布爾公式,
他由布爾輸入變量 \m{x_1,x_2,\ldots,x_k}、
非(\m{\neg})、 AND (\m{\wedge})、 OR (\vee)和括號組成。
如果對輸入變量的每一種 1 和 0 賦值,公式結果都是 1,
則此公式爲{\EMP 重言式}(tautology)。
定義 \ALGO{TAUTOLOGY} 爲由重言式布爾公式所組成的語言。
證明: \m{\text{\ALGO{TAUTOLOGY}}\in \text{\ALGO{co-NP}}}。
\stopEXERCISE

\startANSWER
\TODO{略。}
\stopANSWER

%e34.2-9
\startEXERCISE
證明: \m{\text{\ALGO{P}}\subseteq \text{\ALGO{co-NP}}}。
\stopEXERCISE

\startANSWER
\TODO{略。}
\stopANSWER

%e34.2-10
\startEXERCISE
證明:如果 \m{\text{\ALGO{NP}}\ne \text{\ALGO{co-NP}}},
則 \m{P\ne NP}。
\stopEXERCISE

\startANSWER
\TODO{略。}
\stopANSWER

%e34.2-11
\startEXERCISE
設 \m{G} 爲一個至少包含 3 個頂點的連通無向圖,
並設對 \m{G} 中所有由長度至多爲 3 的路徑連接起來的點對,
將他們直接連接後所性層的圖爲 \m{G^3}。
證明: \m{G^3} 是一個哈密頓圖。
(\hint 爲 \m{G} 構造一棵生成樹,並採用歸納法進行證明。)
\stopEXERCISE

\startANSWER
\TODO{略。}
\stopANSWER

\stopsection
