\startsection[
  title={Polynomial time},
]

%e34.1-1
\startEXERCISE
定義最優化問題 \ALGO{LONGEST-PATH-LENGTH} 爲一個關係,
他將一個無向圖的每個實例、兩個頂點與這兩個頂點間的
一條最長簡單路徑中所包含的邊數聯繫了起來。
定義判定問題 \m{\text{\ALGO{LONGEST-PATH}} = \{\langle G,u,v,k\rangle\}: \text{
\m{G=(V,E)} 爲一個無向圖, \m{u,v\in V}, \m{k\ge 0} 是一個整數,
且 \m{G} 中存在一條從 \m{u} 到 \m{v} 的簡單路徑,
他至少包含 \m{k} 條邊}\}}。
證明:最優化問題 \ALGO{LONGEST-PATH-LENGTH} 可以在多項式時間內解決,
當且僅當 \m{\text{\ALGO{LONGEST-PATH}}\in P}。
\stopEXERCISE

\startANSWER
\TODO{略。}
\stopANSWER

%e34.1-2
\startEXERCISE
對於在無向圖中尋找最長簡單迴路這一問題,
給出其形式劃的定義並給出其相關的判定問題。
另外,給出與該判定問題對應的語言。
\stopEXERCISE

\startANSWER
\TODO{略。}
\stopANSWER

%e34.1-3
\startEXERCISE
給出一種形式化的編碼,
他利用鄰接矩陣的表示形式,
將有向圖編碼爲二進制串。
另外,再給出利用鄰接表表示的編碼。
論證兩種表示形式是多項式相關的。
\stopEXERCISE

\startANSWER
\TODO{略。}
\stopANSWER

%e34.1-4
\startEXERCISE
\refexercise{16.2-2} 中曾要求讀者給出 0-1 揹包問題的“動態規劃算法”,
他是一個多項式時間的算法嗎?解釋你的答案。
\stopEXERCISE

\startANSWER
\TODO{略。}
\stopANSWER

%e34.1-5
\startEXERCISE
證明:對於一個多項式時間的算法,
當他調用一個多項式時間的子例程時,
如果至多調用常數次,
則此算法以多項式時間運行,
但是,當進行多項式次的子例程調用時,
此算法可能變成一個指數時間的算法。
\stopEXERCISE

\startANSWER
\TODO{略。}
\stopANSWER

%e34.1-6
\startEXERCISE
證明:類 \m{P} 在被看作是一個語言集合時,
在並集、交集、連接、補集和 KIeene 星運算下是封閉的。
也就是說,
如果 \m{L_1,L_2\in P},則 \m{L_1\cup L_2\in P},
 \m{L_1\cap L_2\in P}, \m{L_1L_2\in P}, \m{\overbar{L_1}\in P}, \m{L_1^*\in P}。
\stopEXERCISE

\startANSWER
\TODO{略。}
\stopANSWER

\stopsection
