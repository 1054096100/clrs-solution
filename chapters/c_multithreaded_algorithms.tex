\startcomponent c_multithreaded_algorithms

\startchapter[
  title={Multithreaded Algorithms},
]

%e27.1-1
\startEXERCISE
假設 \ALGO{P-FIB} 中第 4 行派生調用 \ALGO{P-FIB(n-2)},
而不是像源程序中使用普通調用的方法,則漸進工作量、
持續時間和並行度各是多少?
附 \ALGO{P-FIB}:

\CLRSH{P-FIB(n)}
\startCLRS
if n <= 1
	return n
else
	x = spawn P-FIB(n-1)
	y = P-FIB(n-2)
	sync
	return x + y
\stopCLRS
\stopEXERCISE

\startANSWER
沒有變化。
\stopANSWER

%e27.1-2
\startEXERCISE
請畫出運行 \ALGO{P-FIB(5)} 的計算有向無環圖。
假設計算中的每個鏈消耗單位時間,
則該計算的工作量、持續時間和並行度各是多少?
如何在 3 個處理器上調度這個計算有向無環圖,
要求使用貪心調度並用執行中的時間步給每個鏈做標記。
\stopEXERCISE

\startANSWER
工作量: \m{T_1 = 29}。

持續時間: \m{T_{\infty} = 10}。

並行度: \m{T_1 / T_{\infty} \approx 2.9}。

\externalfigure[output/e27_1_2-1][factor=fit]
\stopANSWER

%e27.1-3
\startEXERCISE
證明:貪心調度可以達到下面的時間界,
該時間界稍微強於定理 27.1 給出的界:
\startformula
T_p \le \frac{T_1 - T_{\infty}}{P} + T_{\infty}
\stopformula
\stopEXERCISE

\startANSWER
\m{T_{\infty}} 已經包括了所有非完全步和完全步。
在有 \m{P} 個處理器的系統上,一個完全步會分配 \m{P} 個線程。
回顧 \m{P\times T_P \ge T_1}: \m{P} 個處理器所完成的工作量至少與一個處理完完成的一樣多。

在定理 27.1 中,我們分別計算完全步和非完全步的界,
現在我們嘗試另一種方法。

根據定義, \m{T_{\infty}} 是計算圖中最長路徑。
在此路徑上,每一個節點都代表分配一個線程:
如果準備好執行的線程數少於 \m{P},則是非完全步,否則是完全步。

一個完全步可以分配 \m{kP+j} 股,其中 \m{0\le j < P}, \m{k\ge 1},
如果 \m{j = 0},則(可能)可以分解成一個完全步以及後續多個個完全步,
否則(可能)可以分解成一個非完全步以及後續多個完全步。
我們稱第一個非完全步爲{\EMP 僞}非完全步,第一個完全步爲{\EMP 僞}完全步。
所有非完全步、僞非完全步和僞完全步的總數 \m{\le T_{\infty}},
在任何一個點上我們分配的股數都不會多於 \m{P}。

\m{T_1 - T_{\infty}} 就是最長路徑之外的所有股數。
其中每一股都屬於下面四種情況之一:
 \m{T_i},非完全步中的股;
 \m{T_{pi}},僞非完全步中的股;
 \m{T_{pc}},僞完全步中的股;
 \m{T_c},在僞完全步或僞非完全步之後所執行的分配 \m{P} 個線程所得股。

\m{|T_i| + |T_{pi}| + |T_{pc}| \ge T_{\infty}},
最長路徑中的每一股都屬於這三個集合中的一個。
這意味着剩下的那個集合 \m{|T_c|\le T_1 - T_{\infty}},
根據上面所述, \m{|T_c|} 是 \m{P} 的倍數。

\startformula
\frac{|T_c|}{P} + T_{\infty}\le \frac{T_1 - T_{\infty}}{P} + T_{\infty}
\stopformula
\stopANSWER

\startsection[
  title={The basics of dynamic multithreading},
]
\stopsection

\startsection[
  title={Multithreadedmatrix multiplication},
]
\stopsection

\startsection[
  title={Multithreaded merge sort},
]
\stopsection

\startsubject[
  title={Problems},
]

\stopsubject%Problems

\stopchapter
\stopcomponent
