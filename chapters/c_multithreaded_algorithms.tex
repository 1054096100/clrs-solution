\startcomponent c_multithreaded_algorithms

\startchapter[
  title={Multithreaded Algorithms},
]

%e27.1-1
\startEXERCISE
假設 \ALGO{P-FIB} 中第 4 行派生調用 \ALGO{P-FIB(n-2)},
而不是像源程序中使用普通調用的方法,則漸進工作量、
持續時間和並行度各是多少?
附 \ALGO{P-FIB}:

\CLRSH{P-FIB(n)}
\startCLRS
if n <= 1
	return n
else
	x = spawn P-FIB(n-1)
	y = P-FIB(n-2)
	sync
	return x + y
\stopCLRS
\stopEXERCISE


\startsection[
  title={The basics of dynamic multithreading},
]
\stopsection

\startsection[
  title={Multithreadedmatrix multiplication},
]
\stopsection

\startsection[
  title={Multithreaded merge sort},
]
\stopsection

\startsubject[
  title={Problems},
]

\stopsubject%Problems

\stopchapter
\stopcomponent
