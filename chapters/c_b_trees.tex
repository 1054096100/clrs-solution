\startcomponent c_b_trees

\startchapter[
  title={B-Trees},
]

\startsection[
  title={Definition of B-trees},
]
%e18.1-1
\startEXERCISE
爲什麼不允許最小度數 \m{t=1}?
\stopEXERCISE

\startANSWER
根據定義,最小度數 \m{t} 意味着所有節點(根節點除外)都至少含有 \m{t-1} 個關鍵字。
從而使得根節點以外的內部節點都至少有 \m{t} 個孩子節點。
所以,如果 \m{t=1},則意味着根節點之外的節點最少可以只有 \m{t-1=0} 個關鍵字,
根節點之外的內部節點最少可以只有 \m{t=1} 個孩子節點。

因此,最小度數不能是 1。
\stopANSWER

%e18.1-2
\startEXERCISE
當 \m{t} 取何值時,圖 18-1 所示的樹是一棵合法的 B 樹?
\stopEXERCISE

\startANSWER
根節點之外的節點至少要有 \m{t-1} 個關鍵字,最多有 \m{2t-1} 個關鍵字。
根據圖 18-1,有 \m{t-1\le 2} 且 \m{3\le 2t-1},求解得 \m{2\le \le 3},
因此 \m{t} 可以是 2 或 3。
\stopANSWER

%e18.1-3
\startEXERCISE
請給出表示 \m{\{1,2,3,4,5\}} 的最小度數爲 2 的所有合法 B 樹。
\stopEXERCISE

\startANSWER
略。
\stopANSWER

%e18.1-4
\startEXERCISE
一棵高度爲 \m{h} 的 B 樹中,可以存儲最多多少個關鍵字?
用最小度數 \m{t} 的函數表示。
\stopEXERCISE

\startANSWER
每個節點最多包含 \m{2t-1} 個關鍵字。每個節點最多有 \m{2t} 個孩子。
\startformula\startmathalignment
\NC \NC (2t-1)[(2t)^0 + (2t)^1 + \ldots + (2t)^h] \NR
\NC = \NC (2t-1)\sum_{i=0}^{h}(2t)^i \NR
\NC = \NC (2t-1)\frac{(2t)^{h+1} - 1}{2t-1} \NR
\NC = \NC (2t)^{h+1} - 1 \NR
\stopmathalignment\stopformula
\stopANSWER

%18.1-5
\startEXERCISE
如果紅黑樹中每個黑節點吸收他的紅色孩子,
並把他們的孩子併入作爲自己的孩子,
描述這個結果的數據結構。
\stopEXERCISE

\startANSWER
對於黑色節點而言,他可能有一個或兩個紅孩子,也可能沒有,合併後,此節點將有 1、 2 或 3 個關鍵字。
此後所有葉子節點將具有同樣的高度(根據紅黑樹性質 5:從一個節點到所有其後代葉子節點路徑上的黑色節點數目相同)。
因此,這棵紅黑樹變成了一棵 \m{t=2} 的 B 樹,即 2-3-4 樹。
\stopANSWER

\stopsection

\startsection[
  title={Basic operations on B-trees},
]

%e18.2-1
\startEXERCISE
請給出關鍵字 \m{F}、 \m{S}、 \m{Q}、 \m{K}、 \m{C}、 \m{L}、 \m{H}、 \m{T}、 \m{V}、 \m{W}、
 \m{M}、 \m{R}、 \m{N}、 \m{P}、 \m{A}、 \m{B}、 \m{X}、 \m{Y}、 \m{D}、 \m{Z}、 \m{E} 依序
插入一棵最小度數爲 2 的空 B 樹的結果。
只要畫出在某些節點分裂之前的結構以及最終的結構。
\stopEXERCISE

\startANSWER
\externalfigure[output/e18_2_1-1]
\stopANSWER

%e18.2-2
\startEXERCISE
請解釋在什麼情況下(如果有的話),
在調用 \ALGO{B-TREE-INSERT} 的過程中,
會執行冗餘的 \ALGO{DISK-READ} 或 \ALGO{DISK-WRITE} 操作。
(所謂冗餘的 \ALGO{DISK-READ},是指對已經在主存中的某頁做 \ALGO{DISK-READ}。
冗餘的 \ALGO{DISK-WRITE} 是指將已經存在於磁盤上某頁又完全相同的重寫一遍)
\stopEXERCISE

\startANSWER
若所要插入的節點是滿的,但其父節點未滿,我們需要做如下操作:
\startigBase[2]
\item \ALGO{DISK-READ}:放置關鍵字;
\item \ALGO{DISK-WRITE}:節點裂變;
\item \ALGO{DISK-READ}:讀取父節點;
\item \ALGO{DISK-READ}:填充父節點。
\stopigBase

如果父節點也是滿的,則會重複第二和第三步。也就是說不會出現冗餘。
\stopANSWER

%e18.2-3
\startEXERCISE
請說明如何在一棵 B 樹中找出最小關鍵字,
以及如何找出某一給定關鍵字的前驅。
\stopEXERCISE

\startANSWER
找最小關鍵字的過程與二叉搜索樹類似。最小關鍵字即最左葉子節點的第一個關鍵字。

找前驅要首先找到給定的關鍵字,假設爲 \m{x.key_i}。
\startigBase[2]
\item 如果 \m{x} 不是葉子節點,則要找到以 \m{x.c_i} 爲根的子樹的最大關鍵字。
\item 如果 \m{x} 是葉子節點,且 \m{i>1},則返回 \m{x.key_{i-1}}。
\item 如果 \m{x} 是葉子節點,且 \m{i=1},則自底向上搜索,
找到最後一個節點 \m{y},使得 \m{x.key_i} 是以 \m{y.c_j} 爲根的最左關鍵字(\m{j>0})。
如果 \m{j=1},則返回 \ALGO{NIL},這時 \m{x.key_i} 是整個 B 樹的最小關鍵字,沒有前驅;
否則,返回 \m{y.key_{j-1}}。
\stopigBase
\stopANSWER

%e18.2-4
\startEXERCISE
假設將關鍵字 \m{\{1,2,\ldots,n\}} 插入一棵最小度數爲 2 的空 B 樹中,
那麼最終的 B 樹有多少個節點?
\stopEXERCISE

\startANSWER
\m{O(n)}。
\stopANSWER

%e18.2-5
\startEXERCISE
因爲葉子節點無需指向孩子節點的指針,
那麼對於同樣大小的磁盤頁面,可選用一個與內部節點不同的(更大的) \m{t} 值。
請說明如何修改 B 樹的創建和插入過程來處理這個變化。
\stopEXERCISE

\startANSWER
最小度數爲 \m{t} 的 B 樹,一個節點最多可存儲 \m{2t-1} 個關鍵字以及 \m{2t} 個指針。
而葉子節點沒有子節點,因此 \m{2t} 個指針可以用來存儲關鍵字。
若將新的 \m{t} 值記爲 \m{t'},總空間要足夠容納 \m{2t'-1} 個關鍵字,
即根據關鍵字和指針各佔空間多少,即可求出 \m{t'} 的最大值。

B 樹的創建過程當無需修改(空節點)。
插入過程需要修改。
葉子節點分裂出來的節點仍是葉子節點,且仍只分裂出一個葉子節點,
不會出現一個葉子節點一次分裂出多個葉子節點的情況。
插入過程需要修改兩處:1)葉子節點和內部節點的分裂條件不同。
2)分裂後葉子節點和內部節點需要複製的關鍵字數目不同。
\stopANSWER

%e18.2-6
\startEXERCISE
假設 \ALGO{B-TREE-SEARCH} 的實現是在每個節點內採用二分查找,而不是線性查找。
證明:
無論怎樣選擇 \m{t} (\m{t} 爲 \m{n} 的函數),
這種實現所需的 CPU 時間都爲 \m{O(\lg n)}。
\stopEXERCISE

\startANSWER
令 B 樹高爲 \m{h},所含關鍵字數目爲 \m{n},
則 \m{O(h)=O(\log_{t}n)},且 \m{h\le \log_{t}\frac{n+1}{2}}。
每個節點中關鍵字數目不會大於 \m{2t-1},二分查找用時 \m{O(\lg t)}。
因此總時間爲:
\startformula
O(\lg t \times \log_{t}n) = O(\lg t \times \frac{\lg n}{\lg t}) = O(\lg n)
\stopformula
\stopANSWER

%e18.2-7
\startEXERCISE
假設磁盤硬件允許我們任意選擇磁盤頁面的大小,
但讀取磁盤頁面的時間是 \m{a+bt},
其中 \m{a} 和 \m{b} 爲規定的常數, \m{t} 爲確定磁盤頁大小後的 B 樹的最小度數。
請描述如何選擇 \m{t} 以(近似的)最小化 B 樹的查找時間。
對 \m{a=5ms} 和 \m{b=10ms},請給出 \m{t} 的一個最優值。
\stopEXERCISE

\startANSWER
令 B 樹高爲 \m{h},關鍵字數目爲 \m{n},搜索過程最多需要讀取次數爲 \m{h},
則讀頁面總用時爲 \m{(a+bt)\times h},對 \m{t} 求導數。
\stopANSWER

\stopsection

\startsection[
  title={Deleting a key from a B-tree},
]

%e18.3-1
\startEXERCISE
請說明依次從圖 18-8(f) 中刪除 \m{C}、 \m{P} 和 \m{V} 後的結果。
附圖 18-8(f):

\externalfigure[output/e18_3_1-1]
\stopEXERCISE

\startANSWER
\externalfigure[output/e18_3_1-2]

\externalfigure[output/e18_3_1-3]

\externalfigure[output/e18_3_1-4]
\stopANSWER

%e18.3-2
\startEXERCISE
請寫出 \ALGO{B-TREE-DELETE} 的僞碼。
\stopEXERCISE

\startANSWER
\CLRSH{B-TREE-MERGE(T, x, i)}
\startCLRS
y = x.c[i]
z = x.c[i+1]

++y.n
y.key[y.n] = x.key[i]
for j = 1 upto z.n
	++y.n
	y.key[y.n] = z.key[j]
	if not y.leaf
		y.c[y.n] = z.c[j]
if not y.leaf
	y.c[y.n + 1] = z.c[z.n + 1]
free z

for j = i upto x.n - 1
	x.key[i] = x.key[i+1]
	x.c[i+1] = x.c[i+2]
--x.n

if x.n == 0 and T.root == x
	T.root = y
\stopCLRS

\CLRSH{B-TREE-DELETE(T, x, k)}
\startCLRS
i = 1
while i <= x.n and x.key[i] < k
	++i
if i <= x.n and x.key[i] == k
	if x.leaf
		delete x from x
	else
		y = x.c[i]
		z = x.c[i+1]
		if y.n >= T.t
			y' = y
			while not y'.leaf
				y' = y'.c[y'.n + 1]
			x.key[i] = y'.key[y'.n]
			B-TREE-DELETE(T, y, x.key[i])
		else if z.n >= T.t
			z' = z
			while not z'.leaf
				z' = z'.c[1]
			x.key[i] = z'.key[1]
			B-TREE-DELETE(T, z, x.key[i])
		else
			B-TREE-MERGE(T, x, i)
			B-TREE-DELETE(T, y, k)
else
	if x.leaf
		print not found k
	else
		if x.c[i].n == T.t - 1
			if i <= x.n and x.c[i+1].n >= T.t
				y = x.c[i]
				z = x.c[i+1]
				++y.n
				y.key[y.n] = x.key[i]
				if not y.leaf
					y.c[y.n+1] = z.c[1]
				x.key[i] = z.key[1]
				for j = 1 upto z.n - 1
					z.key[j] = z.key[j+1]
					if not z.leaf
						z.c[j] = z.c[j+1]
				if not z.leaf
					z.c[z.n] = z.c[z.n+1]
				--z.n
			else if i > 1 and x.c[i-1].n >= T.t
				y = x.c[i-1]
				z = x.c[i]
				++z.n
				for j = z.n downto 2
					z.key[j] = z.key[j-1]
					if not z.leaf
						z.c[j+1] = z.c[j]
				z.key[1] = x.key[i]
				if not z.leaf
					z.c[2] = z.c[1]
					z.c[1] = y.c[y.n+1]
				x.key[i] = y.key[y.n]
				--y.n
			else
				if i > x.n
					--i
				B-TREE-MERGE(T, x, i)
		B-TREE-DELETE(T, x.c[i], k)
\stopCLRS
\stopANSWER

\stopsection

\startsubject[
  title={Problems},
]

%p18-1
\startPROBLEM
(stacks on secondary storage)
對於一臺計算機,如果較快的主存較少,而較慢的磁盤存儲空間較多,
如何在其上實現一個棧。
 \ALGO{PUSH} 和 \ALGO{POP} 的操作對象爲單字。
我們希望計算機支持的棧可以增長的很大,
甚至無法全部裝入主存,即大部分放在磁盤上。

一種簡單但低效的實現方法是將整個棧放在磁盤上。
在主存中保持一個棧的指針,指向棧頂元素的磁盤地址。
如果該指針的值爲 \m{p},
則棧頂元素是磁盤的 \m{\left\lfloor \frac{p}{m}\right\rfloor} 頁上的第 \m{p \mod m} 個字,
這裏 \m{m} 爲每頁所含字數。

爲了實現 \ALGO{PUSH} 操作,我們增加棧指針,
從磁盤將適當的頁讀到主存中後,
複製要被壓入棧的元素到該頁上適當字的位置,
最後將該頁寫回到磁盤。
 \ALGO{POP} 操作與之類似。
我們減小棧指針,從磁盤上讀入所需的頁,再返回棧頂元素。
我們不需要寫回該頁,因爲他沒有被修改。

由於磁盤操作代價相對較高,
我們統計任何實現的兩部分代價:
總的磁盤存取次數和總的 CPU 時間。
任何對一個包含 \m{m} 個字的頁面的磁盤存取,
都會引起一次磁盤存取 \m{\Theta(m)} 的 CPU 時間。

\startigBase[a]\startitem
從漸進意義上看,使用這種簡單實現,最壞情況下, \m{n} 個棧操作需要存取多少次磁盤?
 CPU 時間又是多少?
(用 \m{m} 和 \m{n} 來表示這個問題及後面幾個問題的答案)
\stopitem\stopigBase

\startANSWER
\m{2n} 次, \m{\Theta(mn)}。
\stopANSWER

現在考慮棧的另一種實現,
即在主存中始終保持存放棧中的一頁。
(還用少量的主存來記錄當前哪一頁在主存中)
只有相關的磁盤頁面駐留在主存中,才能執行棧操作。
如果需要,可以將當前主存中的頁寫回磁盤,
並且可以從磁盤讀入新的一頁放入主存。
如果相關磁盤頁已經在主存中,那麼就無需任何磁盤存取。

\startigBase[continue]\startitem
最壞情況下, \m{n} 個 \ALGO{PUSH} 操作所需的磁盤存取次數是多少?
所需的 CPU 時間是多少?
\stopitem\stopigBase

\startANSWER
\m{n/m} 次, \m{\Theta(n)}。
\stopANSWER

\startigBase[continue]\startitem
最壞情況下, \m{n} 個棧操作所需的磁盤存取次數是多少?
所需的 CPU 時間是多少?
\stopitem\stopigBase

\startANSWER
\m{n/2} 次, \m{\Theta(mn)}。
\stopANSWER

假設在另一種實現方式中,在主存中保持棧的 2 頁(此外還有少量的字來記錄哪些頁在主存中)。
\startigBase[continue]\startitem
請描述如何管理棧頁,使得任何棧操作的攤還磁盤存取次數爲 \m{O(1/m)},
攤還 CPU 時間爲 \m{O(1)}。
\stopitem\stopigBase

\startANSWER
\TODO{略。}
\stopANSWER

\stopPROBLEM

%p18-2
\startPROBLEM
(joining and splitting 2-3-4 tree)
{\EMP 連接}:輸入爲兩個動態集合 \m{S'} 和 \m{S''},以及一個元素 \m{x},
且對任何 \m{x'\in S'} 和 \m{x''\in S''},
都有 \m{x'.key < x.key < x''.key};
返回一個集合 \m{S=S'\cup \{x\} \cup S''}。

{\EMP 分裂}:{\EMP 連接}的逆操作。
給定一個動態集合 \m{S} 和一個元素 \m{x\in S},
他創建了一個集合 \m{S'},
其包含 \m{S-\{x\}} 中所有關鍵字小於 \m{x.key} 的元素;
同時創建了一個集合 \m{S''},
其包含 \m{S - \{x\}} 中所有關鍵字大於 \m{x.key} 的元素。
在這個問題中,我們討論如何在 2-3-4 樹上實現這些操作。
爲方便起見,假定所有元素都只包含關鍵字,
並且所有的關鍵字都不相同。

\startigBase[a]\startitem
對 2-3-4 樹中的每個節點 \m{x},
說明如何將以 \m{x} 爲根的子樹的高度作爲一個屬性 \m{x.height} 來維護。
要確保所給出的實現不影響查找、插入和刪除的漸進運行時間。
\stopitem\stopigBase

\startANSWER
插入:分裂時給新根節點的 \m{height} 賦值。
\stopANSWER

\startigBase[continue]\startitem
說明如何實現連接操作。
給定兩棵 2-3-4 樹 \m{T'}、 \m{T''} 以及一個關鍵字 \m{k},
連接操作應在 \m{O(1+|h'-h''|)} 運行時間內完成,
其中 \m{h'} 和 \m{h''} 分別是樹 \m{T'} 和 \m{T''} 的高度。
\stopitem\stopigBase

\startANSWER
以較高的樹,從根節點開始沿最左側子樹搜索,
直到 height 屬性比較低樹的根節點 height 大 1,記爲節點 \m{x};
對於路徑上任何一個節點,只要是滿的,就將其分裂。
然後將較低的樹與新關鍵字 \m{k} 加入 \m{x}。
\stopANSWER

\startigBase[continue]\startitem
考慮從一棵 2-3-4 樹 \m{T} 的根到一個給定關鍵字 \m{k} 的簡單路徑 \m{p},
 \m{T} 中小於 \m{k} 的關鍵字集合 \m{S'},
以及 \m{T} 中大於 \m{k} 的關鍵字集合 \m{S''}。
證明: \m{p} 將 \m{S'} 分爲一個樹的集合 \m{\{T_0',T_1',\ldots,T_m'\}} 和
一個關鍵字的集合 \m{\{k_1',k_2',\ldots,k_m'\}},
且對任何關鍵字 \m{y\in T_{i-1}'} 和 \m{z\in T_i'} (\m{i=1,2,\ldots,m}),
都有 \m{y<k_i'<z}。
 \m{T_{i-1}'} 和 \m{T_i'} 的高度之間有什麼關係?
請說明 \m{p} 是如何將 \m{S''} 分爲樹集合和關鍵字集合的。
\stopitem\stopigBase

\startANSWER
從根節點開始沿路徑 \m{p} 搜索 \m{k},
對於路徑上每一個節點 \m{x},假設 \m{x.c_i} 在路徑 \m{p} 上:
如果 \m{i = 1},則分出的子樹爲空,響應關鍵字也爲空;
否則,則分出以 \m{x.c_{i-1}} 爲根的子樹,以及關鍵字 \m{x.key_{i-1}}。
所得即爲樹的集合 \m{\{T_0',T_1',\ldots,T_m'\}} 和
關鍵字的集合 \m{\{k_1',k_2',\ldots,k_m'\}}。

對稱的可得樹的集合 \m{\{T_0'',T_1'',\ldots,T_m''\}} 和
關鍵字的集合 \m{\{k_1'',k_2'',\ldots,k_m''\}}。

\m{T_{i-1}'} 的高度大於等於 \m{T_i'} 的高度。
\stopANSWER

\startigBase[continue]\startitem
請說明如何實現 \m{T} 上的分裂操作。
利用連接操作將 \m{S'} 中的關鍵字拼成一棵簡單的 2-3-4 樹 \m{T'},
將 \m{S''} 中的關鍵字拼成一棵簡單的 2-3-4 樹 \m{T''}。
分裂操作的運行時間要求爲 \m{O(\lg n)},
這裏 \m{n} 是 \m{T} 中關鍵字數目。
(\hint 連接的代價應是套迭的)
\stopitem\stopigBase

\startANSWER
沿到某一個關鍵字的路徑進行分裂。參考上一項的答案。
然後將分裂出的集合進行合併。
所用時間爲:
\startformula
\sum_{i=1}^{m}(1+|h(T_{i-1}') - h(T_i')|)
= m + \sum_{i=1}^{m}(h(T_{i-1}') - h(T_i'))
= m + h(T_0') - h(T_m')
\in O(\lg n)
\stopformula
\stopANSWER

\stopPROBLEM

\stopsubject%Problems

\stopchapter
\stopcomponent
