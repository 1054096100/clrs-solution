\startsection[
  title={Johnson’s algorithm for sparse graphs},
]

%e25.3-1
\startEXERCISE
請在圖 25-2 上使用 Johnson 算法來找到所有節點對之間的最短路徑。
給出算法計算出 \m{h} 和 \m{\hat{\omega}}。附圖 25-2:

\externalfigure[output/e25_3_1-1]
\stopEXERCISE

\startANSWER
\startcombination[2*1]
{\startformula
h = \left(\startmatrix
\NC 0 \NR
\NC -5 \NR
\NC -3 \NR
\NC 0 \NR
\NC -1 \NR
\NC -6 \NR
\NC -8 \NR
\stopmatrix\right)
\stopformula}{}

{\startformula
\hat{\omega} = \left(\startmatrix
\NC \infty\NC \infty\NC \infty\NC \infty\NC 0\NC \infty \NR
\NC 3\NC \infty\NC \infty\NC 0\NC \infty\NC \infty \NR
\NC \infty\NC 5\NC \infty\NC \infty\NC \infty\NC 0 \NR
\NC 0\NC \infty\NC \infty\NC \infty\NC 8\NC \infty \NR
\NC \infty\NC 4\NC \infty\NC \infty\NC \infty\NC \infty \NR
\NC \infty\NC 0\NC 2\NC \infty\NC \infty\NC \infty \NR
\stopmatrix\right)
\stopformula}{}
\stopcombination
\stopANSWER

%e25.3-2
\startEXERCISE
在算法 \ALGO{JOHNSON} 裏,給集合 \m{V} 加入了新節點 \m{s},
形成了新集合 \m{V'},這樣做的目的是什麼?
\stopEXERCISE

\startANSWER
爲了使所有節點都可達。
\stopANSWER

%e25.3-3
\startEXERCISE
假定對於所有的邊 \m{(u,v)\in E},
我們有 \m{\omega(u,v)\ge 0}。
請問權重函數 \m{\omega} 和 \m{\hat{\omega}} 之間是什麼關係?
\stopEXERCISE

\startANSWER
\m{h(u)=h(v)=0}, \m{\omega=\hat{\omega}}。
\stopANSWER

%e25.3-4
\startEXERCISE
Greenstreet 教授聲稱,
他有一種新方法給邊的權重重新賦值,比 \ALGO{JOHNSON} 算法中的更簡單。
設 \m{\omega^*=\min_{(u,v)\in E}\{\omega(u,v)\}},
只要對所有的邊 \m{(u,v)\in E},
定義 \m{\hat{\omega}(u,v)=\omega(u,v)-\omega^*} 即可。
請問這種方法錯在哪?
\stopEXERCISE

\startANSWER
這樣 \m{\hat{\omega}(p)=\omega(p)-(k-1)\omega^*},
也就是說與路徑上的邊數有關了。
對於一個路徑而言, \m{\omega^*} 最小,不能說明這條路徑最短。
\stopANSWER

%e25.3-5
\startEXERCISE
假定在一個權重函數爲 \m{\omega} 的有向圖 \m{G} 上運行 \ALGO{JOHNSON}。
證明:如果圖 \m{G} 包含一條權重爲 0 的環路 \m{c},
那麼對於環路 \m{c} 上的每條邊 \m{(u,v)}, \m{\hat{\omega}(u,v)=0}。
\stopEXERCISE

\startANSWER
\startformula\startmathalignment
\NC \delta(s,v) \le \NC \delta(s,u) + \omega(u,v) \NR
\NC \delta(s,u) \le \NC \delta(s,v) + (0 - \omega(u,v)) \NR
\stopmathalignment\stopformula
根據上面兩式,有 \m{\delta(s,v) = \delta(s,u) + \omega(u,v)}。
因此 \m{\hat{\omega(u,v)} = \omega(u,v) + \delta(s,u) - \delta(s,v) = 0}。
\stopANSWER

%e25.3-6
\startEXERCISE
Michener 教授聲稱,
沒有必要在 \ALGO{JOHNSON} 的第 1 行創建一個新的源節點。
他主張可以使用 \m{G'=G},並設 \m{s} 爲任意節點。
請給出一個帶權重的有向圖例子,
如果按這種方法做會導致錯誤的結果。
然後證明:如果圖 \m{G} 是強連通的(每個節點都可以從其他每個節點到達),
那麼使用這位教授的方法不影響結果的正確性。
\stopEXERCISE

\startANSWER
假定 \m{\infty-\infty} 未定義,而不是 0。
令 \m{G=(V,E)},其中 \m{V=\{s,u\}}, \m{E=\{(u,s)\}},
且 \m{\omega(u,s)=0}。也就是說 \m{G} 中只有一條邊。

從 \m{s} 開始運行 \ALGO{BELLMAN-FORD} 時,
 \m{h(s)=\delta(s,s)=0}, \m{h(u)=\delta(s,u)=\infty}。
在重新給權重賦值時, \m{\hat{\omega}(u,s)=0+\infty-0=\infty}。
從而計算出 \m{\hat{\delta}(u,s)=\infty}, \m{d_{us}=\infty+0-\infty\ne 0}。
但由於 \m{\delta(u,s)=0},所以結果是錯的。

而如果 \m{G} 是強連通的,則對於所有節點 \m{v\in V},
都有 \m{h(v)=\delta(s,v)<\infty}。
因此,根據三角不等式,對於所有 \m{(u,v)\in E},
都有 \m{h(v)\le h(u)+\omega(u,v)},從而有 \m{\hat{\omega}(u,v)=\omega(u,v)+h(u)-h(v)\ge 0}。
滿足算法 \ALGO{JOHNSON} 的要求。
後面 \m{\hat{\delta}(u,v)<\infty} 對所有 \m{(u,v)\in E} 也成立。
最後 \m{d_{uv}=\hat{\delta}(u,v)+h(v)-h(u) < \infty} 也成立。
\stopANSWER

\stopsection

