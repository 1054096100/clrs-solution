\startsubject[
  title={Problems},
]

%p25-1
\startPROBLEM
(Transitive closure of a dynamic graph)
假定我們希望給有向圖 \m{G=(V,E)} 插入新邊的時候維持其傳遞閉包,
即在插入每條邊後,更新圖的傳遞閉包。
假定圖 \m{G} 開始時不包含任何邊,並且傳遞閉包用布爾矩陣來表示。
\startigBase[a]\startitem
說明給圖 \m{G} 加入一條新邊時,
如何在 \m{O(V^2)} 時間內更新圖 \m{G=(V,E)} 的傳遞閉包 \m{G^*=(V,E^*)}。
\stopitem\stopigBase

\startANSWER
令 \m{T} 爲圖 \m{G} 的傳遞閉包矩陣:
\startformula
T[i,j] = \startcases
\NC 1 \NC if \m{i = j} \NR
\NC 0 \NC otherwise \NR
\stopcases
\stopformula

\CLRSH{TRANSITIVE-CLOSURE-UPDATE(u,v)}
\startCLRS
for i = 1 upto |V|
	for j = 1 upto |V|
		if T[i,u] == 1 and T[v,j] == 1
			T[i,j] = 1
\stopCLRS

\stopANSWER

\startigBase[continue]\startitem
給出一個圖 \m{G} 與一條邊 \m{e},
使得將 \m{e} 插入 \m{G} 時,不管使用什麼算法,
更新其傳遞閉包的時間複雜度都是 \m{\Omega(V^2)}。
\stopitem\stopigBase

\startANSWER
假定圖 \m{G} 中只有一條從 \m{v_1} 到 \m{v_n} 的一條直線:
 \m{v_1\rightarrow v_2\rightarrow \ldots \rightarrow v_n},
其中 \m{n=|V|}。在插入新邊之前, \m{T} 中只有 \m{n(n+1)/2} 個 \m{1},
即對角線及上半部分的元素。
插入新邊後,變成了一個環路,任意兩點間均可達,
即 \m{T} 中會有 \m{n^2} 個 \m{1}。
新增加的 \m{1} 個數爲 \m{n^2-n(n+1)/2 = n(n-1)/2}。
因此無論用什麼算法時間都會是 \m{\Omega(n^2)}。
\stopANSWER

\startigBase[continue]\startitem
描述一個有效的算法,
使得在將邊加入到圖 \m{G} 中時更新傳遞閉包。
對於任意 \m{n} 次插入的序列,
算法總運行時間爲 \m{\sum_{i=1}^{n}t_i=O(V^3)},
其中 \m{t_i} 是插入第 \m{i} 條邊時更新傳遞閉包所用的時間。
請證明你的算法確實達到了這個時間效率。
\stopitem\stopigBase

\startANSWER
如果用 a)中的方式插入 \m{\Theta(V^2)} 條邊,則總時間爲 \m{\Theta(V^4)}。

\CLRSH{TRANSITIVE-CLOSURE-UPDATE'(u,v)}
\startCLRS
for i = 1 upto |V|
	if T[i,u] == 1 and T[i,v] == 0
		for j = 1 upto |V|
			if T[v,j] == 1
				T[i,j] = 1
\stopCLRS

前兩行執行時間爲 \m{O(nV)=O(V^3)}。
後三行執行時間爲 \m{O(V)},共執行 \m{O(V^2)} 次,
因爲這三行執行一遍至少會將 \m{T[i,v]} 改成 \m{1},
而整個矩陣只有 \m{O(V^2)} 個元素,也就是說最後三行最多執行 \m{O(V^2)} 次。
所以總時間爲 \m{O(V^3)}。
\stopANSWER

\stopPROBLEM

%p25-2
\startPROBLEM
(Shortest paths in \m{\epsilon}-dense graphs)
對於圖 \m{G=(V,E)} 而言,如果 \m{|E|=\Theta(V^{1+\epsilon})},
則圖 \m{G} 爲{\EMP \m{\epsilon} 稠密圖},其中 \m{\epsilon} 爲某個常數,
且 \m{0<\epsilon \le 1}。
如果在 \m{\epsilon} 稠密圖的最短路徑算法中使用 \m{d} 叉最小堆
(請參閱\refproblem{6-2}),
則能使算法的運行時間相當於基於 Fibonacci 堆的算法的運行時間,
卻無需引入後者所用的複雜數據結構。

\startigBase[a]\startitem
\ALGO{INSERT}、 \ALGO{EXTRACT-MIN}、 \ALGO{DECREASE-KEY} 的漸進運行時間是多少?
請以 \m{d} 和元素個數 \m{n} 爲參數來表示。
如果選擇 \m{d=\Theta(n^\alpha)},其中 \m{0<\alpha\le 1},
這些運行時間又是多少?
與 Fibonacci 堆的攤還代價相比如何?
\stopitem\stopigBase

\startANSWER
\ALGO{INSERT}: \m{\Theta(\log_d n = \Theta(1/\alpha)}

\ALGO{EXTRACT-MIN}: \m{\Theta(d\log_d n) = \Theta(n^\alpha / \alpha)}

\ALGO{DECREASE-KEY}: \m{\Theta(\log_d n) = \Theta(1/\alpha)}
\stopANSWER

\startigBase[continue]\startitem
說明如何在 \m{O(E)} 時間內,
在一個 \m{\epsilon} 稠密的有向圖中計算除單源最短路徑,
這裏界定該圖不包含權重爲負值的邊。
(\hint 選一個以 \m{\epsilon} 爲自變量的函數作爲 \m{d})
\stopitem\stopigBase

\startANSWER
用算法 \ALGO{DIJKSTRA},如果 \m{d=V^\epsilon},則:
\startformula\startmathalignment
\NC \NC O(d\log_d V\cdot V + \log_d V\cdot E) \NR
\NC = \NC O(V^\epsilon \cdot V/\epsilon + E / \epsilon) \NR
\NC = \NC O((V^{1+\epsilon} + E)/\epsilon) \NR
\NC = \NC O((E+E)/\epsilon) \NR
\NC = \NC O(E) \NR
\stopmathalignment\stopformula
\stopANSWER

\startigBase[continue]\startitem
說明如何在 \m{O(VE)} 時間內,
在一個 \m{\epsilon} 稠密的有向圖中計算除所有節點對之間的最短路徑,
這裏假定該圖不包含權重爲負值的邊。
\stopitem\stopigBase

\startANSWER
運行 \m{|V|} 次 \ALGO{DIJKSTRA},根據 b)可知總時間爲 \m{O(VE)}。
\stopANSWER

\startigBase[continue]\startitem
說明如何在 \m{O(VE)} 時間內,
在一個 \m{\epsilon} 稠密的有向圖中計算除所有節點對之間的最短路徑,
這裏假定圖中可以包含權重爲負值的邊,
但不包含權重爲負值的環路。
\stopitem\stopigBase

\startANSWER
用 \ALGO{JOHNSON} 重新計算權重,總時間爲 \m{O(VE)}。
\stopANSWER

\stopPROBLEM

\stopsubject%Problems

