\startsection[
  title={The Floyd-Warshall algorithm},
  reference=section:floyd_warshall,
]

%e25.2-1
\startEXERCISE
在圖 25-2 所示的帶權重有向圖上運行算法 \ALGO{FLOYD-WARSHALL},
給出外層循環的每一次迭代所成城的矩陣 \m{D^{(k)}}。
\stopEXERCISE

\startANSWER
\startcombination[2*3]
{\startformula

D^{(1)} = \left(\startmatrix
\NC 0\NC \infty\NC \infty\NC \infty\NC -1\NC \infty \NR
\NC 1\NC 0\NC \infty\NC 2\NC 0\NC \infty \NR
\NC \infty\NC 2\NC 0\NC \infty\NC \infty\NC -8 \NR
\NC -4\NC \infty\NC \infty\NC 0\NC -5\NC \infty \NR
\NC \infty\NC 7\NC \infty\NC \infty\NC 0\NC \infty \NR
\NC \infty\NC 5\NC 10\NC \infty\NC \infty\NC 0 \NR
\stopmatrix\right)

\stopformula}{}{\startformula

D^{(2)} = \left(\startmatrix
\NC 0\NC \infty\NC \infty\NC \infty\NC -1\NC \infty \NR
\NC 1\NC 0\NC \infty\NC 2\NC 0\NC \infty \NR
\NC 3\NC 2\NC 0\NC 4\NC 2\NC -8 \NR
\NC -4\NC \infty\NC \infty\NC 0\NC -5\NC \infty \NR
\NC 8\NC 7\NC \infty\NC 9\NC 0\NC \infty \NR
\NC 6\NC 5\NC 10\NC 7\NC 5\NC 0 \NR
\stopmatrix\right)

\stopformula}{}{\startformula

D^{(3)} = \left(\startmatrix
\NC 0\NC \infty\NC \infty\NC \infty\NC -1\NC \infty \NR
\NC 1\NC 0\NC \infty\NC 2\NC 0\NC \infty \NR
\NC 3\NC 2\NC 0\NC 4\NC 2\NC -8 \NR
\NC -4\NC \infty\NC \infty\NC 0\NC -5\NC \infty \NR
\NC 8\NC 7\NC \infty\NC 9\NC 0\NC \infty \NR
\NC 6\NC 5\NC 10\NC 7\NC 5\NC 0 \NR
\stopmatrix\right)

\stopformula}{}{\startformula

D^{(4)} = \left(\startmatrix
\NC 0\NC \infty\NC \infty\NC \infty\NC -1\NC \infty \NR
\NC -2\NC 0\NC \infty\NC 2\NC -3\NC \infty \NR
\NC 0\NC 2\NC 0\NC 4\NC -1\NC -8 \NR
\NC -4\NC \infty\NC \infty\NC 0\NC -5\NC \infty \NR
\NC 5\NC 7\NC \infty\NC 9\NC 0\NC \infty \NR
\NC 3\NC 5\NC 10\NC 7\NC 2\NC 0 \NR
\stopmatrix\right)

\stopformula}{}{\startformula

D^{(5)} = \left(\startmatrix
\NC 0\NC 6\NC \infty\NC 8\NC -1\NC \infty \NR
\NC -2\NC 0\NC \infty\NC 2\NC -3\NC \infty \NR
\NC 0\NC 2\NC 0\NC 4\NC -1\NC -8 \NR
\NC -4\NC 2\NC \infty\NC 0\NC -5\NC \infty \NR
\NC 5\NC 7\NC \infty\NC 9\NC 0\NC \infty \NR
\NC 3\NC 5\NC 10\NC 7\NC 2\NC 0 \NR
\stopmatrix\right)

\stopformula}{}{\startformula

D^{(6)} = \left(\startmatrix
\NC 0\NC 6\NC \infty\NC 8\NC -1\NC \infty \NR
\NC -2\NC 0\NC \infty\NC 2\NC -3\NC \infty \NR
\NC -5\NC -3\NC 0\NC -1\NC -6\NC -8 \NR
\NC -4\NC 2\NC \infty\NC 0\NC -5\NC \infty \NR
\NC 5\NC 7\NC \infty\NC 9\NC 0\NC \infty \NR
\NC 3\NC 5\NC 10\NC 7\NC 2\NC 0 \NR
\stopmatrix\right)

\stopformula}{}
\stopcombination
\stopANSWER

%e25.2-2
\startEXERCISE
說明如何使用\refsection{25.1} 的技術來計算傳遞閉包。
\stopEXERCISE

\startANSWER
將 \ALGO{EXTEND-SHORTEST-PATHS} 中的“\m{\min}”換成“\m{\lor}”,
將“\m{+}”換成“\m{\land}”。
\stopANSWER

%e25.2-3
\startEXERCISE
請修改 \ALGO{FLOYD-WARSHALL} 算法,
以便根據式 25.6 和 25.7 計算出矩陣 \m{\prod^{(k)}}。
再請嚴格證明:
對於所有節點 \m{i\in V},
前驅子圖 \m{G_{\pi,i}} 是一棵根節點爲 \m{i} 的最短路徑樹。
(\hint 爲了證明 \m{G_{\pi,i}} 無環,
可以首先證明:根據 \m{G_{\pi,i}} 的定義,
 \m{\pi_{ij}^{(k)} = l} 蘊含 \m{d_{ij}^{(k)}\ge d_{il}^{(ik)} + \omega_{lj}}。
然後,再採用引理 24.16 的證明。)
附式 25.6:
\startformula
\pi_{ij}^{(0)} = \startcases
\NC NIL \NC 若 \m{i=j} 或 \m{\omega_{ij} = \infty} \NR
\NC i   \NC 若 \m{i\ne j} 且 \m{\omega_{ij} < \infty} \NR
\stopcases
\stopformula

式 25.7:
\startformula
\pi_{ij}^{(k)} = \startcases
\NC \pi_{ij}^{(k-1)} \NC 若 \m{d_{ij}^{(k-1)} \le d_{ik}^{(k-1)} + d_{kj}^{(k-1)}} \NR
\NC \pi_{kj}^{(k-1)} \NC 若 \m{d_{ij}^{(k-1)} > d_{ik}^{(k-1)} + d_{kj}^{(k-1)}} \NR
\stopcases
\stopformula
\stopEXERCISE

\startANSWER
\CLRSH{FLOYD-WARSHALL(W)}
\startCLRS
n = W.rows
D[0] = W
for i = 1 to n
	for j = 1 to n
		if d[0][i][j] != infinity
			pi[0][i][j] = i
		else
			pi[0][i][j] = NIL
for k = 1 to n
	for i = 1 to n
		for j = 1 to n
			d[k][i][j] = min(d[k-1][i][j], d[k-1][i][k] + d[k-1][k][j])
			if d[k-1][i][j] <= d[k-1][i][k] + d[k-1][k][j]
				pi[k][i][j] = pi[k-1][i][j]
			else
				pi[k][i][j] = pi[k-1][k][j]
return D[n], Pi[n]
\stopCLRS

最短路徑樹的證明略。
\stopANSWER

%e25.2-4
\startEXERCISE
如前所述, \ALGO{FLOYD-WARSHALL} 的空間需求爲 \m{\Theta(n^3)},
因爲要計算 \m{d_{ij}^{(k)}},其中 \m{i,j,k=1,2,\ldots,n}。
請證明下面算法的正確性,下面算法跟 \ALGO{FLOYD-WARSHALL} 的區別就是沒有上標了,
其空間需求可以降低到 \m{\Theta(n^2)}。

\CLRSH{FLOYD-WARSHALL'(W)}
\startCLRS
n = W.rows
D = W
for k = 1 to n
	for i = 1 to n
		for j = 1 to n
			d[i][j] = min(d[i][j], d[i][k] + d[k][j])
return D
\stopCLRS
\stopEXERCISE

\startANSWER
就這個動態規劃算法而言,下一個狀態只跟上一個狀態有關,沒有必要保存其他狀態。
\stopANSWER

%e25.2-5
\startEXERCISE
假定我們修改式 25.7 對等式的處理辦法如下:
\startformula
\pi_{ij}^{(k)} = \startcases
\NC \pi_{ij}^{(k-1)} \NC 若 \m{d_{ij}^{(k-1)} < d_{ik}^{(k-1)} + d_{kj}^{(k-1)}} \NR
\NC \pi_{kj}^{(k-1)} \NC 若 \m{d_{ij}^{(k-1)} \ge d_{ik}^{(k-1)} + d_{kj}^{(k-1)}} \NR
\stopcases
\stopformula
請問這種前驅矩陣 \m{\prod} 的定義正確嗎?
\stopEXERCISE

\startANSWER
正確。
\stopANSWER

%e25.2-6
\startEXERCISE
我們怎樣才能使用 \ALGO{FLOYD-WARSHALL} 算法的輸出來檢測負權重環路?
\stopEXERCISE

\startANSWER
在算法最後再跑一遍循環,如果還有矩陣元素可以更新,則表明有負權重環路。
\stopANSWER

%e25.2-7
\startEXERCISE
在算法 \ALGO{FLOYD-WARSHALL} 中構建最短路徑的另一種辦法是使用 \m{\phi_{ij}^{(k)}},
其中 \m{i,j,k=1,2,\ldots,n},
 \m{\phi_{ij}^{(k)}} 是從節點 \m{i} 到節點 \m{j} 的最短路徑上編號最大的中間節點,
這條路徑上中間所有節點都取自集合 \m{\{1,2,\ldots,k\}}。
請給出 \m{\phi_{ij}^{(k)}} 的一個遞迴公式,
並修改 \ALGO{FLOYD-WARSHALL} 計算 \m{\phi_{ij}^{(k)}}。
然後重寫 \ALGO{PRINT-ALL-PAIRS-SHORTEST-PATH},
其輸入爲矩陣 \m{\Phi=(\phi_{ij}^{(n)})}。
矩陣 \m{\Phi} 與\refsection{15.2} 所討論的鏈式矩陣乘法中的表格有哪些相似之處?
\stopEXERCISE

\startANSWER
\startformula
\phi_{ij}^{(k)} = \startcases
\NC \phi_{ij}^{(k-1)} \NC 若 \m{d_{ij}^{(k-1)} \le d_{ik}^{(k-1)} + d_{kj}^{(k-1)}} \NR
\NC k \NC 若 \m{d_{ij}^{(k-1)} > d_{ik}^{(k-1)} + d_{kj}^{(k-1)}} \NR
\stopcases
\stopformula

\CLRSH{PRINT-ALL-PAIRS-SHORTEST-PATH(\phi, i, j)}
\startCLRS
if i == j
	print i
else if phi[i][j] = NIL
	print no path from i to j exist
else
	PRINT-ALL-PAIRS-SHORTEST-PATH(phi, i, phi[i][j])
	print phi[i][j]
	PRINT-ALL-PAIRS-SHORTEST-PATH(phi, phi[i][j], i)
\stopCLRS
\stopANSWER

%e25.2-8
\startEXERCISE
給出一個 \m{O(VE)} 時間複雜度的算法來計算有向圖 \m{G=(V,E)} 的傳遞閉包。
\stopEXERCISE

\startANSWER
\startCLRS
for each vertex s in G.V
	R[s][s] = 1
for each edge (u,v) in G.E
	for each vertex s in G.V
		if R[s][u] = 1
			R[s][v] = 1
\stopCLRS
\stopANSWER

%e25.2-9
\startEXERCISE
假定我們可以在 \m{f(|V|,|E|)} 的時間內計算出一個有向無環圖的傳遞閉包,
其中 \m{f} 是一個自變量爲 \m{|V|} 和 \m{|E|} 的單調遞增函數。
證明:對於一個常規的有向圖 \m{G=(V,E)},計算其傳遞閉包 \m{G^*=(V,E^*)} 的
時間複雜度爲 \m{f(|V|,|E|)+O(V+E^*)}。
\stopEXERCISE

\startANSWER
先計算 \m{G} 的強連通分量圖 \m{G_{SCC}},用時 \m{O(V+E)}。
在這個過程中需要記錄 \m{G} 中的節點與 \m{G_{SCC}} 中節點的關係,
對於 \m{G} 中任一節點 \m{v},都可以通過一個函數 \m{g(v)} 得到 \m{G_{SCC}} 中與其對應的節點。

然後在圖 \m{G_{SCC}} 上運行算法找到其傳遞閉包。
用時 \m{O(f(|V|,|E|)}。
剩下的就是將 \m{G_{SCC}} 的傳遞閉包轉換回 \m{G} 的傳遞閉包。

當且僅當 \m{G} 的傳遞閉包中有邊 \m{(u,v)} 時,
 \m{G_{SCC}} 的傳遞閉包中纔有 \m{(g(u),g(v)}。
據此,我們可以遍歷 \m{G_{SCC}} 的傳遞閉包中的邊 \m{(a,b)},
對於每個邊 \m{(a,b)},找到所有的 \m{x\in g^{-1}(a)} 和 \m{y\in g^{-1}(b)}。
那麼 \m{(x,y)} 就屬於 \m{E^*}。用時爲 \m{O(E^*)}。

由於 \m{E<E^*},所以總時間爲 \m{O(V+E) + O(f(|V|,|E|)) + O(E^*) = O(f(|V|,|E|) + V + E^*)}。
\stopANSWER

\stopsection

