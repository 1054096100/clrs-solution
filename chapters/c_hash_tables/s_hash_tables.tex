\startsection[
  title={Hash tables},
]
%e11.2-1
\startEXERCISE
假設用一個散列函數 h 將 n 個不同的關鍵字散列到一個長度爲 m 的數列 T 中。
假設採用的是簡單均勻散列,那麼期望的衝突數是多少?
更準確地,集合 \m{\{\{k,l\}: k\ne l, \text{且 }h(k)=h(l)\}} 基的期望值是多少?
\stopEXERCISE

\startANSWER
對於每對關鍵字 k 和 l,其中 \m{k\ne l},
定義指示器隨機變量 \m{X_{kl}=I\{h(k)=h(l)\}}。
由於我們假設簡單均勻散列,
因此 \m{\Pr\{X_{kl}=1\} = \Pr\{h(k)=h(l)\} = 1/m},
從而 \m{E[X_{kl}] = 1/m}。

現在定義指示器隨機變量 Y,用來表示衝突的總數,
因此 \m{Y=\sum_{k\ne l}X_{kl}}。衝突數的期望值爲:
\startformula\startmathalignment
\NC E[Y]
    \NC = E\left[ \sum_{k\ne l} X_{kl} \right] \NR
\NC \NC = \sum_{k\ne l} E\left[ X_{kl} \right] \NR
\NC \NC = \binom{n}{2}\frac{1}{m} \NR
\NC \NC = \frac{n(n-1)}{2}\times \frac{1}{m} \NR
\NC \NC = \frac{n(n-1)}{2m} \NR
\stopmathalignment\stopformula
\stopANSWER

%e11.2-2
\startEXERCISE
對於一個用鏈接法解決衝突的散列表,
說明將關鍵字 5、 28、 19、 15、 20、 33、 12、 17、 10 插入到該表中的過程。
設該表中有 9 個槽位,並設其散列函數爲 \m{h(k)=k \mod 9}。
\stopEXERCISE

\startANSWER
\midaligned{
\startcombination[3*3]
{\externalfigure[output/e11_2_2-1]}{a}
{\externalfigure[output/e11_2_2-2]}{b}
{\externalfigure[output/e11_2_2-3]}{c}
{\externalfigure[output/e11_2_2-4]}{d}
{\externalfigure[output/e11_2_2-5]}{e}
{\externalfigure[output/e11_2_2-6]}{f}
{\externalfigure[output/e11_2_2-7]}{g}
{\externalfigure[output/e11_2_2-8]}{h}
{\externalfigure[output/e11_2_2-9]}{i}
\stopcombination
}
\stopANSWER

%e11.2-3
\startEXERCISE
Marley 教授做了這樣一個假設,即如果將鏈模式改動一下,
使得每個鏈表都能保持已排好序的順序,
散列的性能就可以有較大的提高。
 Marley 教授的改動對成功查找、不成功查找、插入和刪除操作的運行時間有何影響?
\stopEXERCISE

\startANSWER
對於成功查找,平均情況下的運行時間沒什麼變化,還是 \m{\Theta(1+\alpha)}。
要檢查的元素數目比鏈表中小於目標的元素數目多一個。

不成功查找的平均時間也是 \m{\Theta(1+\alpha)},但是比成功查找要少一半。
因爲鏈表是排好序的,目標元素按數序所在位置是隨機的,平均情況下要搜索鏈表元素的一半。

插入元素的平均時間爲 \m{\Theta(1+\alpha)},相對於原來的 \m{\Theta(1)} 有所下降。
因爲鏈表需要保持有序。於不成功查找的過程類似。

刪除元素於成功查找類似,也是 \m{\Theta(1+\alpha)}。
\startformula
E[n_j] = \alpha = \frac{n}{2m}
\stopformula
\stopANSWER

%e11.2-4
\startEXERCISE
通過將所有空閒槽位組成一個空閒鏈表,如何爲散列表中的元素分配、釋放存儲空間?
假設一個槽位中可以存儲一個標記,外加一個元素和一個指針,或者兩個指針。
所有字典、空閒鏈表操作的期望時間應爲 \m{O(1)}。
空閒鏈表是否需要是雙向鏈表,或者說單向空閒鏈表是否足夠?
\stopEXERCISE

\startANSWER
用這個標記來指示槽位是否空閒。

空閒槽位處於空閒鏈表中,通過雙線鏈接將所有空閒槽位組成空閒鏈表。
空閒槽位中含有兩個指針。

而非空閒槽位中除了標記以外,還含有一個元素值和一個指針(可能是 NIL),這個指針只想散列到此槽位的下一個元素。
(當然,這個指針指向的其實是另一個槽位)

{\EMP 插入:}

如果新元素散列到空閒槽位上,則將此槽位從空閒鏈表中移除,將此元素的值填如,並將指針賦值爲 NIL。
爲了使其時間滿足 \m{O(1)},鏈表需爲雙向。

而如果新元素散列到了一個已用的槽位 j 上,則檢查此槽位原有元素 x 是否應當散列到此槽位上。

如果是,則將新元素加入原由鏈表中即可。需要先分配一個空閒槽位(即取出空閒鏈表的頭部節點),填充數據,
並將此槽位加入槽位 j 中的鏈表。

否則,新分配一個槽位,用來在 j 所屬鏈表中替換 j。並將 j 作爲新鏈表的頭節點,填入新元素的值,並將指針賦值爲 NIL。
其中替換 j 時,需要從 j 所屬鏈表的頭部順序查找,才能找到 j 的前驅節點。

{\EMP 刪除:}

令要刪除的元素爲 x,其散列到槽位 j 上。

如果 x 是 j 中的元素,且 j 沒有後繼節點,則釋放 j 即可。

如果 x 是 j 中的元素,但 j 有後繼節點,則交換 j 與其後繼節點的所有內容,並釋放其後繼節點。

否則需要從 j 開始,在鏈表中搜索 x,找到後直接將其從鏈表中剔除,釋放即可。

{\EMP 查找:}

檢查關鍵字所散列到的槽位,如果此槽位的原有元素也是散列到此位置上,則在其鏈表中搜索(包括此槽位自身);
否則查找直接失敗。

所有操作期望時間均爲 \m{O(1)}。
在鏈表中查找的期望時間爲 \m{O(1+\alpha)},於鏈表的存儲位置無關,
所有元素都在表中則意味着 \m{\alpha\le 1}。
而如果空閒鏈表是單向的,就無法在 \m{O(1)} 時間內刪除任意槽位了。
\stopANSWER

%11.2-5
\startEXERCISE
將 n 個關鍵字存儲在大小爲 m 的散列表中。證明:
如果這些關鍵字源於全域 U,且 \m{|U| > nm},
且 U 中有大小爲 n 的子集,其中關鍵字會散列到散列表中的同一槽位上,
那麼用鏈表解決衝突的情況下,最壞情況的查找時間爲 \m{\Theta(n)}。
\stopEXERCISE

\startANSWER
必有一個槽位的元素數量大於 n (鴿籠原理)。
\stopANSWER

%e11.2-6
\startEXERCISE
將 n 個關鍵字存儲在大小爲 m 的散列表中,以鏈表解決衝突,
我們直到每個鏈表的長度,並且知道最長鏈表的長度 L。
如何從所有關鍵字中均勻隨機選取其一,
並在期望時間 \m{O(L\cdot (1+1/\alpha))} 內完成。
\stopEXERCISE

\startANSWER
首先,隨機選擇一個槽位,令其所含元素數目爲 k。

然後,在 \m{1,2,\ldots,L} 中均勻隨機選擇一個,令其爲 p。

如果 \m{p\le k} 則返回 k 個元素中的第 p 個,否則重復執行以上步驟。

上述過程所返回每個元素的概率是相同的。
每個槽位所含元素數目的期望值爲 \m{n/m}。
採樣嘗試成功的概率爲 \m{(n/m)/L}。
採樣成功所要進行的嘗試次數期望值爲 \m{L \times m / n}。
加上在具體槽位上獲取最終元素的開銷 \m{O(L)},
總的期望時間爲 \m{O(L\times (1+m/n))}。

請參考 \goto{stackexchange}[url(http://stackoverflow.com/questions/8629447/efficiently-picking-a-random-element-from-a-chained-hash-table)]。
\stopANSWER

\stopsection
