\startsection[
  title={Direct-address tables},
]

%e11.1-1
\startEXERCISE
假設一動態集合 S 用一個長度爲 m 的直接尋址表 T 來表示。
請給出一個查找 S 中最大元素的過程。
你所給出的過程在最壞情況下的運行時間如何?
\stopEXERCISE

\startANSWER
我們從表的底部(最大元素)查起,並向後掃描整張表,直到找到非空的槽。
最壞情況下運行時間爲 \m{\Theta(m)},此時最大元素在表的起始位置,或者動態集合爲空。
\stopANSWER

%e11.1-2
\startEXERCISE
{\EMP 位向量}({\EMP bit vector})是一個僅包含 0 和 1 的數列。
長度爲 m 的位向量所佔空間比 m 個指針要少的多。
試說明如何用位向量表示動態集合,此集合中的元素互不相同,且沒有衛星數據。
字典操作的運行時間應爲 \m{O(1)}。
\stopEXERCISE

\startANSWER
動態集合的元素必須是整數。

對於位向量中的每一位,如果是 1 則表示集合中有此元素,如果是 0 則表示集合中沒有此元素。
在沒有衛星數據的情況下這足夠了。
修改或查找操作需要用到位操作,如 \&、 \|。

字典操作都是直接獲得結果。只有向量大小稍微有點復雜。
如果動態集合中有 1000 個元素,則位向量中至少要有 1000 位才行。
\stopANSWER

%e11.1-3
\startEXERCISE
如何實現直接尋址表,才能使得元素的關鍵字可以重復且可以有衛星數據?
三個字典操作(\ALGO{INSERT}、 \ALGO{DELETE} 和 \ALGO{SEARCH})的運行時間均爲 \m{O(1)}。
(不要忘記 \ALGO{DELETE} 的參數是要刪除對象的指針而不是關鍵字)
\stopEXERCISE

\startANSWER
用雙向鏈表。

\startigBase[a]
\item \ALGO{INSERT} 插入鏈表需要常數時間;
\item \ALGO{DELETE} 由鏈表移除元素也需常數時間;
\item \ALGO{SEARCH} 返回鏈表中的第一個元素,也是常數時間。
\stopigBase
\stopANSWER

%e11.1-4
\startEXERCISE\DIFFICULT
我們想在一個{\EMP 非常大}的數列上用直接尋址來實現一個字典。
一開始,數列中可能包含一些無用信息,並且由於他非常大,初始化整個數列也不切實際。
請給出你的方案。
所存儲的每個對象所佔空間爲 \m{O(1)};
且 \ALGO{SEARCH}、 \ALGO{INSERT} 和 \ALGO{DELETE} 所需時間也是 \m{O(1)}。
(\hint 用一個附加數列,將其視爲棧,其大小就是字典中實際存儲的關鍵字的數目,
用他來確定數列中的某一項是否有效。)
\stopEXERCISE

\startANSWER
只需將大數列中的元素和棧中的元素建立關聯,如:
棧中元素存儲數列元素的地址,而數列元素存儲棧上元素的索引。
 \ALGO{SEARCH} 時需要檢查上述關聯是否成立,若成立則有效,否則無效。
 \ALGO{INSERT} 對棧執行 \ALGO{PUSH};
 \ALGO{DELETE} 時則將棧當成普通數列,如果要刪除的不是棧頂元素,則將棧頂元素轉移到所刪除的元素處,
並同步更新大數列中的相應元素。
\stopANSWER

\stopsection
