\startsubject[
  title={Problems},
]

%p11-1
\startPROBLEM
(Longest-probe bound for hashing)
採用開放尋址法,用一個大小爲 m 的散列表存儲 n (\m{n\le m/2})個數據項。
\startigBase[a]
\startitem
假設採用均勻散列,證明:
對於 \m{i=1,2,\ldots,n},第 i 次插入需要嚴格多於 k 次探查概率至多爲 \m{2^{-k}}。
\stopitem

\startANSWER
\startformula\startmathalignment
\NC \Pr\{x>k\}
    \NC = \Pr\{x\ge k+1\} \NR
\NC \NC = \frac{n}{m} \cdot \frac{n-1}{m-1} \cdot \ldots \cdot \frac{n-k+1}{m-k+1} \qquad \text{參見定理 11.6} \NR
\NC \NC < (\frac{n}{m})^k \NR
\NC \NC \le (\frac{1}{2})^k \NR
\NC \NC = 2^{-k} \NR
\stopmathalignment\stopformula
\stopANSWER

\startitem
證明:對於 \m{i=1,2,\ldots,n},第 i 次插入需要多於 \m{2\lg{n}} 次探查的概率爲 \m{O(1/n^2)}。
\stopitem

\startANSWER
根據上一項的結果:
\startformula
\Pr\{x>k\} \le 2^{-k} = 2^{-2\lg{n}} = \frac{1}{n^2}
\stopformula
\stopANSWER

設隨機變量 \m{X_i} 表示第 i 次插入所需的探查次數。
在上面 (b) 中已證明 \m{\Pr\{X_i>2\lg{n} = O(1/n^2)\}}。
設隨機變量 \m{X=\max_{1\le i\le n}X_i} 表示 n 次插入過程中所需探查數的最大值。

\startitem
證明: \m{\Pr\{X>2\lg{n}\} = O(1/n)}。
\stopitem

\startANSWER
\startformula
\Pr\{X>2\lg{n}\}
  = \sum_{i=1}^{n}\Pr\{X_i>2\lg{n}\}
  \le \sum_{i=1}^{n}\frac{1}{n^2}
  = n \cdot \frac{1}{n^2}
  = \frac{1}{n}
\stopformula
\stopANSWER

\startitem
證明:最長探查序列的期望長度爲 \m{E[x]=O(\lg{n})}。
\stopitem

\startANSWER
參見 \simpleurl{http://blog.csdn.net/z84616995z/article/details/21329071}。
\startformula\startmathalignment
\NC E[x]
    \NC = \sum_{k=1}^{n}k\Pr\{X=k\} \NR
\NC \NC = \sum_{k=1}^{2\lg{n}}k\Pr\{X=k\} + \sum_{k=2\lg{n}+1}^{n}k\Pr\{X=k\} \NR
\NC \NC < 2\lg{n}\sum_{k=1}^{2\lg{n}}\Pr\{X=k\} + n\sum_{k=2\lg{n}+1}^{n}\Pr\{X=k\} \NR
\NC \NC = 2\lg{n}\Pr\{X\le 2\lg{n}\} + n\Pr\{X>2\lg{n}\} \qquad \text{其中 }
	  \Pr\{X\le 2\lg{n}\}\le 1 \qquad
	  \Pr\{X>2\lg{n}\} \le \frac{1}{n^2} \NR
\NC \NC \le 2\lg{n} + n \cdot \frac{1}{n^2} \NR
\NC \NC < 2\lg{n} + 1 \NR
\NC \NC = O(\lg{n}) \NR
\stopmathalignment\stopformula
\stopANSWER

\stopigBase
\stopPROBLEM

%p11-2
\startPROBLEM
(Slot-size bound for chaining)
假設有一個含 n 個槽的散列表,向表中插入 n 個關鍵字,並用鏈接法解決衝突問題。
每個關鍵字被等可能地散列到每個槽中。
插入所有關鍵字後,設 M 是各槽中所含關鍵字數目的最大值。
讀者的任務是證明 M 的期望值 \m{E[M]} 的一個上界爲 \m{O(\lg{n}/\lg\lg{n})}。
\startigBase[a]
\startitem
證明:正好有 k 個關鍵字被散列到某一特定槽中的概率 \m{Q_k} 爲:
\startformula
Q_k = (\frac{1}{n})^k (1-\frac{1}{n})^{n-k} \binom{n}{k}
\stopformula
\stopitem

\startANSWER
先在 n 個關鍵字中選出 k 個,
這 k 個關鍵字都會散列到特定槽中(每次散列的概率爲 \m{1/n}),
其他關鍵字都散列到其他槽中(每次散列的概率爲 \m{1- 1/n})。
\stopANSWER

\startitem
設 \m{P_k} 爲 \m{M=k} 的概率,
即包含最多關鍵字的槽中有 \m{k} 個關鍵字的概率。
證明: \m{P_k\le n Q_k}。
\stopitem

\startANSWER
所選特定槽可能是 n 個槽中的任何一個,且概率相等。
\stopANSWER

\startitem
應用 Stirling 近似公式 3.18 證明: \m{Q_k < e^k / k^k}。
附 Stirling 近似公式:
\startformula
n! = \sqrt{2\pi n} (\frac{n}{e})^n (1 + \Theta(\frac{1}{n}))
\stopformula
\stopitem

\startANSWER
\startformula\startmathalignment
\NC Q_k
    \NC = (\frac{1}{n})^k (1-\frac{1}{n})^{n-k} \binom{n}{k} \NR
\NC \NC < \frac{1}{n^k} \frac{n!}{k!(n-k)!} \NR
\NC \NC < \frac{1}{k!} \NR
\NC \NC < \frac{1}{(\frac{k}{e})^k} \NR
\NC \NC = \frac{e^k}{k^k} \NR
\stopmathalignment\stopformula
\stopANSWER

\startitem
證明:存在常數 \m{c>1},使得 \m{Q_{k_0} < 1/n^3} 對 \m{k_0 = c \lg{n}/\lg\lg{n}} 成立。
並有結論:對 \m{k\ge k_0 = c\lg{n}/\lg\lg{n}}, \m{P_k < 1/n^2} 成立。
\stopitem

\startANSWER
\startformula\startmathalignment[n=1]
\NC \frac{e^k}{k^k} < \frac{1}{n^3} \NR
\NC (\frac{k_0}{e})^{k_0} > n^3 \qquad \text{兩邊同取對數} \NR
\NC k_0 (\lg{k_0} - 1) > 3 \lg{n} \NR
\NC c \frac{\lg{n}}{\lg\lg{n}} (\lg(\frac{c\lg{n}}{\lg\lg{n}}) - 1) > 3 \lg{n} \NR
\NC c \frac{\lg{n}}{\lg\lg{n}} (\lg{c} + \lg\lg{n} - \lg\lg\lg{n} - 1) > 3 \lg{n} \NR
\NC c (\lg{c} + \lg\lg{n} - \lg\lg\lg{n} - 1) > 3 \lg\lg{n} \NR
\NC c (\lg{c} - \lg\lg\lg{n} - 1) + c \lg\lg{n} > 3 \lg\lg{n} \NR
\NC c (\lg{c} - \lg\lg\lg{n} - 1) > 0 \qquad c > 3 \NR
\NC \lg{c} - \lg\lg\lg{n} > 1 \qquad c > 3 \NR
\NC c / \lg\lg{n} > e \qquad c > 3 \NR
\NC c > e \lg\lg{n} \qquad c > 3 \NR
\stopmathalignment\stopformula
\stopANSWER

\startitem
證明:
\startformula
E[M] \le \Pr\{ M>\frac{c\lg{n}}{\lg\lg{n}} \} \cdot n
         + \Pr\{ M\le \frac{c\lg{n}}{\lg\lg{n}} \} \cdot \frac{c\lg{n}}{\lg\lg{n}}
\stopformula
並有結論: \m{E[M] = O(\lg{n}/\lg\lg{n})}。
\stopitem

\startANSWER
令 \m{k_0 = \frac{c\lg{n}}{\lg\lg{n}}},則:
\startformula\startmathalignment
\NC E[M]
    \NC = \sum_{k=1}^{n}k\Pr\{M=k\} \NR
\NC \NC = \sum_{k=1}^{k_0}k\Pr\{M=k\}
          + \sum_{k=k_0+1}^{n}k\Pr\{M=k\} \NR
\NC \NC < k_0 \sum_{k=1}^{k_0}\Pr\{M=k\}
          + n\sum_{k=k_0+1}^{n}\Pr\{M=k\} \NR
\NC \NC = k_0 \Pr\{M\le k_0\}
          + n\Pr\{M>k_0\} \qquad \text{其中 }
	  \Pr\{M\le k_0\}\le 1 \qquad
	  \Pr\{M>k_0\} = P_k < \frac{1}{n^2} \NR
\NC \NC < k_0 + n \cdot \frac{1}{n^2} \NR
\NC \NC < k_0 + 1 \NR
\NC \NC = O(k_0) \NR
\stopmathalignment\stopformula
\stopANSWER

\stopigBase
\stopPROBLEM

%p11-3
\startPROBLEM
(Quadratic probing)
假設要在一個散列表中查詢關鍵字 k,散列表中各個位置爲 \m{0,1,\ldots,m-1}。
假設散列函數 h 可以將關鍵字空間映射到集合 \m{0,1,\ldots,m-1} 上,按如下方法進行查找:
\startigBase[n]
\item 計算值 \m{j=h(k)},並置 \m{i=0}。
\item 在位置 j 上搜索關鍵字 k,無論是找到了 k 還是該位置爲 空,都結束查找。
\item 置 \m{i=i+1}。如果 \m{i=m},則表示找不到 k,終止查找;否則,令 \m{j=(i+j)\mod m},跳轉到步驟 2。
\stopigBase

假設 m 是 2 的冪。
\startigBase[a]
\startitem
利用等式 11.5,給出合適的 \m{c_1} 和 \m{c_2},證明該方案是一般的“二次探查”法的一個實例。
附等式 11.5:
\startformula
h(k,i) = (h'(k) + c_1 i + c_2 i^2) \mod m
\stopformula
\stopitem

\startANSWER
\startformula
j_i
  = (j_{i-1} + i) \mod m
  = (j_1 + 1 + 2 + \ldots + i) \mod m
  = (h(k) + \frac{1}{2}i^2 + \frac{1}{2}i) \mod m
\stopformula
因此 \m{c_1 = 1/2, c_2 = 1/2}。
\stopANSWER

\startitem
證明:在最壞情況下,這個算法要檢查表中的每一個位置。
\stopitem

\startANSWER
假設兩次的值分別爲 \m{h(k,i)} 和 \m{h(k,j)} (\m{0<i<j<m})。
\startformula\startmathalignment[n=1]
\NC h(k,i) = (h(k) + i^2/2 + i/2) \mod m \NR
\NC h(k,j) = (h(k) + j^2/2 + j/2) \mod m \NR
\stopmathalignment\stopformula
如果二者相等,則:
\startformula\startmathalignment[n=1]
\NC (h(k) + i^2/2 + i/2) \equiv (h(k) + j^2/2 + j/2) (\mod m) \NR
\NC (i^2/2 + i/2) \equiv (j^2/2 + j/2) (\mod m) \NR
\NC (j^2/2 + j/2 - i^2/2 - i/2) \equiv 0 (\mod m) \NR
\NC (j-i)(j+i+1)/2 = km \qquad \text{令 }m=2^t\NR
\NC (j-i)(j+i+1) = k 2^{t+1} \NR
\stopmathalignment\stopformula

另外 \m{j-i} 和 \m{j+i+1} 定是一個奇數、另一個偶數。
其中奇數不含因子 2,所以所有的因子 2 都在偶數中。即 \m{2^{t+1} = 2m} 能整除其中的偶數。

即設 \m{j-i} 是偶數,則 \m{j-i < m},而 \m{2m | (j-i)} 不成立。
而如果 \m{j+i+1} 是偶數,則 \m{j+i+1 < 2m},而 \m{2m | (j-i+1)} 不成立。
因此 \m{h(k,i)\ne h(k,j)}。
即 m 次循環所得散列值各部相等,又因爲都小於 m,所以覆蓋了整個散列表。
\stopANSWER

\stopigBase
\stopPROBLEM

%p11-4
\startPROBLEM
(Hashing and authentication)
令 \m{{\cal H}} 爲一組散列函數,
其中每個函數 \m{h\in{\cal H}} 都可以將關鍵字的全域 U 映射到 \m{\{0,1,\ldots,m-1\}} 上。
如果對於任一 k 個互不相同的關鍵字構成的的固定序列 \m{\langle x^{(1)},x^{(2)},\ldots,x^{(k)} \rangle},
以及從 \m{{\cal H}} 中隨機選出的 \m{h},
序列 \m{\langle h(x^{(1)}),h(x^{(2)}),\ldots,h(x^{(k)}) \rangle} 是 \m{m^k} 個長度爲 \m{k} 的序列
(其元素取自 \m{\{0,1,\ldots,m-1\}})中任意一個,其可能性相同,
我們稱 \m{{\cal H}} 是 {\EMP k 全域}(k-universal)的。
\startigBase[a]
\startitem%a
證明:如果散列函數族 \m{{\cal H}} 是 2 全域的,則他是全域的。
\stopitem

\startANSWER
如果 \m{{\cal H}} 是 2 全域的,則對於兩個特定關鍵字 \m{x\ne y},
序列 \m{\langle x, y\rangle} 是 \m{m^2} 種序列中的任一個,概率相同。
因此,由於 \m{h} 是 \m{{\cal H}} 中的任意一個,
產生衝突 \m{h(x) = h(y)} 的概率爲 \m{(1/m)|{\cal H}|},
所以 \m{{\cal H}} 是全域的。

參見 \simpleurl{http://ripcrixalis.blog.com/2011/02/08/clrs-chapter-11/}。
\stopANSWER

\startitem%b
設全域 U 爲取自 \m{\integers_p = \{0,1,\ldots,p-1\}} 中數值的 n 元組集合,
其中 p 是素數。
考慮元素 \m{x=\langle x_0,x_1,\ldots,x_{n-1} \rangle \in U}。
對於任意 n 元組 \m{a=\langle a_0,a_1,\ldots,a_{n-1} \rangle \in U},
定義散列函數 \m{h_a} 爲:
\startformula
h_a(x) = \left( \sum_{j=0}^{n-1}a_j x_j \right) \mod p
\stopformula
並設 \m{{\cal H}=\{h_a\}}。
證明: \m{{\cal H}} 是全域的,但不是 2 全域的。
(\hint 找一個關鍵字,使得 \m{{\cal H}} 中所有散列函數對其都得到相同的值。)
\stopitem

\startANSWER
假設在 \m{{\cal H}} 中選擇函數 \m{h},
如果 \m{a=\langle 0,0,\ldots,0\rangle},則對於序列 \m{\langle x,y\rangle},
有 \m{h_a(x)=h_a(y) = 0 \mod p = 0},即這種情況下無論關鍵字是什麼,映射對結果永遠是 \m{\langle 0,0\rangle},所以不是 2 全域的。

參見 \simpleurl{http://ripcrixalis.blog.com/2011/02/08/clrs-chapter-11/}。
\stopANSWER

\startitem%c
即設將 (b) 中的 \m{{\cal H}} 略作修改:對任意 \m{a\in U} 和任意 \m{b\in\integers_p},定義:
\startformula
h'_{ab}(x) = \left( \sum_{j=0}^{n-1}a_j x_j + b \right) \mod p
\stopformula
且 \m{{\cal H}'=\{h'_{ab}\}}。論證 \m{{\cal H}'} 是 2 全域的。
(\hint 慮固定的 n 元組 \m{x\in U} 和 \m{y\in U},
對某個 i 有 \m{x_i\ne y_i}。當 \m{a_i} 和 b 包括 \m{\integers_p} 時,
 \m{h'_{ab}(x)} 和 \m{h'_{ab}(y)} 會如何?)
\stopitem

\startANSWER
對於每一對關鍵字 \m{\langle x,y\rangle},其中 \m{x\ne y},
我們要證明所有值對 \m{(h_{ab}(x),h_{ab}(y))} 的概率相同,
其中 \m{h_{ab}} 是從 \m{{\cal H}} 中隨機選出,
即 \m{\langle a_0,a_1,\ldots,a_{n-1}} 和 b 是隨機選出的。

既然 \m{x\ne y},那麼肯定存在 i 使得 \m{x_i\ne y_i},
假設 \m{i=0},令:
\startformula
\alpha = h_{ab}(x)
\qquad \beta = h_{ab}(y)
\stopformula
且:
\startformula
X=\sum_{j=1}^{n-1}a_j x_j
\qquad Y=\sum_{j=1}^{n-1}a_j y_j
\stopformula
則:
\startformula
\alpha = (a_0 x_0 + b + X) \mod p
\qquad \beta = (a_0 y_0 + b + Y) \mod p
\stopformula
由於 \m{x_0 \ne y_0} 且 p 是素數,那麼對於 \m{\alpha} 和 \m{\beta} 而言,
上式有唯一解 \m{a_0} 和 \m{b}。

更明顯一點,考慮我們要生成所有可能的值對 \m{\alpha,\beta},
他與值對 \m{\alpha,\alpha-\beta} 是雙射的。而:
\startformula
\alpha-\beta=a_0(x_0-y_0) + X - Y (\mod p)
\stopformula
所以:
\startformula
a_0 = (\alpha - \beta -X + Y) (x_0 - y_0)^{(-1)} (\mod p)
\stopformula

由於 \m{x_0 \ne y_0} 且 p 是素數,則 \m{(x_0 - y_0)^{(-1)}} 肯定存在且唯一。
對於特定 \m{a_0} 而言,我們可以隨意選擇 \m{\alpha-\beta} 的值。
對於特定 \m{b} 而言,我們可以隨意選擇 \m{\alpha} 的值。
即對於特定的 \m{a_0} 和 \m{b},我們可以隨意選擇 \m{\alpha} 和 \m{\beta},
只要保證 \m{(\alpha-\beta)\mod p} 的值相同即可。

因此,對於任意給定的 \m{\langle a_0,a_1,\ldots,a_{n-1}\rangle},
我們可以找到散列函數 \m{h_{ab}},通過選擇合適的 \m{a_0} 和 \m{b},
使其可以生成任何可能的 \m{\langle\alpha,\beta\rangle}。

由於 \m{a_0} 和 \m{b} 的選擇共有 \m{p^2} 種,
並且 \m{\langle\alpha,\beta\rangle} 也有 \m{p^2} 種,
每個 \m{\langle\alpha,\beta\rangle} 都對應唯一的 \m{\langle a_0,b\rangle}。
這對 \m{a_1,\ldots,a_{n-1}} 的 \m{p^{n-1}} 種選擇均成立。

因此,有 \m{p^{n-1}} 種函數 \m{h_{ab}} 可以生成每個值對 \m{\langle\alpha,\beta\rangle}。
如果 \m{h_{ab}} 是隨機選擇的,則每個值對的概率相同,因此 \m{{\cal H}} 是 2 全域的。

參見 \simpleurl{http://ripcrixalis.blog.com/2011/02/08/clrs-chapter-11/}。
\stopANSWER

\startitem%d
假設 Alice 和 Bob 悄悄地約定了一個取自 2 全域散列函數族 \m{{\cal H}} 中的散列函數 h。
每個 \m{h\in {\cal H}} 將關鍵字全域 U 映射到 \m{\integers_p} 上,此處 p 爲素數。
後來, Alice 通過互聯網向 Bob 發送了一個消息 m,其中 \m{m\in U}。
她同時還通過發送一個認證標記 \m{t=h(m)} 來向 Bob 認證這一消息,
而 Bob 則要檢查他所收到的 \m{(m,t)} 對是否確實滿足 \m{t=h(m)}。
假設某一對手半路截獲了 \m{(m,t)},並試圖將該值對替換爲令一值對 \m{(m',t')} 來欺騙 Bob。
論證無論該對手的計算機性能多好,他成功欺騙 Bob 接受 \m{(m',t')} 的概率至多爲 \m{1/p},
即使他知道所用的散列函數族 \m{{\cal H}}。
\stopitem

\startANSWER
對於關鍵字對 \m{\langle m,m'\rangle},如果 h 是隨機選取的,
則所有散列值對 \m{\langle h(m),h(m')\rangle} 的概率相同。
這就是 \m{{\cal H}} 爲 2 全域的意思。

即 p 種值對 \m{\langle t,t'\rangle} 的概率相同,
所以即使他知道 \m{{\cal H}},根據 m 和 t 也無法得出 \m{h(m')}。
由於 p 種情況的概率相同,且總和爲 1,
任一種可能的 \m{h(m')} 概率均爲 \m{1/p},
所以他只能靠猜,沒有更好的辦法。

參見 \simpleurl{http://ripcrixalis.blog.com/2011/02/08/clrs-chapter-11/}。
\stopANSWER
\stopigBase
\stopPROBLEM

\stopsubject%Problems
