\startsubject[
  title={Problems},
]

%p9-1
\startPROBLEM
(找出最大的 \m{i} 個元素並排序)
給定一個包含 \m{n} 個元素的集合,
我們希望利用基於比較的算法找出最大的 \m{i} 個元素並排序。
請找出下列各種方法中,具有最佳漸進最壞情況運行時間的算法,
以 \m{n} 和 \m{i} 來表示算法的運行時間:
\startigBase[a]
\startitem
對輸入數據排序,並找出最大的 \m{i} 個數;
\stopitem

\startANSWER
\startformula
O(n\lg{n} + i)
\stopformula
\stopANSWER

\startitem
對輸入數據建立一個最大優先對列,並調用 \ALGO{EXTRACT-MAX} 過程 \m{i} 次;
\stopitem

\startANSWER
\startformula
O(n + i\lg{n})
\stopformula
\stopANSWER

\startitem
利用一個順序統計量算法找到第 \m{i} 大的元素,然後用他作爲主元劃分輸入數列,
再對最大的 \m{i} 個數排序。
\stopitem

\startANSWER
\startformula
O(n + i\lg{i})
\stopformula
\stopANSWER
\stopigBase
\stopPROBLEM

%p9-2
\startPROBLEM
(Weighted median)
有 \m{n} 個互異元素 \m{x_1,x_2,\ldots,x_n},其權重分別爲 \m{w_1,w_2,\ldots,w_n},
且 \m{\sum_{i=1}^{n}w_i=1},{\EMP 帶權中位數}(較小的那個) \m{x_k} 滿足下列條件:
\startformula
\sum_{x_i<x_k}w_i < \frac{1}{2}
\stopformula
和
\startformula
\sum_{x_i>x_k}w_i \le \frac{1}{2}
\stopformula

例如,如果這些元素爲 \m{0.1,0.35,0.05,0.1,0.15,0.05,0.2},
且每個元素的權重與其值相等(即, \m{w_i=x_i}, \m{i=1,2,\ldots,7}),
則中位數爲 0.1,但帶權中位數爲 0.2。
\startigBase[a]
\startitem
討論:如果 \m{x_i} 的權重 \m{w_i=1/n},其中 \m{i=1,2,\ldots,n},則帶權中位數就是中位數。
\stopitem

\startANSWER
如果所有權重均爲 \m{1/n},則小於中位數的元素權重和爲 \m{\left\lfloor \frac{n-1}{2}\right\rfloor \frac{1}{n}},
大於中位數的元素權重和爲 \m{\left\lceil \frac{n-1}{2}\right\rceil \frac{1}{n}}。
滿足帶權中位數的要求。
\stopANSWER

\startitem
如何在最壞情況下以時間 \m{O(n\lg{n})} 計算 \m{n} 個元素的帶權中位數。
\stopitem

\startANSWER
先對輸入進行排序,從左至右進行掃描,並累加各元素的權重,一旦權重累計值等於或超過 \m{1/2},則對應元素即爲所求。
\stopANSWER

\startitem
如何利用線性時間的中位數算法(如\refsection{worstcase_linear_selection} 中
的 \ALGO{SELECT})在最壞情況下能以時間 \m{\Theta(n)} 計算帶權中位數。
\stopitem

\startANSWER
與 \ALGO{SELECT} 類似,只是目標不再是主元的位置,而是主元兩側元素權重的和。
\stopANSWER
\stopigBase

郵局位置問題(post-office location problem)定義如下:
有 \m{n} 個點 \m{p_1,p_2,\ldots,p_n},其權重分別爲 \m{w_1,w_2,\ldots,w_n}。
我們要找到一個點 \m{p} (不必非得是輸入中的元素),
以使 \m{\sum_{i=1}^{n}w_i d(p,p_i)} 最小,
其中 \m{d(a,b)} 是點 \m{a} 和點 \m{b} 間的距離。
\startigBase[a,continue]
\startitem
討論:帶權中位數是解決一維郵局位置問題最好的方法,
其中所有點都是實數, \m{d(a,b)=|a-b|}。
\stopitem

\startANSWER
參考\refexercise{oil_well},
假定我們得到了最終答案,就是帶權中位數。
這時無論是將結果左移還是右移,都將離加權和小於 \m{1/2} 的元素越來越近,
而離加權和大於 \m{1/2} 的元素越來越遠,從而增大 \m{\sum_{i=1}^{n}w_i d(p,p_i)}。
\stopANSWER

\startitem
如何解決二維郵局位置問題?其中每個點都是 \m{(x,y)} 坐標,
兩個點 \m{a=(x_1,y_1)} 和 \m{b=(x_2,y_2)} 的距離爲{\EMP 曼哈頓距離(Manhattan distance)},
即 \m{d(a,b)=|x_1-x_2| + |y_1 - y_2|}。
\stopitem

\startANSWER
根據曼哈頓距離的定義,可以將其分成相互獨立的兩部分 \m{d_x(a,b)=|x_1-x_2|} 和 \m{d_y(a,b)=|y_1-y_2|},
要想 \m{\sum_{i=1}^{n}w_i d(p,p_i)} 最小,即 \m{\sum_{i=1}^{n}w_i d_x(p,p_i)} 和 \m{\sum_{i=1}^{n}w_i d_y(p,p_i)} 都最小(\m{d_x} 和 \m{d_y} 均大於零)。
因此我們只需針對 x 坐標和 y 坐標分別找出其帶權中位數即爲所求。
\stopANSWER
\stopigBase

\stopPROBLEM

%e9-3
\startPROBLEM
(Small order statistics)
我們已經知道,用 \ALGO{SELECT} 找出 \m{n} 個元素中的第 \m{i} 個順序統計量,
最壞情況所用的時間滿足 \m{T(n)=\Theta(n)},
但是 \m{\Theta} 所隱含的常數因子非常大。
當 \m{i} 相對 \m{n} 比較小時,
我們可以重新實現他,並將 \ALGO{SELECT} 作爲子過程,以減少最壞情況下的比較次數。
\startigBase[a]
\startitem
設計一個算法,可以找出 \m{n} 個元素中第 \m{i} 小的元素,所用比較次數 \m{U_i(n)} 滿足:
\startformula
U_i(n)=\startmathcases
\NC T(n) \NC 若 \m{i\ge n/2}; \NR
\NC \left\lfloor n/2 \right\rfloor + U_i(\left\lceil n/2 \right\rceil) + T(2i) \NC 否則。 \NR
\stopmathcases
\stopformula
(\hint 相互獨立的\m{\left\lfloor n/2\right\rfloor} 對元素比較開始,
然後在規模更小的子集上進行遞迴)
\stopitem

\startANSWER
修改過的 \ALGO{SELECT},不但能找到第 \m{i} 個順序統計量,同時會劃分數列,找到 \m{i-1} 個更小的元素。
\startigNum
\item 如果 \m{i\ge n/2},直接調用 \ALGO{SELECT};
\item 否則,將數列劃分成一對對元素,每一對元素間進行比較;
\item 選取每一對元素中較小的那個,但是跟蹤另外一個(即保持成對的關聯);
\item 遞迴查找較小元素集合中的最小的 \m{i} 個元素,這樣連同每個元素所對應的元素,一共有 \m{2i} 個元素;
\item 前 \m{i} 個順序統計量肯定在這 \m{2i} 個元素中,在這 \m{2i} 個元素上調用 \ALGO{SELECT}。結果即爲所求。
\stopigNum
其中第三步會將較小元素集合劃分成兩部分 \m{S_1} 和 \m{S_2},
在較大元素集合中分別對應 \m{B_1} 和 \m{B_2},
則 \m{S_1} 和 \m{B_1} 中的元素個數均爲 \m{i},四個集合元素總數爲 \m{n}。
由於 \m{S_2} 和 \m{B_2} 中的元素均大於 \m{S_1} 中的所有元素,
即 \m{S_2} 和 \m{B_2} 中的元素在所有 \m{n} 個元素中排序的序號均大於 \m{i}。
所以前 \m{i} 個順序統計量肯定在 \m{S_1} 和 \m{B_1} 中。
\stopANSWER

\startitem
證明:如果 \m{i<n/2},則 \m{U_i(n)=n+O(T(2i)\lg(n/i))}。
\stopitem

\startANSWER
代入法:
\startformula\startmathalignment
\NC U_i(n)
    \NC = \lfloor n/2 \rfloor + U_i(\lceil n/2 \rceil) + T(2i) \NR
\NC \NC = \lfloor n/2 \rfloor + \lceil n/2 \rceil +
             O(T(2i)\lg(\lfloor n/2 \rfloor / i)) + T(2i) \NR
\NC \NC = n + O(T(2i)\lg(n/i)) + T(2i) \NR
\NC \NC = n + O(T(2i)\lg(n/i)) \NR
\stopmathalignment\stopformula
\stopANSWER

\startitem
證明:如果 \m{i} 是小於 \m{n/2} 的常數,則 \m{U_i(n)=n+O(\lg{n})}。
\stopitem

\startANSWER
\startformula\startmathalignment
\NC U_i(n)
    \NC = n + O(T(2i)\lg(n/i)) \NR
\NC \NC = n + O(O(1)\lg(n/i)) \NR
\NC \NC = n + O(\lg{n} - \lg{i}) \NR
\NC \NC = n + O(\lg{n} - O(1)) \NR
\NC \NC = n + O(\lg{n}) \NR
\stopmathalignment\stopformula
\stopANSWER

\startitem
證明:如果 \m{i=n/k},其中 \m{k\ge 2},則 \m{U_i(n)=n+O(T(2n/k)\lg{k})}。
\stopitem

\startANSWER
\startformula\startmathalignment
\NC U_i(n)
    \NC = n + O(T(2i)\lg(n/i)) \NR
\NC \NC = n + O(T(2n/k)\lg(n/(n/k))) \NR
\NC \NC = n + O(T(2n/k)\lg{k}) \NR
\stopmathalignment\stopformula
\stopANSWER
\stopigBase
\stopPROBLEM

\startPROBLEM
(Alternative analysis of randomized selection)
在這個問題中,我們時用指示器隨機變量來分析 \ALGO{RANDOMIZED-SELECT} 過程,
與節7.4.2 中分析 \ALGO{RANDOMIZED-QUICKSORT} 的方法類似。

在快速排序的分析總,我們假設所有元素互異,
並且將輸入數列 \m{A} 中的元素重命名爲 \m{z_1,z_2,\ldots,z_n},
其中 \m{z_i} 是第 \m{i} 小元素。
因此,調用 \ALGO{RANDOMIZED-SELECT(A, 1, n, k)} 返回 \m{z_k}。

對於 \m{1\le i < j \le n},令
\startformula
X_{ijk}=I\{\text{在查找 \m{z_k} 的算法執行過程中會比較 \m{z_i} 和 \m{z_j}}\}
\stopformula

\startigBase[a]
\startitem
給出 \m{E[X_{ijk}]} 的表達式。
(\hint 達式的值可能會隨 \m{i}、 \m{j} 和 \m{k} 的不同而不同。)
\stopitem

\startANSWER
與快速排序的分析類似,盡管需要考慮 \m{k}。
如果 \m{z_i} 和 \m{z_j} 中任意一個是包含 \m{i}、 \m{j} 和 \m{k} 的最小區間中的第一個,
則要比較 \m{z_i} 和 \m{z_j}。準確的表達式依賴於 \m{k} 與 \m{i}、 \m{j} 的相對位置:
\startformula
E[X_{ijk}] = \startmathcases
\NC 2 / (k - i + 1) \NC 若 \m{i < j \le k} \NR
\NC 2 / (j - i + 1) \NC 若 \m{i \le k \le j} \NR
\NC 2 / (j - k + 1) \NC 若 \m{k \le i < j} \NR
\stopmathcases
\stopformula
\stopANSWER

\startitem
在數列 \m{A} 中查找 \m{z_k} 時,令 \m{X_k} 指示總的比較次數。證明:
\startformula
E[X_k] \le 2 \left(
       \sum_{i=1}^k \sum_{j=k}^n \frac{1}{j - i + 1} +
       \sum_{j=k+1}^n \frac{j - k - 1}{j - k + 1} +
       \sum_{i=1}^{k-2} \frac{k - i - 1}{k - i + 1}
       \right)
\stopformula
\stopitem

\startANSWER
\startformula\startmathalignment
\NC E[X_k]
    \NC = \sum_{i=1}^{n-1}   \sum_{j=i+1}^n E[X_{ijk}] \NR
\NC \NC = \sum_{i=1}^k       \sum_{j=i+1}^n E[X_{ijk}]
            + \sum_{i=k+1}^{n-1} \sum_{j=i+1}^n E[X_{ijk}] \NR
\NC \NC = \sum_{i=1}^k \left(\sum_{j=i+1}^{k-1}E[X_{ijk}]
                                 + \sum_{j=k}^nE[X_{ijk}] \right)
            + \sum_{i=k+1}^{n-1}\sum_{j=i+1}^nE[X_{ijk}] \NR
\NC \NC = \sum_{i=1}^k       \sum_{j=i+1}^{k-1} E[X_{ijk}]
            + \sum_{i=1}^k       \sum_{j=k}^n       E[X_{ijk}]
            + \sum_{i=k+1}^{n-1} \sum_{j=i+1}^n     E[X_{ijk}] \NR
\NC \NC = \sum_{i=1}^{k-2}   \sum_{j=i+1}^{k-1} E[X_{ijk}]
            + \sum_{i=1}^k       \sum_{j=k}^n       E[X_{ijk}]
            + \sum_{i=k+1}^{n-1} \sum_{j=i+1}^n     E[X_{ijk}] \NR
\NC \NC = \sum_{i=1}^{k-2}   \sum_{j=i+1}^{k-1} \frac{2}{k - i + 1}
            + \sum_{i=1}^k       \sum_{j=k}^n       \frac{2}{j - i + 1}
            + \sum_{i=k+1}^{n-1} \sum_{j=i+1}^n     \frac{2}{j - k + 1} \NR
\NC \NC = 2\left(
                \sum_{i=1}^k       \sum_{j=k}^n       \frac{1}{j - i + 1}
              + \sum_{i=k+1}^{n-1} \sum_{j=i+1}^n     \frac{1}{j - k + 1}
              + \sum_{i=1}^{k-2}   \sum_{j=i+1}^{k-1} \frac{1}{k - i + 1}
              \right) \NR
\NC \NC = 2\left(
                \sum_{i=1}^k       \sum_{j=k}^n       \frac{1}{j - i + 1}
              + \sum_{i=k+1}^{n-1} \sum_{j=i+1}^n     \frac{1}{j - k + 1}
              + \sum_{i=1}^{k-2}   \frac{k - i - 1}{k - i + 1}
              \right) \NR
\NC \NC = 2\left(
                \sum_{i=1}^k       \sum_{j=k}^n       \frac{1}{j - i + 1}
              + \sum_{j=k+2}^n     \sum_{i=k+1}^{j-1} \frac{1}{j - k + 1}
              + \sum_{i=1}^{k-2}   \frac{k - i - 1}{k - i + 1}
              \right) \qquad \text{(見下面)} \NR
\NC \NC = 2\left(
                \sum_{i=1}^k       \sum_{j=k}^n       \frac{1}{j - i + 1}
              + \sum_{j=k+2}^n     \frac{j - k - 1}{j - k + 1}
              + \sum_{i=1}^{k-2}   \frac{k - i - 1}{k - i + 1}
              \right) \NR
\NC \NC \le 2\left(
                \sum_{i=1}^k       \sum_{j=k}^n       \frac{1}{j - i + 1}
              + \sum_{j=k+1}^n     \frac{j - k - 1}{j - k + 1}
              + \sum_{i=1}^{k-2}   \frac{k - i - 1}{k - i + 1}
              \right) \NR
\stopmathalignment\stopformula

其中帶注釋的那一步:
\startformula\startmathalignment
\NC   \NC [k+1 \le i \le n - 1][i+1 \le j \le n] \NR
\NC = \NC [k+1 \le i < i + 1 < j \le n] \NR
\NC = \NC [k + 1 < j \le n][k + 1 \le i < j] \NR
\stopmathalignment\stopformula
\stopANSWER

\startitem
證明 \m{E[X_k]\le 4n}。
\stopitem

\startANSWER
後兩項的和:
\startformula
\sum_{j=k+1}^n\frac{j-k-1}{j-k+1} + \sum_{i=1}^{k-2}\frac{k-i-1}{k-i+1}
   \le \sum_{j=k+1}^n 1 + \sum_{i=1}^{k-2} 1
   = n - k + k - 2
   \le n
\stopformula
對於第一項,則可由下列矩陣來說明:
\startformula\startmatrix
\NC i\backslash j  \NC k \NC k+1 \NC \ldots \NC n-1 \NC n \NR
\NC k  \NC 1 \NC \frac{1}{2} \NC \ldots \NC \frac{1}{n-k} \NC \frac{1}{n-k+1} \NR
\NC k-1\NC \frac{1}{2} \NC \frac{1}{3} \NC \ldots \NC \frac{1}{n-k+1} \NC \frac{1}{n-k+2} \NR
\NC \cdots \NC \cdots \NC \cdots\NC \ddots \NC \cdots \NC \cdots \NR
\NC 2  \NC \frac{1}{k-1} \NC \frac{1}{k}   \NC \ldots \NC \frac{1}{n-2} \NC \frac{1}{n-1} \NR
\NC 1  \NC \frac{1}{k}   \NC \frac{1}{k+1} \NC \ldots \NC \frac{1}{n-1} \NC \frac{1}{n}   \NR
\stopmatrix\stopformula
上面矩陣中,行坐標爲 \m{i=k,k-1,\ldots,1},列坐標爲 \m{j=k,k+1,\ldots,n},
矩陣元素即爲 \m{\frac{1}{j - i + 1}}。
左下到右上斜對角線上的元素相同均爲 \m{\frac{1}{m}} 的形式,其中 \m{m=1,2,\ldots,n}。
每個 \m{\frac{1}{m}} 的個數都小於等於 \m{m}。因此整個矩陣元素的和小於 \m{n}。

因此:
\startformula
E[X_k]\le
  2\left(\sum_{i=1}^k       \sum_{j=k}^n       \frac{1}{j - i + 1}
       + \sum_{j=k+1}^n     \frac{j - k - 1}{j - k + 1}
       + \sum_{i=1}^{k-2}   \frac{k - i - 1}{k - i + 1}
   \right) \le 2(n+n) = 4n
\stopformula
\stopANSWER

\startitem
證明:假設數列 \m{A} 中的所有元素互異,則 \ALGO{RANDOMIZED-SELECT} 的期望運行時間爲 \m{O(n)}。
\stopitem

\startANSWER
\ALGO{RANDOMIZED-SELECT} 中的運算跟比較次數成線性關系,
而比較次數的期望值又和 \m{n} 成線性關系,
因此期望運行時間爲線性的 \m{O(n)}。
\stopANSWER
\stopigBase
\stopPROBLEM

\stopsubject%Problems
