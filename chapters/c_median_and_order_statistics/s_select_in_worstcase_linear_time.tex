\startsection[
  title={Selection in worst-case linear time},
  reference=section:worstcase_linear_selection,
]

\startEXERCISE
在算法 \ALGO{SELECT} 中,輸入元素被分爲每組 5 個元素。
如果他們被分爲每組 7 個元素,該算法人然會是線性時間嗎?
證明:如果分成每組 3 個元素, \ALGO{SELECT} 的運行時間不是線性的。
\stopEXERCISE

\startANSWER
每組 7 個元素仍然是線性的。每次劃分時,小於或大於 \m{x} 的元素數組至少爲:
\startformula
4 \left(\left\lceil \frac{1}{2} \left\lceil \frac{n}{7} \right\rceil \right\rceil
           - 2 \right) \ge \frac{2n}{7} - 8
\stopformula

劃分將問題規模減小爲最大 \m{5n/7+8}。則有如下遞迴式:
\startformula
T(n) = \startmathcases
 \NC O(1) \MC \text{若 } n < n_0 \NR
 \NC T(\lceil n/7 \rceil) + T(5n/7 + 8) + O(n) \MC \text{若 } n \ge n_0 \NR
\stopmathcases
\stopformula

猜測 \m{T(n)\le cn},令非遞迴項爲 \m{an}:
\startformula\startmathalignment
\NC T(n) \NC \le c\lceil n/7 \rceil + c(5n/7 + 8) + an \NR
\NC \NC \le cn/7 + c + 5cn/7 + 8c + an \NR
\NC \NC = 6cn/7 + 9c + an \NR
\NC \NC = cn + (-cn/7 + 9c + an) \NR
\NC \NC \le cn \NR
\NC \NC = O(n) \NR
\stopmathalignment\stopformula

當 \m{(-cn/7 + 9c + an) \le 0} 時,最後一步成立。因此:
\startformula\startmathalignment[n=1]
\NC -cn/7 + 9c + an \le 0 \NR
\NC \Downarrow \NR
\NC c(n/7 - 9) \ge an \NR
\NC \Downarrow \NR
\NC \frac{c(n - 63)}{7} \ge an \NR
\NC \Downarrow \NR
\NC c \ge \frac{7an}{n - 63} \NR
\stopmathalignment\stopformula

如果令 \m{n_0=126}, \m{n\le n_0},則 \m{n/(n-63)\le 2}。因此只需 \m{c\ge 14a} 即可。

而對於 3 個一組,則大於或小於中位數的中位數的元素數目至少爲:
\startformula
2 \left(\left\lceil \frac{1}{2} \left\lceil \frac{n}{3} \right\rceil \right\rceil
           - 2 \right) \ge \frac{n}{3} - 4
\stopformula

遞迴式如下:
\startformula
T(n) = T(\lceil n/3 \rceil) + T(2n/3 + 4) + O(n)
\stopformula

用代入法證明 \m{T(n)=\omega(n)}。猜測 \m{T(n) > cn},令非遞迴項爲 \m{an}:
\startformula\startmathalignment
\NC T(n) \NC > c\lceil n/3 \rceil + c(2n/3 + 2) + an \NR
\NC \NC > cn/3 + c + 2cn/3 + 2c + an \NR
\NC \NC = cn + 3c + an \qquad c>0,a>0,n>0\NR
\NC \NC > cn \NR
\NC \NC = \omega(n) \NR
\stopmathalignment\stopformula

上式對所有 \m{c>0} 都成立。
\stopANSWER

% e9.3-2
\startEXERCISE[exercise:partition_1_4]
分析 \ALGO{SELECT},並證明:
如果 \m{n\ge 140},則至少 \m{\lceil n/4 \rceil} 個元素大於中位數的中位數 \m{x},
至少 \m{\lceil n/4 \rceil} 個元素小於 \m{x}。
\stopEXERCISE

\startANSWER
\startformula\startmathalignment[n=1]
\NC \frac{3n}{10} - 6 \ge \left\lceil \frac{n}{4} \right\rceil \NR
\NC \Downarrow \NR
\NC \frac{3n}{10} - 6 \ge \frac{n}{4} + 1 \NR
\NC \Downarrow \NR
\NC \frac{3n}{10} - 7 \ge \frac{n}{4} \NR
\NC \Downarrow \NR
\NC 12n - 280 \ge 10n \NR
\NC \Downarrow \NR
\NC 2n \ge 280 \NR
\NC \Downarrow \NR
\NC n \ge 140 \NR
\stopmathalignment\stopformula
\stopANSWER

%e9.3-3
\startEXERCISE
假設所有元素都是互異的,說明在最壞情況下,
如何才能使快速排序的運行時間爲 \m{O(n\lg{n})}。
\stopEXERCISE

\startANSWER
用 \ALGO{SELECT} 的方式重寫 \ALGO{PARTITION},
則時間變爲 \m{O(n)};
但是當 \m{n} 足夠大時,最小的劃分將是整個輸入的四分之一(參見\refexercise{partition_1_4})。
遞迴式爲:
\startformula
T(n)=T(n/4)+T(3n/4)+O(n)
\stopformula
由\refexercise{partition_alpha} 可知,解爲 \m{\Theta(n\lg{n})}。

這樣我們就可以改善快速排序的漸進時間,雖然常數因子太大以致不太實用。

另一種方式就是在線性時間內找到中位數(用 \ALGO{SELECT})並以此進行劃分,
這樣會得到一個均勻的劃分。
\stopANSWER

%e9.3-4
\startEXERCISE\DIFFICULT
對一個包含 \m{n} 個元素的集合,
假設一個算法只使用比較來確定第 \m{i} 小的元素,證明:
無需額外的比較操作,他也能找到小於他的 \m{i-1} 個元素和大於他的 \m{n-i} 個元素。
\stopEXERCISE

\startANSWER
爲了找到第 \m{i} 小的元素,任何一個算法都需要通過某種方式找到小於他的 \m{i-1} 個元素,
以及大於他的 \m{n-i} 個元素。

通過比較可以將輸入劃分成兩個集合,小於目標元素的在一個集合中,大於目標元素的在另一個集合中,
這兩個集合即爲所求。(快速排序中的劃分)
\stopANSWER

%e9.3-5
\startEXERCISE
假設你已經有了一個最壞情況下是線性時間的用於求解中位數的“黑箱”子過程。
設計一個能在線性時間內解決任意順序統計量的選擇問題算法。
\stopEXERCISE

\startANSWER
如果 \m{i=\left\lceil n/2\right\rceil},則只需調用一次子過程,顯然是線性時間。
否則,只需針對所劃分成的兩部分中的一個調用子過程(視 \m{i} 的大小而定),
因此有如下遞迴式:
\startformula
T(n)=T(n/2)+O(n)
\stopformula

由主定理可知上限爲 \m{O(n)}。
\stopANSWER

%e9.3-6
\startEXERCISE
對一個包含 \m{n} 個元素的集合來說,第 \m{k} 個{\EMP 分位數}是指
能把排序後的集合分成 \m{k} 個大小相同的集合的 \m{k-1} 個順序統計量。
給出一個能找出某一集合的 \m{k} 分位數的 \m{O(n\lg{k})} 時間的算法。
\stopEXERCISE

\startANSWER
\startigNum
\item 如果 \m{k=1},則返回空;
\item 如果 \m{k} 是偶數,我們先找到中位數,用中位數將集合劃分成兩部分,
然後對兩個規模爲 \m{\left\lfloor n/2 \right\rfloor} 的子問題求解,最後加上中位數即可;
\item 如果 \m{k} 是奇數,則先找到 \m{\left\lfloor k/2\right\rfloor} 和 \m{\left\lceil k/2\right\rceil} 兩個邊界,
然後對兩個規模均小於 n/2 的子問題求解。
\stopigNum

最壞情況的遞迴式爲:
\startformula
T(n,k)=2T(\left\lfloor n/2\right\rfloor,k/2)+O(n)
\stopformula
其解爲 \m{O(n\lg{k})}。

如果 \m{n=ak+k-1},其中 \m{a} 爲正整數,毫無疑問,所有子集合元素個數均爲 \m{a},
否則需要注意舍入問題,所有子集合元素個數差異不能超過 1。
\stopANSWER

%e9.3-7
\startEXERCISE
設計一個 \m{O(n)} 時間的算法,對於一個給定的包含 \m{n} 個互異元素的結合 \m{S} 和一個正整數 \m{k\le n},
該算法能夠確定 \m{S} 中最接近中位數的 \m{k} 個元素。
\stopEXERCISE

\startANSWER
\startigNum
\item 先在線性時間內找出中位數;
\item 計算出其他所有元素與中位數的差異;
\item 在上一步結果中,以線性時間找出第 \m{k} 個順序統計量,
\item 距離小於等於第 \m{k} 個順序統計量的元素即爲所求。
\stopigNum
\stopANSWER

%e9.3-8
\startEXERCISE[exercise:9.3-8]
設 \m{X[1..n]} 和 \m{Y[1..n]} 爲兩個有序數列,
每個都包含 \m{n} 個元素。
請設計一個 \m{O(\lg{n})} 時間的算法來找出數列 \m{X} 和 \m{Y} 中所有 \m{2n} 個元素的中位數。
\stopEXERCISE

\startANSWER
\startigNum
\item 如果 \m{n=1},我們選擇小的那個;
\item 否則,分別找出兩個數列的中位數;
\item 中位數較小的那個數列留下大於中位數的元素,另一個數列留下小於其中位數的元素,
這樣問題規模就變爲 \m{\left\lfloor n/2 \right\rfloor};
\item 求兩個新數列的中位數。
\stopigNum

遞迴式爲:
\startformula
T(n)=T(n/2) + O(n)
\stopformula
\stopANSWER

%e9.3-9
\startEXERCISE[exercise:oil_well]
Olay 教授是一家石油公司的顧問。
這家公司正在計劃建造一條從東到西的大型輸油管道,
這一管道將穿越一個有 \m{n} 口油井的油田。
公司希望每口油井都有一條管道支線沿着最短路徑連接到主管道(方向或南或北),如下圖所示。
給定每口油井的 \m{x} 和 \m{y} 坐標,
教授應該如何選擇主管道的最優位置,使得各支線的總長度最小?
證明:該最優位置可以在線性時間內確定。

\externalfigure[output/e9_3_9-1]
\stopEXERCISE

\startANSWER
只需關心 \m{y} 坐標, \m{x} 坐標沒有影響。
如果 \m{n} 是奇數,則選取所有油井 \m{y} 坐標的中位數,作爲主管道的 \m{y} 坐標,
即主管道穿過此油井。這樣主管道兩側的油井數目相同。
對於任兩口油井而言,只要主管道在他們中間通過,那麼這兩口油井的支線管道總長度是不變的。

如果 \m{n} 是偶數,則需要所有油井 \m{y} 坐標的兩個中位數,主管道的 \m{y} 坐標在這兩個 \m{y} 坐標中間即可。
\stopANSWER

\stopsection
