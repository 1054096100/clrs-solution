\startsection[
  title={Bucket sort},
]

\startEXERCISE
參照圖 8-4 的方法,說明 \ALGO{BUCKET-SORT} 在數列 \m{A = \langle .79, .13, .16, .64,
.39, .20, .89, .53, .71, .42 \rangle} 上的操作過程。
\stopEXERCISE

\startANSWER
\externalfigure[output/e8_4_1-1]
\stopANSWER

\startEXERCISE
解釋爲什麼桶排序在最壞情況下運行時間是 \m{\Theta(n^2)}?
我們應該如何修改算法,使其在保持平均情況爲線性時間代價的同時,
最壞情況下時間代價爲 \m{O(n\lg{n})}?
\stopEXERCISE

\startANSWER
如果所有元素全部落在同一桶中,並且是逆序的,則需要對這個桶中的 \m{n} 個逆序元素進行插入排序,
時間爲 \m{\Theta(n^2)}。

我們可以用堆排序或歸並排序來代替插入排序,以減少最壞情況的運行時間。
之所以選擇插入排序,是因爲他可以在鏈表上很好的工作。
如果用其他算法,則可能需要將鏈表改稱數組,這可能會降低算法的速度。
\stopANSWER

\startEXERCISE
設 \m{X} 是一個隨機變量,用於表示在將一枚硬幣拋擲兩次時,正面朝上的次數。
 \m{E[X^2]} 是多少? \m{E^2[X]} 是多少?
\stopEXERCISE

\startANSWER
\startformula
E[X] = 2 \cdot \frac{1}{4} + 1 \cdot \frac{1}{2} + 0 \cdot \frac{1}{4} = 1
\stopformula

\startformula
E[X^2] = E[X] \cdot E[X] = 1
\stopformula

\startformula
E^2[X] = E[X] \cdot E[X] = 1
\stopformula
\stopANSWER

\startEXERCISE\DIFFICULT
在單位圓內給定 \m{n} 個點, \m{p_i=(x_i,y_i)},
對所有 \m{i=1,2,\ldots,n},有 \m{0<x_i^2+y_i^2\le 1}。
假設所有點服從均勻分布,即在單位元的任一區域內找到給定點的概率與該區域的面積成正比。
請設計一個在平均情況下有 \m{\Theta(n)} 時間代價的算法,
他能夠按照點到原點之間的距離 \m{d_i=\sqrt{x_i^2+y_i^2}} 對這 \m{n} 個點進行排序。
\stopEXERCISE

\startANSWER
單位圓面積爲 \m{\pi 1^2=\pi},將其 \m{n} 等分,每塊面積爲 \m{\pi/n}。
如果圓環的內徑爲 \m{a},外徑爲 \m{b},則其面積爲 \m{\pi(b^2-a^2)}。

選取半徑序列 \m{a_0,a_1,a_2,\ldots,a_n} 將單位圓劃分成面積相等的 \m{n} 等份。
第 \m{i} 個圓環內外徑分別爲 \m{a_{i-1}} 和 \m{a_i}。
已知 \m{a_0=0} 且 \m{a_n=1}。則:
\startformula\startmathalignment[n=1]
\NC \pi^2 \pi(a_i^2 - a_{i-1}^2) = \pi(a_j^2 - a_{j-1}^2) = \frac{\pi}{n} \NR
\NC \Downarrow \NR
\NC a_i^2 - a_{i-1}^2 = a_j^2 - a_{j-1}^2 = \frac{1}{n} \NR
\NC \Downarrow \NR
\NC a_i^2 = \frac{1}{n} + a_{i-1}^2 \NR
\stopmathalignment\stopformula
得到如下遞迴式:
\startformula\startmathalignment
\NC a_0 \NC= 0 \NR
\NC a_i \NC= \sqrt{1/n + a_{i-1}^2} \NR
\stopmathalignment\stopformula
解得:
\startformula
a_i=\frac{\sqrt{i}}{\sqrt{n}}
\stopformula

因此,桶的半徑區間爲:
\startformula
\left(0, \frac{1}{\sqrt n} \right],
   \left(\frac{1}{\sqrt n}, \frac{\sqrt 2}{\sqrt n}\right], \cdots,
   \left(\frac{\sqrt{n-1}}{\sqrt n}, 1 \right]
\stopformula

半徑爲 \m{d} 的元素應該放入第 \m{k} 個桶中(從 1 開始計數), \m{k} 爲:
\startformula
k = \left\lceil d^2n \right\rceil
\stopformula
\stopANSWER

\startEXERCISE\DIFFICULT
定義隨機變量 \m{X} 的{\EMP 概率分布函數} \m{P(x)} 爲 \m{P(x)=\Pr\{X\le x\}}。
假設有 \m{n} 個隨機變量 \m{X_1,X_2,\ldots,X_n} 服從一個連續概率分布函數 \m{P},
且他可以在 \m{O(1)} 時間內被計算得到。
設計一個算法,使其能在平均情況下以線性時間對這些數進行排序。
\stopEXERCISE

\startANSWER
與上一練習類似,關鍵在於如何構造“桶”。
\stopANSWER

\stopsection
