\startsubject[
  title={Problems},
]

%p8-1
\startPROBLEM
(Probabilistic lower bounds on comparison sorting)
在這一問題中,我們將證明對於給定的 \m{n} 個互異的輸入元素,
任何確定或隨機的排序比較算法,其概率運行時間都有下界 \m{\Omega(n\lg{n})}。
首先來分析一個確定的比較排序算法 \m{A},其決策樹爲 \m{T_A}。
假設 \m{A} 的輸入的所有排列情況可能性相同。
\startigBase[a]
\startitem
假設 \m{T_A} 的每個葉子節點都標有在給定的隨機輸入情況下到達該節點的概率。
證明:恰有 \m{n!} 個葉子節點標有 \m{1/n!},其他的葉子節點標記爲 \m{0}。
\stopitem

\startANSWER
有 \m{n!} 種排序結果,每一種都對應 \m{n!} 種輸入中的一種。
每種排序結果都對應一個葉子節點,由於各種輸入的可能性相同,
到達具體排序結果的概率均爲 \m{1/n!}。
除了這 \m{n!} 種結果所對應的葉子節點,即使決策樹有其他有更多葉子節點,也無法到達。

當然以上僅憑直覺,正式的證明卻有點麻煩。
\stopANSWER

\startitem
定義 \m{D(T)} 表示一顆決策樹 \m{T} 的外部路徑長度,
即 \m{D(T)} 是 \m{T} 的所有葉子節點深度的和。
假設 \m{T} 爲一顆有 \m{k>1} 個葉子節點的決策樹,
 \m{LT} 和 \m{RT} 分別是 \m{T} 的左子樹和右子樹。
證明:
\startformula
D(T) = D(LT) + D(RT) + k。
\stopformula
\stopitem

\startANSWER
而子樹上的葉子節點在原樹上也是葉子節點。
原樹上的葉子節點也是子樹上的葉子節點(不是左子樹的葉子,就是右子樹的葉子)。
因此 \m{LEAVES(LT) + LEAVES(RT) = k}。
對於子樹上的每個葉子節點而言,在子樹上的深度比在原樹上的深度小 1。
因此:
\startformula
\m{D(T) = D(LT) + LEAVES(LT) + D(RT) + LEAVES(RT) = D(LT) + D(RT) + k}
\stopformula
\stopANSWER

\startitem
定義 \m{d(k)} 爲所有具有 \m{k>1} 個葉子節點的決策樹 \m{T} 的最小 \m{D(T)} 值。
證明:
\startformula
d(k) = \min_{1\le i\le {k-1}}\left\{d(i)+d(k-i)+k\right\}
\stopformula
(\hint 慮一顆能夠取得最小值的、有 \m{k} 個葉子節點的決策樹 \m{T}。
設 \m{i_0} 是 \m{LT} 中的葉子節點數, \m{k-i_0} 是 \m{RT} 中的葉子節點數。)
\stopitem

\startANSWER
對於一顆具有 \m{k} 個葉子節點的決策樹,由上一項可知 \m{D(T) = D(LT) + D(RT) + k}。
而左子樹上葉子節點有從 \m{1} 到 \m{k-1} 共 \m{k-1} 種可能。
由 \m{d(k)} 的定義即得所要證明的結論。
\stopANSWER

\startitem
證明: \m{d} 對於給定的 \m{k} 和 \m{i} (\m{k>1} 且 \m{1\le i\le {k-1}}),
函數 \m{i\lg{i}+(k-i)\lg{k-i}} 在 \m{i=k/2} 處取得最小值,
並有結論 \m{d(k)=\Omega(k\lg{k})}。
\stopitem

\startANSWER
\startformula\startmathalignment
\NC f(i) \NC= i\lg{i} + (k-i)\lg(k-i) \NR
\NC f'(i) \NC= \lg{i} + 1 - \lg(k-i) - 1 = \lg\frac{i}{k-i} \NR
\NC f'(i) = 0 \NC \Leftrightarrow \lg\frac{i}{k-i} = 0 \Rightarrow i/(k-i) = 1 \Rightarrow i = \frac k 2 \NR
\stopmathalignment\stopformula

由於 \m{f'(i)} 單調遞增,因此 \m{f(i)} 在 \m{i=k/2} 處取得局部最小值。

就直覺而言,應該是左右兩顆子樹大小相同時取得最小值。
\stopANSWER

\startitem
證明: \m{D(T_A)=\Omega(n!\lg(n!))},且在平均情況下,
對 \m{n} 個元素進行排序的時間代價爲 \m{\Omega(n\lg{n})}。
\stopitem

\startANSWER
在 \m{T_A} 上有 \m{n!} 個葉子節點,
由上一項可知 \m{D(n)>d(k)=\Omega(n!\lg(n!))}。
每種排列的概率都是 \m{1/n!},
因此排序的期望時間爲:
\startformula
\frac{\Omega(n!\lg(n!))}{n!} = \Omega(\lg(n!)) = \Omega(n\lg{n})
\stopformula
\stopANSWER
\stopigBase

現在考慮一個隨機化的比較排序 \m{B}。
通過引入兩種節點,
我們可以將決策樹模型擴展來處理隨機化的情況。
這兩種節點是:
普通哦年個的比較節點和“隨機化”節點。
隨機化節點刻劃了算法 \m{B} 種所做的形如 \ALGO{RANDOM(1, r)} 的隨機選擇情況。
該類節點有 \m{r} 個子節點,
在算法執行過程中,每一個子節點等概率地被選擇。
\startigBase[a,continue]
\startitem
證明:對任何隨機化比較排序算法 \m{B},
總存在一個確定的比較排序算法 \m{A},
其期望的比較次數不多於 \m{B} 的比較次數。
\stopitem

\startANSWER
隨機算法 \m{B} 所對應的確定算法 \m{A} 可以認爲只是進行了預先的“隨機”選擇。
我們可以通過將隨機節點替換爲我們所選的一個子節點,來構造確定算法的決策樹。
隨機節點的所有子樹的期望值要大於等於其中最小的那棵子樹。
由於確定算法時間代價爲 \m{\Omega(n\lg{n})},
因此隨機算法的時間代價亦爲 \m{\Omega(n\lg{n})}。
\stopANSWER
\stopigBase

\stopPROBLEM

\startPROBLEM
(Sorting in place in linear time)
假設有一個包含 \m{n} 個待排序數據記錄的數列,
且每條記錄的關鍵字值爲 \m{0} 或 \m{1}。
對這樣一組記錄進行排序的算法可能具備如下三中特性中的一部分:
\startigBase[n]
\item 算法的時間代價是 \m{O(n)};
\item 算法是穩定的;
\item 算法是原址排序,除了輸入數列之外,算法只需要固定的額外存儲空間。
\stopigBase

\startigBase[a]
\startitem
給出一個滿足上述條件 1 和條件 2 的算法。
\stopitem

\startANSWER
計數排序。
\stopANSWER

\startitem
給出一個滿足上述條件 1 和條件 3 的算法。
\stopitem

\startANSWER
改造計數排序,不求穩定,即可做到原址排序。
\stopANSWER

\startitem
給出一個滿足上述條件 2 和條件 3 的算法。
\stopitem

\startANSWER
冒泡排序。
\stopANSWER

\startitem
你設計的上面三個算法中是否可也將其中一個用於 \ALGO{RADIX-SORT} 的第 2 行作爲基礎排序方法,
從而使 \ALGO{RADIX-SORT} 在排序有 \m{b} 位關鍵字的 \m{n} 條記錄時的時間代價位 \m{O(bn)}?
如果可以,請解時應如何處理;如果不行,請說明原因。
\stopitem

\startANSWER
只有第一項的計數排序可以。基數排序要求穩定,另外基礎排序要求時間代價爲 \m{\Theta(n)}。
\stopANSWER

\startitem
假設有 \m{n} 條記錄,其中所有關鍵字的值都在 \m{1} 到 \m{k} 的區間內。
你應該如何修改計數排序,
使得他可以在 \m{O(n+k)} 的時間內完成對 \m{n} 條記錄的原址排序。
除輸入數列外,
你可以使用 \m{O(k)} 大小的額外存儲空間。
你給出的算法是穩定的嗎?
(\hint \m{k=3} 時,你應該如何做?)
\stopitem

\startANSWER
\startCLRS
for i = 0 upto range
	cnt[i] = counts[i]
while i ≤ A.length
	key = A[i]
	if cnt[key - 1] < i and cnt[key] <= i
		++i
	else
		exchange A[i] with A[counts[key]]
		--counts[key]
\stopCLRS
\stopANSWER
\stopigBase
\stopPROBLEM

\startPROBLEM
(Sorting variable-length items)
\startigBase[a]
\startitem
給定一個整數數列,其中不同的整數所包含的數字的位數可能不同,
但該數列中,所有整數中包含的總數字位數爲 \m{n}。
設計一個算法,使其可以在 \m{O(n)} 時間內對該數列進行排序。
\stopitem

\startANSWER
先將元素按數字位數分組,然後在組內進行基數排序。
令 \m{G_i} 表示具有 \m{i} 位的元素集合,令 \m{c_i=\left|G_i\right|},則:
\startformula
T(n) = \sum_{i=1}^{n}c_i\cdot i = n
\stopformula
\stopANSWER

\startitem
給定一個字串數列,其中不同的字串所包含的字符數可能不同,
但所有字串中的總字符數爲 \m{n}。
設計一個算法,使其可以在 \m{O(n)} 時間內對該數列進行排序。
({\EMP 注意:}此處的順序是指標準的字典序,例如 \m{a<ab<b}。)
\stopitem

\startANSWER
根據上一項的分析撰寫算法。
\stopANSWER
\stopigBase
\stopPROBLEM

\startPROBLEM
(Water jugs)
假設給了你 \m{n} 個紅色的水壺和 \m{n} 個藍色的水壺。
他們的形狀和尺寸都各不相同。
所有的紅色水壺中所盛的水都不一樣多,
藍色水壺亦是如此。
而且,對於每一個紅色水壺來說,
都有一個對應的藍色水壺,兩者盛有一樣多的水;反之亦然。

你的任務是找出所有的所盛水量一樣多的紅色水壺和藍色水壺,並將他們配成一對。
爲此,可以執行如下操作:
挑出一對水壺,其中一個是紅色的,一個是藍色的,
將紅色水壺中倒滿水,再將水倒入藍色的水壺中。
通過這一操作,可以判斷出這個紅色水壺是否比藍色水壺盛的水更多,或者兩者是一樣多的。
假設這樣的比較需要花費一個單位時間。
你的目標是找出一個算法,他能夠用最少的比較次數來確定所有水壺的配對。
注意,你不能直接比較兩個紅色或兩個藍色的水壺。

\startigBase[a]
\startitem
設計一個確定性算法,能用 \m{\Theta(n^2)} 次比較來完成所有水壺的配對。
\stopitem

\startANSWER
任取一紅色水壺,將其與所有藍色水壺比較,直至找到與之配對的藍色水壺。
待爲所有紅色水壺都找到了與之配對的藍色水壺,算法終止。
\stopANSWER

\startitem
證明:解決該問題算法的比較次數下界爲 \m{\Omega(n\lg{n})}。
\stopitem

\startANSWER
参考\refsection{lower_bound_sorting}。
只是决策树上的每个节点有三个子节点,即是{\EMP 三叉树}。
如果两种水壶各自都是有序的,则一种配对即一个 \m{n} 项排列,将红蓝水壶进行映射。
由于有 \m{n!} 种结果,每种结果都是决策树中的一个叶子节点。
假如树中共有 \m{l} 个节点:
\startformula
n! \le l \le 3^h
\stopformula
则:
\startformula
h \ge \lg(n!) = \Omega(n\lg{n})
\stopformula
\stopANSWER

\startitem
設計一個隨機演算法,其期望的比較次數為 \m{O(n\lg{n})},
並證明這個界是正確的。
對你的演算法而言,最壞情況下的比較次數是多少?
\stopitem

\startANSWER
可以時用與快速配序類似的方式。以如下步驟進行劃分:
\startigBase[n]
\item 隨機選擇一個紅色水壺作為紅色主元;\m{O(1)}
\item 選擇紅色主元所對應的藍色水壺作為藍色主元;\m{O(n)}
\item 以紅色主元劃分藍色水壺,並將藍色主元歸位;\m{O(n)}
\item 以藍色主元劃分紅色水壺,並將紅色主元歸位。\m{O(n)}
\stopigBase

只在紅藍水壺間進行比較。
\stopANSWER
\stopigBase
\stopPROBLEM

\startPROBLEM
(Average sorting)
假設我們不需要完全排序一個數列,
而值是要求數列中的元素在平均情況下是升序的。
更準確地說,如果對所有的 \m{i=1,2,\ldots,n-k} 有下式成立,
我們就稱一個包含 \m{n} 個元素的數列 \m{A} 是 {\EMP \m{k} 排序}(\m{k}-sorted)的:
\startformula
\frac{\sum_{j=i}^{i+k-1}A[j]}{k} \le \frac{\sum_{j=i + 1}^{i+k}A[j]}{k}
\stopformula

\startigBase[a]
\startitem
一個數列是 1 排序的,表示什麼含義?
\stopitem

\startANSWER
將 \m{k=1} 代入可得: \m{A[j] \le A[j+1]},其中 \m{1 \le j \le n-1}。即整個數列是完全升序的。
\stopANSWER

\startitem
給出對數字 \m{1,2,\ldots,10} 的一個排列,他是 2 排序的,但不是完全有序的。
\stopitem

\startANSWER
\m{2, 1, 4, 3, 6, 5, 8, 7, 10, 9}。
\stopANSWER

\startitem
證明:一個包含 \m{n} 個元素的數列是 \m{k} 排序的,
當且僅當對所有 \m{i=1,2,\ldots,n-k},有 \m{A[i]\le A[i+k]}。
\stopitem

\startANSWER
\startformula\startmathalignment[n=1]
\NC \frac{\sum_{j=i}^{i+k-1}A[j]}{k} \le \frac{\sum_{j=i + 1}^{i+k}A[j]}{k} \NR
\NC \Updownarrow \NR
\NC \frac{A[i] + \sum_{j=i+1}^{i+k-1}A[j]}{k} \le
     \frac{\sum_{j=i+1}^{i+k-1}A[j] + A[i+k]}{k} \NR
\NC \Updownarrow \NR
\NC \frac{A[i]}{k} \le \frac{A[i+k]}{k} \NR
\NC \Updownarrow \NR
\NC A[i] \le A[i+k] \NR
\stopmathalignment\stopformula
\stopANSWER

\startitem
設計一個算法,他能在 \m{O(n\lg(n/k))} 時間內對一個包含 \m{n} 個元素的數列進行 \m{k} 排序。
\stopitem

\startANSWER
將數列劃分成 \m{k} 個子數列,記爲 \m{A_s},其中 \m{s=1,2,\ldots,k}。

而 \m{A_s[i] = A[s + i \times k]},其中 \m{i=0,1,2,\ldots,n/k-1}。

只需將這 \m{k} 個子數列排序即可。每個子數列元素個數爲 \m{n/k},
總時間爲
\startformula
O((n/k)\lg(n/k)) \times k = O(n\lg(n/k))
\stopformula
\stopANSWER
\stopigBase

當 \m{k} 是一個常數時,也可以給出 \m{k} 排序算法的下界。
\startigBase[a,continue]
\startitem
證明:我們可以在 \m{O(n\lg{k})} 時間內對一個長度爲 \m{n} 的 \m{k} 排序數列進行全排序。
\hint 以利用\refexercise{k_merge} 的結果。)
\stopitem

\startANSWER
即總元素個樹爲 \m{n} 的 \m{k} 個有序數列,進行歸並排序,參見\refexercise{k_merge}。
\stopANSWER

\startitem
證明:當 \m{k} 是一個常數時,對包含 \m{n} 個元素的數列進行 \m{k} 排序需要 \m{\Omega(n\lg{n})} 的時間。
(\hint 以利用千米解決比較排序的下界的方法。)
\stopitem

\startANSWER
還是解決 \m{k} 個規模爲 \m{n/k} 的子問題,每個子問題最少比較次數爲 \m{\Omega((n/k)\lg(n/k))},
總共 \m{\Omega(n\lg(n/k))}。
\stopANSWER
\stopigBase
\stopPROBLEM

%p8-6
\startPROBLEM
(Lower bound on merging sorted lists)
我們經常遇到合並兩個有序列表的問題。
作爲 \ALGO{MERGE-SORT} 的子過程,
我們在節 2.3.1 中已經遇到過這一問題。
對這一問題,我們將證明在最壞情況下,
合並兩個都包含 \m{n} 個元素的有序列表所需的比較次數下界爲 \m{2n-1}。

首先,利用決策樹來說明比較次數有一個下界 \m{2n-o(n)}。

\startigBase[a]
\startitem
給定 \m{2n} 個數,請算出共有多少種可能的方式將他們劃分成兩個有序列表,
其中每個列表都包含 \m{n} 個數。
\stopitem

\startANSWER
\startformula
\binom{2n}{n}
\stopformula
\stopANSWER

\startitem
利用決策樹和上一項的結果,正面:
任何能夠正確合併兩個有序列表的算法都至少要進行 \m{2n-o(n)} 次比較。
\stopitem

\startANSWER
由\refexercise{c_2n_n} 可知:
\startformula
\binom{2n}{n} = \frac{2^{2n}}{\sqrt{\pi n}}(1 + O(1/n))
\stopformula

如果決策樹中有 \m{l} 個葉子節點,且高度爲 \m{h},則:
\startformula
\frac{2^{2n}}{\sqrt{\pi n}}(1 + O(1/n)) \le l \le 2^h
\stopformula

不等式兩端取對數:
\startformula\startmathalignment
\NC h \NC \ge \lg\left(\frac{2^{2n}}{\sqrt{\pi n}}(1 + O(1/n))\right) \NR
\NC   \NC = \lg{2^{2n}} - \lg\sqrt{\pi n} + \lg(1 + O(1/n)) \NR
\NC   \NC = 2n - O(n) \NR
\stopmathalignment\stopformula
\stopANSWER
\stopigBase

現在我們給出一個更緊確的界 \m{2n-1}。
\startigBase[a,continue]
\startitem
請說明:如果兩個元素在有序序列中是連續的,且分別來自不同的列表,則他們必須進行比較。
\stopitem

\startANSWER
如果不進行比較,我們不知道這兩個元素誰先誰後。
當然如果這兩個元素來自同一個列表,則無需比較。
\stopANSWER

\startitem
利用上一項的結論,說明合併兩個有序列表時的比較次數下界爲 \m{2n-1}。
\stopitem

\startANSWER
如果兩個有序列表 \m{\langle a_1,a_2,\ldots,a_n\rangle} 和 \m{\langle b_1,b_2,\ldots,b_n} 合併後
爲 \m{\langle a_1,b_1,a_2,b_2,\ldots,a_n,b_n},
則任意相鄰兩個元素均須比較,共需比較次數爲 \m{2n-1}。
這是最壞情況。
\stopANSWER
\stopigBase
\stopPROBLEM

%p8-7
\startPROBLEM
(The 0-1 sorting lemma and columnsort)
對兩個數列元素 \m{A[i]} 和 \m{A[j]} (其中 \m{i < j})進行{\EMP 比較交換}({\EMP compare-exchange})的形式如下:

\CLRSH{COMPARE-EXCHANGE(A, i, j)}
\startCLRS
if A[i] > A[j]
	exchange A[i] with A[j]
\stopCLRS

比較交換後,則 \m{A[i] \le A[j]}。

{\EMP 遺忘比較交換算法(oblivious compare-exchange algorithm)}指算法只按照事先
定義好的操作執行,
即需要比較的位置下標必須事先確定好。
雖然算法可能依靠待排序元素個數,但他不能依賴待排序元素的值,
也不能依賴任何之前的比較交換操作的結果。
例如,下面是一個基於遺忘比較交換算法的插入排序:

\CLRSH{INSERTION-SORT(A)}
\startCLRS
for j = 2 to A.length
	for i = j - 1 downto 1
		COMPARE-EXCHANGE(A, i, i + 1)
\stopCLRS

{\EMP 0-1 排序引理(0-1 sorting lemma)}提供了有力的方法來證明
一個遺忘比較交換算法可以產生正確的排序結果。
該引理表明,如果一個遺忘比較交換算法能夠對所有只包含 0 和 1 的輸入序列排序,
那麼他就可以對包含任意值的輸入序列排序。

你可以用反例來證明 0-1 排序引理:
如果一個遺忘比較交換算法不能對一個包含任意值的序列進行排序,
那麼他也不能對某個 0-1 序列進行排序。
不妨假設一個遺忘比較交換算法 \m{X} 未能對數列 \m{A[1..n]} 排序。
設 \m{A[p]} 是算法 \m{X} 未能將其放到正確位置上的最小元素,
而 \m{A[q]} 是被算法 \m{X} 放在 \m{A[p]} 原本應該所在位置上的元素。
定義一個只包含 0 和 1 的數列 \m{B[1..n]} 如下:
\startformula
B[i] = \startmathcases
\NC 0	\MC \text{若 \m{A[i]\le A[p]},} \NR
\NC 1	\MC \text{若 \m{A[i] > A[p]}。} \NR
\stopmathcases
\stopformula

\startigBase[a]
\startitem
討論:當 \m{A[q]>A[p]} 時,有 \m{B[p]=0} 和 \m{B[q]=1}。
\stopitem

\startANSWER
\m{A[p]} 和 \m{A[q]} 所放位置均不正確,且 \m{A[p]} 是位置不正確的最小元素,
因此 \m{A[q] > A[p]}。將 \m{i=p,q} 代入即得 \m{B[p]=0} 和 \m{B[q]=1}。
\stopANSWER

\startitem
爲完成 0-1 排序引理的正面,請先證明算法 \m{X} 不能對數列 \m{B} 正確排序。
\stopitem

\startANSWER
參見 \simpleurl{http://www.iti.fh-flensburg.de/lang/algorithmen/sortieren/networks/nulleinsen.htm}。
\stopANSWER
\stopigBase

現在,需要用 0-1 排序引理來正面一個特別的排序算法的正確性。
{\EMP 列排序}算法用於包含 \m{n} 個元素的矩形數列的排序。
這一矩形數列有 \m{r} 行 \m{s} 列(因此 \m{n=rs}),
滿足下列三個限制條件:
\startigNum[2]
\item \m{r} 必須是偶數;
\item \m{s} 必須是 \m{r} 的因子;
\item \m{r\ge 2s^2};
\stopigBase
當排序完成時,矩形數列是{\EMP 列優先有序}的:
按照列從上到下,從左到右,都是單調遞增的。

如果不包括 \m{n} 的值的計算,列排序需要 8 步操作。
所有奇數步都一樣:對每一列單獨進行排序。
每一個偶數步是一個特定的排列。具體如下:
\startigNum[n]
\item 對每一列進行排序;
\item 將矩形數列轉置,並重新規整化爲 \m{r} 行 \m{s} 列的形式。
也就是說,首先將最左邊的一列放在前 \m{r/s} 行,
然後將下一列放在第二個 \m{r/s} 行,以此類推;
\item 對每一列進行排序;
\item 執行第 2 步排列操作的逆操作;
\item 對每一列進行排序;
\item 將每一列的上半部分移到同一列的下半部分位置,
將每一列的下半部分移到下一列的上半部分,
並將最左邊一列的上半部分置空。
此時,最後一列的下半部分成爲新的最右列的上半部分,新的最右列的下半部分爲空;
\item 對每一列進行排序;
\item 執行第 6 步排列操作的逆操作。
\stopigNum

下面爲對一個 \m{r=6,s=3} 的輸入矩形數列的排序過程(即使違背了 \m{r\ge 2s^2} 的條件,算法仍然有效)。
\midaligned{
\startcombination[nx=5,ny=2, distance=3em, align=middle]
{
\startformula\startmatrix
\NC 10 \NC 14 \NC  5 \NR
\NC  8 \NC  7 \NC 17 \NR
\NC 12 \NC  1 \NC  6 \NR
\NC 16 \NC  9 \NC 11 \NR
\NC  4 \NC 15 \NC  2 \NR
\NC 18 \NC  3 \NC 13 \NR
\stopmatrix\stopformula
}{(a)}{
\startformula\startmatrix
\NC  4 \NC  1 \NC  2 \NR
\NC  8 \NC  3 \NC  5 \NR
\NC 10 \NC  7 \NC  6 \NR
\NC 12 \NC  9 \NC 11 \NR
\NC 16 \NC 14 \NC 13 \NR
\NC 18 \NC 15 \NC 17 \NR
\stopmatrix\stopformula
}{(b)}{
\startformula\startmatrix
\NC  4 \NC  8 \NC 10 \NR
\NC 12 \NC 16 \NC 18 \NR
\NC  1 \NC  3 \NC  7 \NR
\NC  9 \NC 14 \NC 15 \NR
\NC  2 \NC  5 \NC  6 \NR
\NC 11 \NC 13 \NC 17 \NR
\stopmatrix\stopformula
}{(c)}{
\startformula\startmatrix
\NC  1 \NC  3 \NC  6 \NR
\NC  2 \NC  5 \NC  7 \NR
\NC  4 \NC  8 \NC 10 \NR
\NC  9 \NC 13 \NC 15 \NR
\NC 11 \NC 14 \NC 17 \NR
\NC 12 \NC 16 \NC 18 \NR
\stopmatrix\stopformula
}{(d)}{
\startformula\startmatrix
\NC  1 \NC  4 \NC 11 \NR
\NC  3 \NC  8 \NC 14 \NR
\NC  6 \NC 10 \NC 17 \NR
\NC  2 \NC  9 \NC 12 \NR
\NC  5 \NC 13 \NC 16 \NR
\NC  7 \NC 15 \NC 18 \NR
\stopmatrix\stopformula
}{(e)}{
\startformula\startmatrix
\NC  1 \NC  4 \NC 11 \NR
\NC  2 \NC  8 \NC 12 \NR
\NC  3 \NC  9 \NC 14 \NR
\NC  5 \NC 10 \NC 16 \NR
\NC  6 \NC 13 \NC 17 \NR
\NC  7 \NC 15 \NC 18 \NR
\stopmatrix\stopformula
}{(f)}{
\startformula\startmatrix
\NC    \NC  5 \NC 10 \NC 16 \NR
\NC    \NC  6 \NC 13 \NC 17 \NR
\NC    \NC  7 \NC 15 \NC 18 \NR
\NC  1 \NC  4 \NC 11 \NC    \NR
\NC  2 \NC  8 \NC 12 \NC    \NR
\NC  3 \NC  9 \NC 14 \NC    \NR
\stopmatrix\stopformula
}{(g)}{
\startformula\startmatrix
\NC    \NC  4 \NC 10 \NC 16 \NR
\NC    \NC  5 \NC 11 \NC 17 \NR
\NC    \NC  6 \NC 12 \NC 18 \NR
\NC  1 \NC  7 \NC 13 \NC    \NR
\NC  2 \NC  8 \NC 14 \NC    \NR
\NC  3 \NC  9 \NC 15 \NC    \NR
\stopmatrix\stopformula
}{(h)}{
\startformula\startmatrix
\NC  1 \NC  7 \NC 13 \NR
\NC  2 \NC  8 \NC 14 \NR
\NC  3 \NC  9 \NC 15 \NR
\NC  4 \NC 10 \NC 16 \NR
\NC  5 \NC 11 \NC 17 \NR
\NC  6 \NC 12 \NC 18 \NR
\stopmatrix\stopformula
}{(i)}
\stopcombination}

\startigBase[continue]
\startitem
討論:即使不知道奇數步採用了什麼排序算法,
我們也可以把列排序看作一種遺忘比較交換算法。
\stopitem
\stopigBase

雖然似乎很難讓人相信列排序也能實現排序,
但是你可以利用 0-1 排序引理來證明這一點。
因爲列排序可以看作是一種遺忘比較交換算法,
所以我們可以使用 0-1 排序引理。
下面一些定義有助於你試用這一引理。
如果數列仲的某個區域只包含全 0 或者全 1,
我們定義這個區域是{\EMP 乾淨}的。
否則稱這個區域是{\EMP 髒}的。
這裏,假設輸入數據只包含 0 和 1,
且輸入數據能轉換成 \m{r} 行 \m{s} 列。
\startigBase[continue]
\startitem
證明:經過第 1~3 布,數列由三部分組成:
頂部一些由全 0 組成的乾淨行,底部一些由全 1 組成的乾淨行,
以及中間最多 \m{s} 行髒的行。
\stopitem

\startANSWER
第一步後,每一列均爲連續 0 後緊跟連續 1,只有一處 0 到 1 的跳變。
由於 \m{s} 整除 \m{r},第二步會將第一步的每一列轉換爲 \m{r/s} 行。
在這 \m{r/s} 行中最多只有一行有 0 到 1 的跳變,其他行都是 0 或者都是 1。
即每一列所轉換的行中最多有一行是髒的。

第三步會將所有 0 行上移,所有 1 行下移,即最多只會剩下 \m{s} 行髒的行。
\stopANSWER

\startitem
證明:經過第 4 步後,如果按照列有限原則讀取數列,
先讀到的是全 0 的乾淨區域,最後是全 1 的乾淨區域,
中間是由最多 \m{s^2} 個元素組成的髒區域。
\stopitem

\startANSWER
第 4 步後按列優先讀數據,相當於在第 3 步後按行優先讀數據,參考上一項的證明,結論是顯然的。
\stopANSWER

\startitem
證明:第 5~8 步產生一個全排序的 0-1 輸出,
並得到結論:列排序可以正確地對任意輸入值排序。
\stopitem

\startANSWER
第四步後, 髒序列中最多有 \m{s^2} 個元素,爲一列元素的一半,
可能在一列中,也可能分散於相鄰兩列中。左邊的左右列爲 0,右邊的所有列爲 1。
如果髒序列在同一列中,則第 5 步即可保證每列有序,後續步驟不受影響。
如果髒序列分散於相鄰兩列中,則第 6 步會將髒序列轉移到同一列中。
\stopANSWER

\startitem
現在假設 \m{s} 不能被 \m{r} 整除。
證明:經過第 1~3 步,數列的頂部有一些全 0 的乾淨行,
底部有一些全 1 的乾淨行,中間最多 \m{2s-1} 行髒行。
那麼與 \m{s} 相比,在 \m{s} 不能被 \m{r} 整除時, \m{r} 至少要多大才能保證配序的正確性?
\stopitem

\startANSWER
如果 \m{s} 不能整除 \m{r},一行可能不僅包含 0 到 1 的跳變,也包含 1 到 0 的跳變。
最多有 \m{s - 1}  行,即最多有 \m{2s-1} 個髒區域。
 \m{r} 至少要是 \m{2(2s-1)^2}。
\stopANSWER

\startitem
對第 1 步做一個簡單修改,使得我們可以在 \m{s} 不能被 \m{r} 整除時,也保證 \m{r\ge 2s^2},
並且證明在這一修改後,列排序仍然正確。
\stopitem

\startANSWER
第一步中,可以用 \m{+\infty} 補齊數列使得 \m{s} 可以整除 \m{r},
或者可以將數列截斷一小部分,對其獨立排序。後者會更有效,因爲無需移動數列。
其實這些均無必要,列排序對是否整除沒有嚴格要求,
細節參見\simpleurl{http://www.cs.dartmouth.edu/reports/TR2003-444.pdf}。
\stopANSWER
\stopigBase

\stopPROBLEM

\stopsubject
