\startsection[
  title={Dynamic order statistics},
]

%e14.1-1
\startEXERCISE
對於圖 14-1 中的紅黑樹 T,說明執行 \ALGO{OS-SELECT(T.root, 10)} 的過程。
\stopEXERCISE
\startANSWER
略。
\stopANSWER

%e14.1-2
\startEXERCISE
對於圖 14-1 中的紅黑樹 T 和關鍵字 \m{x.key} 爲 35 的節點 x,說明執行 \ALGO{OS-RANK(T, x)} 的過程。
\stopEXERCISE
\startANSWER
略。
\stopANSWER

%e14.1-3
\startEXERCISE
寫出 \ALGO{OS-SELECT} 的非遞迴版本。
\stopEXERCISE
\startANSWER
\CLRSH{OS-SELECT(x, i)}
\startCLRS
root = x
idx = i
while root != NIL
	r = root.left.size + 1
	if idx == r
		return x
	elseif idx < r
		root = root.left
	else
		root = root.right
		idx = idx - r
\stopCLRS
\stopANSWER

%14.1-4
\startEXERCISE
寫出一個遞迴過程 \ALGO{OS-KEY-RANK(T, k)},
以一棵順序統計樹 T 和一個關鍵字 k 作爲輸入,
要求返回 k 在由 T 表示的動態集合中的秩。
假設 T 的所有關鍵字都不相同。
\stopEXERCISE
\startANSWER
\CLRSH{OS-KEY-RANK(T, k)}
\startCLRS
if T == NIL
	return 0
if k == T.key
	if T.left == NIL
		return 1
	else
		return 1 + T.left.size
elseif k < T.key
	return OS-KEY-RANK(T.left, k)
else
	ret = OS-KEY-RANK(T.right, k)
	if T.left == NIL
		return ret + 1
	else
		return ret + T.left.size
\stopCLRS
\stopANSWER

%e14.1-5
\startEXERCISE
給定一個元素 x 和一個自然數 i,
其中 x 屬於一個含有 n 個元素的順序統計樹,
如何在 \m{O(\lg n)} 的時間內確定 x 在該樹線性序中的第 i 個後繼?
\stopEXERCISE
\startANSWER
先計算 x 的秩,記爲 r,然後在此樹中搜索秩爲 \m{r+i} 的節點。
\stopANSWER

%e14.1-6
\startEXERCISE
在 \ALGO{OS-SELECT} 或 \ALGO{OS-RANK} 中,
注意到無論什麼時候引用節點的 size 屬性都是爲了計算秩。
相應地,假設每個節點都存儲他在以自己爲根的子樹中的秩。
試說明在插入和刪除時,如何維護這個信息。
(注意,這兩種操作都可能引起旋轉)
\stopEXERCISE
\startANSWER
略。
\stopANSWER

%e14.1-7
\startEXERCISE
說明如何在 \m{O(n\lg n)} 時間內,
利用順序統計樹對大小爲 n 的數列中的逆序對(參見問題 2-4)進行計數。
\stopEXERCISE
\startANSWER
\TODO{}
\stopANSWER

%e14.1-8
\startEXERCISE
現有一個圓上的 n 條弦,每條弦都由其端點來定義。
請給出一個能在 \m{O(n\lg n)} 時間內確定圓內相交弦對數的算法。
(例如,如果 n 條弦都爲直徑,他們相交於圓心,
則正確的答案爲 \m{\binom{n}{2}})
假設任意兩條弦都不會共享端點。
\stopEXERCISE
\startANSWER
\hint 逆序對。
\stopANSWER

\stopsection
