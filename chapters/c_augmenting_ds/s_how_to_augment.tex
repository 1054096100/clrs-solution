\startsection[
  title={How to augment a data structure},
]

%14.2-1
\startEXERCISE[exercise:add_pointer_for_node]
通過爲節點增加指針的方式,試說明如何在擴張的順序統計樹上,
動態集合的查詢操作 \ALGO{MINIMUM}、 \ALGO{MAXIMUM}、 \ALGO{SUCCESSOR} 和 \ALGO{PREDECESSOR} 都能在
最壞時間 \m{O(1)} 內完成。
順序統計樹上的其他操作漸進性能不應受影響。
\stopEXERCISE
\startANSWER
在每個節點中加入兩個指針數據項,分別指向左子樹中最小的元素和右子樹中最大的元素。
\startformula\startmathalignment
\NC MINIMUM(x) \NC = x.min\NR
\NC MAXIMUM(x) \NC = x.max\NR
\NC SUCCESSOR(x) \NC = x.right.min\NR
\NC PREDECESSOR(x) \NC = x.left.max\NR
\stopmathalignment\stopformula
\stopANSWER

%e14.2-2
\startEXERCISE
能否在不影響紅黑樹任何操作漸進性能的前提下,
將節點的黑高作爲樹中節點的一個屬性來維護?
說明如何做,如果不能請說明理由。
如何維護節點的深度?
\stopEXERCISE
\startANSWER
可以,在節點中添加屬性來記錄黑高。
每個節點的黑高可以根據自身顏色和兩個子節點的黑高來確定。
根據定理 14.1,插入、刪除仍然是 \m{O(\lg n)}。

深度則不行,因爲節點的深度由其父節點的深度決定。
當一個節點的深度發生變化後,最壞情況下需要的時間可能要比 \m{O(n\lg n)} 多。
\stopANSWER

%e14.2-3
\startEXERCISE
設 \m{\otimes} 爲一個滿足結合律的二元運算符,
 \m{a} 爲紅黑樹中每個節點上的一個要維護的屬性。
假設在每個節點 \m{x} 上增加一個屬性 \m{f},
使得 \m{x.f=x_1\cdot a \otimes x_2\cdot a \otimes \cdots \otimes x_m\cdot a},
其中 \m{x_1,x_2,\cdots,x_m} 是以 \m{x} 爲根的子樹中按中序次序排列的所有節點。
說明在一次旋轉後,如何在 \m{O(1)} 時間內更新屬性 \m{f}。
對你的擴張稍作修改,使得他能夠應用到順序統計樹的屬性 \m{size} 中。
\stopEXERCISE
\startANSWER
根據結合律可知: \m{x.f = x.left.f \otimes x \cdot a \otimes x.right.f}。

考慮下面的左旋操作:

\startcombination[2*1]
{\externalfigure[output/e14_2_3-1]}{a}
{\externalfigure[output/e14_2_3-2]}{b}
\stopcombination

可得:
\startformula\startmathalignment
\NC x'.f \NC = \alpha.f \otimes x \cdot a \otimes \beta.f \NR
\NC y'.f \NC = x.f \NR
\stopmathalignment\stopformula
\stopANSWER

%e14.2-4
\startEXERCISE
希望設計一個操作 \ALGO{RB-ENUMERATE(x, a, b)},來擴張紅黑樹。
該操作輸出所有的關鍵字 \m{k},
使得在以 \m{x} 爲根的紅黑樹中有 \m{a\le k \le b}。
描述如何在 \m{\Theta(m+\lg n)} 時間內實現 \ALGO{RB-ENUMERATE},
其中 \m{m} 爲輸出的關鍵字數目,
 \m{n} 爲樹中的內部節點數。
(\hint 不需要給紅黑樹增加新屬性)
\stopEXERCISE
\startANSWER
先調用 \ALGO{TREE-SEARCH(x, a)},對其稍作修改,使其返回大於等於 a 的最小元素,
如果沒有就返回空。此步需要時間 \m{O(\lg n)}。
然後不停調用 \ALGO{TREE-SUCCESSOR(x)},直到遇到比 b 大的節點。共需 m 次。
根據\refexercise{cont_successor},這 m 次調用需要時間爲 \m{O(m+h)}。
因此總時間爲 \m{\Theta(m + \lg n)}。
我們也可以根據\refexercise{add_pointer_for_node} 使用指向前驅、後繼節點的指針,但總的時間還是 \m{O(m+\lg n)}。
\stopANSWER

\stopsection
