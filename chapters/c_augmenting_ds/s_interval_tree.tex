\startsection[
  title={Interval trees},
]

%e14.3-1
\startEXERCISE
寫出用於左旋區間樹的 \ALGO{LEFT-ROTATE} 的僞碼,使其可以在 \m{O(1)} 時間內更新 \m{\max} 屬性。
\stopEXERCISE
\startANSWER
\TODO{略。}
\stopANSWER

%e14.3-2
\startEXERCISE
改寫 \ALGO{INTERVAL-SEARCH},使其在所有區間均爲開區間的情況下也能正確工作。
\stopEXERCISE
\startANSWER
\CLRSH{INTERVAL-SEARCH(T, i)}
\startCLRS
x = T.root
while x != T.nil and i does not overlap x.int
	if x.left != T.nil and x.left.max > i.low
		x = x.left
	else x = x.right
return x
\stopCLRS
\stopANSWER

%e14.3-3
\startEXERCISE
請給出一個有效算法,對一個給定區間 i,返回一個與 i 重疊,且具有最小左端點的區間;
如果不存在在這樣的區間,則返回 T.nil。
\stopEXERCISE
\startANSWER
\CLRSH{MIN_INTERVAL_SEARCH(x, i)}
\startCLRS
if x == NIL
	return NIL
if x.left != NIL and x.left.max >= i.low
	ret = MIN_INTERVAL_SEARCH(x.left, i)
	if ret != NIL
		return ret
if x.int overlap i
	return x
return MIN_INTERVAL_SEARCH(x.right, i)
\stopCLRS
\stopANSWER

%e14.3-4
\startEXERCISE
給定一棵區間樹 T 和一個區間 i,
請描述如何在 \m{O(\min(n, k\lg n))} 時間內列出 T 中所有與 i 重疊的區間,
其中 k 爲輸出的區間數。
(\hint 一種簡單的方法是做若干次查詢,並且在這些查詢操作中修改樹,
另一種略爲複雜點的方法是不對樹進行修改。)
\stopEXERCISE
\startANSWER
\CLRSH{INTERVAL-OVERLAP-LIST(T, x, i)}
\startCLRS
if x.int overlap i
	print x
if x.left != T.nil and x.left.max >= i.low
	INTERVAL-OVERLAP-LIST(T, x.left, i)
if x.right != T.nil and x.low <= i.high && x.right.max >= i.low
	INTERVAL-OVERLAP-LIST(T, x.right, i)
\stopCLRS
\stopANSWER

%e14.3-5
\startEXERCISE
對區間樹 T 和一個區間 i,
請修改有關區間樹的過程來支持新的操作 \ALGO{INTERVAL-SEARCH-EXACTLY(T, i)},
他返回一個指向 T 中節點 x 的指針,使得 \m{x.int.low = i.low} 且 \m{x.int.high = i.high};
或者,如果 T 不包含這樣的區間時返回 T.nil。
所有的操作(包括 \ALGO{INTERVAL-SEARCH-EXACTLY})對於包含 n 個節點的區間樹的運行時間都應爲 \m{O(\lg n)}。
\stopEXERCISE
\startANSWER
先調用 \ALGO{TREE-SEARCH} 找到 x,滿足 \m{x.int.low = i.low},
然後檢查 \m{x.int.hight = i.high} 是否成立,成立則返回 x,否則返回 T.nil。
顯然所需時間爲 \m{O(h)} 即 \m{O(\lg n)}。
\stopANSWER

%e14.3-6
\startEXERCISE
說明如何來維護一個支持操作 \ALGO{MIN-GAP} 的一些數的動態集 Q,
使得該操作能給出 Q 中兩個最接近的數之間的差值。
例如, \m{Q=\{1,5,9,15,18,22\}},則 \ALGO{MIN-GAP} 返回 \m{18-15=3},
因爲 15 和 18 是 Q 中兩個最接近的數。
要使得操作 \ALGO{INSERT}、 \ALGO{DELETE}、 \ALGO{SEARCH} 和 \ALGO{MIN-GAP} 儘可能高效,
並分析他們的運行時間。
\stopEXERCISE
\startANSWER
給紅黑樹節點增加屬性: min-gap,min,max。葉子節點的 min-gap 爲 \m{\infty}。
這三個屬性均可由節點自身和其左右孩子節點的信息算出來。
根據定理 4.1,不會影響插入、刪除等操作的運行時間。

\ALGO{MIN-GAP} 的運行時間爲 \m{O(1)}。
\stopANSWER

%e14.3-7
\startEXERCISE\DIFFICULT
VLSI 數據庫通常將一塊集成電路表示成一組矩形,
假設每個矩形的邊都平行於 x 軸或者 y 軸,
這樣可以用矩形的最小、最大 x 軸 y 軸座標來表示一個矩形。
請給出一個時間爲 \m{O(n\lg n)} 的算法,
來確定 n 個這種表示的矩形集合中是否存在兩個重疊的矩形。
你的算法不一定要輸出所有重疊的矩形,
但對於一個矩形完全覆蓋另一個的情況要給出正確的判斷(邊界線不一定相交)。
(\hint 移動一條“掃描”線,穿過所有的矩形。)
\stopEXERCISE
\startANSWER
從上至下移動掃描線。每次遇到一個矩形,假定是其最小的 y 座標,
我們需要檢查其 x 座標區間是否與樹中的任何矩形重疊。
如果沒有找到,則將此矩形插入樹中(以 x 座標區間爲關鍵字)。

對於區間樹的操作(插入、刪除、檢查是否重疊),其運行時間爲 \m{O(\lg n)}。
實際上還需要 \m{O(n\lg n)} 時間將矩形按 y 座標排序(用於掃描)。
\stopANSWER

\stopsection
