\startsection[
  title={Determining whether any pair of segments intersects},
]

%e33.2-1
\startEXERCISE
試說明在 \m{n} 條線段的集合中,可能有 \m{\Theta(n^2)} 個交點。
\stopEXERCISE

\startANSWER
五角星,每一條線段都與其他所有線段相交。
\stopANSWER

%e33.2-2
\startEXERCISE[33.2-2]
已知兩條在 \m{x} 處可比的線段 \m{a} 和 \m{b},
試說明如何在 \m{O(1)} 時間內確定 \m{a \succeq_x b} 和 \m{b\succeq_x a} 中哪一個成立。
假定這兩條線段都不是垂直的。
(\hint 如果 \m{a} 和 \m{b} 不相交,利用叉積即可。
如果 \m{a} 和 \m{b} 相交(當然頁可以用叉積來確定),
仍然可以只利用加、減、乘這幾種運算,無需使用除法。
當然,在應用 \m{\succeq_x} 關係時,如果 \m{a} 和 \m{b} 相交,
就可以停下來並聲明已找到了一個交點。)
\stopEXERCISE

\startANSWER
令 \m{a = \overbar{p_1 p_2}, b=\overbar{p_3,p_4}},
其中 \m{p_i=(x_i,y_i),1\le i\le 4},
 \m{x_1 < x_2, x_3 < x_4}。
即 \m{a} 左端點爲 \m{p_1},右端點爲 \m{p_2};
 \m{b} 左端點爲 \m{p_3},右端點爲 \m{p_4}。

如果 \m{a} 和 \m{b} 不相交,考慮兩條線段 \m{\vec{p_1 p_2}} 和 \m{\vec{p_2 p_3}},
如果在 \m{p_2} 處左轉,則 \m{b\succeq_x a};
如果在 \m{p_2} 處右轉,則 \m{a\succeq_x b}。

如果 \m{a} 和 \m{b} 交點位於 \m{x} 左邊,
那麼可以根據右端點判斷 \m{a} 和 \m{b} 的關係;
如果交點位於 \m{x} 右邊,
那麼可以根據左端點判斷 \m{a} 和 \m{b} 的關係。
\stopANSWER

%e33.2-3
\startEXERCISE
Mason 教授建議修改過程 \ALGO{ANY-SEGMENTS-INTERSECT},
使其不是找出一個交點後返回,而是輸出相交的線段,
再繼續進行 \m{for} 循環的下一次迭代。
他把這樣得到的過程稱爲 \ALGO{PRINT-INTERSECTING-SEGMENTS},
並聲稱該過程能夠按照線段在集合中出現的次序,
從左到右輸出所有的交點。
 Dixon 教授不同意 Mason 教授,稱其做法有誤。
哪位教授的說法是正確的?
 \ALGO{PRINT-INTERSECTING-SEGMENTS} 所找出的第一個相交點
總是最左邊的交點嗎?
他總能找出所有的相交點嗎?
\stopEXERCISE

\startANSWER
不正確,不是,不能。
\stopANSWER

%e33.2-4
\startEXERCISE
寫出一個運行時間爲 \m{O(n\lg n)} 的算法,
以確定由 \m{n} 個頂點組成的多邊形是否是簡單多邊形。
\stopEXERCISE

\startANSWER
用掃除線算法判斷是否有兩條邊相交,
忽略相鄰兩線段的交點。
\stopANSWER

%e33.2-5
\startEXERCISE
寫出一個運行時間爲 \m{O(n\lg n)} 的算法,
以確定共有 \m{n} 個頂點的兩個簡單多邊形是否相交。
\stopEXERCISE

\startANSWER
用掃除線算法判斷是否有兩條邊相交,
忽略相鄰兩線段的交點。
\stopANSWER

%e33.2-6
\startEXERCISE
一個圓面是由一個圓加上其內部組成,用圓心和半徑表示。
如果兩個圓面有公共點,則稱這兩個圓面相交。
寫出一個運行時間爲 \m{O(n\lg n)} 的算法,
以確定 \m{n} 個圓面中是否有任何兩個圓面相交。
\stopEXERCISE

\startANSWER
參見\goto{stackoverflow}
[url(http://stackoverflow.com/questions/19926054/segment-intersection)]

\TODO{略。}
\stopANSWER

%e33.2-7
\startEXERCISE
已知 \m{n} 條線段中共有 \m{k} 個相交點,
試說明如何在 \m{O((n+k)\lg n)} 時間內輸出全部 \m{k} 個交點。
\stopEXERCISE

\startANSWER
將交點作爲事件點。
\stopANSWER

%e33.2-8
\startEXERCISE
論證即使有三條或更多的線段相交於同一點,
過程 \ALGO{ANY-SEGMENTS-INTERSECT} 也能正確執行。
\stopEXERCISE

\startANSWER
\TODO{略。}
\stopANSWER

%e33.2-9
\startEXERCISE
證明:在有垂直線段的情況下,
如果將某一垂直線段的底部端點當作是左端點,
其頂部端點當作右端點,
則過程 \ALGO{ANY-SEGMENTS-INTERSECT} 也能正確執行。
如果允許有垂直線段,
對\refexercise{33.2-2} 的回答應如何修改。
\stopEXERCISE

\startANSWER
\TODO{略。}
\stopANSWER

\stopsection
