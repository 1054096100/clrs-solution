\startsection[
  title={Finding the closest pair of points},
]

%e33.4-1
\startEXERCISE
Williams 教授提出了一個方案,可以在最近點對算法中,
只檢查數組 \m{Y'} 中每個點後面的 5 個點,
其思想是,總是將直線 \m{l} 上的點放入集合 \m{P_L} 中。
那麼,直線 \m{l} 上就不可能有一個點屬於 \m{P_L},
另一個點屬於 \m{P_R} 的重合點對。
因此,至多可能有 6 個點處於 \m{\delta\times 2\delta} 的矩形內。
這種方案的缺陷何在?
\stopEXERCISE

\startANSWER
\TODO{略。}
\stopANSWER

%e33.4-2
\startEXERCISE
試說明只檢查數組 \m{Y'} 中跟隨在每個點後的 5 個數組位置就足夠了。
\stopEXERCISE

\startANSWER
\TODO{略。}
\stopANSWER

%e33.4-3
\startEXERCISE
除歐幾里德距離外,還有其他方法定義兩點間的距離。
在平面上,點 \m{p_1} 和 \m{p_2} 之間的{\EMP \m{L_m} 距離}由下式給出:
\startformula
(|x_1-x_2|^m + |y_1-y_2|^m)^{1/m}
\stopformula
因此,歐幾里德距離實際上是 \m{L_2} 距離。
修改最近點對算法,使其適用於 \m{L_1} 距離,
也稱爲{\EMP 曼哈頓距離}(Manhattan distance)。
\stopEXERCISE

\startANSWER
\TODO{略。}
\stopANSWER

%e33.4-4
\startEXERCISE
已知平面上兩個點 \m{p_1} 和 \m{p_2},
他們之間的 \m{L_\infty} 距離爲 \m{\max(|x_1-x_2|,|y_1-y_2|)}。
修改最近點對算法,使其適用於 \m{L_\infty} 距離。
\stopEXERCISE

\startANSWER
\TODO{略。}
\stopANSWER

%e33.4-5
\startEXERCISE
假設最近點對算法裏 \m{\Omega(n)} 對點是共垂線的。
試說明如何確定集合 \m{P_L} 和 \m{P_R} 以及如何確定 \m{Y} 中的
每個點是在 \m{P_L} 中還是在 \m{P_R} 中,
從而使最近點對算法的運行時間保持 \m{O(n\lg n)}。
\stopEXERCISE

\startANSWER
\TODO{略。}
\stopANSWER

%e33.4-6
\startEXERCISE
對最近點對算法進行修改,
使其能避免對數組 \m{Y} 進行預排序,
但仍能使算法的運行時間保持爲 \m{O(n\lg n)}。
(\hint 將已排序的數組 \m{Y_L} 和 \m{Y_R} 加以合併,
以形成有序數組 \m{Y}。)
\stopEXERCISE

\startANSWER
\TODO{略。}
\stopANSWER

\stopsection
