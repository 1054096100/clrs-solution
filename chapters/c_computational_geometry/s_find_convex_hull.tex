\startsection[
  title={Finding the convex hull},
]

%e33.3-1
\startEXERCISE
證明:在過程 \ALGO{GRAHAM-SCAN} 中,
點 \m{p_1} 和 \m{p_m} 必定是 \ALGO{CH(Q)} 的頂點。
\stopEXERCISE

\startANSWER
\TODO{略。}
\stopANSWER

%e33.3-2
\startEXERCISE
考慮一個能支持加法、比較和懲罰運算的計算模型,
用該模型對 \m{n} 個數進行排序時,
其下界爲 \m{\Omega(n\lg n)}。
證明:
當在這樣的一個模型中有序地計算出由 \m{n} 個點組成的集合的凸包時,
其下界爲 \m{\Omega(n\lg n)}。
\stopEXERCISE

\startANSWER
\TODO{略。}
\stopANSWER

%e33.3-3
\startEXERCISE
已知一個點集 \m{Q},
證明彼此相距最遠的點對必定是 \ALGO{CH(Q)} 中的頂點。
\stopEXERCISE

\startANSWER
反證法,令相距最遠的兩個點爲 \m{p_1,p_2},
其連線延長線與 \ALGO{CH(Q)} 交於兩點 \m{p_3,p_4}。

如果 \m{p_3,p_4} 都是 \ALGO{CH(Q)} 的頂點,
則 \m{\vec{p_3p_4}} 的長度大於 \m{\vec{p_1p_2}},矛盾。

如果 \m{p_3,p_4} 只有一個是 \ALGO{CH(Q)} 的頂點,
假定 \m{p_3} 是 \ALGO{CH(Q)} 的頂點,
 \m{p_4} 所在的邊爲 \m{\vec{p_5,p_6}},
則 \m{\vec{p_3 p_5}} 和 \m{\vec{p_3 p_6}} 中必有一個長度大於 \m{\vec{p_3 p_4}},
當然也就大於 \m{\vec{p_1 p_2}} 的長度,矛盾。

如果 \m{p_3,p_4} 都不是 \ALGO{CH(Q)} 的頂點,
令 \m{p_3} 所在的邊爲 \m{\vec{p_5 p_6}},
 \m{p_4} 所在的邊爲 \m{\vec{p_7 p_8}},
其中 \m{p_5,p_7} 均在 \m{\vec{p_3p_4}} 的左側,
 \m{p_6,p_8} 均在 \m{\vec{p_3p_4}} 的右側。
那麼 \m{\vec{p_5p_7}} 和 \m{\vec{p_6p_8}} 中必有一個長度大於 \m{\vec{p_3 p_4}},
當然也就大於 \m{\vec{p_1 p_2}} 的長度,矛盾。
\stopANSWER

%e33.3-4
\startEXERCISE
對一個給定的多邊形 \m{P} 和在其邊界上的一個點 \m{q},
 \m{q} 的{\EMP 投影}是點 \m{r} 的集合,其中點 \m{r} 滿足如下條件:
線段 \m{\overbar{qr}} 完全在 \m{P} 的邊界上或內部。
正如圖 33-10 所示,
如果在 \m{P} 的內部存在一個點 \m{p},
他處於 \m{P} 的邊界上每個點的投影中,
則多邊形 \m{P} 是{\EMP 星形多邊形}。
所有滿足這種條件的點 \m{p} 的集合稱爲 \m{P} 的{\EMP 內核}。
給定一個 \m{n} 個頂點的星型多邊形 \m{P},
他的各個頂點已按逆時針方向排序,
試說明如何在 \m{O(n)} 時間內計算出 \ALGO{CH(Q)}。
\stopEXERCISE

\startANSWER
\TODO{略。}
\stopANSWER

%e33.3-5
\startEXERCISE
在{\EMP 聯機凸包問題}(on-line convex-hull problem)中,
每次只給出由 \m{n} 個點所組成的集合 \m{Q} 中的一個點。
在接收到每個點後,就計算出當前所有點的凸包。
顯然,可以對每個點運行一次 Graham 掃描算法,
總時間爲 \m{O(n^2\lg n)}。
試說明如何在 \m{O(n^2)} 時間內解決聯機凸包問題。
\stopEXERCISE

\startANSWER
\TDDO{略。}
\stopANSWER

%e33.3-6
\startEXERCISE\DIFFICULT
試說明如何實現增量法,以在 \m{O(n\lg n)} 時間內計算出 \m{n} 個點的凸包。
\stopEXERCISE

\startANSWER
\TDDO{略。}
\stopANSWER

\stopsection
