\startsection[
  title={Basicmatrix properties},
]

%eD.2-1
\startEXERCISE
證明:矩陣的逆唯一,
即如果 \m{B} 和 \m{C} 均爲 \m{A} 的逆,則 \m{B=C}。
\stopEXERCISE

\startANSWER
\startformula\startmathalignment[n=1]
\NC AB = BA = I \NR
\NC AC = CA = I \NR
\stopmathalignment\stopformula

\startformula
B = BI = B(AC) = (BA)C = IC = C
\stopformula
\stopANSWER

%eD.2-2
\startEXERCISE
證明:下三角矩陣或上三角矩陣的行列式與其對角線元素之積相等。

證明:一個下三角矩陣的逆(如果存在)也是下三角矩陣。
\stopEXERCISE

\startANSWER
根據行列式的定義:
\startformula
\det(A) = \startcases
\NC a_{11} \NC 如果 \m{n=1} \NR
\NC \sum_{j=1}^{n} (-1)^{1+j} a_{1j} \det(A_{[1j]}) \NC 如果 \m{n>1} \NR
\stopcases
\stopformula
由於第一行只有 \m{a_{11}} 不是 0,因此:
\startformula
\sum_{j=1}^{n} (-1)^{1+j} a_{1j} \det(A_{[1j]})
  = a_{11} \cdot \det(A_{[1j]})
  = a_{11} \cdot a_{22} \cdot \ldots \cdot a_{nn}
\stopformula

令 \m{L^{-1} = [y_1 \ldots y_n]},其中 \m{y_k} 是 \m{n\times 1} 矩陣。
根據定義 \m{LL^{-1} = I = [e_1\ldots e_n]},其中 \m{e_k} 也是 \m{n\times 1} 矩陣,
其只有第 \m{k} 行爲 1,其他都是 0。
由於
\startformula
LL^{-1}=L[y_1\ldots y_n] = [Ly_1\ldots Ly_n]
\stopformula
所以 \m{Ly_k = e_k},其中 \m{1\le k\le n}。

由於 \m{L} 是下三角矩陣:
\startformula
Ly_k =
\left[\startmatrix[align=left]
\NC L_{11} Y_{1k} \NR
\NC L_{21} Y_{1k} + L_{22} Y_{2k} \NR
\NC \vdots \NR
\NC L_{k-1,1} Y_{1k} + L_{k-1,2} Y_{2k} + \ldots + L_{k-1,k-1} Y_{k-1,k} \NR
\NC L_{k,1} Y_{1k} + L_{k,2} Y_{2k} + \ldots + L_{k,k-1} Y_{k-1,k} + L_{k,k}Y_{k,k} \NR
\NC \vdots \NR
\stopmatrix\right]
=
\left[\startmatrix
\NC 0 \NR
\NC 0 \NR
\NC \vdots \NR
\NC 0 \NR
\NC 1 \NR
\NC \vdots \NR
\stopmatrix\right]
=e_k
\stopformula
由第一行可知 \m{Y_{1k} = 0},代入第二行,可得 \m{Y_{2k}=0},以此類推,
可知 \m{i < k, Y_{ik} = 0},即 \m{Y} 也是下三角矩陣。
\stopANSWER

%eD.2-3
\startEXERCISE
證明:如果 \m{P} 是一個排列矩陣,則 \m{P} 是可逆的,
他的逆是 \m{P^T},且 \m{P^T} 也是一個排列矩陣。
\stopEXERCISE

\startANSWER
\startformula
P^{-1} = P^T
\stopformula
\stopANSWER

%eD.2-4
\startEXERCISE
令 \m{A} 和 \m{B} 是 \m{n\times n} 矩陣, 且 \m{AB=I}。
證明:若矩陣 \m{A'} 是將矩陣 \m{A} 第 \m{j} 行加到第 \m{i} 行所得,
則將 \m{B} 中第 \m{j} 列減去第 \m{i} 列所得的 \m{B'} 爲 \m{A'} 的逆矩陣。
\stopEXERCISE

\startANSWER
\TODO{略。}
\stopANSWER

%eD.2-5
\startEXERCISE
令 \m{A} 是一個非奇異 \m{n\times n} 複數矩陣。
證明: \m{A^{-1}} 中每個元素均爲實數,
當且僅當 \m{A} 中每個元素是實數。
\stopEXERCISE

\startANSWER
\TODO{略。}
\stopANSWER

%eD.2-6
\startEXERCISE
證明:若 \m{A} 是一個非奇異的 \m{n\times n} 對稱矩陣,則 \m{A^{-1}} 是對稱的。

證明:若 \m{B} 是任意 \m{m\times n} 矩陣,
則 \m{m\times m} 矩陣 \m{BAB^T} 也是對稱的。
\stopEXERCISE

\startANSWER
\TODO{略。}
\stopANSWER

%eD.2-7
\startEXERCISE
證明定理 D.2。
即證明:
一個矩陣 \m{A} 是列滿秩的,
當且僅當 \m{Ax=0} 蘊含 \m{x=0}。
(\hint 將一列與另一列線性相關表示爲矩陣——向量等式。)
\stopEXERCISE

\startANSWER
\TODO{略。}
\stopANSWER

%eD.2-8
\startEXERCISE
對於任意兩個相容矩陣 \m{A} 和 \m{B},
\startformula
\rank(AB)\le \min(\rank(A),\rank(B))
\stopformula
其中,如果 \m{A} 或 \m{B} 是非奇異方陣,則大呢公式成立。
(\hint 使用矩陣秩的另一種定義。)
\stopEXERCISE

\startANSWER
\TODO{略。}
\stopANSWER

\stopsection
