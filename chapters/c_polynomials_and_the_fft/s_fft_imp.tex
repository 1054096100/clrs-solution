\startsection[
  title={Efficient FFT implementations},
]

%e30.3-1
\startEXERCISE
請說明如何用 \ALGO{ITERATIVE-FFT} 計算出輸入向量 \m{(0,2,3,-1,4,5,7,9)} 的 \ALGO{DFT}。
\stopEXERCISE

\startANSWER
\TODO{略。}
\stopANSWER

%e30.3-2
\startEXERCISE
請說明如何實現一個 \ALGO{FFT} 算法,
注意把位逆序置換放在計算的最後而不是開始。
(\hint 考慮逆 \ALGO{DFT})
\stopEXERCISE

\startANSWER
\TODO{略。}
\stopANSWER

%e30.3-3
\startEXERCISE
在每個階段中, \ALGO{ITERATIVE-FFT} 計算旋轉銀子多少次?
重寫 \ALGO{ITERATIVE-FFT},使其在階段 \m{s} 中計算旋轉銀子 \m{2^{s-1}} 次。
\stopEXERCISE

\startANSWER
\TODO{略。}
\stopANSWER

%e30.3-4
\startEXERCISE\DIFFICULT
假設 \ALGO{FFT} 電路的蝴蝶操作中加法器有時會發生錯誤:
不論輸入如何,他們的輸出總是爲 0。
假設卻又一個加法器失效,但你並不知道是哪一個。
描述你如何能夠通過給整個 \ALGO{FFT} 電路提供輸入值並觀察其輸出,
找到那個失效的加法器。
你的方法效率如何?
\stopEXERCISE

\startANSWER
\TODO{略。}
\stopANSWER

\stopsection
