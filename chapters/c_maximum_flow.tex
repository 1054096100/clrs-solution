\startcomponent c_maximum_flow

\startchapter[
  title={Maximum Flow},
]

\startsection[
  title={Flow networks},
  reference=section:flow_networks,
]

%e26.1-1
\startEXERCISE
證明:在一個流網絡中,
將一條邊分解爲兩條邊所得到的是一個等價的網絡。
更形式化地說,假定網絡 \m{G} 包含邊 \m{(u,v)},
我們以如下方式創建一個新的流網絡 \m{G'}:
創建一個新節點 \m{x},
用新的邊 \m{(u,x)} 和 \m{(x,v)} 來替換原來的邊 \m{(u,v)},
並設置 \m{c(u,x)=c(x,v)=c(u,v)}。
證明: \m{G'} 中的一個最大流與 \m{G} 中的一個最大流具有相同的值。
\stopEXERCISE

\startANSWER
\m{f(u,x)=f(x,v)}。
\stopANSWER

%e26.1-2
\startEXERCISE
將流的性值和定義推廣到多源點和多匯點的流問題上。
證明:在多源點多匯點流網絡中,
通過增加一個超級源點和超級匯點,可以所形成一個單源點單匯點流網絡,
新網絡與原網絡中的流是一一對應的。
\stopEXERCISE

\startANSWER
容量限制:對於所有 \m{u,v\in V},都有 \m{0\le f(u,v)\le c(u,v)};

流量守恆:對於所有 \m{u\in V-S-T},都有 \m{\sum_{v\in V}f(v,u)=\sum_{v\in V}f(u,v)}。
\stopANSWER

%e26.1-3
\startEXERCISE
假定流網絡 \m{G=(V,E)} 違反了如下假設:對於所有節點 \m{v\in V},
網絡必須包括一條路徑 \m{s\leadsto v\leadsto t}。
設節點 \m{u} 滿足:不存在路徑 \m{s\leadsto u\leadsto t}。
證明: \m{G} 中必然存在一個最大流 \m{f},
使得對於所有節點 \m{v\in V}, \m{f(u,v)=f(v,u)=0}。
\stopEXERCISE

\startANSWER
\m{u} 只能扇入或只能扇出,因此只能是既無扇入亦無扇出。
\stopANSWER

%e26.1-4
\startEXERCISE
設 \m{f} 爲網絡中的一個流, \m{\alpha} 爲一實數,
將 \m{\alpha f} 稱爲{\EMP 標量流積(scalar flow product)},
是從 \m{V\times V} 到 \m{R} 的函數,定義如下:
\startformula
(\alpha f)(u,v) = \alpha \cdot f(u,v)
\stopformula
證明:網絡中的流形成一個{\EMP 凸集(convex set)}。
也就是說,證明:如果 \m{f_1} 和 \m{f_2} 爲兩個流,
則 \m{\alpha f_1 + (1-\alpha)f_2} 也是一個流,這裏 \m{0\le \alpha \le 1}。
\stopEXERCISE

\startANSWER
\m{0\le \alpha f_1 + (1-\alpha)f_2 \le \max(f_1,f_2)}。
\stopANSWER

%e26.1-5
\startEXERCISE
將最大流問題表述爲一個線性規劃問題。
\stopEXERCISE

\startANSWER
\startformula\startmathalignment
\NC \max \NC (\sum_{v\in V}f(s,v) - \sum_{v\in V}f(v,s)) \NR
\NC s.t. \NC 0\le f(u,v) \le c(u,v) \NR
\NC \NC \sum_{v\in V}f(v,u) - \sum_{v\in V}f(u,v) = 0 \NR
\stopmathalignment\stopformula
\stopANSWER

%e26.1-6
\startEXERCISE
Adam 教授有兩個兒子,可不幸的是,他們互相討厭對方。
隨着時間的推移,問題變得如此嚴重,
他們不僅不願意一起到學校,
而且都拒絕走對方當天所走過的街區。
兩個孩子並不在意自己走的路徑與對方所走的路徑在街角交叉。
幸運的是,教授的房子和學校都位於街角。
但除此之外,教授不能肯定是否可以在滿足上述條件的情況下
把兩個小孩送到同一所學校。
教授有一份小鎮的地圖,試說明如何將這個問題轉換爲一個最大流問題,
以便決定是否可以將孩子送到同一所學校。
\stopEXERCISE

\startANSWER
每條路的容量都是 1,最大流必須大於 1。
\stopANSWER

%e26.1-7
\startEXERCISE
假定除邊的容量外,流網絡還有{\EMP 節點容量}。
即對於每個節點 \m{v},有一個極限值 \m{l(v)},
這是可以流經此節點的最大流量。
請說明如何將一個帶有節點容量的流網絡 \m{G=(V,E)} 轉換
爲一個等價的但沒有節點容量的流網絡 \m{G'=(V',E')},
使得 \m{G'} 中的最大流與 \m{G} 中的最大流取值相同。
圖 \m{G'} 裏有多少個節點和多少條邊?
\stopEXERCISE

\startANSWER
將每個節點 \m{v} 轉換成一條邊 \m{v,v‘},其容量爲 \m{l(v)}。
 \m{V' = 2V}, \m{E'=V+E}。
\stopANSWER

\stopsection

\startsection[
  title={The Ford-Fulkerson method},
]

%e26.2-1
\startEXERCISE
證明式 26.6 中的和值等於式 26.7 中的和值。
\stopEXERCISE

\startANSWER
\startformula\startmathalignment
\NC |f\uparrow f'| \NC = \sum_{v\in V_1}f(s,v) - \sum_{v\in V_2}f(v,s)
  + \sum_{v\in V_1\cup V_2}f'(s,v) - \sum_{v\in V_1\cup V_2}f'(v,s) \NR
\NC \NC = \sum_{v\in V}f(s,v) - \sum_{v\in V}f(v,s)
  + \sum_{v\in V}f'(s,v) - \sum_{v\in V}f'(v,s) \NR
\NC \NC = |f| + |f'| \NR
\stopmathalignment\stopformula
\stopANSWER

%e26.2-2
\startEXERCISE
在圖 26-1(b)中,橫跨切割(\m{\{s,v_2,v_4\}}, \m{\{v_1,v_3,t\}})的流是多少?
該切割的容量又是多少?附圖 26-1:

\externalfigure[output/e26_2_2-1]
\stopEXERCISE

\startANSWER
\startformula
f(s,t) = f(s,v_1) + f(v_2,v_1) + f(v_4,v_3) + f(v_4,t) - f(v_3,v_2) = 19
\stopformula

\startformula
f(s,t) = c(s,v_1) + c(v_2,v_1) + c(v_4,v_3) + c(v_4,t) = 31
\stopformula
\stopANSWER

%e26.2-3
\startEXERCISE
在圖 26-1(a)所示的流網絡上演示 \ALGO{EDMONDS-KARP} 的執行過程。
附圖 26-1(a):

\externalfigure[output/e26_2_3-1]
\stopEXERCISE

\startANSWER
\startcombination[nx=2]
{\externalfigure[output/e26_2_3-2]}{}
{\externalfigure[output/e26_2_3-3]}{}
\stopcombination
\startcombination[nx=2]
{\externalfigure[output/e26_2_3-4]}{}
{\externalfigure[output/e26_2_3-5]}{}
\stopcombination
\startcombination[nx=2]
{\externalfigure[output/e26_2_3-6]}{}
{\externalfigure[output/e26_2_3-7]}{}
\stopcombination
\startcombination[nx=2]
{\externalfigure[output/e26_2_3-8]}{}
{}{}
\stopcombination
\stopANSWER

%e26.2-4
\startEXERCISE
在圖 26-6 的例子中,對應圖中所示最大流的最小切割是什麼?
在例子中出現的增廣路徑裏,哪一條路徑抵消了先前被傳輸的流?
\stopEXERCISE

\startANSWER
最小切割爲 \m{(\{s,v_1,v_2,v_4\},\{v_3,t\})}。
(c)中的 \m{(v_1,v_2)} 抵消了(b)中的 \m{(v_2,v_1)}。
(c)中的 \m{(v_2,v_3)} 抵消了(a)中的 \m{(v_3,v_2)}。
\stopANSWER

%e26.2-5
\startEXERCISE
在\refsection{flow_networks} 中,我們通過增加容量無限的邊,
把多源點多匯點的流網絡轉換爲單源點單匯點的流網絡。
證明:
如果原來的網絡容量是有限的,則轉換後的網絡中任何一個流的值都是有限的。
\stopEXERCISE

\startANSWER
因爲我們可以找到一個最小切割,只包含有限個有限權重的邊。
\stopANSWER

%e26.2-6
\startEXERCISE
假定在一個多源點多匯點的流網絡中,
每個源節點 \m{s_i} 生產出恰好 \m{p_i} 個單位的流,
因此, \m{\sum_{v\in V}f(s_i,v)=p_i}。
假定每個匯點 \m{t_j} 恰好消費 \m{q_j} 個單位的流,
因此 \m{\sum_{v\in V}f(v,t_j)=q_j},其中 \m{\sum_{i}p_i = \sum_{j}q_j}。
說明如何在此網絡中尋找最大流 \m{f} (可以先將其轉換爲一個單源點單匯點的流網絡)。
\stopEXERCISE

\startANSWER
新添加源點 \m{s} 和匯點 \m{t}。
邊 \m{(s,s_i)} 的容量爲 \m{c(s,s_i) = p_i}。
邊 \m{(t_j,t)} 的容量爲 \m{c(t_j,t) = q_j}。
\stopANSWER

%e26.2-7
\startEXERCISE
證明引理 26.2。附引理 26.2:

設 \m{G=(V,E)} 爲一個流網絡,
 \m{f} 是圖 \m{G} 的一個流,
設 \m{p} 爲殘存網路 \m{G_f} 中的一條增光路徑。
定義一個函數 \m{f_p: V\times V\rightarrow R} 如下:
\startformula
f_p(u,v)=\startcases
\NC c_f(p) \MC \text{如果 \m{(u,v)} 在 \m{p} 上} \NR
\NC 0 \MC 否則 \NR
\stopcases
\stopformula
則 \m{f_p} 是殘存網絡 \m{G_f} 中的一個流,
其值爲 \m{|f_p| = c_f(p) > 0}。
\stopEXERCISE

\startANSWER
容量限制:由於 \m{c_f(p)} 是 \m{p} 上所有邊的最小權重,
因此滿足 \m{0\le f_p(u,v) = c_p(u,v) \le c_f(p)}。

流量守恆: \m{\sum_{v\in V}f(v,u) = c_f(p) = \sum_{v\in V}f(u,v)}。
\stopANSWER

%e26.2-8
\startEXERCISE
假定我們重新定義殘存網絡,
進制一切進入源節點 \m{s} 的邊。
算法 \ALGO{FORD-FULKERSON} 所計算出的最大流是否仍然正確?
\stopEXERCISE

\startANSWER
正確。
\stopANSWER

%e26.2-9
\startEXERCISE
假定 \m{f} 和 \m{f'} 都是流網絡 \m{G} 中的流,
計算流 \m{f\uparrow f'}。
加增後的流滿足流量守恆嗎?
滿足容量限制嗎?
\stopEXERCISE

\startANSWER
滿足流量守恆,不滿足容量限制。
\stopANSWER

%e26.2-10
\startEXERCISE
說明在流網絡 \m{G=(V,E)} 中,
如何使用一個最多包含 \m{|E|} 條增廣路徑的序列找到最大流。
(\hint 找到最大流後再確定路徑)
\stopEXERCISE

\startANSWER
先找到最大流,可以得到網絡的最小切割。
在最小切割的基礎上,每次增加一條邊,上限爲 \m{|E|}。
\stopANSWER

%e26.2-11
\startEXERCISE
無向圖的{\EMP 邊連通性}是指使圖變爲非連通圖所需要刪除的最少邊數 \m{k}。
例如,樹的邊連通性爲 \m{1},環路的邊連通性爲 \m{2}。
請說明如何在最多 \m{|V|} 個流網絡上運行最大流算法,
從而確定無向圖 \m{G=(V,E)} 的邊連通性,
這裏的每個流網絡節點數爲 \m{O(V)},邊數爲 \m{O(E)}。
\stopEXERCISE

\startANSWER
先將圖 \m{G} 轉換成帶權有向圖 \m{G'},兩圖頂點相同。
 \m{G} 中的每條邊 \m{(u,v)} 對應 \m{G'} 中的兩條邊 \m{(u,v)} 和 \m{(v,u)}。
 \m{G'} 中所有邊權重均爲 \m{1}。
在 \m{G'} 基礎上構造 \m{|V|} 個網絡流,
分別令每個頂點作爲匯點,源點可任意選取,只要不是匯點即可。
依次求這 \m{|V|} 個流網絡的最大流。
所有最大流的最小值即爲 \m{G} 的邊連通性。
\stopANSWER

%e26.2-12
\startEXERCISE
給定一個流網絡 \m{G}, \m{G} 中包含進入源節點 \m{s} 的邊。
設 \m{f} 爲網絡 \m{G} 中的一個流,
在該流中,其中一條進入源點的邊 \m{(v,s)} 有 \m{f(v,s)=1}。
證明:圖 \m{G} 中必存在另一個流 \m{f'},
滿足 \m{f'(v,s)=0},使得 \m{|f|=|f'|}。
給出一個 \m{O(E)} 時間複雜度的算法,
在給定流 \m{f} 的情況下計算 \m{f'},
這裏假定所有邊的容量都是正整數。
\stopEXERCISE

\startANSWER
在流網絡 \m{G=(V,E)} 中,對於節點 \m{v\in V-s},
如果沒有路徑 \m{s\leadsto v},則 \m{f(v,s)=0}。

令從 \m{s} 可達的節點集合爲 \m{Y},
從 \m{s} 不可達的節點集合爲 \m{Z},有:
\startformula\startmathalignment
\NC \sum_{z\in Z} \sum_{x\in V} f(x,z) \NC = \sum_{z\in Z}\sum_{x\in V} f(z,x) \qquad \text{流量守恆} \NR
\NC \sum_{z\in Z}\sum_{x\in Z}f(x,z) + \sum_{z\in Z}\sum_{x\in Y}f(x,z) \NC
  \sum_{z\in Z}\sum_{x\in Z}f(z,x) + \sum_{z\in Z}\sum_{x\in Y}f(z,x) \NR
\stopmathalignment\stopformula

根據流量守恆有 \m{\sum_{z\in Z}\sum_{x\in Z}f(x,z) = \sum_{z\in Z}\sum_{x\in Z}f(z,x)}。
又由於 \m{\sum_{z\in Z}\sum_{x\in Y} f(x,z) = 0},所以:
對於所有 \m{z\in Z} 和 \m{x\in Y},都有 \m{f(z,x) = 0}。
因此,對於所有 \m{v\in Z} 和 \m{s\in Y},都有 \m{f(v,s) = 0}。

而如果 \m{f(v,s)=1},那麼 \m{v\notin Z},即肯定存在路徑 \m{s\leadsto v}。
從而有環路 \m{s\leadsto v \leadsto s},給其中每條邊都減 \m{1},形成新的流 \m{f'},
不會影響總流量,即 \m{|f| = |f'|}。

首先搜索節點 \m{v},使得 \m{(v,s)\in E} 且 \m{f(v,s)=1},
可以通過 \ALGO{BFS} 搜索路徑 \m{s\leadsto s}。
確保路徑上所有邊 \m{(u,v)} 都有 \m{f(u,v)\ge 1}。

然後將所得路徑上所有邊 \m{(u,v)} 的 \m{f(u,v)} 都減去 \m{1}。得到 \m{f'},返回即可。
\stopANSWER

%e26.2-13
\startEXERCISE
假定我們希望找到流網絡 \m{G} 的一個最小切割,使其邊數是所有最小切割中最少的。
這裏假定 \m{G} 的所有容量都是整數。
說明如何修改 \m{G} 的容量來創建一個新的流網絡 \m{G'},
使得 \m{G'} 中的任一最小切割都是 \m{G} 中邊數最少的最小切割。
\stopEXERCISE

\startANSWER
\m{G'} 的頂點、邊與 \m{G} 的相同,但邊的容量有所區別。
對於所有 \m{(u,v)\in E},有 \m{c_{G'}(u,v) = c_{G}(u,v) + \sigma}。
其中 \m{\sigma} 是一個常數,保證最小切割的邊數最多不會超過 \m{G} 中邊數最小的最小切割。
同時 \m{\sigma} 也不能太大,因爲在減少最小切割邊數的同時可能導致切割不再是最小切割。

\startformula
\sigma = \frac{m}{2|E|}
\stopformula
其中 \m{m} 是 \m{G} 中最小切割與非最小切割的最小差異。
要證明其正確性,我們需要證明:

1) \m{G} 中邊數最少的最小切割是 \m{G'} 中的唯一最小切割。

2) \m{G} 中非最小切割不是 \m{G'} 的最小切割。

先證明(1),令 \m{(S,T),(X,Y)} 是 \m{G} 的最小切割, \m{|(S,T)|<|(X,Y)|},有:
\startformula\startmathalignment
\NC c'(S,T) \NC = c(S,T) + |(S,T)|\sigma \NR
\NC \NC < c(X,Y) + |(X,Y)|\sigma \qquad c(S,T) = c(X,Y) \text{且} |(S,T)| < |(X,Y)| \NR
\NC \NC = c'(X,Y) \NR
\stopmathalignment\stopformula

再證明(2),令 \m{(S,T)} 爲 \m{G} 的最小切割, \m{c(S,T)<c(X,Y)},
我們來證明 \m{c'(X,Y)} 不是 \m{G'} 的最小切割:
\startformula\startmathalignment
\NC c'(S,T) \NC = c(S,T) + |(S,T)|\sigma \NR
\NC \NC \le c(S,T) + |E|\sigma \qquad |(S,T)| \le |E| \NR
\NC \NC = c(S,T) + \frac{m}{2} \NR
\NC \NC \le c(X,Y) \qquad c(X,Y) - c(S,T) \ge m \NR
\NC \NC < c'(X,Y) \NR
\stopmathalignment\stopformula

\stopANSWER

\stopsection

\startsection[
  title={Maximum bipartite matching},
]

%e26.3-1
\startEXERCISE
在圖 26-8(c)上運行算法 \ALGO{FORD-FULKERSON},
給出每次流量遞增後的殘存網絡。
將集合 \m{L} 中的節點從上至下編號 1~5,
集合 \m{R} 中的編號從上至下編號 6~9。
對於每次迭代,選擇字典次序最小的增廣路徑。
附圖 26-8(c):

\externalfigure[output/e26_3_1-1]
\stopEXERCISE

\startANSWER
\startcombination[nx=4]
{\externalfigure[output/e26_3_1-2]}{}
{\externalfigure[output/e26_3_1-3]}{}
{\externalfigure[output/e26_3_1-4]}{}
{\externalfigure[output/e26_3_1-5]}{}
\stopcombination
\startcombination[nx=4]
{\externalfigure[output/e26_3_1-6]}{}
{\externalfigure[output/e26_3_1-7]}{}
{\externalfigure[output/e26_3_1-8]}{}
{}{}
\stopcombination
\stopANSWER

%e26.3-2
\startEXERCISE
證明定理 26.10。
附定理 26.10:

如果容量函數 \m{c} 只能取整數值,
則 \ALGO{FORD-FULKERSON} 所生成的最大流 \m{f} 滿足: \m{|f|} 是整數值。
而且,對於所有節點 \m{u} 和 \m{v}, \m{f(u,v)} 的值都是整數。
\stopEXERCISE

\startANSWER
算法 \ALGO{FORD-FULKERSON} 中增廣路徑中邊的最小容量肯定是整數,
流的容量每次的變化都是整數,因此最終的最大流 \m{|f|} 是整數,且任一邊的流量都是整數。
\stopANSWER

%e26.3-3
\startEXERCISE
設 \m{G=(V,E)} 是一個二分圖,
其節點劃分爲 \m{V=L\cup R},設 \m{G'=(V',E')} 爲其對應的流網絡。
在 \ALGO{FORD-FULKERSON} 執行過程中,
對在 \m{G'} 中找出的任意增廣路徑的長度給出一個適當的上界。
\stopEXERCISE

\startANSWER
根據定義,殘存網絡 \m{G'_{f}} 的增廣路徑是 \m{s\leadsto t} 的簡單路徑。
在 \m{G} 中不存在兩端均在 \m{L} 或 \m{R} 中的邊,
 \m{G'} 和 \m{G'_f} 中也沒有這樣的邊。
只有 \m{s} 連接 \m{L} 中的節點, \m{R} 中的節點連接到 \m{t}。
 \m{G'} 中的邊均從 \m{L} 到 \m{R},
而 \m{G'_f} 中的邊則可能從 \m{R} 到 \m{L}。
所以增廣路徑只能是:
\startformula
s\rightarrow L\rightarrow R\rightarrow \ldots \rightarrow L\rightarrow R\rightarrow t
\stopformula
在 \m{L} 和 \m{R} 中來回穿梭,但每個節點只能通過一次。
總節點數爲 \m{2 + 2\times \min(|L|,|R|)}。
因此增廣路徑的長度上限爲 \m{2\times \min(|L|,|R|) + 1}。
\stopANSWER

%e26.3-4
\startEXERCISE\DIFFICULT
{\EMP 完全匹配(perfect matching)}指包含圖中所有節點的匹配。
設 \m{G=(V,E)} 是節點劃分爲 \m{V=L\cup R} 的無向二分圖,
其中 \m{|L|=|R|}。
對於任意 \m{X\subseteq V},定義 \m{X} 的{\EMP 鄰居(neighborhood)}爲:
\startformula
N(X)=\{y\in V:\exists x\in X, (x,y)\in E\}
\stopformula
即集合中的元素由 \m{X} 中的節點可達。
請證明 Hall 定理:

圖 \m{G} 中存在一個完全匹配當且僅當對於每個子集 \m{A\subseteq L},
有 \m{|A|\le |N(A)|}。
\stopEXERCISE

\startANSWER
先證明:如果存在完全匹配,則對於任一 \m{A\subseteq L},有 \m{|A|\le |N(A)|}。

根據完全匹配的定義, \m{L} 中的每個節點都連接到了 \m{R} 中的節點,且互不相同。
即 \m{|A|\le |N(A)|}。

現在看另一面:如果所有 \m{A\subseteq L},都有 \m{|A|\le |N(A)|},
則存在完全匹配。

如果 \m{|L|=|R|=1},很顯然存在完全匹配,
即 \m{G} 中只有一條邊。假定上述假設對 \m{|L|=|R|=1,2,\ldots,n-1} 均成立。
當 \m{|L|=|R|=n} 時,分兩種情況考慮:

第一種情況:對於所有 \m{A\subseteq L},都有 \m{|A|<|N(A)|}。
在 \m{L} 中任取一個節點 \m{u},然後在 \m{R} 中選取 \m{u} 的任一鄰居,
然後將這兩個點從 \m{G} 中移除。
剩下的爲圖 \m{G'},仍然能保證:對於所有 \m{A\subseteq L},都有 \m{|A|\le |N(A)|}。
由於在 \m{G'} 中, \m{|L|=|R|=n-1},根據歸納假設,有完全匹配 \m{M'},
在此基礎上添加 \m{(u,v)},即爲 \m{G} 的完全匹配。

第二種情況:至少存在一個 \m{A\subset L},滿足 \m{|A|=|N(A)|}。
根據歸納假設,在 \m{A} 和 \m{N(A)} 間存在完全匹配 \m{M_A}。
從 \m{G} 中移除 \m{A} 和 \m{N(A)}。
在剩下的 \m{G'} 中,仍然滿足:對於所有 \m{B\subseteq L - A},都有 \m{|B|\le |N(B)|}。
用反證法。假如 \m{|B|>|N(B)|},那麼 \m{|A\cup B|>|N(A\cup B)|},顯然矛盾。
因此, \m{G'} 有完全匹配 \m{M'}。
結合 \m{M_A} 和 \m{M'},就可以得到 \m{G} 的完全匹配。
\stopANSWER

%e26.3-5
\startEXERCISE\DIFFICULT
將圖 \m{G=(V,E)} 中的節點劃分爲 \m{V=L\cup R},
如果二分圖中的每個節點 \m{v} 的度數都是 \m{d},
則稱該二分圖是 {\EMP \m{d} 正則的(\m{d}-regular)}。
對於每個 \m{d} 正則的二分圖,都有 \m{|L|=|R|}。
證明:
每個 \m{d} 正則二分圖的匹配基數都是 \m{|L|}。
(\hint 證明對應的流網絡最小切割的容量爲 \m{|L|})
\stopEXERCISE

\startANSWER
\m{L} 的總出度與 \m{R} 的總入度相同,每個節點的度數又都相同,所以 \m{|L|=|R|}。

最大匹配的基數就是 \m{G'} 中最大流的值。
定義 \m{G'} 的一個流:
\startformula
f(u,v)=\startcases
\NC 1/d \MC \text{如果 \m{(u,v)\in E}} \NR
\NC 1 \MC \text{如果 \m{u=s} 或者 \m{v=t}} \NR
\NC 0 \MC \text{其他} \NR
\stopcases
\stopformula
則 \m{f} 是 \m{G'} 的一個流,其值就是完全匹配的基數。
\stopANSWER

\stopsection

\startsection[
  title={Push-relabel algorithms},
]

%e26.4-1
\startEXERCISE
證明:算法 \ALGO{INITIALIZE-PREFLOW(G,s)} 終止後,
有 \m{s.e\le -|f^*|},其中 \m{f^*} 是流網絡 \m{G} 的一個最大流。
\stopEXERCISE

\startANSWER
\startformula
s.e = -\sum_{(s,v)\in E}c(s,v)
\stopformula
\stopANSWER

%e26.4-2
\startEXERCISE
說明如何實現通用的推送-重貼標籤算法,
使得每個重貼標籤的操作成本爲 \m{O(V)},
每個推送操作的成本爲 \m{O(1)},
並且可以在 \m{O(1)} 時間內選擇一個合適的操作,
從而使得整個算法運行時間爲 \m{O(V^2E)}。
\stopEXERCISE

\startANSWER
\m{s} 和 \m{t} 不會溢出,也就不會被重貼標籤。
 \m{t} 可以吸收無限多流,也就不會推送。

有邊 \m{(u,v)\in E_f}, \m{u.h = v.h + 1}。
則可以進行推送,一旦調用 \ALGO{PUSH(u,v)},若 \m{u.e = 0},則爲非飽和推送,
在下一次迭代中我們不會對 \m{u} 進行推送或重貼標籤,因爲 \m{u} 不再溢出。
而如果推送後 \m{u.e > 0},則爲飽和推送,
在下一次迭代中我們會對 \m{u} 進行推送或重貼標籤。
除此之外, \m{v.e} 一定大於零,
因爲 \m{v.e} 一直非負,在執行 \ALGO{PUSH(u,v)} 時還會增大。
至於 \m{v},是推送,還是重貼標籤,就很難判斷了。

令 \m{G} 和 \m{G_f} 分別代表流和殘存網絡。
在 \m{G_f} 中,對於每個節點 \m{u\in V_f},
令 \m{u.r} 表示 \m{G_f} 中 \m{u} 的一個鄰接點,即 \m{(u,u.r)\in E_f},
且 \m{u.h = u.r.h + 1}。
如果不存在這樣的節點,則 \m{u.r = NIL},
 \ALGO{RELABEL(u)} 會正確設置 \m{u.r}。
除此之外,我們用兩個鏈表實現通用的推送重貼標籤算法。

\m{L_1} 是待處理鏈表,對於所有 \m{u\in L_1},都有 \m{u.e\le 0}。
根據之前的討論,在下一次迭代中,我們不會對這些節點進行推送或重貼標籤,
因爲他們沒有溢出。

\m{L_2} 是推送重貼標籤鏈表,對於所有 \m{u\in L_2},都有 \m{u.e > 0}。
即這些節點全部溢出,我們會對其進行推送或重貼標籤。
如果 \m{L_2} 爲空,則算法終止。

初始化的時候進行如下操作。
在執行完 \ALGO{INITIALIZE-PREFLOW} 後,對於所有節點 \m{u\in V},
如果 \m{u.e\le 0},則將其移入 \m{L_1},否則將其移入 \m{L_2}。

從 \m{L_2} 中取出一個節點 \m{u} 後,
根據 \m{u.r} 的值進行不同的處理。
如果 \m{u.r} 是 \m{NIL},則重貼標籤,然後插入到 \m{L_2} 的最前端。
否則令 \m{v = u.r},
先檢查 \m{u.h = v.h + 1} 是否成立,
如果不成立,還是重貼標籤,然後插入到 \m{L_2} 的最前端,
否則調用 \ALGO{PUSH(u,v)}。
調用 \ALGO{PUSH(u,v)} 之後,如果 \m{u.e = 0},
則將 \m{u} 移入 \m{L_1};
如果 \m{u.e > 0},則先將 \m{u.h} 重置爲 \m{NIL},
然後將其插入到 \m{L_2} 末尾。
但 \m{v.e} 還是正值,如果下一次迭代中可以推送 \m{v},就必須推送。
幸運的是,通過跟蹤 \m{v.r},很容易做到這一點。
只需在執行完 \ALGO{PUSH(u,v)} 後,將 \m{v} 移到 \m{L_2} 的最前端,
下一次迭代時,算法會自動判斷。

我們知道每個重貼標籤操作的代價是 \m{O(V)},
每個推送操作的代價是 \m{O(1)},
選擇操作需要 \m{O(1)}。

然而我們還有點不放心:
(1)調用完 \ALGO{PUSH(u,v)} 後,
如果 \m{u.e=0},則會將其移到正確位置。
而如果 \m{u.e>0},我們將其移到了 \m{L_2} 尾端,
如果還可以繼續推送 \m{u} (但沒有立刻推送),會出現什麼情況?
最壞情況下,調用完 \ALGO{PUSH(u,v)} 後(正好是一個飽和推送),
我們會額外執行一個重貼標籤操作。
根據引理 26.22,最多會有 \m{2|V||E|} 個飽和推送,
因此總代價最多爲 \m{2|V||E| \times O(V) = O(V^2 E)}。
(2)從 \m{L_2} 中取出一個節點 \m{u},
如果 \m{u.r = v},且 \m{u.h \ne v.h+1},
則從 \m{v} 到 \m{u} 迴流,
即 \m{u} 仍然溢出,不能再從 \m{u} 推送到 \m{v}。
對於每條邊 \m{(u,v)\in E_f},
這種情況最多有 \m{2|V|} 次。
 \m{u.r=v} 意味着之前進行過一次非飽和推送 \m{(u,v)},
即 \m{u.h = v.h+1}。
但現在 \m{u.h\ne v.h + 1},即 \m{u.h\le v.h}, \m{v.h} 增大了。
下一次執行 \ALGO{PUSH(u,v)} 時會發現仍然有 \m{u.h=v.h + 1},
即 \m{u.h} 又增大了。
因此對於每條邊 \m{(u,v)\in E_f},
最多有 \m{2|V|} 次重貼標籤操作。
總代價爲 \m{|E|\times (2|V|) \times O(V)=O(V^2 E)}。
\stopANSWER

%e26.4-3
\startEXERCISE
證明:在通用推送重貼標籤算法中,
所有 \m{O(V^2)} 個重貼標籤操作,總共只用了 \m{O(VE)} 的時間。
\stopEXERCISE

\startANSWER
每次調用 \ALGO{RELABEL(u)},我們會檢查所有邊 \m{(u,v)\in E_f}。
對於每個節點而言,最多重貼標籤 \m{2|V|-1} 次,
而對於每條邊 \m{(u,v)},在重貼標籤過程中,
最多會檢查 \m{4|V|-2} 次,
其中一半用於 \m{u},另一半用於 \m{v}。
殘存網絡中最多有 \m{2|E|=O(E)} 條邊,
因此重貼標籤總時間爲 \m{O(VE)}。
\stopANSWER

%e26.4-4
\startEXERCISE
假定使用推送重貼標籤算法找到了流網絡 \m{G=(V,E)} 的一個最大流,
如何快速找到 \m{G} 的一個最小切割。
\stopEXERCISE

\startANSWER
可以在 \m{O(V)} 時間內做到。

首先,找到 \m{\hat{h}},滿足 \m{0<\hat{h}<|V|},
且算法終止時沒有節點高度爲 \m{\hat{h}}。

由於 \m{s.h = |V|},且 \m{t.h = 0},
我們僅需考慮 \m{|V|-2} 個節點。
也就是說 \m{\hat{h}} 的值有 \m{|V|-1} 種可能,
我們知道在 \m{1,2,\ldots,|V|-1} 中,至少有一個值不是任何節點的高。
因此如何選取 \m{\hat{h}},就很明確了,
很容易通過一個布爾數組,在 \m{O(V)} 時間內就可以找到 \m{\hat{h}}。

令 \m{S=\{u\in V: u.h > \hat{h}\}}, \m{T=\{v\in V: v.h<\hat{h}\}}。
由於 \m{s.h=|V|>\hat{h}},所以 \m{s\in S},
又由於 \m{t.h=0<\hat{h}},所以 \m{t\in T},
這也符合切割的要求。

我們需要使得對於所有 \m{u\in S} 和 \m{v\in T}, \m{f(u,v)=c(u,v)}。
一旦如此,則 \m{f(S,T)=c(S,T)},
根據推論 26.6, \m{(S,T)} 就是最小切割。

爲了反證,設存在節點 \m{u\in S} 和 \m{v\in T}, \m{(u,v)\in E_f}。
因爲 \m{h} 始終是高度函數(引理 26.17),有 \m{u.h\le v.h + 1}。
但同時 \m{v.h < \hat{h} < u.h},
又因爲所有高度都是整數, \m{v.h\le u.h - 2}。
因此, \m{u.h\le v.h + 1\le u.h - 2 + 1 = u.h - 1},
即 \m{0\le -1},顯然不成立。
因此 \m{(S,T)} 就是最小切割。

\stopANSWER

%e26.4-5
\startEXERCISE
給出一個有效的推送重貼標籤算法,
使其可以在一個二分圖中找到一個最大匹配,
並分析其效率。
\stopEXERCISE

\startANSWER
\TODO{略。}
\stopANSWER

%e26.4-6
\startEXERCISE
假定在流網絡 \m{G=(V,E)} 中所有邊的容量都在集合 \m{\{1,2,\ldots,k\}} 中。
分析通用推送重貼標籤算法的運行時間,
請以 \m{|V|}、 \m{|E|} 和 \m{k} 來予以表示。
(\hint 每條邊在變爲飽和之前可以支持多少次非飽和推送操作)
\stopEXERCISE

\startANSWER
\TODO{略。}
\stopANSWER

%e26.4-7
\startEXERCISE
證明:我們可以將 \ALGO{INITIALIZE-PREFLOW} 的第 6 行改爲:
\startformula
s.h = |G.V| - 2
\stopformula
而不會影響通用推送重貼標籤算法的正確性和漸進性能。
\stopEXERCISE

\startANSWER
如果設置 \m{s.h = |V|-2},則需要修改高度函數的定義。

要證明正確性,只需要更新引理 26.18 的證明。
原始證明推導出的是 \m{s.h\le k < |V|} 與 \m{s.h=|V|} 矛盾。
改成 \m{s.h=|V|-2} 之後,這個矛盾不存在了。

在原始證明中,假定有一簡單增廣路徑 \m{\langle v_0,v_1,\ldots,v_k\rangle},
其中 \m{v_0=s}, \m{v_k = t},因此 \m{k<|V|}。
 \m{(s,v_1)} 怎樣才能成爲殘存邊?
此邊在 \ALGO{INITIALIZE-PREFLOW} 中已經飽和了,
這意味着曾經有從 \m{v_1} 推送到 \m{s} 的操作。
爲了做到這一點,必須滿足 \m{v_1.h = s.h + 1}。
如果我們設置了 \m{s.h=|V|-2},
也就意味着 \m{v_1.h} 是 \m{|V|-1}。
在那之後, \m{v_1.h} 不再減小,因此 \m{v_1.h\ge |V|-1}。
回溯增廣路徑,有 \m{v_{k-i}.h\le t.h + i},
對於 \m{i=0,1,\ldots,k} 均成立。
回顧假設,由於增廣路徑是簡單路徑, \m{k<|V|}。
令 \m{i=k-1},有 \m{v_1.h\le t.h+k-1 < 0+|V|-1}。
現在就有矛盾了, \m{v_1.h \ge |V|-1} 和 \m{v_1.h < |V|-1}。
也就是說,引理 26.18 依然成立。

漸進分析沒什麼變化。
\stopANSWER

%e26.4-8
\startEXERCISE
設 \m{\delta_{f}(u,v)} 爲殘存網絡 \m{G_f} 中從節點 \m{u} 到節點 \m{v} 的距離(邊的條數)。
證明: \ALGO{GENERIC-PUSH-RELABEL} 維持性質 \m{u.h < |V|} 意味着 \m{u.h\le \delta_{f}(u,t)},
維持性質 \m{u.h \ge |V|} 意味着 \m{u.h-|V|\le \delta_{f}(u,s)}。
\stopEXERCISE

\startANSWER
\TODO{略。}
\stopANSWER

%e26.4-9
\startEXERCISE\DIFFICULT
如前一個練習,設 \m{\delta_{f}(u,v)} 爲殘存網絡 \m{G_f} 中從節點 \m{u} 到節點 \m{v} 的距離。
請說明如何修改通用推送重貼標籤算法,
使其維持性質 \m{u.h<|V|} 意味着 \m{u.h=\delta_{f}(u,t)},
維持性質 \m{u.h\ge |V|} 意味着 \m{u.h-|V|=\delta_{f}(u,s)}。
你所設計的算法中,維持該性質所用總時間應爲 \m{O(VE)}。
\stopEXERCISE

\startANSWER
\TODO{略。}
\stopANSWER

%e26.4-10
\startEXERCISE\DIFFICULT
證明:在流網絡 \m{G=(V,E)} 上運行 \ALGO{GENERIC-PUSH-RELABEL},
非飽和推送的次數爲 \m{4|V|^{2}|E|},假定 \m{|V|\ge 4}。
\stopEXERCISE

\startANSWER
\TODO{略。}
\stopANSWER

\stopsection

\startsection[
  title={The relabel-to-front algorithm},
]
\stopsection

\startsubject[
  title={Problems},
]

\stopsubject%Problems

\stopchapter
\stopcomponent
