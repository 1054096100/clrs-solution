\startsection[
  title={Aggregate analysis},
]
%e17.1-1
\startEXERCISE
如果棧操作包括 \ALGO{MULTIPUSH},
他將 \m{k} 個數據壓入棧中,
那麼棧操作的攤還代價的界還是 \m{O(1)} 嗎?
\stopEXERCISE

\startANSWER
不是了, \ALGO{MULTIPUSH} 序列攤還代價的界是 \m{O(k)}。
\stopANSWER

%e17.1-2
\startEXERCISE
證明:如果 \m{k} 位計數器的例子中允許 \ALGO{DECREMENT} 操作,
那麼 \m{n} 個操作的運行時間可能達到 \m{\Theta(nk)}。
\stopEXERCISE

\startANSWER
例如: \m{2^k -1 + 1 -1 + 1 - \ldots}。
\stopANSWER

%e17.1-3
\startEXERCISE[exercise:power_cost]
假定我們對一個數據結構執行一個由 \m{n} 個操作組成的操作序列,
當 \m{i} 爲 \m{2} 的冪時,第 \m{i} 個操作的代價爲 \m{i};否則代價爲 \m{1}。
使用聚合分析確定每個操作的攤還代價。
\stopEXERCISE

\startANSWER
總代價小於 \m{3n},攤還代價爲 \m{3}。
歸納遞推:
\startformula\startmathalignment
\NC \sum_{i=1}^{2^i} d_i \NC < 3 \times 2^i \NR
\NC  \NC \sum_{i=1}^{2^{i+1}} d_i \NR
\NC =\NC \sum_{i=1}^{2^i} d_i + 1 + 1 + \ldots + 1 + 2^{i+1} \NR
\NC <\NC 3\times 2^i + (2^i - 1) + 2^{i+1} \NR
\NC =\NC 6\times 2^i - 1 \NR
\NC <\NC 3\times 2^{i+1} \NR
\stopmathalignment\stopformula
\stopANSWER

\stopsection
