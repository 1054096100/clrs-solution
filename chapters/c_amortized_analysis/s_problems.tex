\startsubject[
  title={Problems},
]

%p17-1
\startPROBLEM
(Bit-reversed binary counter)
\refchapter{poly_fft}介紹了一個稱爲快速傅立葉變換(Fast Fourier Transform, FFT)的重要算法。
 FFT 算法的第一步是對一個輸入數組 \m{A[0..n-1]} 執行一個稱爲{\EMP 位逆序置換}(bit-reversal permutation)的操作,
數組長度爲 \m{n=2^k}, \m{k} 是一個非負整數。
這個置換操作將下標的二進制表示互爲逆序的數組元素進行交換。

我們將每個下標 \m{a} 表示爲一個 \m{k} 位二進制序列 \m{\langle a_{k-1},a_{k-2},\ldots,a_0\rangle},
其中 \m{a=\sum_{i=0}^{k-1}a_i 2^i}。我們定義:
\startformula
rev_k(\langle a_{k-1},a_{k-2},\ldots,a_0\rangle) = \langle a_0,a_1,\ldots,a_{k-1}\rangle
\stopformula
因此:
\startformula
rev_k(a) = \sum_{i=0}^{k-1}a_{k-i-1} 2^i
\stopformula
例如,若 \m{n=16}(或等價地, \m{k=4}),則 \m{rev_k(3) = 12},
因爲 3 的 4 位二進制表示爲 \m{0011},其逆序爲 \m{1100},
是 12 的 4 位二進制表示。
\startigBase[a]
\startitem
設計一個運行時間爲 \m{\Theta(k)} 的函數 \m{rev_k},
編寫算法在 \m{O(nk)} 時間內對長度爲 \m{n=2^k} 的數組執行位逆序置換。
\stopitem

\startANSWER
略。
\stopANSWER
\stopigBase

我們可以使用一個基於攤還分析的算法來改進位逆序置換操作的運行時間。
我們可以維護一個“位逆序計數器”,
並設計一個算法 \ALGO{BIT-REVERSED-INCREMENT},
當給定一個位逆序計數器值 \m{a},
該過程能得到 \m{rev_k(rev_k(a)+1)}。
例如,若 \m{k=4},
位逆序計數器從 0 開始,
則連續調用 \ALGO{BIT-REVERSED-INCREMENT} 會得到序列:
\startformula
0000,1000,0100,1100,0010,1010,\ldots = 0,8,4,12,2,10,\ldots
\stopformula
\startigBase[a,continue]
\startitem
假定你的計算機的每個機器字保存 \m{k} 位二進制數,
而一個機器字中的值進行一次任意偏移量的左/右移位、按位與、按位或等操作只需單位時間。
設計一個算法 \ALGO{BIT-REVERSED-INCREMENT},
能使一個 \m{n} 個元素的數組上的位逆序置換操作在 \m{O(n)} 時間內完成。
\stopitem

\startANSWER
與二進制計數器的程序一樣,只是迭代時方向正好相反。
\stopANSWER

\startitem
假定在單位時間內你只能完成左/右移一位的操作。
還可能實現 \m{O(n)} 時間的位逆序置換操作嗎?
\stopitem

\startANSWER
可以。
\stopANSWER
\stopigBase
\stopPROBLEM

%p17-2
\startPROBLEM
(making binary search dynamic)
有序數組上的二分查找花費對數時間,
但插入一個新元素的時間與數組規模呈線性關係。
我們可以通過維護多個有序數組來提高插入性能。

具體地,假定我們希望支持 \m{n} 元集合上的 \ALGO{SEARCH} 和 \ALGO{INSERT} 操作。
令 \m{k=\lceil \lg{(n+1)}\rceil},
令 \m{n} 的二進制表示爲 \m{\langle n_{k-1},n_{k-2},\ldots,n_0\rangle}。
我們維護 \m{k} 個有序數組 \m{A_0,A_1,\ldots,A_{k-1}},
對 \m{i=0,1,\ldots,{k-1}},數組 \m{A_i} 的長度爲 \m{2^i}。
每個數組或滿或空,
取決於 \m{n_i=1} 還是 \m{n_i=0}。
因此,所有 \m{k} 個數組中保存的元素總數爲 \m{\sum_{i=0}^{k-1}n_i 2^i = n}。
雖然單獨每個數組都是有序的,
但不同數組中的元素之間不存在特定的大小關係。

\startigBase[a]
\startitem
設計算法,實現這種數據結構上的 \ALGO{SEARCH} 操作,分析其最壞情況運行時間。
\stopitem

\startANSWER
對於每個數組都用二分查找進行搜索,知道找到想要的元素。
最壞情況下需要搜索所有數組:
\startformula\startmathalignment
\NC T(n) \NC = \Theta(\lg 1 + \lg 2 + \lg 2^2 + \ldots + \lg 2^{k-1}) \NR
\NC \NC = \Theta(0+1+\ldots+(k-1)) \NR
\NC \NC = \Theta(\frac{1}{2}k(k-1)) \NR
\NC \NC = \Theta(\lg^2 n) \NR
\stopmathalignment\stopformula

從按數組大小遞減順序搜索,在某些情況下性能會更好一些,但不影響最壞情況的結果。
\stopANSWER

\startitem
設計 \ALGO{INSERT} 算法。分析最壞情況運行時間和攤還時間。
\stopitem

\startANSWER
找到最小的空數組 \m{A_i},將新元素插入 \m{A_i},
並將 \m{A_0,A_1,\ldots,A_{i-1}} 中所有元素全部轉移到 \m{A_i} 中。
最壞情況下,需要移動所有 \m{n+1} 個元素。

用聚合法分析攤還代價。
對於 \m{n} 次插入操作,數組 \m{A_i} 會改變 \m{n/2^i} 次,
此處所謂改變,指數組由空變滿,或由滿變空。
因此:
\startformula
\sum_{i=0}^{k-1}\frac{n}{2^i}2^i = \sum_{i=0}^{k-1}n = nk \in O(n\lg n)
\stopformula
即攤還代價爲 \m{O(\lg n)}。
\stopANSWER

\startitem
討論如何實現 \ALGO{DELETE}。
\stopitem

\startANSWER
\ALGO{DELETE} 花的時間不可能比 \ALGO{SEARCH} 更少。
實際上, \ALGO{DELETE} 最快也需要線性時間,除非我們以其他方式描述元素間的關係。
考慮以下,如果要刪除最長數組 \m{A_{k-1}} 正中間的元素。
刪除後,至少得有一個元素 \m{e} 從其他數組中轉移到 \m{A_{k-1}} 中。
但由於我們不知道 \m{e} 與 \m{A_{k-1}} 中其他元素的關係,
重排 \m{A_{k-1}} 需要時間爲 \m{2^{k-1}\in O(n)}。
由於每次刪除都需要這麼多時間,攤還代價也是 \m{O(n)}。

找到最小非空數組 \m{A_i},並找到要刪除的元素。
假定要刪除的元素在數組 \m{A_j} 中,將其刪除。
從 \m{A_i} 中轉移一個元素到 \m{A_j} 中(如果 \m{i=j},則保持不變)。
然後將 \m{A_i} 中剩餘元素轉移到 \m{A_0,A_1,\ldots,A_{i-1}} 中。
最後,重排 \m{A_j}。
\stopANSWER
\stopigBase
\stopPROBLEM

%p17-3
\startPROBLEM
(amortized weight-balanced tree)
考慮擴充普通而叉搜索樹,
爲每個節點 \m{x} 增加屬性 \m{x.size},
此屬性給出了根爲 \m{x} 的子樹中關鍵字的數量。
另 \m{\alpha} 是 \m{1/2\le \alpha < 1} 之間的一個常數。
當 \m{x.left.size\le \alpha\cdot x.size} 且 \m{x.right.size\le \alpha\cdot x.size} 時,
我們稱節點 \m{x} 是 {\EMP \m{\alpha} 平衡的}。
如果樹中每個節點都是 \m{\alpha} 平衡的,
則稱樹整體是 {\EMP \m{\alpha} 平衡的}。
 G.Varghese 提出了如下攤還方法來維護加權平衡樹。
\startigBase[a]\startitem
在某種意義上,一棵 \m{1/2} 平衡樹達到了極限的平衡。
給定任意一棵而叉搜索樹中的一個節點 \m{x},
證明:如何重建以 \m{x} 爲根的子樹,
使得他變爲 \m{1/2} 平衡的。
你的算法運行時間應爲 \m{\Theta(x.size)},
可以使用 \m{O(x.size)} 的輔助空間。
\stopitem\stopigBase

\startANSWER
首先中序遍歷子樹,並按順序將所有元素存儲在輔助空間中。
然後選擇數組中中間元素作爲子樹的根,並依此遞歸構建其左、右兩棵子樹。
所用時間以及所用輔助空間均爲 \m{O(x.size)}。
\stopANSWER

\startigBase[continue]\startitem
證明:在一棵 \m{n} 個節點的 \m{\alpha} 平衡二叉搜索樹中執行一次搜索操作的最壞情況時間爲 \m{O(\lg n)}。
\stopitem\stopigBase

\startANSWER
令 \m{f(n)} 爲所需時間,則 \m{f(n)\le f(\alpha n) + O(1)}。
求解得 \m{f(n)\in O(\lg_{1/\alpha}n) = O(\lg n)}。
\stopANSWER

對於本問題的剩餘部分,假定常數 \m{\alpha} 嚴格大於 \m{1/2}。
假定你實現的 \ALGO{INSERT} 和 \ALGO{DELETE} 算法與普通 \m{n} 節點二叉搜索樹的算法是一樣的,
差別僅在於,如果發現樹中任何節點不再是 \m{\alpha} 平衡的,
則在最高的不平衡節點,
對以他爲根的子樹執行“重建”,
使其變爲 \m{1/2} 平衡的。

我們用勢能法分析此重建方法。
對於二叉搜索樹 \m{T} 中的一個節點 \m{x},定義:
\startformula
\Delta(x) = |x.left.size - x.right.size|
\stopformula
定義 \m{T} 的勢函數爲:
\startformula
\Phi(T) = c \sum_{x\in T: \Delta(x)\ge 2}\Delta(x)
\stopformula
其中 \m{c} 是一個足夠大的常數,他依賴於 \m{\alpha}。

\startigBase[continue]\startitem
證明:任意二叉搜索樹的勢都是非負的, \m{1/2} 平衡樹的勢爲 \m{0}。
\stopitem\stopigBase

\startANSWER
略。
\stopANSWER

\startigBase[continue]\startitem
假定 \m{m} 個單位的勢夠支付重建 \m{m} 節點子樹的代價。
相對於 \m{\alpha} 來說, \m{c} 應該多大才能使得重建一棵非 \m{\alpha} 平衡的子樹攤還時間爲 \m{O(1)}。
\stopitem\stopigBase

\startANSWER
重建前:
\startformula
\Delta(x)
= |x.left.size - x.right.size|
\ge \alpha \cdot x.size - (1-\alpha)\cdot x.size
= (2\alpha - 1)\cdot x.size
\stopformula
即重建前, \m{\Phi(T) \ge c \cdot \Delta(x) \ge c\cdot (2\alpha - 1)\cdot x.size}。

而重建後 \m{\Phi(T) = 0}。因此必須滿足 \m{\Phi(T) \ge x.size},其中等式右側爲重建代價。

如果可以保證 \m{c\dot(2\alpha-1) \cdots x.size \ge x.size},則就可以滿足要求。
求解得:
\startformula
c\ge \frac{1}{2\alpha -1}
\stopformula
\stopANSWER

\startigBase[continue]\startitem
證明:在一棵 \m{n} 節點的 \m{\alpha} 平衡樹中插入一個節點或刪除一個節點的攤還時間爲 \m{O(\lg n)}。
\stopitem\stopigBase

\startANSWER
令攤還代價爲 \m{\hat{c_i}}。
如果沒有觸發重構,則:
\startformula
\hat{c_i} = c_i + \Phi(D_i) - \Phi(D_{i-1})
\stopformula
由 (b) 可知, \m{c_i \in O(\lg n)}。
類似,樹高也是 \m{O(\lg n)}。
無論是插入還是刪除節點 \m{v} 都會改變從他到根節點路徑上所有節點的 \m{\Delta(x)}。
因此:
\startformula
\Phi(D_i) - \Phi(D_{i-1}) \le c\cdot O(\lg n) \in O(\lg n)
\stopformula

如果觸發了重構,令 \m{c_i = c_i' + c_i''},
其中 \m{c_i'} 是插入或刪除的代價,而 \m{c_i''} 是重構的代價。
令 \m{D_i'} 是插入或刪除之後、重構之前的狀態。那麼:
\startformula
\hat{c_i}=c_i' + c_i'' + (\Phi(D_i) - \Phi(D_i')) + (\Phi(D_i') - \Phi(D_{i-1}))
\stopformula

\m{\Phi(D_i')} 是重構之前的勢, \m{\Phi(D_i)} 是刪除之後的勢,
 \m{\Phi(D_i')-\Phi(D_i)} 是重構後減少的勢。根據上一個問題:
\startformula
c_i'' \le \Phi(D_i')-\Phi(D_i)
\stopformula
因此:
\startformula\startmathalignment
\NC \hat{c_i} \NC = c_i' + c_i'' + (\Phi(D_i) - \Phi(D_i')) + (\Phi(D_i') - \Phi(D_{i-1})) \NR
\NC \NC \le O(\lg n) + c_i'' - (\Phi(D_i')) - \Phi(D_i)) + O(\lg n) \NR
\NC \NC \le O(\lg n) \NR
\stopmathalignment\stopformula

\stopANSWER

\stopPROBLEM

%p17-4
\startPROBLEM
(The cost of restructuring red-black tree)
紅黑樹有 4 種基本的{\EMP 結構性修改}(structural modification)操作:
插入、刪除、旋轉及更改顏色。
我們已經看到 \ALGO{RB-INSERT} 和 \ALGO{RB-DELETE} 操作僅使用 \m{O(1)} 次旋轉、
插入和刪除操作來維持紅黑樹的性質,
但可能會多次更改顏色。
\startigBase[a]\startitem
設計一個 \m{n} 節點的合法的紅黑樹,
使得調用 \ALGO{RB-INSERT} 添加第 \m{n+1} 個節點會引起 \m{\Omega(\lg n)} 次顏色更改。
然後設計一個 \m{n} 節點的合法紅黑樹,
使得調用 \ALGO{RB-DELETE} 刪除一個特定節點會引起 \m{\Omega(\lg n)} 次顏色更改。
\stopitem\stopigBase

\startANSWER
略。
\stopANSWER

雖然最壞情況下,每個操作更改顏色的次數可能是對數的,
但我們可以證明,在一個空紅黑樹上執行任意 \m{m} 個 \ALGO{RB-INSERT} 和 \ALGO{RB-DELETE} 操作序列,
最壞情況也只會引起 \m{O(m)} 次結構性修改。
注意:每次更改顏色都算作一次結構性修改。

\startigBase[continue]\startitem
\ALGO{RB-INSERT-FIXUP} 和 \ALGO{RB-DELETE-FIXUP} 的主循環都處理一些{\EMP 終止}條件:
一旦條件得到滿足,會導致循環在常數次操作後終止。
對於 \ALGO{RB-INSERT-FIXUP} 和 \ALGO{RB-DELETE-FIXUP} 中處理的各種情況,
指出其中哪些會導致循環終止,哪些不會。
(\hint 參見圖 13-5、圖 13-6 和圖 13-7)
\stopitem\stopigBase

\startANSWER
略。
\stopANSWER

我們首先分析僅僅執行插入操作時引起的結構性修改。
另 \m{T} 爲一棵紅黑樹,
定義 \m{\Phi(T)} 是 \m{T} 中紅節點數量。
假定一個單位的勢可以支付 \ALGO{RB-INSERT-FIXUP} 的三種情況的任意一種所引起的結構性修改的代價。

\startigBase[continue]\startitem
令 \m{T'} 表示對 \m{T} 應用 \ALGO{RB-INSERT-FIXUP} 中 case 1 得到結果。
證明: \m{\Phi(T')=\Phi(T) - 1}。
\stopitem\stopigBase

\startANSWER
略。
\stopANSWER

\startigBase[continue]\startitem
當使用 \ALGO{RB-INSERT} 向一棵紅黑樹中插入一個節點時,
我們可以將操作分解爲三部分。
列出 \ALGO{RB-INSERT} 的第 1 ~ 16 行引起的結構性改變和勢的變化,
以及 \ALGO{RB-INSERT-FIXUP} 中各種情況引起的變化。
\stopitem\stopigBase

\startANSWER
略。
\stopANSWER

\startigBase[continue]\startitem
用(d)證明:
任意一次 \ALGO{RB-INSERT} 所導致的結構性修改的攤還次數爲 \m{O(1)}。
\stopitem\stopigBase

\startANSWER
略。
\stopANSWER

我們現在希望證明既有插入又有刪除時,所引起的結構性修改次數仍爲 \m{O(m)}。
對每個節點 \m{x},我們定義:
\startformula
\omega(x) = \startcases
\NC 0 \NC \text{若 \m{x} 是紅節點} \NR
\NC 1 \NC \text{若 \m{x} 是黑節點且沒有紅孩子} \NR
\NC 0 \NC \text{若 \m{x} 是黑節點且有一個紅孩子} \NR
\NC 2 \NC \text{若 \m{x} 是黑節點且由兩個紅孩子} \NR
\stopcases
\stopformula
現在定義紅黑樹 \m{T} 的勢函數爲:
\startformula
\Phi(T) = \sum_{x\in T}\omega(x)
\stopformula
且令 \m{T'} 爲對 \m{T} 應用 \ALGO{RB-INSERT-FIXUP} 或 \ALGO{RB-DELETE-FIXUP} 的任意非終結性情況後的結果。

\startigBase[continue]\startitem
證明:
對 \ALGO{RB-INSERT-FIXUP} 的任意非終結性情況,有 \m{\Phi(T')\le \Phi(T) - 1}。
證明: \ALGO{RB-INSERT-FIXUP} 的任意一次調用所引起的結構性修改的攤還次數爲 \m{O(1)}。
\stopitem\stopigBase

\startANSWER
略。
\stopANSWER

\startigBase[continue]\startitem
證明:
對 \ALGO{RB-DELETE-FIXUP} 的任意非終結性情況,有 \m{\Phi(T')\le \Phi(T) - 1}。
證明: \ALGO{RB-DELETE-FIXUP} 的任意一次調用所引起的結構性修改的攤還次數爲 \m{O(1)}。
\stopitem\stopigBase

\startANSWER
略。
\stopANSWER

\startigBase[continue]\startitem
證明:
由任意 \m{m} 個 \ALGO{RB-INSERT-FIXUP} 和 \ALGO{RB-DELETE-FIXUP} 構成的序列最壞情況下執行 \m{O(m)} 次結構性修改。
\stopitem\stopigBase

\startANSWER
略。
\stopANSWER

\stopPROBLEM

%p17-5
\startPROBLEM
(Competitive analysis of self-organizing lists with move-to-front)
{\EMP 自組織列表}是 \m{n} 個元素的鏈表,
每個元素有一個唯一的關鍵字。
當我們在列表中搜索元素時,需喲啊給定一個關鍵字,
我們搜索的是具有這個關鍵字的元素。
自組織列表由兩個重要性質:
\startigBase[n]
\startitem
爲了在列表中查找一個元素,我們必須從表頭開始遍歷列表,
直至遇到具有給定關鍵字的元素。
如果此元素是列表的第 \m{k} 個元素,則查找代價爲 \m{k}。
\stopitem
\startitem
我們可以在任意個操作後根據給定規則重排列表元素,產生一定的代價。
我們可以使用任何我們喜歡的啓發式策略來決定如何重排表。
\stopitem
\stopigBase

假定從一個給定的 \m{n} 個元素的列表開始,
並且給定了一個訪問序列——關鍵字搜索序列 \m{\sigma=\langle \sigma_1,\sigma_2,\ldots,\sigma_m\rangle}。
序列的代價是序列中單個訪問的代價之和。

在多種可能的列表重排方法中,
本問題關注相鄰元素轉置操作——交換相鄰元素在列表中的位置,
一次轉置的代價爲單位時間。
你可以用勢函數證明:
針對移至前端列表的重排問題,
一種特定的啓發式策略的代價最壞情況也不會超過任何其他啓發式策略代價的 4 倍,
即使其他啓發式策略預先知道訪問序列!
我們稱這種分析爲{\EMP 競爭分析}。

對於一個啓發式策略 \m{H} 和列表的一個給定初始順序,
我們將序列 \m{\sigma} 的訪問代價記爲 \m{C_H(\sigma)}。
令 \m{m} 表示 \m{\sigma} 中訪問的數量。

\startigBase[a]\startitem
證明:若啓發式策略 \m{H} 預先不知道訪問序列,
那麼利用 \m{H} 來處理訪問序列 \m{\sigma} 的最壞情況代價爲 \m{C_H(\sigma)=\Omega(mn)}。
\stopitem\stopigBase

\startANSWER
略。
\stopANSWER

當使用{\EMP 移至前端}啓發式策略時,
搜索到元素 \m{x} 後,
我們將 \m{x} 移動到列表的第一個位置(即列表的前端)。

令 \m{rank_L(x)}表示元素 \m{x} 在列表 \m{L} 中的序號,
即 \m{x} 在 \m{L} 中的位置。
例如,若 \m{x} 是 \m{L} 中第 4 個元素,
那麼 \m{rank_L(x)=4}。
令 \m{c_i} 表示用移至前端策略處理訪問 \m{\sigma_i} 的代價,
包括在列表中查找元素的代價和通過一系列相鄰元素轉置操作將其移至列表前端的代價。

\startigBase[continue]\startitem
證明:如果 \m{\sigma_i} 使用移至前端策略在 \m{L} 中訪問元素 \m{x},
則 \m{c_i=2\cdot rank_L(x)-1}。
\stopitem\stopigBase

\startANSWER
略。
\stopANSWER

現在我們比較移至前端策略與其他任何按照上述兩個性質處理訪問序列的啓發式策略 \m{H}。
策略 \m{H} 可能按任何他想用的方式轉置列表中的元素,
他甚至可能預先知道整個訪問序列。

令 \m{L_i} 表示使用移至前端策略處理訪問 \m{\sigma_i} 後得到的列表,
 \m{L_i^*} 表示使用策略 \m{H} 後得到的列表。
我們用 \m{c_i} 表示移動前端策略的代價, \m{c_i^*} 表示策略 \m{H} 的代價。
假定策略 \m{H} 在處理訪問 \m{\sigma_i} 時執行 \m{t_i^*} 次轉置。

\startigBase[continue]\startitem
在(b)中,你證明了 \m{c_i=2\cdot rank_{L_{i-1}}(x)-1}。
現在證明: \m{c_i^* = rank_{L_{i-1}^*}(x) + t_i^*}。
\stopitem\stopigBase

\startANSWER
\stopANSWER

我們定義{\EMP 逆序}(inversion)關係:
 \m{L_i} 中一對元素 \m{y} 和 \m{z},
在 \m{L_i} 中 \m{y} 在 \m{z} 之前,
在 \m{L_i^*} 中 \m{z} 在 \m{y} 之前。
假定處理完訪問序列 \m{\langle \sigma_1,\sigma_2,\ldots,\sigma_i\rangle},
 \m{L_i} 中有 \m{q_i} 個逆序關係。
然後,我們定義一個勢函數 \m{\Phi},
將 \m{L_i} 映射到實數 \m{\Phi(L_i)=2q_i}。
例如,如果 \m{L_i} 有元素 \m{\langle e,c,a,d,b\rangle},
而 \m{L_i^*} 有元素 \m{\langle c,a,b,d,e\rangle},
那麼 \m{L_i} 有 5 個逆序 \m{((e,c),(e,a),(e,d),(e,b),(d,b))},
因此 \m{\Phi(L_i)=10}。
觀察到對所有 \m{i},都有 \m{\Phi(L_i)\ge 0},
並且如果移至前端策略和策略 \m{H} 從相同的列表 \m{L_0} 開始,
那麼 \m{\Phi(L_0)=0}。

\startigBase[continue]\startitem
證明:轉置操作要麼將勢增加 2,要麼減少 2。
\stopitem\stopigBase

\startANSWER
略。
\stopANSWER

假定訪問 \m{\sigma_i} 查找元素 \m{x}。
爲了理解勢是如何根據 \m{\sigma_i} 來變化的,
我們將除 \m{x} 之外的元素劃分爲 4 個集合,
劃分的依據是在第 \m{i} 次訪問之前他們在列表中的位置:

集合 \m{A} 包含在 \m{L_{i-1}} 和 \m{L_{i-1}^*} 中都位於 \m{x} 之前的元素。

集合 \m{B} 包含在 \m{L_{i-1}} 中位於 \m{x} 之前的元素,在 \m{L_{i-1}^*} 中位於 \m{x} 之後的元素。

集合 \m{C} 包含在 \m{L_{i-1}} 中位於 \m{x} 之後的元素,在 \m{L_{i-1}^*} 中位於 \m{x} 之前的元素。

集合 \m{D} 包含在 \m{L_{i-1}} 和 \m{L_{i-1}^*} 中都位於 \m{x} 之後的元素。

\startigBase[continue]\startitem
證明: \m{rank_{L_{i-1}}(x) = |A| + |B| + 1} 且 \m{rank_{L_{i-1}^*}(x) = |A| + |C| + 1}。
\stopitem\stopigBase

\startANSWER
略。
\stopANSWER

\startigBase[continue]\startitem
證明:處理訪問 \m{\sigma_i} 會引起勢的變化:
\startformula
\Phi(L_i) - \Phi(L_{i-1}) \le 2(|A| - |B| + t_i^*)
\stopformula
其中, \m{t_i^*} 表示用啓發式策略 \m{H} 處理訪問 \m{\sigma_i} 期間執行的轉置操作次數。
\stopitem\stopigBase

\startANSWER
略。
\stopANSWER

我們定義處理訪問 \m{\sigma_i} 的攤還代價爲 \m{\hat{c_i}=c_i + \Phi(L_i) - \Phi(L_{i-1})}。
\startigBase[continue]\startitem
證明:處理訪問 \m{\sigma_i} 的攤還代價上界爲 \m{4c_i^*}。
\stopitem\stopigBase

\startANSWER
略。
\stopANSWER
\startigBase[continue]\startitem
證明:使用移至前端策略處理訪問序列 \m{\sigma} 的代價 \m{C_{MTF}(\sigma)} 至多
是用其他任何啓發式策略 \m{H} 處理 \m{\sigma} 的代價 \m{C_H(\sigma)} 的 4 倍,
假定兩種啓發式策略都是從相同的列表開始處理訪問序列。
\stopitem\stopigBase

\startANSWER
略。
\stopANSWER

\stopPROBLEM

\stopsubject%Problems
