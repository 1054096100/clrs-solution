\startcomponent c_single_source_shortest_paths

\startchapter[
  title={Single-Source Shortest Paths},
]

\startsection[
  title={The Bellman-Ford algorithm},
  reference=section:24.1,
]

%e24.1-1
\startEXERCISE
在圖 24-4 上運行 Bellman-Ford 算法,
使用節點 \m{z} 作爲源節點。
在每一遍鬆弛過程中,以圖中相同的次序對每條邊進行鬆弛,
給出每遍鬆弛操作後的 \m{d} 值和 \m{\pi} 值。
然後,把邊 \m{(z,x)} 的權重改爲 4,
再次運行該算法,這次使用 \m{s} 作爲源節點。
\stopEXERCISE

\startANSWER
從 \m{z} 點開始:

\startcombination[nx=3,ny=2]
{\externalfigure[output/e24_1_1-1]}{}
{\externalfigure[output/e24_1_1-2]}{}
{\externalfigure[output/e24_1_1-3]}{}
{\externalfigure[output/e24_1_1-4]}{}
{\externalfigure[output/e24_1_1-5]}{}
\stopcombination

改了 \m{(z,x)} 的權重後,會出現總權重爲負值的環路。

\externalfigure[output/e24_1_1-10]
\stopANSWER

%e24.1-2
\startEXERCISE
證明推論 24.3。附推論 24.3:

設 \m{G=(V,E)} 是一帶權重的源節點爲 \m{s} 的有向圖,
其權重函數爲 \m{\omega: E\rightarrow R}。
假定圖 \m{G} 不包含從 \m{s} 可以到達的權重爲負值的環路,
則對於所有節點 \m{v\in V},
存在一條從源節點 \m{s} 到節點 \m{v} 的路徑
當且僅當 \ALGO{BELLMAN-FORD} 算法終止時有 \m{v.d < \infty}。
\stopEXERCISE

\startANSWER
首先,假設 \m{G} 中存在 \m{s} 到 \m{v} 的路徑,
證明算法終止時  \m{v.d < \infty}。
根據引理 24.2 可知 \m{v.d=\delta(s,v)},滿足 \m{v.d < \infty}。

現在,假定算法 \ALGO{BELLMAN-FORD} 終止時 \m{v.d<\infty}。
而根據引理 24.2,可知 \m{v.d=\infty} 或者 \m{v.d=\delta(s,v)}。
根據假設可知 \m{v.d=\delta(s,v)}。
因此 \m{G} 中存在從 \m{s} 到 \m{v} 的路徑。

附引理 24.2:

設 \m{G=(V,E)} 爲一個帶權重的源節點爲 \m{s} 的有向圖,
其權重函數爲 \m{\omega: E\rightarrow R}。
假定圖 \m{G} 不包含從 \m{s} 可以到達的權重爲負值的環路。
那麼算法 \ALGO{BELLMAN-FORD} 的第 2~4 行的 {\EMP for} 循環
執行了 \m{|V|-1} 次之後,
對於所有從源節點 \m{s} 可以到達的節點 \m{v},
我們有 \m{v.d=\delta(s,v)}。
\stopANSWER

%e24.1-3
\startEXERCISE
給定有向圖 \m{G=(V,E)} 帶權重且沒有權重爲負值的環路,
對於所有節點 \m{v\in V},
從源節點 \m{s} 到節點 \m{v} 之間的最短路徑中,
包含邊的條數的最大值爲 \m{m}。
(這裏,判斷最短路徑的根據是權重,不是邊的條數。)
請修改算法 \ALGO{BELLMAN-FORD},
讓其可以在 \m{m+1} 遍鬆弛操作後終止,
即使事先不知道 \m{m} 的值。
\stopEXERCISE

\startANSWER
如果一次遍歷過程中沒有任何節點需要鬆弛操作,則終止循環。
\stopANSWER

%e24.1-4
\startEXERCISE[exercise:24.1-4]
修改算法 \ALGO{BELLMAN-FORD},使其對於所有節點 \m{v} 來說,
如果從源節點 \m{s} 到節點 \m{v} 的一條路徑上存在權重爲負值的環路,
則將 \m{v.d} 的值設置爲 \m{-\infty}。
\stopEXERCISE

\startANSWER
\CLRSH{BELLMAN-FORD-B(G,w,s)}
\startCLRS
INITIALIZE-SINGLE-SOURCE(G,s)
for i = 1 to |G.V| - 1
	for each edge (u,v) in G.E
		RELAX(u,v,w)
for each edge (u,v) in G.E
	if v.d > u.d + w(u,v)
		v.d = -infty
		let t = v.pi
		while t != NIL and t.d != -infty
			t.d = -infty
			t = t.pi
for each edge (u,v) in G.E
	if v.d > u.d + w(u,v)
		return FALSE
return TRUE
\stopCLRS
\stopANSWER

%e24.1-5
\startEXERCISE\DIFFICULT
設 \m{G=(V,E)} 爲一個帶權重的有向圖,
其權重函數爲 \m{\omega: E\rightarrow R}。
請給出一個時間複雜度爲 \m{O(VE)} 的算法,
對於每個節點 \m{v\in V},
計算出數值 \m{\delta*(v)=\min_{u\in V}\{\delta(u,v)\}}。
\stopEXERCISE

\startANSWER
\CLRSH{RELAX-MOD(u,v,w)}
\startCLRS
min = w(u,v) + (u.d < 0 ? u.d : 0)
if v.d > min
	v.d = min
\stopCLRS

\ALGO{BELLMAN-FORD} 的時間複雜度不變,仍爲 \m{O(VE)}。
\stopANSWER

%e24.1-6
\startEXERCISE\DIFFICULT
設 \m{G=(V,E)} 爲一個帶權重的有向圖,
且圖中存在權重爲負值的環路。
請給出一個有效的算法來列出所有屬於該環路上的節點。
並證明算法的正確性。
\stopEXERCISE

\startANSWER
利用\refexercise{24.1-4} 的代碼。
\stopANSWER

\stopsection

\startsection[
  title={Single-source shortest paths in directed acyclic graphs},
]

%e24.2-1
\startEXERCISE
請在圖 24-5 上運行 \ALGO{DAG-SHORTEST-PATHS},使用節點 \m{r} 作爲源節點。
\stopEXERCISE

\startANSWER
\externalfigure[output/e24_2_1-6]
\stopANSWER

%e24.2-2
\startEXERCISE
假定將 \ALGO{DAG-SHORTEST-PATHS} 的第 3 行改爲:
\startCLRS
for the first |V| - 1 vertices, taken in topologically sorted order
\stopCLRS
證明:該算法的正確性保持不變。
\stopEXERCISE

\startANSWER
拓撲排序後,沒有從最後一個頂點出發的邊,因此不需要遍歷最後一個頂點。
\stopANSWER

%e24.2-3
\startEXERCISE
上面描述的 PERT 圖的公式有一點不大自然。
在一個更自然的結構下,
圖中的節點代表要執行的工作,
邊代表工作之間的次序限制,
即邊 \m{(u,v)} 表示工作 \m{u} 必須在工作 \m{v} 之前執行。
在這種結構的圖中,我們將權重賦給節點,而不是邊。
請修改 \ALGO{DAG-SHORTEST-PATHS} 過程,
使其可以在線性時間內找出這種有向無環圖一條最長的路徑。
\stopEXERCISE

\startANSWER
替換 \ALGO{DAG-SHORTEST-PATHS} 中的兩個函數:

\CLRSH{NEW-INITIALIZE-SINGLE-SOURCE(G,s,w)}
\startCLRS
for each vertex v in G.V
	v.d = infty
	v.pi = NIL
s.d = w(s)
\stopCLRS

\CLRSH{NEW-RELAX(u,v,w)}
\startCLRS
if v.d > u.d + w(v)
	v.d = u.d + w(v)
	v.pi = u
\stopCLRS
\stopANSWER

%e24.2-4
\startEXERCISE
給出一個有效算法計算有向無環圖中的路徑總數,
並分析其時間複雜度。
\stopEXERCISE

\startANSWER
\m{s.c = 1}, \m{v.c = \sum_{(u,v)\in E}u.c}。

時間複雜度爲 \m{\Theta(V+E)}。

\CLRSH{DAG-PATHS(G,s)}
\startCLRS
topologically sort the vertices of G

for each vertex v in G.V
	v.c = 0
s.c = 1

for each vertex u, taken in topologically sorted order
	for each vertex v in u.Adj
		v.c = v.c + u.c
\stopCLRS
\stopANSWER

\stopsection

\startsection[
  title={Dijkstra’s algorithm},
]

%e24.3-1
\startEXERCISE
在圖 24-2 上運行 Dijkstra 算法,
第一次使用節點 \m{s} 作爲源節點,
第二次使用節點 \m{z} 作爲源節點。
以類似於圖 24-6 的風格,給出每次 {\EMP while} 循環後的 \m{d} 和 \m{\pi},
以及集合 \m{S} 中的所有節點。
\stopEXERCISE

\startANSWER
從 \m{s} 開始:

\startcombination[3*2]
{\externalfigure[output/e24_3_1-1]}{a}
{\externalfigure[output/e24_3_1-2]}{b}
{\externalfigure[output/e24_3_1-3]}{c}
{\externalfigure[output/e24_3_1-4]}{d}
{\externalfigure[output/e24_3_1-5]}{e}
{\externalfigure[output/e24_3_1-6]}{f}
\stopcombination

從 \m{z} 開始:

\startcombination[3*2]
{\externalfigure[output/e24_3_1-7]}{a}
{\externalfigure[output/e24_3_1-8]}{b}
{\externalfigure[output/e24_3_1-9]}{c}
{\externalfigure[output/e24_3_1-10]}{d}
{\externalfigure[output/e24_3_1-11]}{e}
{\externalfigure[output/e24_3_1-12]}{f}
\stopcombination
\stopANSWER

%e24.3-2
\startEXERCISE
請舉出一個包含負權重的有向圖,
使 Dijkstra 算法在騎上運行時將產生不正確的結果。
爲什麼有負權重的情況下,定理 24.6 的證明不能成立?
\stopEXERCISE

\startANSWER
Dijkstra 算法原理:每次新拓展一個最近的點,
就更新與其相鄰的點的距離。
當所有邊權重均爲正值時,不會存在一個距離更短的沒有拓展過的點。
所以這個點的距離永遠不會被改變,因而保證了算法的正確性。
而一旦邊的權重有負值,這個假設就不成立了。
\stopANSWER

%e24.3-3
\startEXERCISE
假定將 Dijkstra 算法第 4 行改爲:
\startCLRS
while |Q| > 1
\stopCLRS
這種改變將讓 {\EMP while} 循環的執行才樹從 \m{|V|} 次降爲 \m{|V|-1} 次。
這樣修改後的算法正確嗎?
\stopEXERCISE

\startANSWER
正確。
\stopANSWER

%e24.3-4
\startEXERCISE
Gaedel 教授寫了一個還曾需,他聲稱該程序實現了 Dijkstra 算法。
對於每個節點 \m{v\in V},
該程序生成 \m{v.d} 和 \m{v.\pi}。
請給出一個時間複雜度爲 \m{O(V+E)} 的算法來檢查教授所編寫程序的輸出。
該算法應該判斷每個節點的 \m{d} 和 \m{\pi} 屬性是否與某棵最短路徑樹中的信息匹配。
這裏可以假設所有邊的權重均非負。
\stopEXERCISE

\startANSWER
沿最短路徑樹上的邊執行鬆弛操作。
\stopANSWER

%e24.3-5
\startEXERCISE
Newman 教授覺得自己發現了 Dijkstra 算法的一個更簡單的證明。
他聲稱 Dijkstra 算法對最短路徑上面的每條邊的鬆弛次序與該條邊在該條最短路徑中的次序相同,
因此,路徑鬆弛性質適用於從源節點可以到達的所有節點。
請構造一個有向圖來說明 Dijkstra 算法並不一定按照最短路徑中邊的出現次序來對邊進行鬆弛,
從而證明教授是錯的。
\stopEXERCISE

\startANSWER
\TODO{略。}
\stopANSWER

%e24.3-6
\startEXERCISE
給定有向圖 \m{G=(V,E)},
每條邊 \m{(u,v)\in E} 有一個關聯值 \m{r(u,v)},
該值是一個實數,範圍爲 \m{0\le r(u,v)\le 1},
表示從節點 \m{u} 到節點 \m{v} 之間通信鏈路的可靠性。
可以認爲, \m{r(u,v)} 代表從 \m{u} 到 \m{v} 的通信鏈路不失效的概率,
並且假設這些概率之間相互獨立。
請給出一個有效的算法找到任意兩個節點之間最可靠的通信鏈路。
\stopEXERCISE

\startANSWER
權重之和最小變成了概率之積最大。
\stopANSWER

%e24.3-7
\startEXERCISE
給定帶權重的有向圖 \m{G=(V,E)},
其權重函數爲 \m{\omega: E\rightarrow \{1,2,\ldots,W\}},
其中 \m{W} 爲某個正整數,
假設圖中從源節點 \m{s} 到任意兩個節點之間的最短路徑權重都不相同。
現在,假設定義一個沒有權重的有向圖 \m{G'=(V\cup V',E')}。
該圖是將每條邊 \m{(u,v)\in E} 予以替換,
替換所用的是 \m{\omega(u,v)} 條具有單位權重的邊。
請問圖 \m{G'} 一共有多少個節點?
現在假設在 \m{G'} 上運行廣度優先搜索算法。
證明: \m{G'} 的廣度優先搜索將 \m{V} 中節點塗上黑色的次序
與 Dijkstra 算法運行在圖 \m{G} 上時從優先隊列中抽取節點的次序相同。
\stopEXERCISE

\startANSWER
對於邊 \m{(u,v)} 會增加 \m{\omega(u,v)-1} 個節點,
因此 \m{G'} 中節點的數目爲:
\startformula
|V| + \sum_{(u,v)\in E}\omega(u,v) - |E|
\stopformula

對於遍歷順序:對於節點 \m{v},假設 Dijkstra 算法算出的值爲 \m{v.d},
則廣度優先搜索時恰好是在第 \m{v.d} 步將 \m{v} 染色。
\stopANSWER

%e24.3-8
\startEXERCISE[exercise:24.3-8]
給定帶權重的有向圖 \m{G=(V,E)},
其權重函數爲 \m{\omega: E\rightarrow \{1,2,\ldots,W\}},
其中 \m{W} 爲某個非負整數。
請修改 Dijkstra 算法來計算從給定源節點 \m{s} 到所有將誒點之間的最短路徑。
該算法時間應爲 \m{O(WV+E)}。
\stopEXERCISE

\startANSWER
用一個數組 \m{A} 實現優先級隊列,下標爲節點的 \m{d} 值,
相應元素存儲的是節點列表,這些節點的 \m{d} 值均與下標相同。

\ALGO{EXTRACT-MIN} 時,所得節點的 \m{d} 值是逐漸增大的,
因此每次從數組中找 \m{d} 值最小元素時只需從上次所得節點的 \m{d} 值開始遍歷即可,
此操作總共需要時間 \m{O(WV)}。

\ALGO{DECREASE-KEY} 總共需要時間 \m{O(E)}。
每次檢查一條邊執行 \ALGO{RELAX} 時,只需根據新的 \m{d} 值移動節點在數組 \m{A} 中的位置即可。
\stopANSWER

%e24.3-9
\startEXERCISE
修改\refexercise{24.3-8} 中的算法,
使其運行時間爲 \m{O((V+E)\lg W)}。
(\hint 在任意時刻,集合 \m{V-S} 裏有多少個不同的最短路徑估計?)
\stopEXERCISE

\startANSWER
改用二叉堆實現優先級隊列。
\stopANSWER

%e24.3-10
\startEXERCISE
假設給定帶權重的有向圖 \m{G=(V,E)},
從源節點 \m{s} 出發的邊的權重可以爲負值,
而其他所有邊的權重全部是非負值,
同時,圖中不包含權重爲負值的環路。
證明: Dijkstra 算法可以正確計算出從源節點 \m{s} 到所有其他節點之間的最短路徑。
\stopEXERCISE

\startANSWER
這種情況下不會破壞 \m{S} 中節點的 \m{d} 值。
\stopANSWER

\stopsection

\startsection[
  title={Difference constraints and shortest paths},
]

%e24.4-1
\startEXERCISE
請給出下面差分約束系統的可行解或證明該系統沒有可行解。
\startformula\startmathalignment
\NC x_1 - x_2 \le \NC 1 \NR
\NC x_1 - x_4 \le \NC -4 \NR
\NC x_2 - x_3 \le \NC 2 \NR
\NC x_2 - x_5 \le \NC 7 \NR
\NC x_2 - x_6 \le \NC 5 \NR
\NC x_3 - x_6 \le \NC 10 \NR
\NC x_4 - x_2 \le \NC 2 \NR
\NC x_5 - x_1 \le \NC -1 \NR
\NC x_5 - x_4 \le \NC 3 \NR
\NC x_6 - x_3 \le \NC -8 \NR
\stopmathalignment\stopformula
\stopEXERCISE

\startANSWER
\externalfigure[output/e24_4_1-8]
\stopANSWER

%e24.4-2
\startEXERCISE
請給出下面差分約束系統的可行解或證明該系統沒有可行解。
\startformula\startmathalignment
\NC x_1 - x_2 \le \NC 4 \NR
\NC x_1 - x_5 \le \NC 5 \NR
\NC x_2 - x_4 \le \NC -6 \NR
\NC x_3 - x_2 \le \NC 1 \NR
\NC x_4 - x_1 \le \NC 3 \NR
\NC x_4 - x_3 \le \NC 5 \NR
\NC x_4 - x_5 \le \NC 10 \NR
\NC x_5 - x_3 \le \NC -4 \NR
\NC x_5 - x_4 \le \NC -8 \NR
\stopmathalignment\stopformula
\stopEXERCISE

\startANSWER
無解,因爲形成了權重爲負值的環路。

\externalfigure[output/e24_4_2-7]
\stopANSWER

%e24.4-3
\startEXERCISE
約束圖中從新節點 \m{v_0} 到其他節點之間的最短路徑權重能夠爲正值嗎?請解釋。
\stopEXERCISE

\startANSWER
不會,因爲從 \m{v_0} 直接到達其他節點的邊權重爲 0,最短路徑的權重不能大於這個值,否則這條邊就是最短路徑了。
\stopANSWER

%e24.4-4
\startEXERCISE
請將單源目的地最短路徑問題表示爲一個線性規劃問題。
\stopEXERCISE

\startANSWER
\TODO{略。}
\stopANSWER

%e24.4-5
\startEXERCISE
請修改 \ALGO{BELLMAN-FORD} 算法,
使其能在 \m{O(nm)} 時間內解決由 \m{n} 個未知變量和 \m{m} 個約束條件所構成的差分約束系統問題。
\stopEXERCISE

\startANSWER
額外添加的 \m{v_0} 及其 \m{n} 條權值爲 0 的邊沒有意義。
我們可以在開始將所有節點 \m{v} 的 \m{d} 初始化爲 0。
\stopANSWER

%e24.4-6
\startEXERCISE
假定在除差分約束系統外,
我們希望處理形式爲 \m{x_i=x_j+b_k} 的{\EMP 相等約束}。
請說明如何修改算法 \ALGO{BELLMAN-FORD} 來解決這種約束系統。
\stopEXERCISE

\startANSWER
每一個等式轉換成兩條邊: \m{x_i-x_j\le b_k} 和 \m{x_j-x_i\le -b_k}。
\stopANSWER

%e24.4-7
\startEXERCISE
說明如何在一個沒有額外節點 \m{v_0} 的約束圖上運行類似 \ALGO{BELLMAN-FORD} 來求解差分約束系統。
\stopEXERCISE

\startANSWER
額外添加的 \m{v_0} 及其 \m{n} 條權值爲 0 的邊沒有意義。
我們可以在開始將所有節點 \m{v} 的 \m{d} 初始化爲 0。
\stopANSWER

%e24.4-8
\startEXERCISE\DIFFICULT
設 \m{Ax\le b} 爲一個有 \m{n} 個變量和 \m{m} 個約束條件的差分約束系統。
證明:在對應的約束圖上運行 \ALGO{BELLMAN-FORD} 將獲得 \m{\sum_{i=1}^{n}x_i} 的最大值,
這裏 \m{Ax\le b} 並且 \m{x_i\le 0}。
\stopEXERCISE

\startANSWER
此算法的解中最大的那個肯定是 0,根據 \m{x_i\le 0} 可知已經最大,其他 \m{x} 根據不等式依次可知均已最大。
因此總和亦爲最大。
\stopANSWER

%e24.4-9
\startEXERCISE\DIFFICULT
設 \m{Ax\le b} 爲一個有 \m{n} 個變量和 \m{m} 個約束條件的差分約束系統。
證明:在對應的約束圖上運行 \ALGO{BELLMAN-FORD} 將獲得 \m{\max\{x_i\} - \min\{x_i\}} 的最小值,
其中 \m{Ax\le b}。
如果該算法被用於安排建設工程的進度,請說明如何應用上述事實。
\stopEXERCISE

\startANSWER
根據上一題可知,如果 \m{x_i\le 0},則得到的所有 \m{x} 都是最大的,
因此 \m{\max\{x_i\} - \min\{x_i\}} 是最小的。
\stopANSWER

%e24.4-10
\startEXERCISE
假定線性規劃問題 \m{Ax\le b} 的矩陣 \m{A} 中每一行對應一個約束條件,
具體來說,對應的是一個形式爲 \m{x_i\le b_k} 的單個變量的約束條件,
或一個形式爲 \m{-x_i\le b_k} 的單變量約束條件。
請說明如何修改算法 \ALGO{BELLMAN-FORD} 來解決這個差分約束系統問題。
\stopEXERCISE

\startANSWER
將新節點 \m{v_0} 加入單變量約束條件。初始化的時候 \m{v_0.d = 0}。

\m{x_i\le b_k}: \m{x_i - x_0 \le b_k}。

\m{-x_i\le b_k}: \m{x_0 - x_i \le b_k}。
\stopANSWER

%e24.4-11
\startEXERCISE
請給出一個有效算法來解決 \m{Ax\le b} 的差分約束系統問題,
這裏 \m{b} 的所有元素爲實數,所有的變量 \m{x_i} 都是整數。
\stopEXERCISE

\startANSWER
將 \m{b} 向下取整。
\stopANSWER

%e24.4-12
\startEXERCISE\DIFFICULT
請給出一個有效算法來解決 \m{Ax\le b} 的差分約束系統問題,
這裏 \m{b} 的所有元素爲實數,所有的變量 \m{x_i} 中某個給定的子集是整數。
\stopEXERCISE

\startANSWER
\TODO{略。}
\stopANSWER

\stopsection

\startsection[
  title={Proofs of shortest-paths properties},
]

%e24.5-1
\startEXERCISE
圖 24-2 中有兩棵最短路徑樹,請給出與其不同的另外兩棵最短路徑樹。
\stopEXERCISE

\startANSWER
\startcombination[nx=2]
{\externalfigure[output/e24_5_1-1]}{a}
{\externalfigure[output/e24_5_1-2]}{b}
\stopcombination
\stopANSWER

%e24.5-2
\startEXERCISE
\m{G=(V,E)} 是一個帶權重的有向圖,
權重函數爲 \m{\omega:E\rightarrow R}。
設 \m{s\in V} 爲某個源節點。
請舉出一個例子,使得圖 \m{G} 滿足下列條件:
對於每條邊 \m{(u,v)\in E},
存在一棵根節點爲 \m{s} 的包含邊 \m{(u,v)} 的最短路徑樹,
也包含一棵根節點爲 \m{s} 的不包含邊 \m{(u,v)} 的最短路徑樹。
\stopEXERCISE

\startANSWER
也許只有存在權重爲 0 的邊才行。

\externalfigure[output/e24_5_2-1]
\stopANSWER

%e24.5-3
\startEXERCISE
對引理 24.10 的證明進行改善,使其可以處理最短路徑權重爲 \m{\infty} 和 \m{-\infty} 的情況。
附引理 24.10:

(三角不等式)設 \m{G=(V,E)} 爲一個帶權重的有向圖,
其權重函數爲 \m{\omega: E\rightarrow R},源節點爲 \m{s}。
那麼對於所有的邊 \m{(u,v)\in E},有
\startformula
\delta(s,v) \le \delta(s,u) + \omega(u,v)
\stopformula
\stopEXERCISE

\startANSWER
如果從 \m{s} 無法到達 \m{v},但從 \m{s} 可以到達 \m{u},
則肯定無法從 \m{u} 到達 \m{v}。

\m{\infty + x = \infty}
\stopANSWER

%e24.5-4
\startEXERCISE
設 \m{G=(V,E)} 是一個帶權重的有向圖,
權重函數爲 \m{\omega:E\rightarrow R}。
設 \m{s\in V} 爲某個源節點。
調用 \ALGO{INITIALIZE-SINGLE-SOURCE(G,s)} 對其進行初始化。
證明:如果一系列鬆弛操作將 \m{s.\pi} 的值設爲一個非空值,
則圖 \m{G} 包含一個權重爲負值的環路。
\stopEXERCISE

\startANSWER
假定是遍歷節點 \m{u} 的時候設置 \m{s.\pi},
那麼 \m{\delta(s,u) + \delta(u,s) < 0},命題得證。
\stopANSWER

%e24.5-5
\startEXERCISE
設 \m{G=(V,E)} 是一個帶權重的有向圖,沒有負值環路。
設 \m{s\in V} 爲源節點。
對於節點 \m{v\in V-\{s\}},如果從源節點 \m{s} 可達,
我們允許 \m{v.\pi} 是節點 \m{v} 在任意一條最短路徑上的前驅;
如果不可達,則 \m{v.\pi} 爲 NIL。
請舉出一個圖例和一種 \m{\pi} 的賦值,
使得 \m{G_\pi} 中形成一條環路。
(根據引理 24.16,這樣的一種賦值不可能由一系列鬆弛操作生成。)
\stopEXERCISE

\startANSWER
\externalfigure[output/e24_5_5-1]
\stopANSWER

%e24.5-6
\startEXERCISE
設 \m{G=(V,E)} 爲一個帶權重的有向圖,
權重函數爲 \m{\omega: E\rightarrow R},
且不包含權重爲負值的環路。
設 \m{s\in V} 爲源節點,
假定圖 \m{G} 由 \ALGO{INITIALIZE-SINGLE-SOURCE(G,s)} 進行初始化。
證明:對於每個節點 \m{v\in V_\pi}, \m{G_\pi} 中存在一條從源節點 \m{s} 到節點 \m{v} 的路徑,
並且該性質在任何鬆弛操作序列中維持爲不變式。
\stopEXERCISE

\startANSWER
\TODO{略。}
\stopANSWER

%e24.5-7
\startEXERCISE
設 \m{G=(V,E)} 爲一個帶權重的有向圖,
且不包含權重爲負值的環路。
設 \m{s\in V} 爲源節點,
假定圖 \m{G} 由 \ALGO{INITIALIZE-SINGLE-SOURCE(G,s)} 進行初始化。
證明:對於所有節點 \m{v\in V_\pi},
存在一個由 \m{|V|-1} 個鬆弛步驟所組成的鬆弛序列來生成 \m{v.d=\delta(s,v)}。
\stopEXERCISE

\startANSWER
\TODO{略。}
\stopANSWER

%e24.5-8
\startEXERCISE
設 \m{G=(V,E)} 爲一個帶權重的有向圖,
且包含一個從源節點 \m{s} 可達的權重爲負值的環路。
請說明如何構造一個 \m{E} 的鬆弛操作的無線序列,
使得每一步鬆弛操作都能更新一個最短路徑的估值。
\stopEXERCISE

\startANSWER
\TODO{略。}
\stopANSWER

\stopsection

\startsubject[
  title={Problems},
]

%p24-1
\startPROBLEM
(Yen’s improvement to Bellman-Ford)
設 \ALGO{BELLMAN-FORD} 算法的輸入圖爲 \m{G=(V,E)},
在第一遍鬆弛前,
我們給 \m{V} 的所有節點賦予一個隨機的線性次序 \m{v_1,v_2,\ldots,v_{|V|}}。
然後,將 \m{E} 劃分爲 \m{E_f\cup E_b},
這裏 \m{E_f=\{(v_i,v_j)\in E: i<j\}}, \m{E_b=\{(v_i,v_j)\in E: i > j\}}。
(假定圖 \m{G} 不包含自循環,
因此一條邊要麼屬於 \m{E_f},要麼屬於 \m{E_b}。)
定義 \m{G_f=(V,E_f)} 和 \m{G_b=(V,E_b)}。
\startigBase[a]\startitem
證明: \m{G_f} 是無環的,且其拓撲排序爲 \m{\langle v_1,v_2,\ldots,v_{|V|}\rangle};
 \m{G_b} 是無環的,且其拓撲排序爲 \m{\langle v_{|V|},v_{|V|-1},\ldots,v_1\rangle}。
\stopitem\stopigBase

\startANSWER
如果有環會違反節點編號的大小關係。
\stopANSWER

假定用以下方式實現每一邊鬆弛操作:
以 \m{v_1,v_2,\ldots,v_{|V|}} 的次序訪問所有節點,
並對從每個節點發出的 \m{E_f} 進行鬆弛操作。
然後,再以次序 \m{\langle v_{|V|},v_{|V|-1},\ldots,v_1\rangle} 訪問所有節點,
並對從每個節點發出的 \m{E_b} 進行鬆弛操作。

\startigBase[continue]\startitem
證明:在上述操作方式下,如果 \m{G} 不包含從源節點 \m{s} 可達的負權重環路,
則在 \m{\lceil |V|/2\rceil} 邊鬆弛操作後,
對於所有節點 \m{v\in V},有 \m{v.d=\delta(s,v)}。
\stopitem\stopigBase

\startANSWER
令一個 FB 塊爲 \m{f_1 \ldots f_i b_1\ldots b_j},
其中 \m{f_1,\ldots,f_i\in E_f}, \m{b_1,\ldots,b_j\in E_b}。
一個 FB 塊也可以只包含 \m{f_1\ldots f_i} 或者 \m{b_1\ldots b_j}。
對於任一條從 \m{s} 到 \m{x} 的最短路徑 \m{P},
令其所含不相交的 FB 塊數目最小爲 \m{B(P)}。
則 \m{|P| < |V|}, \m{B(P)\le \lceil |V|/2\rceil}。

每次遍歷時,可以完成一個 FB 塊中邊的鬆弛操作。
最多遍歷 \m{\lceil |V|/2\rceil} 次,
就可以完成對任一最短路徑中所有邊的鬆弛操作。
\stopANSWER

\startigBase[continue]\startitem
上述算法是否改善了 \ALGO{BELLMAN-FORD} 算法的漸進運行時間?
\stopitem\stopigBase

\startANSWER
每次遍歷時間不變,但遍歷次數減半,漸進時間不變。
\stopANSWER
\stopPROBLEM

%p24-2
\startPROBLEM
(Nesting boxes)
假定有很多維度爲 \m{d} 的盒子,
對於盒子 \m{x=\langle x_1,x_2,\ldots,x_d\rangle} 和 \m{y=\langle y_1,y_2,\ldots,y_d\rangle} 而言,
如果集合 \m{\{1,2,\ldots,d\}} 存在一個排列 \m{\pi},
使得 \m{x_{\pi(1)} < y_1, x_{\pi(2)} < y_2, \ldots, x_{\pi(d)} < y_d},
則稱盒子 \m{x} {\EMP 嵌套}在盒子 \m{y} 裏面。

\startigBase[a]\startitem
證明嵌套關係是傳遞的。
\stopitem\stopigBase

\startANSWER
假定有三個 \m{d} 維的盒子 \m{X}、 \m{Y} 和 \m{Z} 分別爲:
 \m{\langle x_1,x_2,\ldots,x_d\rangle}、 \m{\langle y_1,y_2,\ldots,y_d\rangle} 和 \m{\langle z_1,z_2,\ldots,z_d\rangle}。
存在 \m{\pi_1} 和 \m{\pi_2} 使得 \m{x_{\pi_1(i)}<y_i} 和 \m{y_{\pi_2(i)}<z_i} 對所有 \m{1\le i\le d} 都成立。
令 \m{\pi_3(i)=\pi_1(\pi_2(i))},可以使得 \m{x_{\pi_3(i)}=x_{\pi_1(\pi_2(i))} < y_{\pi_2(i)} < z_i}。
\stopANSWER

\startigBase[continue]\startitem
給定兩個同維度的盒子 \m{x} 和 \m{y},
請給出一個有效算法來判斷 \m{x} 是否嵌套在 \m{y} 裏面。
\stopitem\stopigBase

\startANSWER
可以先對兩個盒子各自按非遞減順序排序,需要時間 \m{O(d\lg d)}。
然後比較排序後兩個數組中的對應元素即可,比較所需時間爲 \m{O(d)}。
總時間爲 \m{O(d\lg d)}。
\stopANSWER

\startigBase[continue]\startitem
假定有一組 \m{n} 個 \m{d} 維的盒子 \m{\{B_1,B_2,\ldots,B_n\}}。
請給出一個有效算法來找出最長序列 \m{\langle B_{i_1},B_{i_2},\ldots,B_{i_k}\rangle},
使得盒子 \m{B_{i_j}} 嵌套在盒子 \m{B_{i_{j+1}}} 裏,
其中 \m{j=1,2,\ldots,k-1}。
請以 \m{d} 和 \m{n} 來表述算法的運行時間。
\stopitem\stopigBase

\startANSWER
首先對 \m{n} 個盒子進行排序,所用時間 \m{O(nd\lg d)}。
其次構建有向無環圖 \m{G=(V,E)},用時 \m{O(n^2 d},
其中 \m{v_i\in V} 代表盒子 \m{B_i}, \m{1\le i\le n}。
圖中的邊 \m{v_i,v_j} 代表盒子 \m{B_i} 嵌套在盒子 \m{B_j} 裏。

然後爲 \m{G} 增加一個節點 \m{s},並爲每個 \m{v_i} 增加邊 \m{(s,v_i}。
然後再增加一個節點 \m{t},並爲每個 \m{v_i} 增加邊 \m{(v_i,t)}。
所用時間 \m{O(n)}。

現在問題轉化爲:尋找從 \m{s} 到 \m{t} 的最長路徑。
所用時間爲 \m{O(n^2)}。

最後將結果轉換爲 \m{B_i} 序列,用時 \m{O(n)}。

總時間爲 \m{O(n^2 d + nd\lg d)}。
\stopANSWER

\stopPROBLEM

%p24-3
\startPROBLEM
(Arbitrage)
{\EMP 套匯},指利用匯率差異,將一個單位的貨幣轉換爲多於一個單位的同種貨幣的行爲。
例如,假定 1 美元可以購買 49 印度盧比, 1 印度盧比可以購買 2 日元, 1 日元可以購買 0.0107 美元。
那麼通過貨幣間的轉換,一個交易商可以從 1 美元開始,
購買 \m{49\times 2 \times 0.0107 = 1.0486} 美元,從而獲得 4.86\% 的利潤。

假設給定 \m{n} 種貨幣 \m{c_1,c_2,\ldots,c_n} 和一個 \m{n\times n} 的匯率表 \m{R},
一個單位 \m{c_i} 貨幣可以購買 \m{R[i,j]} 單位的 \m{c_j} 貨幣。

\startigBase[a]\startitem
給出一個有效算法,判斷是否存在貨幣序列 \m{\langle c_{i_1},c_{i_2},\ldots,c_{i_k}\rangle},使得:
\startformula
R[i_1,i_2] \cdot R[i_2,i_3]\cdot \ldots \cdot R[i_{k-1},i_k]\cdot R[i_k,i_1] > 1
\stopformula
請分析算法的運行時間。
\stopitem\stopigBase

\startANSWER
將問題轉換爲圖,頂點代表一種貨幣,邊代表兩種貨幣間可以進行交易。
邊的權重定義爲 \m{\omega(i,j)=\log\left(\frac{1}{R[i,j]}\right)}。
\startformula\startmathalignment
\NC \NC R[i_1,i_2] \times R[i_2,i_3]\times \ldots \times R[i_{k-1},i_k]\times R[i_k,i_1] > 1 \NR
\NC \NC \frac{1}{R[i_1,i_2] \times R[i_2,i_3]\times \ldots \times R[i_{k-1},i_k]\times R[i_k,i_1]} < 1 \NR
\NC \NC \frac{1}{R[i_1,i_2]} \times \frac{1}{R[i_2,i_3]} \times \ldots
	\times \frac{1}{R[i_{k-1},i_k]} \times \frac{1}{R[i_k,i_1]} < 1 \NR
\NC \NC \log\left(\frac{1}{R[i_1,i_2]} \times \frac{1}{R[i_2,i_3]} \times \ldots
	\times \frac{1}{R[i_{k-1},i_k]} \times \frac{1}{R[i_k,i_1]}\right) < \log 1 = 0 \NR
\NC \NC \log\frac{1}{R[i_1,i_2]} + \log\frac{1}{R[i_2,i_3]} + \ldots
	+ \log\frac{1}{R[i_{k-1},i_k]} + \log\frac{1}{R[i_k,i_1]} < 0\NR
\stopmathalignment\stopformula

接下來檢查圖中是否有負權重的環路。
由於是完全圖,可以從任何節點開始,用算法 \ALGO{DFS} 搜索是否有負權重的環路。
需要時間爲 \m{O(|V|+|E|)},即 \m{O(V^2)}。
\stopANSWER

\startigBase[continue]\startitem
給出一個有效算法打印這樣一個序列(如果存在的話)。分析算法的運行時間。
\stopitem\stopigBase

\startANSWER
可以給每個節點增加屬性 \m{v.\pi} (前驅)和 \m{v.d} (路徑總權重),
最後將前驅打印出來並反序即可。總時間爲 \m{O(V+2E)=O(V+E)=O(V^2)}。
\stopANSWER

\stopPROBLEM

%p24-4
\startPROBLEM
(Gabow’s scaling algorithm for single-source shortest paths)
伸縮算法解決問題的方式如下:
首先考慮相關輸入(如邊的權重)的最高有效位,
然後通過檢查最高兩個有效位對初始解進行微調。
這種算法漸次檢查更多有效位,每次對解進行微調,
直到對所有輸入位進行檢查並計算出正確解爲止。

本題中,我們通過對邊的權重進行伸縮來計算單源最短路徑。
給定有向圖 \m{G=(V,E)},圖的所有邊權重均爲非負整數 \m{\omega}。
設 \m{W=\max_{(u,v)\in E}\{\omega(u,v)\}}。
我們的目標是設計一個運行時間爲 \m{O(E\lg W)} 的算法來計算最短路徑。
假設所有節點都可從源節點到達。

該算法對邊權重的二進制表示進行逐位檢查,
從最高有效位到最低有效位。
具體來說,設 \m{k=\lceil \lg(W+1)\rceil} 爲 \m{W} 的二進制表示所需要的位數。
並且對於 \m{i=1,2,\ldots,k},設 \m{\omega_i(u,v)=\lfloor \omega(u,v)/2^{k-i}\rceil}。
也就是說, \m{\omega_i(u,v)} 是由 \m{\omega(u,v)} 的前 \m{i} 個最高有效位
給出的“收縮”的 \m{\omega(u,v)} 版本。
因此,對於所有邊 \m{(u,v)\in E},都有 \m{\omega_k(u,v)=\omega(u,v)}。
例如,如果 \m{k=5},且 \m{(u,v)=25},
其二進制表示爲 \m{\langle 11001\rangle},
則 \m{\omega_3(u,v)=\langle 110\rangle=6}。
又例如,如果 \m{\omega(u,v)=\langle 00100\rangle = 4},
則 \m{\omega_3(u,v)=\langle 001\rangle = 1}。
令 \m{\delta_i(u,v)} 爲用權重函數 \m{\omega_i} 得到的 \m{u} 和 \m{v} 間最短路徑的權重,
對於所有節點 \m{u,v\in V},都有 \m{\delta_k(u,v)=\delta(u,v)}。
對於給定源節點 \m{s},該算法首先計算出對於所有節點 \m{v\in V} 的所有最短路徑權重 \m{\delta_1(s,v)},
然後再計算出 \m{\delta_2(s,v)},
這樣一直下去,直到計算出 \m{\delta_k(s,v)}。
假定 \m{|E|\ge |V|-1},我們將看到從 \m{\delta_{i-1}} 計算出 \m{\delta_i} 所需要的時間爲 \m{O(E)},
因此,整個算法運行時間爲 \m{O(kE)=O(E\lg W)}。

\startigBase[a]\startitem
假定對於所有節點 \m{v\in V},有 \m{\delta(s,v)\le |E|}。
證明:可以在 \m{O(E)} 時間內計算出所有的 \m{\delta(s,v)}。
\stopitem\stopigBase

\startANSWER
由於權重非負,我們可以用 \ALGO{DIJKSTRA} 算法搜索最短路徑。
用二叉堆的話,時間複雜度爲 \m{O(E\lg V)}。
其中 \m{\lg V} 對應的是 \ALGO{EXTRACT-MIN},
 \m{E} 對應的是 \ALGO{DECREASE-KEY}。

由於權重是整數,且最短路徑權重上限爲 \m{|E|},還有改進的餘地。
我們可以用計數排序維護節點鏈表。
我們知道路徑權重範圍爲 \m{[0,|E|]},
可以用一個數組 \m{l} 來維護一個鏈表,以存儲權重。
使用算法 \ALGO{DIJKSTRA},我們需要實現 \ALGO{INSERT}、 \ALGO{DECREASE-KEY} 和 \ALGO{EXTRACT-MIN}。

\ALGO{INSERT} 比較簡單,如果一個節點可由權重 \m{i} 到達,
只需將節點插入到 \m{l[i]} 鏈表的頭部即可,時間爲 \m{O(1)}。

而 \ALGO{DECREASE-KEY} 也只需從源鏈表中刪除節點,修改權重,插入到新鏈表中即可,
刪除和插入都是 \m{O(1)}。

對於 \ALGO{EXTRACT-MIN},我們注意到算法運行過程中,
如果上一次所節點路徑權重爲 \m{i},那麼下一次所取節點路徑權重肯定不小於 \m{i}。
即下一次我們只需從 \m{l[i]} 開始搜索。
這樣整個 \m{l} 數組只需掃描一遍。每次提取操作需要 \m{O(1)}。
從而所有 \ALGO{EXTRACT-MIN} 所需時間爲 \m{O(E + V)}。

共 \m{V} 次 \ALGO{INSERT}, \m{E} 次 \ALGO{DECREASE-KEY}, \m{V} 次 \ALGO{EXTRACT-MIN}。
總時間爲 \m{O(V+E)}。又由於 \m{|E|>|V|-1},因此時間爲 \m{O(E)}。
\stopANSWER

\startigBase[continue]\startitem
證明:可以在 \m{O(E)} 時間內計算出所有的 \m{\delta_1(s,v)}。
\stopitem\stopigBase

\startANSWER
利用 a)的方法計算 \m{\delta_1(s,v)}。
由於 \m{\omega_1} 只能是 0 或 1,最短路徑的權重不會超過 \m{|V|-1}。
即 \m{\delta_1(s,v)\le |V| - 1 \le |E|}。
因此根據 a),我們可以在 \m{O(E)} 時間內計算出所有的 \m{\delta_1(s,v)}。
\stopANSWER

\startigBase[continue]\startitem
證明:對於 \m{i=2,3,\ldots,k},
要麼有 \m{\omega_i(u,v)=2\omega_{i-1}(u,v)},
要麼有 \m{\omega_i(u,v)=2\omega_{i-1}(u,v)+1}。
然後證明:對於所有節點 \m{v\in V},有
\startformula
2\delta_{i-1}(s,v)\le \delta_i(s,v)\le 2\delta_{i-1}(s,v) + |V| - 1
\stopformula
\stopitem\stopigBase

\startANSWER
如果第 \m{i} 位爲 0,則 \m{\omega_i(u,v) = 2\omega_{i-1}(u,v)};
如果第 \m{i} 位爲 1,則 \m{\omega_i(u,v) = 2\omega_{i-1}(u,v) + 1}。

\startformula\startmathalignment
\NC \delta_i(s,v) \NC = \min \sum_{e\in Path(s,v)} \omega_i(e) \NR
\NC \NC \ge \min \sum_{e\in Path(s,v)} 2\omega_{i-1}(e) \NR
\NC \NC = 2 \cdot \min \sum_{e\in Path(s,v)} \omega_{i-1}(e) \NR
\NC \NC = 2 \delta_{i-1}(s,v) \NR
\stopmathalignment\stopformula

\startformula\startmathalignment
\NC \delta_i(s,v) \NC = \min \sum_{e\in Path(s,v)} \omega_i(e) \NR
\NC \NC \le \min \sum_{e\in Path(s,v)} (2\omega_{i-1}(e) + 1) \NR
\NC \NC = \min \left( 2 \sum_{e\in Path(s,v)} \omega_{i-1}(e) + \sum_{e\in Path(s,v)} 1\right) \NR
\NC \NC = \min \left( 2 \sum_{e\in Path(s,v)} \omega_{i-1}(e) + |V| - 1\right) \NR
\NC \NC = 2 \delta_{i-1}(s,v) + |V| - 1 \NR
\stopmathalignment\stopformula
\stopANSWER

\startigBase[continue]\startitem
對於所有的 \m{(u,v)\in E} 和 \m{i=2,3,\ldots,k},定義:
\startformula
\hat{\omega_i}(u,v) = \omega_i(u,v) + 2\delta_{i-1}(s,u) - 2\delta_{i-1}(s,v)
\stopformula
證明:對於所有邊 \m{u,v)\in E} 和 \m{i=2,3,\ldots,k},
重新計算過的邊 \m{(u,v)} 權重值 \m{\hat{\omega_i}(u,v)} 是一個非負整數。
\stopitem\stopigBase

\startANSWER
\startformula\startmathalignment
\NC \delta_{i-1}(s,v) \NC \le \delta_{i-1}(s,u) + \omega_{i-1}(u,v) \NR
\NC 2\delta_{i-1}(s,v) \NC \le 2\delta_{i-1}(s,u) + 2\omega_{i-1}(u,v) \NR
\NC 2\delta_{i-1}(s,v) \NC \le 2\delta_{i-1}(s,u) + \omega_i(u,v) \NR
\NC 0 \NC \le \omega_i(u,v) + 2\delta_{i-1}(s,u) - 2\delta_{i-1}(s,v) \NR
\NC 0 \NC \le \hat{\omega_i} \NR
\stopmathalignment\stopformula
\stopANSWER

\startigBase[continue]\startitem
在本題中,我們定義 \m{\hat{\delta_i}(s,v)} 爲用權重函數 \m{\hat{\omega_i}} 時
從 \m{s} 到 \m{v} 最短路徑的權重。
證明:對於所有邊 \m{v\in V} 和 \m{i=2,3,\ldots,k},
有 \m{\delta_i(s,v)=\hat{\delta_i}(s,v) + 2\delta_{i-1}(s,v)},
並且 \m{\hat{\delta_i}(s,v)\le E}。
\stopitem\stopigBase

\startANSWER
\startformula\startmathalignment
\NC \hat{\delta_i(s,v)} = \NC \min \sum_{e\in Path(s,v)} \hat{\omega_i}(e) \NR
\NC = \NC \min(\hat{\omega_i}(s,x_1) + \hat{\omega_i}(x_1,x_2) + \ldots + \hat{\omega_i}(x_n, v)) \NR
\NC = \NC \min((\omega_i(s,x_1) + 2\delta_{i-1}(s,s) - 2\delta_{i-1}(s,x_1)) \NR
\NC   \NC + (\omega_i(x_1,x_2) + 2\delta_{i-1}(s,x_1) - 2\delta_{i-1}(s,x_2)) + \ldots \NR
\NC   \NC + (\omega_i(x_n,v) + 2\delta_{i-1}(s,x_n) - 2\delta_{i-1}(s,v))) \NR
\NC = \NC \min(2\delta_{i-1}(s,s) - 2\delta_{i-1}(s,v) + \sum_{e\in Path(s,v)} \omega_i(e)) \NR
\NC = \NC -2\delta_{i-1}(s,v) + \sum_{e\in Path(s,v)} \omega_i(e)) \NR
\NC = \NC -2\delta_{i-1}(s,v) + \delta_i(s,v) \NR
\stopmathalignment\stopformula
因此有 \m{\delta_i(s,v) = \hat{\delta_i}(s,v) + 2\delta_{i-1}(s,v)}。

\startformula\startmathalignment
\NC \delta_i(s,v) \NC = \hat{\delta_i}(s,v) + 2\delta_{i-1}(s,v) \NR
\NC 2\delta_{i-1}(s,v) + |V| - 1 \NC \ge \hat{\delta_i}(s,v) + 2\delta_{i-1}(s,v) \NR
\NC \hat{\delta_i}(s,v) \NC \le 2\delta_{i-1}(s,v) + |V| - 1 - 2\delta_{i-1}(s,v) \NR
\NC \hat{\delta_i}(s,v) \NC \le |V| - 1 \NR
\NC \hat{\delta_i}(s,v) \NC \le |E| \NR
\stopmathalignment\stopformula
\stopANSWER

\startigBase[continue]\startitem
說明如何在 \m{O(E)} 時間內從 \m{\delta_{i-1}(s,v)} 計算出 \m{\delta_i(s,v)},
並證明:可以在 \m{O(E\lg W)} 時間內算出所有節點 \m{v} 的 \m{\delta(s,v)}。
\stopitem\stopigBase

\startANSWER
得到 \m{\delta_{i-1}(s,v)} 後,可以利用 d)的公式計算 \m{\hat{\omega_i}(u,v)},用時 \m{O(E)}。
在 d)中,我們知道了 \m{\hat{\delta_i}(s,v)} 的上限是 \m{|E|}。
然後可以用 a)的算法計算 \m{\hat{\delta_i(s,v)}},用時 \m{O(E)}。
然後用 e)的結果計算 \m{\delta_i(s,v)},用時 \m{O(V)}。

也就是說,由 \m{\delta_{i-1}(s,v)} 計算 \m{\delta_i(s,v)} 用時 \m{O(E)},
而 \m{1\le i \le \lg W},因此計算 \m{\delta(s,v)} 用時 \m{O(E\lg W)}。
\stopANSWER

\stopPROBLEM

%p24-5
\startPROBLEM
(Karp’s minimum mean-weight cycle algorithm)
設 \m{G=(V,E)} 爲一帶權重有向圖,
權重函數爲 \m{\omega: E\rightarrow R},令 \m{n=|V|}。
定義 \m{E} 中邊的環路 \m{c=\langle e_1,e_2,\ldots,e_k\rangle} 的平均權重爲:
\startformula
\mu(c) = \frac{1}{k}\sum_{i=1}^{k}\omega(e_i)
\stopformula
令 \m{\mu* = \min_{c}\mu(c)},其中 \m{c} 爲 \m{G} 中所有有向環路。
我們稱環路權重 \m{\mu(c) = \mu*} 的環路 \m{c} 爲{\EMP 最小平均權重環路}。
本題要研究的是如何高效計算出 \m{\mu*}。

不失一般性,我們可以假設可以從源節點 \m{s} 到達所有節點 \m{v\in V}。
令 \m{\delta(s,v)} 爲從 \m{s} 到 \m{v} 的最短路徑權重,
令 \m{\delta_k(s,v)} 也是從 \m{s} 到 \m{v} 的最短路徑權重,但路徑中邊數限定爲 \m{k}。
如果從 \m{s} 到 \m{v} 的路徑中沒有邊數恰好爲 \m{k} 的,則 \m{\delta_k(s,v)=\infty}。
\startigBase[a]\startitem
證明:如果 \m{\mu*=0},則圖 \m{G} 中沒有負權重環路,且對於所有節點 \m{v\in V}:
\startformula
\delta(s,v)=\min_{0\le k\le n-1}\delta_k(s,v)
\stopformula
\stopitem\stopigBase

\startANSWER
如果圖中有負權重環路,則此環路的 \m{\mu} 小於 0,與 \m{\mu*=0} 矛盾。

根據定義,對於所有 \m{1\le k\le n-1},都有 \m{\delta_k(s,v) \ge \delta(s,v)}。
\stopANSWER

\startigBase[continue]\startitem
證明:如果 \m{\mu*=0},則對於所有節點 \m{v\in V}:
\startformula
\max_{0\le k\le n-1}\frac{\delta_n(s,v)-\delta_k(s,v)}{n-k} \ge 0
\stopformula
(\hint 使用 a)的兩個屬性)
\stopitem\stopigBase

\startANSWER
由於無負權重環路,有 \m{\delta_n(s,v) \ge \delta(s,v)}。即:
\startformula\startmathalignment
\NC \max_{0\le k\le n-1}\frac{\delta_n(s,v)-\delta_k(s,v)}{n-k} \ge \NC \max_{0\le k\le n-1}\frac{\delta(s,v) - \delta_k(s,v)}{n-k} \NR
\NC \ge \NC \frac{\delta(s,v) - \min_{0\le k\le n-1}\delta_k(s,v)}{n} \NR
\NC = \NC \frac{\delta(s,v) - \delta(s,v)}{n} \NR
\NC = \NC 0 \NR
\stopmathalignment\stopformula
\stopANSWER

\startigBase[continue]\startitem
設環路 \m{c} 權重爲 0,上有兩點 \m{u} 和 \m{v}。
假定 \m{\mu*=0},且 \m{c} 上從 \m{u} 到 \m{v} 簡單路徑的權重爲 \m{x}。
證明: \m{\delta(s,v)=\delta(s,u)+x}。
(\hint 環路上從節點 \m{v} 到 \m{u} 簡單路徑的權重爲 \m{-x}。)
\stopitem\stopigBase

\startANSWER
\startformula\startmathalignment
\NC \delta(s,v) \le \NC \delta(s,u) + \delta(u,v) \le \delta(s,u) + x \NR
\NC \delta(s,u) \le \NC \delta(s,v) + \delta(v,u) \le \delta(s,v) - x \NR
\stopmathalignment\stopformula
兩式聯立可知 \m{\delta(s,v)=\delta(s,u)+x}。
\stopANSWER

\startigBase[continue]\startitem
證明:如果 \m{\mu*=0},則在每個最小平均權重環路上都存在一個節點 \m{v},滿足:
\startformula
\max_{0\le k\le n-1}\frac{\delta_n(s,v) - \delta_k(s,v)}{n-k} = 0
\stopformula
(\hint 說明如何將一條最短路徑擴展到最小平均權重環路上的任意節點,
以找出到環路上下一個節點的最短路徑。)
\stopitem\stopigBase

\startANSWER
將 \m{s} 和 \m{v} 間的最短路徑在環路上進行擴展,使得長度爲 \m{n}。
則 \m{\delta_n(s,v) = \delta(s,v)}。
而 \m{\delta_k(s,v)} 要麼是 \m{\infty},要麼等於 \m{\delta(s,v)}。
命題得證。
\stopANSWER

\startigBase[continue]\startitem
證明:如果 \m{\mu*=0},那麼:
\startformula
\min_{v\in V}\max_{0\le k\le n-1} \frac{\delta_n(s,v)-\delta_k(s,v)}{n-k} = 0
\stopformula
\stopitem\stopigBase

\startANSWER
對於任一節點, d)式要麼是 0,要麼大於 0,因此最小值是 0。
\stopANSWER

\startigBase[continue]\startitem
證明:如果給 \m{G} 的每條邊權重加一個常數 \m{t},
則 \m{\mu*} 也會增加 \m{t}。用該事實證明:
\startformula
\mu*=\min_{v\in V}\max_{0\le k\le n-1} \frac{\delta_n(s,v)-\delta_k(s,v)}{n-k}
\stopformula
\stopitem\stopigBase

\startANSWER
\m{\delta_n(s,v)} 會增加 \m{nt}, \m{\delta_k(s,v)} 會增加 \m{kt},
因此 \m{\frac{\delta_n(s,v)-\delta_k(s,v)}{n-k}} 會增加 \m{t}。
\stopANSWER

\startigBase[continue]\startitem
給出一個時間複雜度爲 \m{O(VE)} 的算法來計算 \m{\mu*}。
\stopitem\stopigBase

\startANSWER
根據以下遞推式計算 \m{\delta_k(s,v)},其中 \m{k=0,1,\ldots,n},
用時 \m{O(VE)}:
\startformula
\delta_{i+1}(s,v) = \min_{u\in V}(\delta_k(s,u) + \omega(u,v))
\stopformula
然後在 \m{O(V^2)} 時間內計算 f)中的極值。
\stopANSWER

\stopPROBLEM

%p24-6
\startPROBLEM
(Bitonic shortest paths)
對於一個序列而言,如果該序列先單調遞增,然後再單調遞減,
或者進行循環移位後可以如此,我們就稱此序列是{\EMP 雙調序列}。
例如,序列 \m{\langle 1,4,6,8,3,-2\rangle}、
 \m{\langle 9,2,-4,-10,-5\rangle} 和 \m{\langle 1,2,3,4\rangle} 都是雙調序列,
但 \m{\langle 1,3,12,4,2,10\rangle} 則不是雙調序列。
(請參閱\refproblem{15-3} 中雙調歐幾里德旅行商問題)

給定有向圖 \m{G=(V,E)},權重函數 \m{\omega:E\rightarrow R},
並且所有權重值唯一。
我們希望能找到從源節點 \m{s} 出發的單源最短路徑。
我們還有一條額外信息:對於每個節點 \m{v\in V},
從 \m{s} 到 \m{v} 的任意最短路徑上的邊權重形成一個雙調序列。

請給出最有效的算法來解決這個問題,並分析其運行時間。
\stopPROBLEM

\startANSWER
雙調序列可能有兩種情況:增減增、減增減。
我們可以先對邊進行排序,用時 \m{O(E\lg E)}。
然後對邊進行鬆弛操作,共 4 遍,
其中第一和第三邊按權重遞增順序,另兩邊按權重遞減順序。
共用時 \m{O(V+E\lg V)}。
總用時爲 \m{O(E\lg E + V + E\lg V)},由於 \m{E = O(V^2)},
因此用時爲 \m{O(E\lg V + V + E) = O(V + E\lg V)}。
\stopANSWER

\stopsubject%Problems

\stopchapter
\stopcomponent
