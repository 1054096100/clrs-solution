\startsection[
  title={Disjoint-set operations},
  reference=section:disjoint_set_op,
]

%e21.1-1
\startEXERCISE
假設 \ALGO{CONNECTED-COMPONENTS} 作用於一個無向圖 \m{G=(V,E)},
這裏 \m{V=\{a,b,c,d,e,f,g,h,i,j,k\}},
且 \m{E} 中的邊處理順序如下:
 \m{(d,i)}, \m{(f,k)}, \m{(g,i)}, \m{(b,g)}, \m{(a,h)}, \m{(i,j)}, \m{(d,k)}, \m{(b,j)}, \m{(d,f)}, \m{(g,j)}, \m{(a,e)}。
請列出每次執行完第 3~5 行後各連通分量的頂點。
\stopEXERCISE

\startANSWER

\startxtable[
    option=max,
    align={middle,lohi},
    split=yes,
    header=repeat,
    footer=repeat,
    offset=.25em,
]

% head
\startxtablehead[frame=off,bottomframe=on]
\startxrow[foregroundstyle=bold,]
  \xcell[rightframe=on]{}\xcell{}
\stopxrow
\stopxtablehead

\define[1]\MSET{%
\ifx&#1&%
\xcell{}%
\else%
\xcell[align={flushleft,lohi}]{\m{\{#1\}}}%
\fi%
}%

% body
\startxtablebody[frame=off]
\startxrow \xcell[rightframe=on]{     }\processcommalist[a,      b,              c,d,      e,f,    g,h,i,j,k]\MSET \stopxrow
\startxrow \xcell[rightframe=on]{(d,i)}\processcommalist[a,      b,              c,{d,i},  e,f,    g,h, ,j,k]\MSET \stopxrow
\startxrow \xcell[rightframe=on]{(f,k)}\processcommalist[a,      b,              c,{d,i},  e,{f,k},g,h, ,j, ]\MSET \stopxrow
\startxrow \xcell[rightframe=on]{(g,i)}\processcommalist[a,      b,              c,{d,g,i},e,{f,k},g,h, ,j, ]\MSET \stopxrow
\startxrow \xcell[rightframe=on]{(b,g)}\processcommalist[a,      {b,d,g,i},      c,       ,e,{f,k}, ,h, ,j, ]\MSET \stopxrow
\startxrow \xcell[rightframe=on]{(a,h)}\processcommalist[{a,h},  {b,d,g,i},      c,       ,e,{f,k}, , , ,j, ]\MSET \stopxrow
\startxrow \xcell[rightframe=on]{(i,j)}\processcommalist[{a,h},  {b,d,g,i,j},    c,       ,e,{f,k}, , , , , ]\MSET \stopxrow
\startxrow \xcell[rightframe=on]{(d,k)}\processcommalist[{a,h},  {b,d,f,g,i,j,k},c,       ,e,     , , , , , ]\MSET \stopxrow
\startxrow \xcell[rightframe=on]{(b,j)}\processcommalist[{a,h},  {b,d,f,g,i,j,k},c,       ,e,     , , , , , ]\MSET \stopxrow
\startxrow \xcell[rightframe=on]{(d,f)}\processcommalist[{a,h},  {b,d,f,g,i,j,k},c,       ,e,     , , , , , ]\MSET \stopxrow
\startxrow \xcell[rightframe=on]{(g,j)}\processcommalist[{a,h},  {b,d,f,g,i,j,k},c,       ,e,     , , , , , ]\MSET \stopxrow
\startxrow \xcell[rightframe=on]{(a,e)}\processcommalist[{a,e,h},{b,d,f,g,i,j,k},c,       , ,     , , , , , ]\MSET \stopxrow
\stopxtablebody

\stopxtable

\stopANSWER

%e21.1-2
\startEXERCISE
證明: \ALGO{CONNECTED-COMPONENTS} 處理完所有的邊後,
兩個頂點在相同的連通分量中當且僅當他們在同一個集合中。
\stopEXERCISE

\startANSWER
\TODO{略。}
\stopANSWER

%e21.1-3
\startEXERCISE
在 \ALGO{CONNECTED-COMPONENTS} 作用於一個有 \m{k} 個連通分量的無向圖 \m{G=(V,E)} 的過程中,
 \ALGO{FIND-SET} 需要調用多少次?
 \ALGO{UNION} 需要調用多少次?
用 \m{|V|}、 \m{|E|} 和 \m{k} 來表示你的答案。
\stopEXERCISE

\startANSWER
\ALGO{FIND-SET} 需要調用 \m{2|E|} 次。

\ALGO{UNION} 需要調用 \m{|V|-k} 次。
\stopANSWER

\stopsection
