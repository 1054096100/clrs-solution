\startsection[
  title={The hiring problem},
]

\startEXERCISE
證明:假設在 HIRE-ASSISTANT 算法的第 4 行中,
我們總能確定哪一個應聘者更佳,
則意味着我們知道應聘者排名的全部次序。

\CLRSH{HIRE-ASSISTANT(n)}
\startCLRS
best = 0	// candidate 0 is a least-qualified dummy candidate
for i = 1 to n
	interview candidate i
	if candidate i is better than candidate best
		best = i
		hire candidate i
\stopCLRS
\stopEXERCISE
\startANSWER
全序是一種包含了所有關系的偏序 \m{\forall a,b \in A: a R b \text{或} b R a}。
一個關系若是偏序關系,他必須是自反的、反對稱的以及可傳遞的。

令這個關系爲“同樣好或更好”:
\startigBase[1]
\item 自反:顯然,任何人與自身相比都滿足次關系;
\item 反對稱:如果 \m{A} 比 \m{B} 好,那麼 \m{B} 不會比 \m{A} 好;
\item 可傳遞:如果 \m{A} 比 \m{B} 好, \m{B} 又比 \m{C} 好,那麼 \m{A} 比 \m{C} 好。
\stopigBase
到此可知,此關系是偏序關系。

又由於我們假設可以比較任意兩個應聘者,也就說包含了所有關系,即全序。
\stopANSWER

\startEXERCISE \DIFFICULT
利用 \ALGO{RANDOM(0,1)} 實現過程 \ALGO{RANDOM(a,b)}。
作爲 \m{a} 和 \m{b} 的函數,其運行時間是多少?
\stopEXERCISE
\startANSWER
令 \m{n=b-a},算法爲:
\startigBase[n]
\item 找到最小整數 \m{c},使得 \m{2^c\ge n} (\m{c=\left\lceil\ln{n}\right\rceil});
\item 調用 \ALGO{RANDOM(0,1)} \m{c} 次,得到一個 \m{c} 比特的二進制數 \m{r};
\item 如果 \m{r>n},跳轉到上一步;
\item 否則返回 \m{a+r}。
\stopigBase
這會在範圍 \m{[a, b]} 內均勻的生成一個隨機數。
但是,第二步可能會重復多次。
有 \m{p=n/2^c} 概率不會重復第二步。
由幾何分布可知有 \m{1/p} 的概率會重復第二步,即 \m{2^c/n}。
由於每次執行第二步,需要調用 \ALGO{RANDOM(0,1)} \m{c} 次,運行時間爲:
\startformula
O(\frac{2^c}{n} c) = O(\frac{\ln(b-a)2^{\ln(b-a)}}{b-a}) = O(\ln(b-a))
\stopformula
\stopANSWER

\startEXERCISE \DIFFICULT
假設你希望以各爲 1/2 的概率輸出 0 和 1。
已有一個可以輸出 0 或 1 的過程 \ALGO{BIASED-RANDOM} 可用。
此過程輸出 1 的概率爲 \m{p},輸出 0 的概率爲 \m{1-p},
其中 \m{0<p<1},但 \m{p} 的值未知。
請給出一個利用 \ALGO{BIASED-RANDOM} 作爲子程式的算法,返回一個無偏的結果,
即各以 1/2 的概率返回 0 和 1。
作爲 \m{p} 的函數,此算法的運行時間爲多少?
\stopEXERCISE
\startANSWER
簡單!
\startigBase[n]
\item 生成兩個隨機數 \m{x} 和 \m{y};
\item 如果這兩個數不同,則返回 \m{x};
\item 否則,重復第一步。
\stopigBase
\m{x=0, y=1} 和 \m{x=1, y=0} 的概率相同。
第二步直接返回的概率是 \m{2pq}。
重試次數的期望值是 \m{1/(2pq)},運行時間爲 \m{O(\frac{1}{p(1-p)})}。
\stopANSWER

\stopsection
