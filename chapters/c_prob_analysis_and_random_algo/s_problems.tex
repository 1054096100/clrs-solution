\startsubject[
  title={Problems},
]

\startPROBLEM(Probabilstic counting)
利用一個 \m{b} 位的計數器,
我們一般只能計數到 \m{2^b -1}。
而用 R.Morris 的{\EMP 概率記數法},
我們可以計數到一個大得多的值,
代價是精度有所損失。

對 \m{i=0,1,\ldots,2^b-1},令計數器的值 \m{i} 表示 \m{n_i} 的計數,
其中 \m{n_i} 構成了一個非負的遞增序列。
假設計數器初值爲 \m{0},表示計數 \m{n_0 = 0}。
INCREMENT 運算單元工作在一個計數器上,
他以概率的方式包含值 \m{i}。
如果 \m{i = 2^b-1},則該運算單元報告溢出錯誤;
否則,該運算單元以概率 \m{1/(n_{i+1}-n_i)} 把計數器增加 \m{1},
以概率 \m{1-1/(n_{i+1}-n_i)} 保持計數器不變。

對所有的 \m{i\ge 0},若選擇 \m{n_i = i},此計數器就退化爲一個普通的計數器。
若選擇 \m{n_i=2^{i-1}(i>0)},或者 \m{n_i=F_i}(第 \m{i} 個 Fibonacci 數,參見\refsection{notationfunction}),
則會出現更多有趣的情形。

對於這個問題,假設 \m{n_{2^b-1}} 已足夠大,發生溢出錯誤的概率可以忽略。

\startigBase[a]
\item 請說明在執行 \m{n} 次 INCREMENT 運算後,計數器所表示的數期望值正好是 \m{n};

\startANSWER
假設第 \m{j} 次 INCREMENT 前,計數器的值爲 \m{i},表示 \m{n_i}。
如果計數器增加,那麼其值會增大 \m{n_{i+1} - n_i}。
其概率爲 \m{1/(n_{i+1} - n_i)},並且:
\startformula\startmathalignment
\NC E[X_j] \NC= 0 \cdot \Pr\{\text{stays same}\} + 1 \cdot \Pr\{\text{increases}\} \NR
\NC        \NC= 0 \cdot \left(1 - \frac{1}{n_{i+1} - n_i}\right) +
                1 \cdot \left((n_{i+1} - n_i) \cdot \frac{1}{n_{i+1} - n_i}\right) \NR
\NC        \NC= 1 \NR
\stopmathalignment\stopformula
這是每次 INCREMENT 的期望值,由於有 \m{n} 次 INCREMNT,其期望值爲 \m{n}。
\stopANSWER

\item 分析計數器表示的計數的方差依賴於 \m{n_i} 序列。
我們來看一個簡單情形:對所有 \m{i\ge 0}, \m{n_i = 100i}。
在執行了 \m{n} 次 INCREMENT 運算後,請估計計數器所表示數的方差。

\startANSWER
單次 INCREMENT 運算的協方差:
\startformula\startmathalignment
\NC Var[X_j] \NC= E[X_j^2] - E^2[X_j] \NR
\NC           \NC= \left(0^2 \cdot \frac{99}{100} + 100^2 \frac{1}{100}\right) - 1 \NR
\NC           \NC= 99 \NR
\stopmathalignment\stopformula

\m{n} 次運算的協方差:
\startformula
Var[X] = Var[X_1 + X_2 + \ldots + X_n] = \sum_{i=1}^n Var[X_i] = 99n
\stopformula
\stopANSWER
\stopigBase
\stopPROBLEM

\startPROBLEM(Searching an unsorted array)
本題將分析三個算法,他們在一個包含 \m{n} 個元素的無序數列 \m{A} 中查找一個值 \m{x}。

考慮如下的隨機策略:隨機挑選 \m{A} 中的一個下標 \m{i}。
如果 \m{A[i] = x},則終止;否則,繼續挑選 \m{A} 中的一個新的隨機下標。
重復隨機挑選下標,知道找到一個下標 \m{j},使得 \m{A[j] = x},
或者知道我們已檢查過 \m{A} 中的所有元素。
注意,我們每次都是從全部下標的結合中挑選,於是可能會不止一次地檢查某個元素。
\startigBase[a]
\item 請寫出過程 \ALGO{RANDOM-SEARCH} 的僞碼實現上述策略。
確保當 \m{A} 中所有下標都被挑選過時,你的算法應停止。
\stopigBase

\startANSWER
\CLRSH{RANDOM-SEARCH(x, A, n)}
\startCLRS
/BTEX\m{v = \emptyset}/ETEX
while /BTEX \m{|v|\ne n} /ETEX
	i = RANDOM(1, n)
	if A[i] = x
		return i
	else:
		/BTEX\m{v = v\cup i}/ETEX
return ␀
\stopCLRS
其中可以多種方式實現 \m{v}——如散列表、樹或位圖。位圖的時間、空間性能應當都是最好的。
\stopANSWER

\startigBase[continue]
\item 假定恰好有一個下標 \m{i} 使得 \m{A[i] = x}。
在我們找到 \m{x} 和 \ALGO{RANDOM-SEARCH} 結束之前,
必須挑選 \m{A} 下標的數目期望值是多少?
\stopigBase

\startANSWER
參考 \goto{Bernoulli trials}[url(http://en.wikipedia.org/wiki/Bernoulli_trial)],期望值爲 \m{n}。
\stopANSWER

\startigBase[continue]
\item 假設有 \m{k\ge 1} 個下標 \m{i} 使得 \m{A[i] = x},
推廣你對上一項的解答。
在找到 \m{x} 或 \ALGO{RANDOM-SEARCH} 結束之前,
必須挑選 \m{A} 的下標的數目期望值是多少?
你的答案應該是 \m{n} 和 \m{k} 的函數。
\stopigBase

\startANSWER
答案爲 \m{n/k}。
\stopANSWER

\startigBase[continue]
\item 假設沒有下標 \m{i} 使得 \m{A[i] = x},
在檢查完 \m{A} 的所有元素或 \ALGO{RANDOM-SEARCH} 結束之前,
我們必須挑選 \m{A} 的下標的數目期望值是多少?
\stopigBase

\startANSWER
答案爲 \m{n(\ln{n} + O(1))},參見節 5.4.2。
\stopANSWER

現在考慮一個確定性的線性查找算法,我們稱之爲 \ALGO{DETERMINISTIC-SEARCH}。
具體地說,這個算法在 \m{A} 中順序查找 \m{x},
考慮 \m{A[1]}、 \m{A[2]}、 \m{A[3]}、 \m{\ldots}、 \m{A[n]},
直到找到 \m{A[i] = x},或者到達數列的末尾。
假設輸入數列的所有排列都是等可能的。
\startigBase[continue]
\item 假設恰好有一個下標 \m{i} 使得 \m{A[i] = x}。
\ALGO{DETERMINISTIC-SEARCH} 的平均運行時間是多少?
最壞運行時間又是多少?
\stopigBase

\startANSWER
最壞運行時間爲 \m{n}。平均運行時間爲 \m{(n+1)/2}。
\stopANSWER

\startigBase[continue]
\item 假設有 \m{k\ge 1} 個下標 \m{i} 使得 \m{A[i] = x},推廣你對上一項的解答。
\ALGO{DETERMINISTIC-SEARCH} 的平均運行時間是多少?最壞運行時間又是多少?
你的答案應該是 \m{n} 和 \m{k} 的函數。
\stopigBase

\startANSWER
最壞運行時間爲 \m{n-k+1}。
平均運行時間爲 \m{(n+1)/(k+1)}。
令 \m{X_i} 爲指示器隨機變量,代表第 \m{i} 個元素匹配,則 \m{\Pr\{X_i\} = 1/(k+1)}。
令 \m{Y} 爲指示器隨機變量,代表在前 \m{n-k+1} 個元素中找到了匹配項(\m{\Pr{\{Y\}}=1})。
因此:
\startformula\startmathalignment
\NC E[X] \NC= E[X_1 + X_2 + \ldots + X_{n-k} + Y] \NR
\NC      \NC= 1 + \sum_{i=1}^{n-k}E[X_i] \NR
\NC      \NC= 1 + \frac{n-k}{k+1} \NR
\NC      \NC= \frac{n+1}{k+1} \NR
\stopmathalignment\stopformula
\stopANSWER

\startigBase[continue]
\item 假設沒有下標 \m{i} 使得 \m{A[i] = x},
\ALGO{DETERMINISTIC-SEARCH} 的平均運行時間是多少?最壞運行時間又是多少?
\stopigBase

\startANSWER
兩種情況都是 \m{n}。
\stopANSWER

最後,考慮一個隨機算法 \ALGO{SCRAMBLE-SEARCH},
他縣將輸入數列隨機變換排列,然後在變換後的數列上,運行上面的確定性線性查找算法。
\startigBase[continue]
\item 設 \m{k} 是滿足 \m{A[i] = x} 的下標的數目,
請給出在 \m{k=0} 和 \m{k=1} 兩種情況下,
算法 \ALGO{SCRAMBLE-SEARCH} 最壞情形的運行時間和運行時間期望值。
推廣你的解答以處理 \m{k\ge 1} 的情況。
\stopigBase

\startANSWER
與 \ALGO{DETERMINISTIC-SEARCH} 一樣,只是將“平均情況”換成了“期望”。
\stopANSWER

\startigBase[continue]
\item 你將會使用 3 種查找算法中的哪一個?解釋你的答案。
\stopigBase

\startANSWER
當然是 \ALGO{DETERMINISTIC-SEARCH}。
雖然 \ALGO{SCRAMBLE-SEARCH} 可以有更好的期望值,
但是排列變換需要額外的時間,而這個時間是線性的操作。
在進行排列變換的時間李,我們可能已經掃描完了整個數列並得到了結果。
\stopANSWER

\stopPROBLEM

\stopsubject
