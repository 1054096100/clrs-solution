\startcomponent c_elementary_graph_algorithms

\startchapter[
  title={Elementary Graph Algorithms},
]

\startsection[
  title={Representations of graphs},
]

%e22.1-1
\startEXERCISE
給定有向圖的鄰接鏈表,需要多長時間才能計算出每個節點的出度(發出的邊的條數)?
多長時間才能計算出每個節點的入度(進入的邊的條數)?
\stopEXERCISE

\startANSWER
\m{\Theta(|V|+|E|)}。

\m{\Theta(|V||E|)}。而如果使用額外 \m{\Theta(|V|)} 空間,可以在 \m{\Theta(|V| + |E|)} 時間內完成。
\stopANSWER

%e22.1-2
\startEXERCISE
給定一棵有 7 個節點的完全二叉樹的鄰接鏈表,
請給出等價的鄰接矩陣表示。
這裏假設節點的編號爲從 1~7。
\stopEXERCISE

\startANSWER

\startformula\startmathalignment
\NC 1 \NC \qquad 2,3 \NR
\NC 2 \NC \qquad  1,4,5 \NR
\NC 3 \NC \qquad  1,6,7 \NR
\NC 4 \NC \qquad  2 \NR
\NC 5 \NC \qquad  2 \NR
\NC 6 \NC \qquad  3 \NR
\NC 7 \NC \qquad  3 \NR
\stopmathalignment\stopformula

\startformula\startmatrix
\NC 0 \NC 1 \NC 1 \NC 0 \NC 0 \NC 0 \NC 0 \NR
\NC 1 \NC 0 \NC 0 \NC 1 \NC 1 \NC 0 \NC 0 \NR
\NC 1 \NC 0 \NC 0 \NC 0 \NC 0 \NC 1 \NC 1 \NR
\NC 0 \NC 1 \NC 0 \NC 0 \NC 0 \NC 0 \NC 0 \NR
\NC 0 \NC 1 \NC 0 \NC 0 \NC 0 \NC 0 \NC 0 \NR
\NC 0 \NC 0 \NC 1 \NC 0 \NC 0 \NC 0 \NC 0 \NR
\NC 0 \NC 0 \NC 1 \NC 0 \NC 0 \NC 0 \NC 0 \NR
\stopmatrix\stopformula

\stopANSWER

%e22.1-3
\startEXERCISE
有向圖 \m{G=(V,E)} 的{\EMP 轉置}是圖 \m{G^T=(V,E^T)},
這裏 \m{E^T=\{(v,u)\in V \times V:(u,v)\in E\}}。
\stopEXERCISE

\startANSWER
\m{O(V+E)},需要額外大小爲 \m{O(V)} 的指針數組。

轉置矩陣即可, \m{O(|V|^2)}。
\stopANSWER

%e22.1-4
\startEXERCISE
給定多圖 \m{G=(V,E)} 的鄰接鏈表(多圖是允許重複邊和自循環邊的圖),
請給出一個時間爲 \m{O(V+E)} 的算法,
用來計算該圖的“等價”無向圖 \m{G'=(V,E')} 的鄰接鏈表表示。
這裏 \m{E'} 是將 \m{E} 中的冗餘邊和自循環邊刪除後餘下的邊。
刪除冗餘邊指的是將兩個節點之間的多條邊替換爲一條邊。
\stopEXERCISE

\startANSWER
遍歷所有頂點 \m{V_i},新建一個空數組,用來記錄是否有 \m{V_i} 到其他節點的邊,
遍歷 \m{V_i} 的鄰接鏈表中所有元素 \m{V_j},如果 \m{V_j} 在新數組中沒有被標記過,則標記,
如果已被標記過,則刪除當前節點。然後清空新數組,遍歷下一個頂點的鄰接鏈表。
\stopANSWER

%e22.1-5
\startEXERCISE
有向圖 \m{G=(V,E)} 的{\EMP 平方圖}是圖 \m{G^2=(V,E^2)},
這裏,邊 \m{(u,v)\in E^2} 當且僅當圖 \m{G} 包含一條最多由兩條邊構成的從 \m{u} 到 \m{v} 的路徑。
請給出一個有效算法來計算圖 \m{G} 的平方圖 \m{G^2}。
這裏圖 \m{G} 既可以以鄰接表表示,
也可以以鄰接矩陣表示。
請分析算法的運行時間。
\stopEXERCISE

\startANSWER
鄰接矩陣: \m{M^2[u,w] = \sum_{v}M[u,v]\cdot M[v,w]},即矩陣相乘;時間複雜度爲 \m{O(V^3)}。
如果採用 Strassen 算法則爲 \m{O(V^{2.376})}。

若採用鄰接鏈表:遍歷所有邊 \m{(u,v)},然後在遍歷 \m{v} 的所有鄰接頂點 \m{w},將 \m{w} 加入到新的鄰接鏈表中(利用額外數組記錄是否已經添加過,以解決衝突)。時間複雜度爲 \m{O(VE)}。
\stopANSWER

%e22.1-6
\startEXERCISE
多數以鄰接矩陣作爲輸入的圖算法的運行時間爲 \m{\Omega(V^2)},但也有例外。
給定圖 \m{G} 的鄰接矩陣表示,
請給出一個 \m{O(V)} 時間的算法來判斷有向圖 \m{G} 是否存在一個 {\EMP 通用匯點}(universal sink)。
通用匯點指的是入度爲 \m{|V|-1} 但出度爲 0 的節點。
\stopEXERCISE

\startANSWER
只需判斷一行加一列元素。
\stopANSWER

%e22.1-7
\startEXERCISE
有向無環圖 \m{G=(V,E)} 的{\EMP 關聯矩陣}(incidence matrix)是一個滿足下述條件的 \m{|V|\times|E|} 矩陣 \m{B=(b_{ij})}:
\startformula
b_{ij} = \startcases
\NC -1 \MC \text{如果邊 \m{j} 從節點 \m{i} 出發} \NR
\NC 1 \MC \text{如果邊 \m{j} 進入節點 \m{i}} \NR
\NC 0 \MC \text{其他} \NR
\stopcases
\stopformula
請說明矩陣乘積 \m{BB^T} 裏的每一個元素代表什麼意思。
這裏 \m{B^T} 是矩陣 \m{B} 的轉置。
\stopEXERCISE

\startANSWER
與頂點 \m{i} 相連的邊的數目。
\stopANSWER

%e22.1-8
\startEXERCISE
假定數組 \m{Adj[u]} 的每個記錄項不是鏈表,而是一個散列表,
裏面包含的是 \m{(u,v)\in E} 的節點 \m{v}。
如果每條邊被查詢的概率相同,
則判斷一條邊是否在圖中的期望時間值是多少?
這種表示方法的缺陷是什麼?
請爲每條邊鏈表給出一個不同的數據結構來解決這個問題。
與散列表相比較,你給出的新方法存在什麼缺陷嗎?
\stopEXERCISE

\startANSWER
每個散列表越大,則期望時間越少,當散列表跟頂點個數一樣大時,期望的查詢時間就是 \m{O(1)} 了。

缺陷是空間佔用大。

用鄰接矩陣。
\stopANSWER

\stopsection

\startsection[
  title={Breadth-first search},
]

%e22.2-1
\startEXERCISE
請計算出在有向圖 22-2(a)上運行廣度優先搜索算法後的 \m{d} 值和 \m{\phi} 值。
這裏假定節點 3 爲算法所用的源節點。
\stopEXERCISE

\startANSWER
\TODO{略。}
\stopANSWER

%e22.2-2
\startEXERCISE
請計算出在有向圖 22-3 所示無向圖上運行廣度優先搜索算法後的 \m{d} 值和 \m{\phi} 值。
這裏假定節點 u 爲算法所用的源節點。
\stopEXERCISE

\startANSWER
\TODO{略。}
\stopANSWER

%e22.2-3
\startEXERCISE
證明:使用單個位來存放每個節點的顏色即可。
這個論點可以通過證明將算法第 18 行的僞碼刪除後, \ALGO{BFS} 過程生成的結果不變得到。
\stopEXERCISE

\startANSWER
算法只關心節點是不是白色,至於是黑還是灰則沒有區別。
\stopANSWER

%e22.2-4
\startEXERCISE
如果將輸入的圖用鄰接矩陣來表示,
並修改算法來應對此種形式的輸入,
請問 \ALGO{BFS} 的運行時間將是多少?
\stopEXERCISE

\startANSWER
\m{O(V^2)}
\stopANSWER

%e22.2-5
\startEXERCISE
證明:在廣度有限搜索算法裏,
賦給節點 \m{u} 的 \m{u.d} 值與節點在鄰接表裏出現的次序無關。
使用圖 22-3 作爲例子,
證明: \ALGO{BFS} 所計算出的廣度優先樹可以因鄰接鏈表中的次序不同而不同。
\stopEXERCISE

\startANSWER
最短路徑長度當然和次序無關。
而最短路徑可能有多條,就和次序有關了。
\stopANSWER

%e22.2-6
\startEXERCISE
舉出一個有向圖 \m{G=(V,E)} 的例子,
對於源節點 \m{s\in V} 和一組樹邊 \m{E_{\pi}\subseteq E},
使得對於每個節點 \m{v\in V},圖 \m{V, E_{\pi}} 中從源節點 \m{s} 到
節點 \m{v} 的唯一簡單路徑也是圖 \m{G} 中的一條最短路徑,
但是,不管鄰接表裏節點之間的次序如何,
邊集 \m{E_\pi} 都不能通過在圖 \m{G} 上運行 \ALGO{BFS} 來獲得。
\stopEXERCISE

\startANSWER
如下圖,實線屬於 \m{E_\pi}。

\externalfigure[output/e22_2_6-1]
\stopANSWER

%e22.2-7
\startEXERCISE
職業摔跤手可以分爲兩種類型:
“娃娃臉”(“好人”)型和“高跟鞋”(“壞人”)型。
在任意一對職業摔跤手之間都有可能存在競爭關係。
假定有 \m{n} 個職業摔跤手,
並且有一個給出競爭關係的 \m{r} 對摔跤手的鏈表。
請給出一個時間爲 \m{O(n+4)} 的算法來判斷是否可以將某些摔跤手劃分爲“娃娃臉”型,
而剩下的劃分爲”高跟鞋“型,
使得所有的競爭關係均只存在於娃娃臉型和高跟鞋型選手之間。
如果可以進行這種劃分,
則算法還應當生成一種這樣的劃分。
\stopEXERCISE

\startANSWER
遍歷邊,給頂點染色。
\stopANSWER

%e22.2-8
\startEXERCISE\DIFFICULT
我們將一棵樹 \m{T=(V, E)} 的{\EMP 直徑}定義爲 \m{\max_{u,v\in V}\delta(u,v)},
也就是說,樹中所有最短路徑距離的最大值即爲樹的直徑。
請給出一個有效算法來計算樹的直徑,
並分析算法的運行時間。
\stopEXERCISE

\startANSWER
\startigBase[n]
\item 選擇任意節點 \m{A},運行 \ALGO{BFS} 找到離 \m{A} 最遠的節點 \m{S};
\item 從 \m{S} 開始運行 \ALGO{BFS},找到最遠節點 \m{D};
\item 從 \m{S} 到 \m{D} 的路徑即爲所求。
\stopigBase

主要是證明 \m{S} 必爲直徑的一個端點。用反證法,參考如下鏈接:

\SIMPLEURL{http://stackoverflow.com/questions/12989578/prove-traversing-a-k-ary-tree-twice-yields-the-diameter}。
\stopANSWER

%e22.2-9
\startEXERCISE
設 \m{G=(V,E)} 爲一個連通無向圖。
請給出一個 \m{O(V+E)} 時間的算法來計算圖 \m{G} 中的一條這樣的路徑:
該路徑正反向通過 \m{E} 中每條邊恰好一次(該路徑通過每條邊兩次,但這兩次的方向相反)。
如果給你大量的分幣作爲獎勵,
請描述如何在迷宮中找出一條路。
\stopEXERCISE

\startANSWER
深度優先進行遍歷,優先訪問沒有訪問過的邊,如果所有相鄰點都訪問過,沿路徑返回上一個節點,繼續執行。
\stopANSWER

\stopsection

\startsection[
  title={Depth-first search},
]

%e22.3-1
\startEXERCISE
劃一個 \m{3\times 3} 的網格,行和列的擡頭分別標記爲白色、灰色和黑色。
對於每個表單元 \m{(i,j)},
請指出在對有向圖進行深度有限搜索的過程中,是否可能存在一條邊,
鏈接一個顏色爲 \m{i} 的節點和一個顏色爲 \m{j} 的節點。
對於每種可能的邊,指明該種邊的類型。
另外,請針對無向圖的深度優先搜索再製作一張這樣的網格。
\stopEXERCISE

\startANSWER
\startigBase[2]
\item T:樹邊
\item B:後向邊
\item F:前向邊
\item C:橫向邊
\stopigBase

有向圖:

\startxtable[
    option=max,
    align={middle,lohi},
    split=yes,
    header=repeat,
    footer=repeat,
    offset=.25em,
]

% head
\startxtablehead[frame=off,bottomframe=on]
\startxrow[foregroundstyle=bold,]
  \xcell{\m{i,j}}\xcell{White}\xcell{Gray}\xcell{Black}
\stopxrow
\stopxtablehead

% body
\startxtablebody[frame=off]
\startxrow \xcell{White}\xcell{TBFC}\xcell{BC}  \xcell{C}   \stopxrow
\startxrow \xcell{Gray} \xcell{TF}  \xcell{TBF} \xcell{TFC} \stopxrow
\startxrow \xcell{Black}\xcell{}    \xcell{B}   \xcell{TBFC}\stopxrow
\stopxtablebody

\stopxtable


無向圖:

\startxtable[
    option=max,
    align={middle,lohi},
    split=yes,
    header=repeat,
    footer=repeat,
    offset=.25em,
]

% head
\startxtablehead[frame=off,bottomframe=on]
\startxrow[foregroundstyle=bold,]
  \xcell{\m{i,j}}\xcell{White}\xcell{Gray}\xcell{Black}
\stopxrow
\stopxtablehead

% body
\startxtablebody[frame=off]
\startxrow \xcell{White}\xcell{TB}\xcell{TB}\xcell{}  \stopxrow
\startxrow \xcell{Gray} \xcell{TB}\xcell{TB}\xcell{TB}\stopxrow
\startxrow \xcell{Black}\xcell{}  \xcell{TB}\xcell{TB}\stopxrow
\stopxtablebody

\stopxtable

\stopANSWER

%e22.3-2
\startEXERCISE[exercise:22.3-2]
給出深度有限搜索算法在圖 22-6 上的運行過程。
假定深度有限搜索算法的第 5~7 行的 {\EMP for} 循環是以字母表順序依次處理每個節點,
並假定每條鄰接鏈表皆以字母表順序對裏面的節點進行了排序。
請給出每個節點的發現時間和完成時間,並給出每條邊的分類。
附圖 22-6:

\externalfigure[output/e22_3_2-1]
\stopEXERCISE

\startANSWER
如下圖,其中紅色的是樹邊,藍色的是後向邊,黑色的是前向邊,虛線是橫向邊。

\externalfigure[output/e22_3_2-2]
\stopANSWER

\stopsection

%e22.3-3
\startEXERCISE
給出圖 22-4 的深度優先搜索的括號化結構。
\stopEXERCISE

\startANSWER
\externalfigure[output/e22_3_3-1]
\stopANSWER

%e22.3-4
\startEXERCISE
證明:使用單個位來存放每個節點的顏色已經足夠。
這一點可以通過證明如下事實來得到:
如果將 \ALGO{DFS-VISIT} 的第 2 行刪除, \ALGO{DFS} 給出的結果相同。
\stopEXERCISE

\startANSWER
\TODO{略。}
\stopANSWER

%e22.3-5
\startEXERCISE
證明邊 \m{(u,v)} 是:
\startigBase[a]
\item 樹邊或前向邊當且僅當 \m{u.d < v.d < v.f < u.f};
\item 後向邊當且僅當 \m{v.d\le u.d < u.f \le v.f};
\item 橫向邊當且僅當 \m{v.d<v.f<u.d<u.f}。
\stopigBase
\stopEXERCISE

\startANSWER
\TODO{略。}
\stopANSWER

%e22.3-6
\startEXERCISE
證明:在無向圖中,根據深度有限搜索算法是先探索 \m{(u,v)} 還是先探索 \m{(v,u)} 来将邊 \m{(u,v)} 分類爲樹邊或者後向邊,
與根據分類列表中的 4 種類型的次序進行分類是等價的。
\stopEXERCISE

\startANSWER
\TODO{略。}
\stopANSWER

%e22.3-7
\startEXERCISE
請重寫 \ALGO{DFS} 算法的僞碼,以便使用棧來消除遞歸調用。
\stopEXERCISE

\startANSWER
\CLRSH{DFS-STACK(G, s)}
\startCLRS
for each vertex u in V(G) - {s}
	u.color = WHITE
	u.pred = NIL
time = 0
S = EMPTY

time = time + 1
s.td = time
s.color = GRAY
PUSH(S, s)
while S != EMPTY
	t = TOP(S)
	if exist v in V(G).Adj[t], v.color == WHITE	// v's adjacency list hasn't been fully examined
		v.pred = t
		time = time + 1
		v.td = time
		v.color = GRAY
		PUSH(S, v)
	else
		t = POP(S)
		time = time + 1
		v.tf = time
		v.color = BLACK
\stopCLRS
\stopANSWER

%e22.3-8
\startEXERCISE
請給出如下猜想的一個反例:
如果有向圖 \m{G} 包含一條從節點 \m{u} 到節點 \m{v} 的路徑,
並且在對圖 \m{G} 進行深度優先搜索時有 \m{u.d < v.d},
則節點 \m{v} 是節點 \m{u} 在深度優先森林中的一個後代。
\stopEXERCISE

\startANSWER
\externalfigure[output/e22_3_8-1]
\stopANSWER

%e22.3-9
\startEXERCISE
請給出如下猜想的一個反例:
如果有向圖 \m{G} 包含一條從節點 \m{u} 到節點 \m{v} 的路徑,
則任何對圖 \m{G} 的深度優先搜索都將導致 \m{v.d\le u.f}。
\stopEXERCISE

\startANSWER
參考上一題。
\stopANSWER

%e22.3-10
\startEXERCISE
修改深度優先搜索的僞碼,
讓其打印出有向圖 \m{G} 的每條邊及其分類。
並指出,如果圖 \m{G} 是無向圖,
要進行何種修改才能達到相同的效果。
\stopEXERCISE

\startANSWER
\TODO{略。}
\stopANSWER

%e22.3-11
\startEXERCISE
請解釋有向圖的一個節點 \m{u} 怎樣才能成爲深度優先樹中的唯一節點,
即使節點 \m{u} 同時有入邊和出邊。
\stopEXERCISE

\startANSWER
有頂點 \m{v_1,v_2,\ldots,v_k},存在邊 \m{(v_i,u)};
有頂點 \m{w_1,w_2,\ldots,w_l},存在邊 \m{(u,w_j)};
在執行 \ALGO{DFS} 時,可能先發現所有 \m{w_j},
然後再以 \m{u} 爲根進行搜索時, \m{u} 就會成爲一個孤立的節點。
\stopANSWER

%e22.3-12
\startEXERCISE
證明:我們可以在無向圖 \m{G} 上使用深度優先搜索來獲得圖 \m{G} 的連通分量,
並且深度優先森林所包含的樹的棵樹與 \m{G} 的連通分量數量相同。
更準確地說,請給出如何修改深度優先搜索來讓其給每個節點賦予一個介於 1 和 \m{k} 之間的整數值 \m{v.cc},
這裏 \m{k} 是 \m{G} 的連通分量數,
使得 \m{u.cc = v.cc} 當且僅當節點 \m{u} 和節點 \m{v} 處於同一個連通分量中。
\stopEXERCISE

\startANSWER
\TODO{略。}
\stopANSWER

%e22.3-13
\startEXERCISE\DIFFICULT
對於有向圖 \m{G=(V,E)} 來說,
如果 \m{u\leadsto v} 意味着圖 \m{G} 至多包含一條從 \m{u} 到 \m{v} 的簡單路徑,
則圖 \m{G} 是{\EMP 單連通圖}(singly connected)。
請給出一個有效算法來判斷一個有向圖是否是單連通圖。
\stopEXERCISE

\startANSWER
\TODO{略。}
\stopANSWER

\startsection[
  title={Topological sort},
]

%e22.4-1
\startEXERCISE
給出算法 \ALGO{TOPOLOGICAL-SORT} 運行於圖 22-8 上時所生成的節點次序。
這裏的所有假設與\refexercise{22.3-2} 一樣。
附圖 22-8:

\externalfigure[output/e22_4_1-1]
\stopEXERCISE

\startANSWER
\m{pnosmryvwzxuqt}。
\stopANSWER

%e22.4-2
\startEXERCISE
請給出線性時間算法,其輸入爲一個有向無環圖 \m{G=(V,E)} 以及兩個節點 \m{s} 和 \m{t},
其輸出是從節點 \m{s} 到節點 \m{t} 之間的簡單路徑的數量。
例如,對於圖 22-8 所示的有向無環圖,
從節點 \m{p} 到節點 \m{v} 一共有 4 條簡單路徑,
分別是 \m{pov}、 \m{poryv}、 \m{posryv} 和 \m{psryv}。
(本題僅要求計數簡單路徑的條數,而不要求將簡單路徑本身列舉出來。)
\stopEXERCISE

\startANSWER
先進行拓撲排序,並取 \m{s} 到 \m{t} 間的子集,令序列爲 \m{v_1, v_2, \ldots, v_n},
其中 \m{v_1=s}, \m{v_n = t}。給序列中每一項加一個屬性用來計數,初始值全爲 0。
然後從左至右遍歷此序列。如果 \m{v_i=s},則 \m{c_i = 1};
否則, \m{c_i = \sum_{v_j:(v_j,v_i)\in E}p[j]}。

\startCLRS
for i = 1 upto n
	if v[i] == s
		c[i] = 1
	else
		c[i] = 0
for i = 1 upto n - 1
	for each (v[i],v[j]) in E and j <= n
		c[j] += c[i]

return c[n]
\stopCLRS
\stopANSWER

%e22.4-3
\startEXERCISE
給出一個算法來判斷給定無向圖 \m{G=(V,E)} 是否包含一個環路。
算法運行時間應該在 \m{O(V)} 數量級,且與 \m{|E|} 無關。
\stopEXERCISE

\startANSWER
運行 \ALGO{DFS},如果發現前向邊,則有環路。
如果沒有環路,則 \m{|E| < |V|}。
如果有環路,則掃描的邊數肯定小於 \m{|V|},因此算法時間複雜度爲 \m{O(V)}。
\stopANSWER

%e22.4-4
\startEXERCISE
證明或反證下述論斷:
如果有向圖 \m{G} 包含環路,
則在算法 \ALGO{TOPOLOGICAL-SORT(G)} 所生成的節點序列裏,
圖 \m{G} 中與所生成序列不一直的“壞”邊的條數最少。
\stopEXERCISE

\startANSWER
不成立。選擇不同的起點,會得到不同的序列,壞邊的數目也不盡相同,因此不能保證壞邊數目最少。
如下圖,若從 \m{a} 點開始,則得到序列 \m{a,b,c},有 1 條壞邊 \m{(c,a)}。
若從 \m{b} 點開始,則得到序列 \m{b,c,a},有 2 條壞邊 \m{(a,c)}、 \m{(a,b)}。

\externalfigure[output/e22_4_4-1]
\stopANSWER

%e22.4-5
\startEXERCISE
在有向無環圖 \m{G=(V,E)} 上執行拓撲排序還有一種辦法,
就是重複尋找入度爲 0 的節點,輸出該節點,
將該節點及從其發出的邊從圖中刪除。
請解釋如何在 \m{O(V+E)} 時間內實現這種思想。
如果圖 \m{G} 包含環路,將會發生什麼情況?
\stopEXERCISE

\startANSWER
首先運行 \ALGO{DFS} 或 \ALGO{BFS} 可以在 \m{O(V+E)} 時間內統計出每個點的入度和出度,
以後在刪除邊的時候要維護這些信息。
每次輸出入度爲 0 的那個點並刪除其出邊,並更新相應節點的入度。
要執行 \m{O(V)} 次輸出和 \m{O(E)} 次的刪除。
總時間爲 \m{O(V+E)}。

如果有環路,那可能沒有入度爲 0 的點。
\stopANSWER

\stopsection

\startsection[
  title={Strongly connected components},
]

%e22.5-1
\startEXERCISE
如果在圖 \m{G} 中加入一條新的邊, \m{G} 中的強連通分量的數量會發生怎樣的變化?
\stopEXERCISE

\startANSWER
如果增加的邊位於某一強連通分量內部,則不會發生變化。
如果增加的邊跨越兩個分量,則分量的數量可能不變,也可能減少,甚至減少到只有一個強連通分量。
\stopANSWER

%e22.5-2
\startEXERCISE
給出算法 \ALGO{STRONGLY-CONNECTED-COMPONENTS} 在圖 22-6 上的運行過程。
具體要求是,給出算法第 1 行所計算出的完成時間和第 3 行所生成的森林。
假定 \ALGO{DFS} 的第 5~7 行的循環是以字母表順序來對節點進行處理,
並且連接鏈表中的節點也是以字母表順序排列好的。
附圖 22-6:

\externalfigure[output/e22_3_2-1]
\stopEXERCISE

\startANSWER
下圖爲 \m{G^T},其中的數字爲第一行所計算出的時間。
黑色邊連接兩個強連通分量,其他顏色的邊標記不同的顏色分量。

\externalfigure[output/e22_5_2-2]
\stopANSWER

%e22.5-3
\startEXERCISE
Bacon 教授聲稱,如果在第二次深度優先搜索時使用原始圖 \m{G} 而不是圖 \m{G} 的轉置圖 \m{G^T},
並且以完成時間的遞增次序來掃描節點,則計算強連通分量的算法將更加簡單。
這個更加簡單的算法總是能計算出正確的結果嗎?
\stopEXERCISE

\startANSWER
不正確。
\stopANSWER

%e22.5-4
\startEXERCISE
證明:對於任意有向圖 \m{G} 來說, \m{((G^T)^{SCC})^T=G^{SCC}}。
也就是說,轉置圖 \m{G^T} 的分量圖的轉置與圖 \m{G} 的分量圖相同。
\stopEXERCISE

\startANSWER
\m{G} 和 \m{G^T} 有相同的強連通分量,但分量間的連接關係正好相反。
\stopANSWER

%e22.5-5
\startEXERCISE
給出一個時間複雜度爲 \m{O(V+E)} 的算法來計算有向圖 \m{G=(V,E)} 的分量圖。
請確保在算法所生成的分量圖中,任意兩個節點之間至多存在一條邊。
\stopEXERCISE

\startANSWER
給各分量編號,記爲 \m{1,2,\ldots,k}, \m{k \le V},
並記錄每一個節點 \m{u} 所屬分量 \m{s[u]}。
遍歷所有邊 \m{(u,v)},如果 \m{s[u] \ne s[v]},則將 \m{(s[u],s[v])} 加入到新集合 \m{T} 中。
對 \m{T} 中元素進行基排序。
將 \m{T} 中所有與前一個元素不同的元素加入到新集合 \m{S} 中,
則 \m{S} 即爲所求。
\stopANSWER

%e22.5-6
\startEXERCISE
給定有向圖 \m{G=(V,E)},
請說明如何創建另一個圖 \m{G'=(V,E')},
使得:
(a) \m{G'} 的強連通分量與 \m{G} 的相同,
(b) \m{G'} 的分量圖與 \m{G} 的相同,
以及(c) \m{E'} 所包含的邊儘可能少。
請給出一個計算圖 \m{G'} 的快速算法。
\stopEXERCISE

\startANSWER
利用上一題的結果,另外每個分量內部進行 \ALGO{DFS},保留一個環路即可。
\stopANSWER

%e22.5-7
\startEXERCISE
給定有向圖 \m{G=(V,E)},
如果對於所有節點對 \m{u,v\in V},
我們有 \m{u\leadsto v} 或 \m{v\leadsto u},
則 \m{G} 是{\EMP 半連通}的。
請給出一個有效的算法來判斷圖 \m{G} 是否是半連通的。
證明算法的正確性並分析其運行時間。
\stopEXERCISE

\startANSWER
先生成強連通分量圖,若所有分量間存在一個線性鏈(開環),則此圖是半連通的。
\stopANSWER

\stopsection

\startsubject[
  title={Problems},
]

%e22-1
\startPROBLEM
(Classifying edges by breadth-first search)
深度優先搜索將圖中的邊分類爲樹邊、後向邊、前向邊和橫向邊。
廣度優先搜索也可以用來進行這種分類。
具體來說,廣度優先搜索將從源節點可以到達的邊劃分爲同樣的 4 種類型。
\startigBase[a]
\startitem
證明在對無向圖進行的廣度優先搜索中,下面的性質成立:
\startigBase[n]
\item 不存在後向邊,也不存在前向邊。
\item 對於每條樹邊 \m{(u,v)},我們有 \m{v.d=u.d+1}。
\item 對於每條橫向邊 \m{(u,v)},我們有 \m{v.d=u.d} 或 \m{v.d=u.d+1}。
\stopigBase
\stopitem

\startANSWER
\TODO{略。}
\stopANSWER

\startitem
證明在對有向圖進行廣度優先搜索時,下面的性質成立:
\startigBase[n]
\item 不存在前向邊。
\item 對於每條樹邊 \m{(u,v)},我們有 \m{v.d=u.d+1}。
\item 對於每條橫向邊 \m{(u,v)},我們有 \m{v.d\le u.d+1}。
\item 對於每條後向邊 \m{(u,v)},我們有 \m{0\le v.d\le u.d}。
\stopigBase
\stopitem

\startANSWER
\TODO{略。}
\stopANSWER

\stopigBase
\stopPROBLEM

%e22-2
\startPROBLEM
(Articulation points, bridges, and biconnected components)
設 \m{G=(V,E)} 爲一個連通無向圖。
圖 \m{G} 的{\EMP 銜接點}是指圖 \m{G} 中的一個節點,
刪除該節點將導致圖不連通。
圖 \m{G} 的{\EMP 橋}是指圖中的一條邊,刪除該邊,圖就不再連通。
圖 \m{G} 的{\EMP 雙連通分量}是指一個最大的邊集合,
裏面的任意兩條邊都處於同一條簡單環路中。
圖 22-10 描述的就是這些概念的定義。
我們可以使用深度優先搜索算法來判斷 \m{G} 的銜接點、橋和雙連通分量。
設 \m{G_\pi = (V, E_\pi)} 來爲圖 \m{G} 的深度優先樹。
附圖 22-10:

\externalfigure[output/p22-2-1]

\startigBase[a]\startitem
證明: \m{G_\pi} 的根節點是圖 \m{G} 的銜接點當且僅當他在 \m{G_\pi} 中至少有兩個子節點。
\stopitem\stopigBase

\startANSWER
如果根節點是銜接點,如果只有一個子節點,那麼刪除根節點,不會影響圖的連通性,則不是銜接點,與假設不符。

如果根節點有至少兩個子節點,由於沒有連接各子樹的橫向邊,那麼刪除根節點會影響圖的連通性,符合銜接點的定義。
\stopANSWER

\startigBase[continue]\startitem
設節點 \m{v} 爲 \m{G_\pi} 的一個非根節點。證明:
 \m{v} 是 \m{G} 的銜接點當且僅當節點 \m{v} 有一個子節點 \m{s},
且沒有任何從節點 \m{s} 或任何 \m{s} 的後代節點指向 \m{v} 的真祖先的後向邊。
\stopitem\stopigBase

\startANSWER
有後向邊會導致成環。
\stopANSWER

\startigBase[continue]\startitem
定義:
\startformula
v.low = \min \startcases
\NC v.d \MC \NR
\NC w.d \MC\text{:\m{(u,w)} 是節點 \m{v} 的某個後代節點 \m{u} 的一條後向邊}\NR
\stopcases
\stopformula
請說明如何在 \m{O(E)} 的時間內爲所有節點 \m{v} 計算出 \m{v.low} 的值。
\stopitem\stopigBase

\startANSWER
\CLRSH{DFS-VISIT(u)}
\startCLRS
color[u] = GRAY
time = time + 1
d[u] = time
low[u] = d[u]
for each v Adj[u]
	if color[v] = WHITE
		pi[v] = u
		DFS-VISIT(v)
		if low[v] < low[u]
			low[u] = low[v]
	elseif color[v] = GRAY		// back edge
		if d[v] < low[u]
			low[u] = d[v]
color[u] = BLACK
time = time + 1
f[u] = time
\stopCLRS
\stopANSWER

\startigBase[continue]\startitem
說明如何在 \m{O(E)} 時間內計算出圖 \m{G} 的所有銜接點。
\stopitem\stopigBase

\startANSWER
對於根節點,檢查有幾個子節點,如果多於一個,則爲銜接點,在(a)中已經證明。
對於其他節點 \m{v},如果 \m{v} 有一個孩子 \m{b},滿足 \m{low[b]\ge d[v]},
意味着以 \m{b} 爲根的子樹上沒有指向 \m{v} 的祖先的後向邊,
即 \m{v} 是銜接點,因爲刪除 \m{v} 會破壞圖的連通性。
\stopANSWER

\startigBase[continue]\startitem
證明:圖 \m{G} 的一條邊是橋當且僅當該邊不屬於 \m{G} 中的任何簡單環路。
\stopitem\stopigBase

\startANSWER
根據環路的定義,從環路中的任意節點開始,都可以返回到該節點的路。
因此,移除環路中的任意一條邊都不會孤立這個節點。
如果邊 \m{(u,v)} 沒有在環路中,
則只有一條從 \m{u} 到 \m{v} 的路。
如果可以從 \m{v} 沿其他路徑返回到 \m{u},則會形成一個環路。
因此,移除 \m{(u,v)} 會斷開 \m{u} 和 \m{v} 的連接,因此 \m{(u,v)} 是橋。
\stopANSWER

\startigBase[continue]\startitem
說明如何在 \m{O(E)} 時間內計算出圖 \m{G} 的所有橋。
\stopitem\stopigBase

\startANSWER
在 \ALGO{DFS-VISIT} 中,計算 \m{low[v]} 時,如果發現 \m{u} 的子節點 \m{v} 滿足 \m{low[v]>d[u]},
則我們可以任務此邊是橋,因爲移除他會導致 \m{u} 和 \m{v} 失連,這在(d)中已經證明。
\stopANSWER

\startigBase[continue]\startitem
證明: \m{G} 的雙連通分量是 \m{G} 的非橋邊的一個劃分。
\stopitem\stopigBase

\startANSWER
根據雙連通分量的定義,其內所有邊都不是橋,由由於其“最大”性,外部的邊都是橋。
因此他是 \m{G} 的非橋邊的一個劃分。
\stopANSWER

\startigBase[continue]\startitem
給出一個 \m{O(E)} 時間複雜度的算法來給圖 \m{G} 的每條邊 \m{e} 做出標記。
這個標記是一個正整數 \m{e.bcc} 且滿足 \m{e.bcc = e'.bcc} 當且僅當邊 \m{e} 和邊 \m{e'} 在同一個雙連通分量中。
\stopitem\stopigBase

\startANSWER
移除橋邊,遍歷銜接點。
\stopANSWER

\stopPROBLEM

%p22-3
\startPROBLEM
(Euler tour)
強連通有向圖 \m{G=(V,E)} 中的一個歐拉回路是指一條遍歷圖 \m{G} 中每條邊恰好一次的環路。
不過,這條環路可以多次訪問同一個節點。

\startigBase[a]\startitem
證明:圖 \m{G} 有一條歐拉回路當且僅當對於圖中的每個節點 \m{v},
有 \m{in-degree(v) = out-degree(v)}。
\stopitem\stopigBase

\startANSWER
如果圖 \m{G} 有歐拉回路 \m{C},所有節點的出度和入度顯然是相等的。

反過來,如果所有節點的入度和出度均相等,任取節點 \m{v},
則肯定存在一條路徑從 \m{v} 出發最後又返回到 \m{v}。
不停的刪除簡單環路,並刪除孤立的節點,然後再從被刪除了部分邊但還未被孤立的節點開始,
重新尋找簡單環路……
最終將所有簡單環路組合起來就是一條歐拉回路。
\stopANSWER

\startigBase[continue]\startitem
給出一個複雜度爲 \m{O(E)} 的算法來找出圖 \m{G} 的一條歐拉回路。
(\hint 對邊不相交環路進行歸併。)
\stopitem\stopigBase

\startANSWER
先以 \m{O(|E|)} 時間檢查是不是所有節點的入度都等於其出度。
如果是,然後再以不停搜索、刪除簡單環路的方式給出歐拉回路。
\stopANSWER

\stopPROBLEM

%p22-4
\startPROBLEM
(Reachability)
設 \m{G=(V,E)} 爲一個有向圖,
且每個節點 \m{u\in V} 都標有一個唯一的整數值標記 \m{L(u)},
 \m{L(u)} 的取值爲集合 \m{\{1,2,\ldots,|V|\}}。
對於每個節點 \m{u\in V},
設 \m{R(u)=\{v\in V: u\leadsto v\}} 爲從節點 \m{u} 可以到達的所有節點的集合。
定義 \m{\min(u)} 爲 \m{R(u)} 中標記爲最小的節點,
即 \m{\min(u)} 爲節點 \m{v},滿足 \m{L(v)=\min\{L(w): w\in R(u)\}}。
請給出一個時間複雜度爲 \m{O(V+E)} 的算法來計算所有節點 \m{u\in V} 的 \m{\min(u)}。
\stopPROBLEM

\startANSWER
從標記最小的節點開始反向深度優先搜索,路徑上的所有點的結果均爲起始點的標記值。
完成後,若還有僞確定的點,繼續執行上述過程。
\stopANSWER

\stopsubject%Problems

\stopchapter
\stopcomponent
