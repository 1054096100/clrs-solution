\startsection[
  title={Insertion},
]

%e13.3-1
\startEXERCISE
在 \ALGO{RB-INSERT} 的第 16 行,將新節點 z 着爲紅色,從而可能破壞紅黑樹性質 4。
如果將 z 着爲黑色,則不會破壞紅黑樹的性質 4。
爲什麼不將 z 着爲黑色?
\stopEXERCISE

\startANSWER
如果將 z 着爲黑色,會破壞紅黑樹的性質 5。
\stopANSWER

%e13.3-2
\startEXERCISE
現有一紅黑樹,初始爲空,將關鍵字 41、 38、 31、 12、 19、 8 相繼插入,試畫出結果。
\stopEXERCISE

\startANSWER
請參閱 \simpleurl{https://www.cs.usfca.edu/~galles/visualization/RedBlack.html}。

\externalfigure[output/e13_3_2-1]
\stopANSWER

%e13.3-3
\startEXERCISE
假設圖 13-5 和圖 13-6 中子樹 \m{\alpha}、 \m{\beta}、 \m{\gamma}、 \m{\delta} 和 \m{\epsilon} 的黑高都是 k。
給兩圖中的所有節點都標上黑高,以驗證圖中所示的轉換能保持性質 5。
\stopEXERCISE

\startANSWER
略。
\stopANSWER

%e13.3-4
\startEXERCISE
Teach 教授擔心 \ALGO{RB-INSERT-FIXUP} 會將 \m{T.nil.color} 設爲紅色,
從而在 \m{z} 爲根時,第 1 行的測試不會終止循環。
試說明 \ALGO{RB-INSERT-FIXUP} 永遠不會將 \m{T.nil.color} 設爲紅色。
\stopEXERCISE

\startANSWER
略。
\stopANSWER

%e13.3-5
\startEXERCISE
用 \ALGO{RB-INSERT} 插入 \m{n} 個節點形成一棵紅黑樹。
證明:如果 \m{n>1},則該樹至少有一個紅節點。
\stopEXERCISE

\startANSWER
\m{n=2} 時,有一個內部節點是紅的。
以後每次運行 \ALGO{RB-INSERT-FIXUP} 時,
如果是 case 1,則 \m{z} 保持紅色,case 2 和 case 3 都不會減少紅色節點的數量。
\stopANSWER

%e13.3-6
\startEXERCISE
說明如果紅黑樹中不提供父指針,如果有效地實現 \m{RB-INSERT}。
\stopEXERCISE

\startANSWER
可以採用自頂向下的插入,一旦紅黑樹性質遭到破壞就通過修改顏色、旋轉恢復紅黑樹的性質。
 case 2 和 case 3 都能在本節點完成,
 case 1 會使節點 \m{z} 上升兩層至 \m{z.p.p},
但這時不會再遇到 case 1,因爲向下搜索時已經把這種情況消除了。
\stopANSWER

\stopsection
