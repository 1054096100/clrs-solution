\startsection[
  title={Deletion},
]

%e13.4-1
\startEXERCISE
證明:執行 \ALGO{RB-DELETE-FIXUP} 之後,樹根一定是黑色的。
\stopEXERCISE

\startANSWER
\ALGO{RB-DELETE-FIXUP} 中的 while 循環會向上搜索,直到到達樹根或者一個紅色節點。
如果找到樹根,則會將其染成黑色,無論這個根是怎麼來的。

如果在到達根節點之前先碰到了紅色節點,則根節點還會繼續保持黑色。
\stopANSWER

%e13.4-2
\startEXERCISE
證明:在 \ALGO{RB-DELET} 中,如果 \m{x} 和 \m{x.p} 都是紅色的,
可以通過調用 \ALGO{RB-DELETE-FIXUP(T, x)} 來恢復性質 4。
\stopEXERCISE

\startANSWER
如果 \m{x} 是紅色,則 \m{y} (即 \m{x} 的父節點,
同時也是 \m{z} 或者 \m{z} 的後繼)有至少一個非空孩子(\m{x})且不是紅色
(否則, \m{x} 會是黑色),否則會針對 \m{x} 調用 \ALGO{RB-DELETE-FIXUP}。
 \m{y} 也有一個黑色兄弟節點。

\m{y} 曾經只有一個孩子 \m{x}:如果 \m{y} 跟 \m{z} 一樣,只有一個孩子,
則 \m{z} 只有左孩子, \m{x} 就是左孩子, \m{z} 只有右孩子, \m{x} 就是右孩子;
如果 \m{y} 是 \m{z} 的後繼( \m{z} 有兩個孩子),那麼 \m{x} 就是右孩子。
這兩種情況類似,我們以 \m{x} 是 \m{y} 的左孩子爲例:

結果取決於 \m{x} 的兄弟 \m{w} 是否有兩個黑色孩子,前者是 case 2,後者是 case 3;
然後都會變成 case 4,並恢復性質 4。
\stopANSWER

%e13.4-3
\startEXERCISE
在練習 13.3-2 中,將關鍵字 41、 38、 31、 12、 19、 8 連續插入一棵初始爲空的樹,
最終得到了一棵紅黑樹。請給出從該樹中連續刪除關鍵字 8、 12、 19、 31、 38、 41 後
的紅黑樹。
\stopEXERCISE

\startANSWER
\startcombination[nx=6,location={top}]
{\externalfigure[output/e13_4_3-1]}{}
{\externalfigure[output/e13_4_3-2]}{}
{\externalfigure[output/e13_4_3-3]}{}
{\externalfigure[output/e13_4_3-4]}{}
{\externalfigure[output/e13_4_3-5]}{}
{\externalfigure[output/e13_4_3-6]}{}
\stopcombination
\stopANSWER

%e13.4-4
\startEXERCISE
在 \ALGO{RB-DELETE-FIXUP} 代碼的哪些行中,可能會檢查或修改哨兵 \m{T.nil}?
\stopEXERCISE

\startANSWER
刪除黑色節點後,首先會告訴其孩子(所刪除節點最多只有一個孩子),
如果他沒有孩子,則會檢查或修改 \m{T.nil}。

\m{x} 的兄弟 \m{w} 沒有孩子,或者沒有左孩子但要左旋,或者沒有右孩子卻要右旋。
\stopANSWER

%e13.4-5
\startEXERCISE
在圖 13-7 的每種情況中,給出所示子樹的根節點至每棵子樹 \m{\alpha}、 \m{\beta}、
 \m{\ldots}、 \m{\epsilon} 之間的黑色節點個數,
並驗證他們在轉換後保持不變。
當一個節點的 color 屬性爲 c 或 c' 時,
在計數中用記號 count(c) 或 count(c') 來表示。
\stopEXERCISE

\startANSWER
\TODO{}
\stopANSWER

%e13.4-6
\startEXERCISE
Skelton 和 Baron 教授擔心在 \ALGO{RB-DELETE-FIXUP} 的 case 1 開始時,
節點 \m{x.p} 可能不是黑色的,這樣第 5 ~ 6 行就是錯的。
證明: \m{x.p} 在 case 1 開始時肯定是黑色的。
\stopEXERCISE

\startANSWER
\TODO{}
\stopANSWER

%e13.4-7
\startEXERCISE
假設用 \ALGO{RB-INSERT} 將一個節點 \m{x} 插入一棵紅黑樹,
緊接着又用 \ALGO{RB-DELETE} 將他從樹中刪除。
所得紅黑樹有什麼變化?
\stopEXERCISE

\startANSWER
可能不同。
新加入節點可能會導致某些旋轉操作,刪除後不會旋轉回來。
即使樹的結構不變,節點的顏色也可能發生變化。
\stopANSWER

\stopsection
