\startsection[
  title={Rotations},
]

%e13.2-1
\startEXERCISE
寫出 \ALGO{RIGHT-ROTATE} 的僞碼。
\stopEXERCISE

\startANSWER
\CLRSH{RIGHT-ROTATE(T, y)}
\startCLRS
x = y.left
y.left = x.right
if x.right != T.nil
	x.right.p = y
x.p = y.p
if y.p == T.nil
	T.root = x
elseif y == y.p.left
	y.p.left = x
else y.p.right = x
x.right = y
y.p = x
\stopCLRS
\stopANSWER

%e13.2-2
\startEXERCISE
證明:對於任何一棵二叉搜索樹,如果節點數爲 \m{n},則有 \m{n-1} 種可能的旋轉。
\stopEXERCISE

\startANSWER
只有孩子節點才能旋轉,而根節點不是任何節點的孩子。
\stopANSWER

%e13.2-3
\startEXERCISE
如圖 13-2 中左側樹,在子樹 \m{\alpha}、 \m{\beta}、 \m{\gamma} 中分別任取節點 \m{a}、 \m{b}、 \m{c}。
在將節點 \m{x} 左旋後, \m{a}、 \m{b}、 \m{c} 的深度有什麼變化? 附圖 13-2:

\externalfigure[output/e13_2_3-1]
\stopEXERCISE

\startANSWER
\m{\alpha} 下降一层, \m{\beta} 深度不变, \m{\gamma} 上升一层。
\stopANSWER

%e13.2-4
\startEXERCISE
證明:任一二叉搜索樹,其節點個數爲 \m{n},則可以通過 \m{O(n)} 次旋轉,
轉換爲其他任一含 \m{n} 個節點的二叉搜索樹。
(\hint 先證明至多 \m{n-1} 次右旋足以將樹轉變爲一條右側伸展的鏈。)
\stopEXERCISE

\startANSWER
右旋會使左孩子數少 1,而右孩子多 1。
\stopANSWER

%e13.2-5
\startEXERCISE
有二叉搜索樹 \m{T_1},通過一系列右旋操作變成二叉搜索樹 \m{T_2},
則稱 \m{T_1} 可以{\EMP 右轉(right-converted)}成 \m{T_2}。
試舉反例,描述 \m{T_1} 不能右轉成 \m{T_2} 的情況。
然後證明:如果 \m{T_1} 可以右轉成 \m{T_2},那麼可以通過 \m{O(n^2)} 次右旋來實現。
\stopEXERCISE

\startANSWER
\externalfigure[output/e13_2_5-1]
\externalfigure[output/e13_2_5-2]

\m{T_1} 中右孩子都不可能變成根節點。
所以新的根節點只能是 \m{T_1} 的根節點到最小節點路徑上的節點。
我們可以在 \m{O(n)} 時間解決根節點,然後遞迴解決左右子樹,時間複雜度爲:
\m{T(n)=T(n_L)+T(n_R)+O(n)=O(n^2)},其中 \m{n_L+n_R=n-1}。
\stopANSWER

\stopsection
