\startsection[
  title={Solving systems of linear equations},
  reference=section:solve_system_of_linear_equation,
]

%e28.1-1
\startEXERCISE
採用正向替換法求解下面方程組:
\startformula
\left[\startmatrix
\NC 1 \NC 0 \NC 0 \NR
\NC 4 \NC 1 \NC 0 \NR
\NC -6\NC 5 \NC 1 \NR
\stopmatrix\right]
\left[\startmatrix
\NC x_1 \NR
\NC x_2 \NR
\NC x_3 \NR
\stopmatrix\right]
=
\left[\startmatrix
\NC 3 \NR
\NC 14 \NR
\NC -7 \NR
\stopmatrix\right]
\stopformula
\stopEXERCISE

\startANSWER
\startformula
\left[\startmatrix
\NC x_1 \NR
\NC x_2 \NR
\NC x_3 \NR
\stopmatrix\right]
=
\left[\startmatrix
\NC 3 \NR
\NC 2 \NR
\NC 1 \NR
\stopmatrix\right]
\stopformula
\stopANSWER

%e28.1-2
\startEXERCISE
找到下面矩陣的一個 LU 分解:
\startformula
\left[\startmatrix
\NC 4 \NC -5 \NC 6 \NR
\NC 8 \NC -6 \NC 7 \NR
\NC 12\NC -7 \NC 12 \NR
\stopmatrix\right]
\stopformula
\stopEXERCISE

\startANSWER
\startcombination[3*1]
{\externalfigure[output/e28_1_2-1]}{}
{\externalfigure[output/e28_1_2-2]}{}
{\externalfigure[output/e28_1_2-3]}{}
\stopcombination

\startformula
{\left[\startformula\startmatrix
\NC 4 \NC -5 \NC 6 \NR
\NC 8 \NC -6 \NC 7 \NR
\NC 12\NC -7 \NC 12 \NR
\stopmatrix\stopformula\right]}{}
=
{\left[\startformula\startmatrix
\NC 1 \NC 0 \NC 0 \NR
\NC 2 \NC 1 \NC 0 \NR
\NC 3 \NC 2 \NC 1 \NR
\stopmatrix\stopformula\right]}{}
{\left[\startformula\startmatrix
\NC 4 \NC -5 \NC 6 \NR
\NC 0 \NC 4 \NC -5 \NR
\NC 0 \NC 0 \NC 4 \NR
\stopmatrix\stopformula\right]}{}
\stopformula
\stopANSWER

%e28.1-3
\startEXERCISE
利用 LUP 分解求解如下方程組:
\startformula
\left[\startmatrix
\NC 1 \NC 5 \NC 4 \NR
\NC 2 \NC 0 \NC 3 \NR
\NC 5 \NC 8 \NC 2 \NR
\stopmatrix\right]
\left[\startmatrix
\NC x_1 \NR
\NC x_2 \NR
\NC x_3 \NR
\stopmatrix\right]
=
\left[\startmatrix
\NC 12 \NR
\NC 9 \NR
\NC 5 \NR
\stopmatrix\right]
\stopformula
\stopEXERCISE

\startANSWER
\startcombination[3*2]
{\externalfigure[output/e28_1_3-1][scale=600]}{}
{\externalfigure[output/e28_1_3-2][scale=600]}{}
{\externalfigure[output/e28_1_3-3][scale=600]}{}
{\externalfigure[output/e28_1_3-4][scale=600]}{}
{\externalfigure[output/e28_1_3-5][scale=600]}{}
{\externalfigure[output/e28_1_3-6][scale=600]}{}
\stopcombination

\startformula
\left[\startmatrix
\NC 0 \NC 0 \NC 1 \NR
\NC 1 \NC 0 \NC 0 \NR
\NC 0 \NC 1 \NC 0 \NR
\stopmatrix\right]
\left[\startmatrix
\NC 1 \NC 5 \NC 4 \NR
\NC 2 \NC 0 \NC 3 \NR
\NC 5 \NC 8 \NC 2 \NR
\stopmatrix\right]
=
\left[\startmatrix
\NC 1 \NC 0 \NC 0 \NR
\NC 1/5 \NC 1 \NC 0 \NR
\NC 2/5 \NC -16/17 \NC 1 \NR
\stopmatrix\right]
\left[\startmatrix
\NC 5 \NC 8 \NC 2 \NR
\NC 0 \NC 17/5 \NC 18/5 \NR
\NC 0 \NC 0 \NC 95/17 \NR
\stopmatrix\right]
\stopformula

\startformula
\left[\startmatrix
\left[\startmatrix
\NC 5 \NC 8 \NC 2 \NR
\NC 0 \NC 17/5 \NC 18/5 \NR
\NC 0 \NC 0 \NC 95/17 \NR
\stopmatrix\right]
\NC 1 \NC 0 \NC 0 \NR
\NC 1/5 \NC 1 \NC 0 \NR
\NC 2/5 \NC -16/17 \NC 1 \NR
\stopmatrix\right]
y =
\left[\startmatrix
\NC 0 \NC 0 \NC 1 \NR
\NC 1 \NC 0 \NC 0 \NR
\NC 0 \NC 1 \NC 0 \NR
\stopmatrix\right]
\left[\startmatrix
\NC 12 \NR
\NC 9 \NR
\NC 5 \NR
\stopmatrix\right]
=
\left[\startmatrix
\NC 5 \NR
\NC 12 \NR
\NC 9 \NR
\stopmatrix\right]
\stopformula

\startformula
\left[\startmatrix
\NC 5 \NC 8 \NC 2 \NR
\NC 0 \NC 17/5 \NC 18/5 \NR
\NC 0 \NC 0 \NC 95/17 \NR
\stopmatrix\right]
x =
y =
\left[\startmatrix
\NC 5 \NR
\NC 11 \NR
\NC 295/17 \NR
\stopmatrix\right]
\stopformula

\startformula
x =
\left[\startmatrix
\NC -3/19 \NR
\NC -1/19 \NR
\NC 59/19 \NR
\stopmatrix\right]
\stopformula
\stopANSWER

%e28.1-4
\startEXERCISE
請描述對角矩陣的 LUP 分解。
\stopEXERCISE

\startANSWER
\m{P = I}, \m{L = I}, \m{U = P}。
\stopANSWER

%e28.1-5
\startEXERCISE
請描述置換矩陣的 LUP 分解,並證明其唯一性。
\stopEXERCISE

\startANSWER
\m{P = A^T}, \m{L = I}, \m{U = I}。
\stopANSWER

%e28.1-6
\startEXERCISE
證明:對所有 \m{n\ge 1},存在一個 \m{n\times n} 奇異矩陣,
具有一個 LU 分解。
\stopEXERCISE

\startANSWER
\TODO{略。}
\stopANSWER

%e28.1-7
\startEXERCISE
在 \ALGO{LU-DECOMPOSITION} 中,
當 \m{k=n} 時,是否有必要執行最外層的 for 循環迭代?
 \ALGO{LUP-DECOMPOSITION} 中情況又如何?
\stopEXERCISE

\startANSWER
當 \m{k=n} 時,都沒有必要執行最外層的 for 循環。
\stopANSWER

\stopsection
