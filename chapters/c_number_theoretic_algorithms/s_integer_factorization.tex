\startsection[
  title={Integer factorization},
]

%e31.9-1
\startEXERCISE
在圖 31-7(a)所示的執行過程中,
過程 \ALGO{POLLARD-RHO} 在何時輸出 1387 的因子 73?
\stopEXERCISE

\startANSWER
\TODO{略。}
\stopANSWER

%e31.9-2
\startEXERCISE
假設給定函數 \m{f:\integers_n\rightarrow\integers_n} 和
一個初值 \m{x_0 \in \integers_n}。
定義 \m{x_i=f(x_{i-1})}, \m{i=1,2,\ldots}。
令 \m{t} 和 \m{u>0} 是滿足如下算式的最小值:
\startformula
x_{t+i}=x_{t+u+i}(i=0,1,\ldots)
\stopformula
在 Pollard 的 rho 算法術語中,
 \m{t} 爲 rho 的尾的長度。
試寫出一個計算 \m{t} 和 \m{u} 的值的有效算法,
並分析其運行時間。
\stopEXERCISE

\startANSWER
\TODO{略。}
\stopANSWER

%e31.9-3
\startEXERCISE
爲了發現形如 \m{p^e} 的數(其中 \m{p} 是素數, \m{e>1})的一個因子,
 \ALGO{POLLARD-RHO} 要執行多少步?
\stopEXERCISE

\startANSWER
\TODO{略。}
\stopANSWER

%e31.9-4
\startEXERCISE\DIFFICULT
\ALGO{POLLARD-RHO} 的缺點之一是,
在其遞歸過程的每一步,
都要計算一個 \m{\gcd}。
然而,可以對 \m{\gcd} 的計算進行批處理:
通過累計一行中數個連續的 \m{x_i} 的積,
然後在 \m{\gcd} 計算中使用該積而不是 \m{x_i}。
請詳細描述如何實現這一思想,
爲什麼他是可行的,
以及在處理一個 \m{\beta} 位數 \m{n} 時,
所選取的最有效的批處理規模是多大?
\stopEXERCISE

\startANSWER
\TODO{略。}
\stopANSWER

\stopsection
