\startsection[
  title={The Chinese remainder theorem},
]

%31.5-1
\startEXERCISE
找出所有解,使得 \m{x\equiv 4 (\mod 5)} 和 \m{x\equiv 5 (\mod 11)} 同時成立。
\stopEXERCISE

\startANSWER
\startformula
x \equiv 49 (\mod 55)
\stopformula

\startxtable[
    option=max,
    align={middle,lohi},
    split=yes,
    header=repeat,
    footer=repeat,
    offset=.25em,
]

% head
\startxtablehead[frame=off,bottomframe=on]
\startxrow[foregroundstyle=bold,]
  \xcell[rightframe=on]{}\processcommalist[ 0, 1, 2, 3, 4, 5, 6, 7, 8, 9,10]\xcell
\stopxrow
\stopxtablehead

% body
\startxtablebody[frame=off]
\startxrow \xcell[rightframe=on]{0}\processcommalist[ 0,45,35,25,15, 5,50,40,30,20,10]\xcell \stopxrow
\startxrow \xcell[rightframe=on]{1}\processcommalist[11, 1,46,36,26,16, 6,51,41,31,21]\xcell \stopxrow
\startxrow \xcell[rightframe=on]{2}\processcommalist[22,12, 2,47,37,27,17, 7,52,42,32]\xcell \stopxrow
\startxrow \xcell[rightframe=on]{3}\processcommalist[33,23,13, 3,48,38,28,18, 8,53,43]\xcell \stopxrow
\startxrow \xcell[rightframe=on]{4}\processcommalist[44,34,24,14, 4,{\bfa 49},39,29,19, 9,54]\xcell \stopxrow
\stopxtablebody

\stopxtable

\stopANSWER

%31.5-2
\startEXERCISE
找出被 9、 8、 7 除時,餘數分別爲 1、 2、 3 的所有整數 x。
\stopEXERCISE

\startANSWER
由於 \m{\lcm(9,8,7) = 504},則:
\startformula
x \equiv 10 (\mod 504)
\stopformula
\stopANSWER

%e1.5-3
\startEXERCISE
論證:在定理 31.27 的定義下,如果 \m{\gcd(a,n)=1},則:
\startformula
(a^{-1} \mod n) \leftrightarrow
  ((a_1^{-1} \mod n_1),(a_2^{-1} \mod n_2),\ldots,(a_k^{-1} \mod n_k))
\stopformula
\stopEXERCISE

\startANSWER
\TODO{需要證明}
\stopANSWER

%31.5-4
\startEXERCISE[exercise:31.5-4]
在定理 31.27 的定義下,證明:
對於任意的多項式 f,方程 \m{f(x)\equiv 0 (\mod n)} 的根的個數
等於 \m{f(x)\equiv 0 (\mod n_1), f(x)\equiv 0 (\mod n_2), \ldots, f(x) \equiv 0 (\mod n_k)} 中每個方程根的個數的積。
\stopEXERCISE

\startANSWER
\TODO{需要證明}
\stopANSWER

\stopsection
