\startsection[
  title={The RSA public-key cryptosystem},
]

%e31.7-1
\startEXERCISE
考慮一個 RSA 密鑰集合,
其中 \m{p=11,q=29,n=319,e=3}。
在密鑰中用到的 \m{d} 值應當是多少?
對消息 \m{M=100} 加密後得到什麼消息?
\stopEXERCISE

\startANSWER
\TODO{略。}
\stopANSWER

%e31.7-2
\startEXERCISE
證明:如果 Alice 的公開指數 \m{e} 等於 3,
並且對方獲得了 Alice 的祕密指數 \m{d},
其中 \m{0<d<\phi(n)},
則對方能夠在關於 \m{n} 的位數的多項式時間內
對 Alice 的模 n 進行分解。
(儘管不用證明下列結論,
但你也許會對下列事實感興趣:
即使條件 \m{e=3} 被去除,
上述結論仍然成立。參見 Miller[255]。)
\stopEXERCISE

\startANSWER
\TODO{略。}
\stopANSWER

%e31.7-1
\startEXERCISE
證明:在如下意義中, RSA 是乘法的:
\startformula
P_A(M_1)P_A(M_2) \equiv P_A(M_1 M_2) (\mod n)
\stopformula
利用這個事實證明:
如果對方有一個過程,
對 \m{\integers_n} 中的用 \m{P_A} 加密的消息,
他能夠有效地解密出其中的百分之一,
則他可以運用一種概率性算法,
以較大概率爲每一條用 \m{P_A} 加密的信息進行解密。
\stopEXERCISE

\startANSWER
\TODO{略。}
\stopANSWER

\stopsection
