\startsection[
  title={Primality testing},
]

%e31.8-1
\startEXERCISE
證明:如果一個奇整數 \m{n>1} 不是素數或素數的冪,
則存在一個以 \m{n} 爲模的 1 的非平凡平方根。
\stopEXERCISE

\startANSWER
令 \m{n = (p_1 ^{e_1})(p_2 ^{e_2})\ldots (p_i ^{e_i})},
其中 \m{p_1,p_2,\ldots,p_n \ge 3}。
由於奇整數 \m{n} 不是素數或者素數的冪,
則 \m{n} 必然是奇合數,
至少存在兩個不同的奇素銀子,滿足 \m{i\ge 2},且 \m{p_1 \ne p_2}。

由於 \m{p_1,p_2,\ldots,p_n\ge 3} 爲互不相同的素數,
所以 \m{p_1^{e_1},p_2^{e_2},\ldots,p_i^{e_i}} 是兩兩互質。
根據孫子定理:求 \m{x^2\equiv 1 (\mod n)} 的解,
等價於求方程組 \m{x^2\equiv 1 (\mod p_i^{e_i})} 的解,
其中 \m{i=1,2,\ldots,k}。
根據定理 31.34 對於奇素數 \m{p_i}, \m{e_i\ge 1},
方程 \m{x^2\equiv 1 (\mod p_i^{e_i})} 有且僅有兩個解,
又根據\refexercise{31.5-4} 的結論可知:
由於存在至少兩個素銀子 \m{p_1,p_2},
所以至少存在如下兩個方程:
\startformula
f(x)=x^2 - 1 \equiv 0 (\mod p_1^{e_1})
\stopformula
\startformula
f(x) = x^2 - 1 \equiv 0 (\mod p_2^{e_2})
\stopformula
他們的解的乘積即爲原方程 \m{x^2-1\equiv 0 (\mod n)} 的解,
數目不少於 4,出去平凡根 \m{\pm 1} 外,
肯定還存在非平凡根。
\stopANSWER

%e31.8-2
\startEXERCISE
可以把歐拉定理稍微加強爲如下形式:對所有 \m{a\in \integers_n^*}:
\startformula
a^{\lambda(n)} \equiv 1 (\mod n)
\stopformula
其中 \m{n= p_1^{e_1} p_2^{e_2}\ldots p_r^{e_r}},且 \m{\lambda(n)} 定義爲:
\startformula
\lambda(n) = 1cm(\phi(p_1^{e_1}),\ldots,\phi(p_r^{e_r}))
\stopformula
證明 \m{\lambda(n)|\phi(n)},
如果 \m{\lambda(n)|n-1},
則合數 \m{n} 爲 Carmichael 數。
最小的 Carmichael 數爲 \m{561=3\cdot 11 \cdot 17};
這裏, \m{\lambda(n)=1cm(2,10,16)=80},
他可以整除 560。
證明 Carmichael 數必須既是“無平方數”(不能被任何素數的平方所整除),
又是至少三個素數的積。
(因此, Carmichael 數並不常見。)
\stopEXERCISE

\startANSWER
\TODO{略。}
\stopANSWER

%e31.8-3
\startEXERCISE
證明:如果 \m{x} 是以 \m{n} 爲模的 1 的非平凡平方根,
則 \m{\gcd(x-1,n)} 和 \m{\gcd(x+1,n)} 都是 \m{n} 的非平凡約數。
\stopEXERCISE

\startANSWER
\TODO{略。}
\stopANSWER

\stopsection
