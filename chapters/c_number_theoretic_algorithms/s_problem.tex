\startsubject[
  title={Problems},
]

%p31-1
\startPROBLEM
(Binary gcd algorithm)
與計算餘數的執行速度相比,
大多數計算機執行減法運算、
測試一個二進制整數的奇偶性運算以及
折半運算的執行速度都要更快些。
本題所討論的{\EMP 二進制 gcd 算法}中避免了歐幾里德算法中計算餘數的過程。

\startigBase[a]\startitem
證明:如果 a 和 b 都是偶數,則 \m{\gcd(a,b)=2\cdot \gcd(a/2,b/2)}。
\stopitem\stopigBase

\startANSWER
\TODO{略。}
\stopANSWER

\startigBase[continue]\startitem
證明:如果 a 是奇數, b 是偶數,則 \m{\gcd(a,b)=\gcd(a,b/2)}。
\stopitem\stopigBase

\startANSWER
\TODO{略。}
\stopANSWER

\startigBase[continue]\startitem
證明:如果 a 和 b 都是奇數,則 \m{\gcd(a,b)=\gcd((a-b)/2,b)}。
\stopitem\stopigBase

\startANSWER
\TODO{略。}
\stopANSWER

\startigBase[continue]\startitem
設計一個有效的二進制 \m{\gcd} 算法,
輸入整數爲 a 和 b (\m{a\ge b}),
並且算法的運行時間爲 \m{O(\lg a)}。
假定每個減法運算、測試奇偶性運算以及折半運算都能在單位時間內執行。
\stopitem\stopigBase

\startANSWER
\TODO{略。}
\stopANSWER
\stopPROBLEM

%p31-2
\startPROBLEM
(Analysis of bit operations in Euclid’s algorithm)
\startigBase[a]\startitem
考慮用普通的“紙和筆”算法來實現長除法的運算:
用 a 除以 b,得到商 q 和餘數 r。
證明:這種算法需要執行 \m{O((1+\lg q)\lg b)} 次位操作。
\stopitem\stopigBase

\startANSWER
\TODO{略。}
\stopANSWER

\startigBase[continue]\startitem
定義 \m{\mu(a,b)=(1+\lg a)(1+\lg b)}。
證明:過程 \ALGO{EUCLID} 在把計算 \m{\gcd(a,b)} 的問題轉化爲
計算 \m{\gcd(b,a\mod b)} 的問題時,
所執行的位操作次數至多爲 \m{c(\mu(a,b)-\mu(b,a\mod b)},
其中 \m{c>0} 爲某一個足夠大的常數。
\stopitem\stopigBase

\startANSWER
\TODO{略。}
\stopANSWER

\startigBase[continue]\startitem
證明: \ALGO{EUCLID(a,b)} 通常需要執行 \m{O(\mu(a,b))} 次位操作;
當其輸入爲兩個 \m{\beta} 位數時,
需要執行的位操作次數爲 \m{O(\beta^2)}。
\stopitem\stopigBase

\startANSWER
\TODO{略。}
\stopANSWER
\stopPROBLEM

%p31-3
\startPROBLEM
(Three algorithms for Fibonacci numbers)
在已知 \m{n} 的情況下,
本題對計算第 \m{n} 個 Fibonacci 數 \m{F_n} 的三種算法的效率進行了比較。
假定兩個數的加法、減法和乘法的代價都是 \m{O(1)},與數的大小無關。
\startigBase[a]\startitem
證明:基於遞迴式(3.22)計算 \m{F_n} 的直接遞迴方法的運行時間爲 \m{n} 的冪。
(例如,\refsection{27.1} 的 \ALGO{FIB} 程序。)
\stopitem\stopigBase

\startANSWER
\TODO{略。}
\stopANSWER

\startigBase[continue]\startitem
試說明如何運用記憶法在 \m{O(n)} 時間內計算 \m{F_n}。
\stopitem\stopigBase

\startANSWER
\TODO{略。}
\stopANSWER

\startigBase[continue]\startitem
試說明如何僅用整數加法和乘法運算,
就可以在 \m{O(\lg n)} 的時間內計算 \m{F_n}。
(\hint 考慮如下矩陣及其冪:
\startformula
\left[\startmatrix
\NC 0 \NC 1 \NR
\NC 1 \NC 1 \NR
\stopmatrix\right]
\stopformula)
\stopitem\stopigBase

\startANSWER
\TODO{略。}
\stopANSWER

\startigBase[continue]\startitem
現在假設對兩個 \m{\beta} 位數相加需要 \m{\Theta(\beta)} 時間,
對兩個 \m{\beta} 位數相乘需要 \m{\Theta(\beta^2)} 時間。
如果這樣更合理地估計基本算數運算的代價,
這三種方法的運行時間又是多少?
\stopitem\stopigBase

\startANSWER
\TODO{略。}
\stopANSWER
\stopPROBLEM

%p31-4
\startPROBLEM
(Quadratic residues)
設 \m{p} 是一個奇素數。
如果關於未知量 \m{x} 的方程 \m{x^2=a(\mod p)} 有解,
則數 \m{a\in \integers_p^*} 就是一個{\EMP 二次餘數}。
\startigBase[a]\startitem
證明:對模 \m{p},恰有 \m{(p-1)/2} 個二次餘數。
\stopitem\stopigBase

\startANSWER
\TODO{略。}
\stopANSWER

\startigBase[continue]\startitem
如果 \m{p} 是素數,對 \m{a\in \integers_p^*},
定義 {\EMP  Legendre symbol} \m{\left(\frac{a}{p}\right)},
若 \m{a} 是對模 \m{p} 的二次餘數,則他等於 1;
否則他等於 -1。證明:如果 \m{a\in \integers_p^*},則:
\startformula
\left(\frac{a}{p}\right)\equiv a^{(p-1)/2} (\mod p)
\stopformula
給出一個有效的算法,
使其能確定一個給定的數 a 是否對模 p 的二次餘數。
分析所給算法的效率。
\stopitem\stopigBase

\startANSWER
\TODO{略。}
\stopANSWER

\startigBase[continue]\startitem
證明:如果 \m{p} 是形如 \m{4k+3} 的素數,
且 a 是 \m{\integers_p^*} 中的一個二次餘數,
則 \m{a^{k+1}\mod p} 是對模 \m{p} 的 \m{a} 的平方根。
找出一個以 \m{p} 爲模的二次餘數 \m{a} 的平方根需要多長時間?
\stopitem\stopigBase

\startANSWER
\TODO{略。}
\stopANSWER

\startigBase[continue]\startitem
試描述一個有效的隨機算法,
找出一個以任意素數 \m{p} 爲模的非二次餘數,
也就是指 \m{\integers_p^*} 中不是二次餘數的成員。
所給出的算法平均需要執行多少次算數運算?
\stopitem\stopigBase

\startANSWER
\TODO{略。}
\stopANSWER
\stopPROBLEM

\stopsubject%Problems
