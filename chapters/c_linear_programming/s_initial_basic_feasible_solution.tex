\startsection[
  title={The initial basic feasible solution},
]

%e29.5-1
\startEXERCISE
寫出詳細的僞碼實現 \ALGO{INITIALIZE-SIMPLEX} 的第 5 行和第 14 行。
\stopEXERCISE

\startANSWER
\TODO{略。}
\stopANSWER

%e29.5-2
\startEXERCISE
請說明當 \ALGO{SIMPLEX} 的主題循環部分被 \ALGO{INITIALIZE-SIMPLEX} 運行時,永遠不會返回“無界”。
\stopEXERCISE

\startANSWER
\TODO{略。}
\stopANSWER

%e29.5-3
\startEXERCISE
假設已知一個標準型的線性規劃 \m{L},
並且假設對 \m{L} 與 \m{L} 的對偶問題,
初始鬆弛型相應的基本解都是可行的。
請說明 \m{L} 的最優目標值是 0。
\stopEXERCISE

\startANSWER
\TODO{略。}
\stopANSWER

%e29.5-4
\startEXERCISE
假設在一個線性規劃中嗯我們允許嚴格的不等式。
請說明在這種情況下,線性規劃基本定理不再成立。
\stopEXERCISE

\startANSWER
\TODO{略。}
\stopANSWER

%e29.5-5
\startEXERCISE
用 \ALGO{SIMPLEX} 求解下面的線性規劃:

最大化: \m{x_1+3x_2}

滿足約束:
\startformula\startmathalignment[n=5,
align={right,right,right,right,middle}]
\NC  x_1 \NC - \NC  x_2 \NC \le \NC 8 \NR
\NC -x_1 \NC - \NC  x_2 \NC \le \NC -3 \NR
\NC -x_1 \NC + \NC 4x_2 \NC \le \NC 2 \NR
\NC     \NC x_1, \NC x_2 \NC \ge \NC 0 \NR
\stopmathalignment\stopformula
\stopEXERCISE

\startANSWER
\TODO{略。}
\stopANSWER

%e29.5-6
\startEXERCISE
用 \ALGO{SIMPLEX} 求解下面的線性規劃:

最大化: \m{x_1 - 2x_2}

滿足約束:
\startformula\startmathalignment[n=5,
align={right,right,right,right,middle}]
\NC   x_1 \NC + \NC 2x_2 \NC \le \NC 4 \NR
\NC -2x_1 \NC - \NC 6x_2 \NC \le \NC -12 \NR
\NC       \NC   \NC  x_2 \NC \le \NC 1 \NR
\NC     \NC x_1, \NC x_2 \NC \ge \NC 0 \NR
\stopmathalignment\stopformula
\stopEXERCISE

\startANSWER
\TODO{略。}
\stopANSWER

%e29.5-7
\startEXERCISE
用 \ALGO{SIMPLEX} 求解下面的線性規劃:

最大化: \m{x_1 + 3x_2}

滿足約束:
\startformula\startmathalignment[n=5,
align={right,right,right,right,middle}]
\NC  -x_1 \NC + \NC  x_2 \NC \le \NC -1 \NR
\NC  -x_1 \NC - \NC  x_2 \NC \le \NC -3 \NR
\NC  -x_1 \NC + \NC 4x_2 \NC \le \NC 2 \NR
\NC     \NC x_1, \NC x_2 \NC \ge \NC 0 \NR
\stopmathalignment\stopformula
\stopEXERCISE

\startANSWER
\TODO{略。}
\stopANSWER

%e29.5-8
\startEXERCISE
求解式(29.6)~(29.10)給出的線性規劃。
\stopEXERCISE

\startANSWER
\TODO{略。}
\stopANSWER

%e29.5-9
\startEXERCISE
考慮下面一個變量的線性規劃,我們稱爲 \m{P}:

最大化: \m{tx}

滿足約束:
\startformula\startmathalignment
\NC rx \NC \le s \NR
\NC x \NC \ge 0 \NR
\stopmathalignment\stopformula
其中 \m{r}、 \m{s} 和 \m{t} 是任意的實數。
設 \m{D} 是 \m{P} 的對偶。

敘述對 \m{r}、 \m{s} 和 \m{t} 的哪些值,可以做出如下斷言:
\startigBase[n]
\item P 和 D 都具有優先目標值的最優解。
\item P 是可行的,但 D 是不可行的。
\item D 是可行的,但 P 是不可行的。
\item P 和 D 都是不可行的。
\stopigBase
\stopEXERCISE

\startANSWER
\TODO{略。}
\stopANSWER

\stopsection
