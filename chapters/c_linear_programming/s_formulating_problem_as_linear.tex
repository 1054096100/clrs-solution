\startsection[
  title={Formulating problems as linear programs},
  reference={section:formulate_as_linear},
]

%e29.2-1
\startEXERCISE
請將單對最短路徑線性規劃從式(29.44)~(29.46)轉換稱標準型。
\stopEXERCISE

\startANSWER
\TODO{略。}
\stopANSWER

%e29.2-2
\startEXERCISE
請明確寫出求圖 24-2(a)中從節點 s 到節點 y 的最短路徑的線性規劃。
\stopEXERCISE

\startANSWER
\TODO{略。}
\stopANSWER

%e29.2-3
\startEXERCISE
在單源最短路徑問題中,我們希望從源點 s 到所有頂點 \m{v\in V} 的最短路徑權值。
給定圖 G,請寫出一個線性規劃,其解具有如下性質:
對每個頂點 \m{v\in V}, \m{d_v} 是從 s 到 v 的最短路徑權值。
\stopEXERCISE

\startANSWER
\TODO{略。}
\stopANSWER

%e29.2-4
\startEXERCISE
請明確寫出求圖 26-1(a)中最大流的線性規劃。
\stopEXERCISE

\startANSWER
\TODO{略。}
\stopANSWER

%e29.2-5
\startEXERCISE
請重寫最大流式(29.47)~(29.50)的線性規劃,
使得他只使用 \m{O(V+E)} 個約束。
\stopEXERCISE

\startANSWER
\TODO{略。}
\stopANSWER

%e29.2-6
\startEXERCISE
請寫出一個線性規劃,
給定一個二部圖 \m{G=(V,E)},
其中每條邊 \m{(u,v)\in E} 有一個非負的容量 \m{c(u,v)\ge 0} 和
一個費用 \m{a(u,v)}。
與多商品流問題一樣,
我們已知 \m{k} 種不同商品 \m{K_1,K_2,\ldots,K_k},
其中用三元組 \m{K_i=(s_i,t_i,d_i)} 來詳細說明商品 \m{i}。
與多商品流問題一樣,我們爲商品 \m{i} 定義流 \m{f_i},
在邊 \m{(u,v)} 上定義匯聚流 \m{f_{uv}}。
一個可行流滿足在每條邊 \m{(u,v)} 上匯聚流不超過邊 \m{(u,v)} 的容量。
一個流的費用是 \m{\sum a(u,v)f_{uv}},
目標是尋找具有最小費用的可行流。
請將這個問題表示爲一個線性規劃。
\stopEXERCISE

\startANSWER
\TODO{略。}
\stopANSWER

\stopsection
