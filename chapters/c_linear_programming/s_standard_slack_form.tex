\startsection[
  title={Standard and slack forms},
]

%e29.1-1
\startEXERCISE
如果將式(29.24)~(29.28)中的線性規劃表示成式(29.19)~(29.21)中的緊湊記號形式,
則 \m{n}、 \m{m}、 \m{A}、 \m{b} 和 \m{c} 分別是什麼?
附:

\startformula\startmathalignment[n=6,
  align={right,right,right,middle,left,right}]
\NC 2x_1 - \NC 3x_2 + \NC 3x_3 \NC     \NC    \NC \qquad (29.24) \NR
\NC  x_1 + \NC  x_2 - \NC  x_3 \NC \le \NC  7 \NC \qquad (29.25) \NR
\NC -x_1 - \NC  x_2 + \NC  x_3 \NC \le \NC -7 \NC \qquad (29.26) \NR
\NC  x_1 - \NC 2x_2 + \NC 2x_3 \NC \le \NC  4 \NC \qquad (29.27) \NR
\NC  x_1,  \NC  x_2,  \NC  x_3 \NC \ge \NC  0 \NC \qquad (29.28) \NR
\NC        \NC        \NC c^Tx \NC     \NC    \NC \qquad (29.19) \NR
\NC        \NC        \NC  Ax  \NC \le \NC  b \NC \qquad (29.20) \NR
\NC        \NC        \NC   x  \NC \ge \NC  0 \NC \qquad (29.21) \NR
\stopmathalignment\stopformula
\stopEXERCISE

\startANSWER
\startformula\startmathalignment
\NC n \NC = 3 \NR

\NC m \NC = 3 \NR

\NC A \NC = \left[\startmatrix
\NC 1 \NC 1 \NC -1 \NR
\NC -1 \NC -1 \NC 1 \NR
\NC 1 \NC -2 \NC 2 \NR
\stopmatrix\right] \NR

\NC b \NC = \left[\startmatrix
\NC 7 \NR
\NC -7 \NR
\NC 4 \NR
\stopmatrix\right] \NR

\NC c \NC = \left[\startmatrix
\NC 2 \NR
\NC -3 \NR
\NC 3 \NR
\stopmatrix\right] \NR
\stopmathalignment\stopformula
\stopANSWER

%e29.1-2
\startEXERCISE
請給出式(29.24)~(29.28)中線性規劃的三個可行解。
每個解的目標值是多少?
\stopEXERCISE

\startANSWER
\stopANSWER

\stopsection
