\startsubject[
  title={Problems},
]

\startPROBLEM
(用插入的方式建堆 Building a heap using insertion)
我們可以通過反復調用 \ALGO{MAX-HEAP-INSERT} 不斷向堆中插入元素來構建一個堆。
考慮 \ALGO{BUILD-MAX-HEAP} 如下實現方式:

\CLRSH{BUILD-MAX-HEAP'(A)}
\startCLRS
A.heap-size = 1
for i = 2 to A.length
	MAX-HEAP-INSERT(A, A[i])
\stopCLRS

\startigBase[a]
\startitem
當輸入數據相同時, \ALGO{BUILD-MAX-HEAP} 和 \ALGO{BUILD-MAX-HEAP'} 所生成的堆是否相同?
如果是,請證明;否則請舉出反例。

\startANSWER
不同。例如,輸入數據爲 \m{\langle 1, 2, 3, 4, 5, 6 \rangle} 時,兩個堆分別如下:

\startcombination[2*1]
{\externalfigure[output/p6_1_a-1]}{}
{\externalfigure[output/p6_1_a-2]}{}
\stopcombination

\stopANSWER
\stopitem
\startitem
證明:最壞情況下,調用 \ALGO{BUILD-MAX-HEAP'} 建立包含 \m{n} 個元素的堆,
其時間復雜度爲 \m{\Theta(n\lg{n})}。

\startANSWER
最壞情況下, \ALGO{MAX-HEAP-INSERT} 的運行時間爲 \m{\Theta(\lg{n})},會被調用 \m{n-1} 次。
最壞情況下, \ALGO{MAX-HEAP-INSERT} 會將所有元素都移到堆的根節點上, 即需要 \m{\lg{k}} 次交換,
無論 \m{k} 的值是多少。
如果輸入數列已經是排好序的,則就是最壞情況。時間復雜度爲(參見\refexercise{lg_n_fac}):
\startformula
\sum_{i=2}^{n}\lg{i} = \lg(n!) = \Theta(n\lg{n})
\stopformula
\stopANSWER
\stopitem
\stopigBase
\stopPROBLEM

\startPROBLEM[problem:6-2]
(堆 d 叉堆的分析 Analysis of d -ary heaps)
d 叉堆與二叉堆類似,只是每個非葉子節點有 \m{d} 個子節點,而不是 2 個子節點。
\startigBase[a]
\startitem
如何在一個數列中表示 \m{d} 叉堆?

\startANSWER
需要修改 \ALGO{LEFT}、 \ALGO{RIGHT} 和 \ALGO{PARENT} 的定義。
第 \m{i} 個元素的第 \m{k} 個子節點的索引爲 \m{di + k - 1},
而父節點的索引爲 \m{\lfloor i/d \rfloor}。(索引從 1 開始)
\stopANSWER
\stopitem

\startitem
包含 \m{n} 個元素的 \m{d} 叉堆的高度是多少?用 \m{n} 和 \m{d} 表示。

\startANSWER
\m{\log_d{n}}。
\stopANSWER
\stopitem

\startitem
請給出 \ALGO{EXTRACT-MAX} 在 \m{d} 叉最大堆上的一個有效實現,
並用 \m{d} 和 \m{n} 表示其時間復雜度。

\startANSWER
\ALGO{EXTRACT-MAX} 的時間復雜度爲 \m{O(d\log_d{n})}。
\stopANSWER
\stopitem

\startitem
請給出 \ALGO{INSERT} 在 \m{d} 叉最大堆上的一個有效實現,
並用 \m{d} 和 \m{n} 表示其時間復雜度。

\startANSWER
\ALGO{INSERT} 的時間復雜度爲 \m{O(\log_d{n})}。
\stopANSWER
\stopitem

\startitem
請給出 \ALGO{INCREASE-KEY(A, i, k)} 在 \m{d} 叉最大堆上的一個有效實現。
當 \m{k<A[i]} 時,他會觸發一個錯誤,否則執行 \m{A[i]=k},並更新相應的堆。
並用 \m{d} 和 \m{n} 表示其時間復雜度。

\startANSWER
\ALGO{INCREASE-KEY(A, i, k)} 的時間復雜度爲 \m{O(\log_d{n})}。
\stopANSWER
\stopitem
\stopigBase
\stopPROBLEM

%p6-3
\startPROBLEM
(Young tableaus)
在一個 \m{m\times n} 的{\EMP 楊氏矩陣(Young tableau)}中,
每一行的數據都是從左到右有序的,每一列數據都是從上到下有序的。
其中也會有一些值爲 \m{\infty} 的數據項,用來表示不存在的元素。
因此,楊氏矩陣可以用來存儲 \m{r\le mn} 個有限的數。

\simpleurl{https://en.wikipedia.org/wiki/Young_tableau}

\startigBase[a]\startitem
畫出一個包含元素爲 \m{\{9, 16, 3, 2, 4, 8, 5, 14, 12\}} 的 \m{4\times 4} 楊氏矩陣。
\stopitem\stopigBase

\startANSWER
\startformula\startpmatrix%[location=low]
\NC      2 \NC      3 \NC     12 \NC     14 \NR
\NC      4 \NC      8 \NC     16 \NC \infty \NR
\NC      5 \NC      9 \NC \infty \NC \infty \NR
\NC \infty \NC \infty \NC \infty \NC \infty \NR
\stoppmatrix\stopformula
\stopANSWER

\startigBase[continue]\startitem
對於一個 \m{m\times n} 的楊氏矩陣 \m{Y} 而言,請證明:
如果 \m{Y[1,1]=\infty},則 \m{Y} 爲空;
如果 \m{Y[m,n]<\infty},則 \m{Y} 爲滿(即包含 \m{mn} 個元素)。
\stopitem\stopigBase

\startANSWER
如果 \m{Y[1,1]=\infty},則第一行元素都要大於等於左上角元素,即第一行元素均爲 \m{\infty},
而對於任一列而言,所有元素都要大於此列中第一行的元素,即整個矩陣所有元素均爲 \m{\infty}。
類似,如果 \m{Y[m,n]<\infty},則其他元素均小於等於由下角元素,即所有元素都不是 \m{\infty},即矩陣爲滿。
\stopANSWER

\startigBase[continue]\startitem
請給出一個在 \m{m\times n} 楊氏矩陣上時間復雜度爲 \m{O(m+n)} 的 \ALGO{EXTRACT-MIN} 的算法實現。
可以考慮使用一個遞迴過程,將規模爲 \m{m\times n} 的問題分解爲規模爲 \m{(m-1)\times n} 或
者 \m{m\times(n-1)} 的子問題(\hint 考 \ALGO{MAX-HEAPIFY})。
這裏,定義 \m{T(p)} 用來表示 \m{EXTRACT-MIN} 在任一 \m{m\times n} 的楊氏矩陣上的時間復雜度,
其中 \m{p=m+n}。給出並求解 \m{T(p)} 的遞迴表達式,其結果爲 \m{O(m+n)}。
\stopitem\stopigBase

\startANSWER
\m{A[1,1]} 是最小的,就是返回值,將其替換爲 \m{\infty},這會破壞楊氏矩陣的性質,
用類似 \ALGO{MAX-HEAPIFY} 的過程來維持其性質。
將 \m{A[i,j]} 與其鄰居比較,並將鄰居中最小的與其交換位置。
這樣會使得 \m{A[i,j]} 遵循楊氏矩陣的性質,然後將變成 \m{A[i,j+1]} 或 \m{A[i+1,j]} 的問題。
當 \m{A[i,j]} 比所有鄰居都小時,就終止程序。則:
\startformula
T(p) = T(p - 1) + O(1) = T(p-2) + O(1) + O(1) = \ldots = O(p)
\stopformula
\stopANSWER

\startigBase[continue]\startitem
試說明如何在 \m{O(m+n)} 的時間內,將一個新元素插入到一個未滿的 \m{m\times n} 的楊氏矩陣中。
\stopitem\stopigBase

\startANSWER
與上一題類似,只是改爲從右下角開始,向左向上移動。時間不變。
\stopANSWER

\startigBase[continue]\startitem
在不用其他配序算法的情況下,試說明如何利用一個 \m{n\times n} 的楊氏矩陣在 \m{O(n^3)} 時間內
將 \m{n^2} 個數進行排序。
\stopitem\stopigBase

\startANSWER
矩陣開始爲空,最終爲滿,插入元素 \m{n^2} 個。
每次插入操作都需時間 \m{O(n+n)=O(n)}。復雜度爲 \m{n^2O(n)=O(n^3)}。
然後在矩陣中一個一個的取元素,放入原數列中,時間復雜度一樣。
總共時間爲 \m{O(n^3)}。

如果允許矩陣中只有左上角一部分元素具有楊氏矩陣的性質,則可以原地排序。
\stopANSWER

\startigBase[continue]\startitem
設計一個時間復雜度爲 \m{O(m+n)} 的算法,
用來判斷一個給定的數是否存儲在 \m{m\times n} 的楊氏矩陣中。
\stopitem\stopigBase

\startANSWER
從左下角開始,比較 \m{current} 和 \m{key},
如果 \m{current > key},則上移,如果 \m{current < key},則右移。
如果 \m{current = key},則返回成功,否則直到終止。
\stopANSWER

\stopPROBLEM

\stopsubject
