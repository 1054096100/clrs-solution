\startsection[
  title={Building heap},
]

\startEXERCISE
參照圖 6-3 的方法,
說明 \ALGO{BUILD-MAX-HEAP} 在數列 \m{A = \langle 5, 3, 17, 10, 84, 19, 6, 22, 9 \rangle} 上的操作過程。
\stopEXERCISE

\startANSWER
\startcombination[3*2]
{\externalfigure[output/e6_3_1-1]}{}
{\externalfigure[output/e6_3_1-2]}{}
{\externalfigure[output/e6_3_1-3]}{}
{\externalfigure[output/e6_3_1-4]}{}
{\externalfigure[output/e6_3_1-5]}{}
{}{}
\stopcombination
\stopANSWER

\startEXERCISE
對於 \ALGO{BUILD-MAX-HEAP} 中第 2 行的循環控制變量 \m{i} 而言,
爲什麼要求他從 \m{\left\lfloor A.length/2 \right\rfloor} 到 \m{1} 遞減,
而不是從 \m{1} 到 \m{\left\lfloor A.length/2 \right\rfloor} 遞增?
附:

\CLRSH{BUILD-MAX-HEAP(A)}
\startCLRS
A.heap-size = A.length
for i =  ⌊A.length/2⌋ downto 1
	MAX-HEAPIFY(A, i)
\stopCLRS
\stopEXERCISE

\startANSWER
如果是遞增的話,就不能調用 \ALGO{MAX-HEAPIFY} 了,因爲無法保證子樹是最大堆。
即如果從 \m{1} 開始,無法保證 \m{A[2]} 和 \m{A[3]} 是最大堆的根節點。
\stopANSWER

\startEXERCISE
證明:對於任一含有 \m{n} 個元素的堆,最多有 \m{\left\lceil n/2^{h+1} \right\rceil} 個高度爲 \m{h} 的節點。
\stopEXERCISE

\startANSWER
首先,堆中葉子節點的個數爲 \m{\left\lceil n/2 \right\rceil} \refexercise{heap_leave}。
下面歸納證明 \m{h}:

{\EMP 初始化:} \m{h = 0} 時,葉子節點數目 \m{\left\lceil n/2 \right\rceil = \left\lceil n/2^{0+1} \right\rceil};

{\EMP 保持:}假設對於 \m{h - 1} 結論成立,移除所有葉子節點得到的新堆含有
 \m{n-\left\lceil n/2 \right\rceil = \left\lfloor n/2 \right\rfloor} 個元素,
原堆中高度爲 \m{h} 的節點在新樹中高度爲 \m{h-1};
令新樹中高度爲 \m{h-1} 的節點個數爲 \m{T},則由假設可知:
\startformula
T = \left\lceil \left\lfloor n/2 \right\rfloor / 2^{h-1+1} \right\rceil
     < \left\lceil (n/2)/2^h \right\rceil
     = \left\lceil \frac{n}{2^{h+1}} \right\rceil
\stopformula
即對於 \m{h} 結論依舊成立。
\stopANSWER

\stopsection
