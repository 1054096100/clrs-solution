\startcomponent c_devide_and_conquer

\chapter{Divide-and-Conquer}

\section{The maximum-subarray problem}

\startEXERCISE
當 \m{A} 中所有元素均爲負數時, \ALGO{FIND-MAXIMUM-SUBARRAY} 返回什麼?
\stopEXERCISE
\startANSWER
返回一個單元素數組,即最大負整數。
\stopANSWER

\startEXERCISE
編寫暴力求解方法的僞代碼來解決最大子數列問題,其運行時間應爲 \m{\Theta(n^2)}。
\stopEXERCISE
\startANSWER
\CLRSH{FIND-MAX-SUBARRAY(A, low, high)}
\startCLRS
left = 0
right = 0
sum = -∞
for i = low to high
	current-sum = 0
	for j = i to high
	current-sum += A[j]
	if sum < current-sum
		sum = current-sum
	left = i
	right = j
return (left, right, sum)
\stopCLRS
\stopANSWER

\startEXERCISE
在你的計算機上編碼實現用於解決最大子數列問題的暴力算法和遞迴算法。
請指出兩種算法性能交叉點處的問題規模 \m{n_0} ,即在此處遞迴算法將擊敗暴力算法?
然後,修改遞迴算法,當問題規模小於 \m{n_0} 時改用暴力算法。
修改後,性能交叉點有變化嗎?
\stopEXERCISE
\startANSWER
修改前 \m{n_0 = 30},修改算法後 \m{n_0 = 1}。
\stopANSWER

\startEXERCISE
假設修改最大子數列的定義,允許結果爲空,其和爲 \m{0},
需要如何修改現有算法才能使得結果可以爲空?
\stopEXERCISE
\startANSWER
需修改 \m{high = low + 1} 的分支,即當數組只有一個元素的情況;
若是這個元素小於 \m{0},則結果爲空,否則結果爲此元素。
\stopANSWER

\startEXERCISE
使用如下思想爲最大子數列問題設計一個非遞迴的、線性時間的算法。
從數列左邊界開始,從左至右處理,記錄到目前爲止已經處理過的最大子數列。
若已知 \m{A[1..j]}的最大子數列,基於如下性質將解擴展爲 \m{A[1..j+1]} 的最大子數列:
\m{A[1..j+1]} 的最大子數列要麼是 \m{A[1..j]} 的最大子數列,
要麼是某個子數列 \m{A[i..j+1]},其中 \m{1\leq i\leq j+1}。
在已知 \m{A[1..j]} 的最大子數列的情況下,可以在線性時間內找出形如 \m{A[i..j+1]} 的最大子數列。
\stopEXERCISE
\startANSWER
參考\SIMPLEURL{http://en.wikipedia.org/wiki/Maximum_subarray_problem}。

若允許空數列:
\startCLRS
MaxSoFar = 0
MaxEndingHere = 0
for i = 1 to N
	MaxEndingHere = max(0, MaxEndingHere + A[i])
	MaxSoFar = max(MaxSoFar, MaxEndingHere)
\stopCLRS

若不允許空數列:
\startCLRS
MaxSoFar = A[1]
MaxEndingHere = A[1]
for i = 2 to N
	MaxEndingHere = max(A[i], MaxEndingHere + A[i])
	MaxSoFar = max(MaxSoFar, MaxEndingHere)
\stopCLRS
\stopANSWER

\section{Strassen’s algorithm for matrix multiplication}

\startEXERCISE
用 Strassen 算法計算矩陣積:
\startformula
\startpmatrix%[location=low]
\NC1\NC3\NR
\NC7\NC5\NR
\stoppmatrix
\startpmatrix%[location=low]
\NC6\NC8\NR
\NC4\NC2\NR
\stoppmatrix
\stopformula
\stopEXERCISE

\startANSWER
十個矩陣分別爲:
\startformula\startmathalignment[n=10]
\NC S_1 \NC= 6 \qquad
\NC S_2 \NC=  4 \qquad
\NC S_3 \NC= 12 \qquad
\NC S_4 \NC= -2 \qquad
\NC S_5 \NC= 5 \NR
\NC S_6 \NC= 8 \qquad
\NC S_7 \NC= -2 \qquad
\NC S_8 \NC=  6 \qquad
\NC S_9 \NC= -6 \qquad
\NC S_{10} \NC= 14 \NR
\stopmathalignment\stopformula
七個矩陣積分別爲:
\startformula\startmathalignment[n=8]
\NC P_1 \NC= 1 \cdot 6 = 6 \qquad
\NC P_2 \NC= 4 \cdot 2 = 8 \qquad
\NC P_3 \NC= 6 \cdot 12 = 72 \qquad
\NC P_4 \NC= (-2) \cdot 5 = -10 \NR
\NC P_5 \NC= 6 \cdot 8 = 48 \qquad
\NC P_6 \NC= (-2) \cdot 6 = -12 \qquad
\NC P_7 \NC= (-6) \cdot 14 = -84 \qquad
\NC \NC \NR
\stopmathalignment\stopformula
結果爲:
\startformula\startmathalignment[n=4]
\NC C_{11} \NC = 48 + (-10) - 8 + (-12) = 18 \qquad
\NC C_{12} \NC = 6 + 8 = 14 \NR
\NC C_{21} \NC = 72 + (-10) = 62 \qquad
\NC C_{22} \NC = 48 + 6 - 72 - (-84) = 66 \NR
\stopmathalignment\stopformula
即:
\startformula
\startpmatrix
\NC18\NC44\NR
\NC62\NC66\NR
\stoppmatrix
\stopformula
\stopANSWER

\startEXERCISE
爲 Strassen 算法編寫僞碼。
\stopEXERCISE

\startANSWER
\startCLRS
TODO
\stopCLRS
\stopANSWER

\startEXERCISE
修改 Strassen 算法,使之能適應矩陣規模 \m{n} 不是 \m{2} 的冪的情況?
證明:算法的運行時間爲 \m{\Theta(n^{\lg 7})}。
\stopEXERCISE

\startANSWER
我們可以通過添加值爲 \m{0} 的元素來擴展矩陣,使得矩陣規模 \m{n} 是 \m{2} 的冪。
運行時間顯然爲 \m{\Theta(n^{\lg 7})}。
\stopANSWER

\startEXERCISE
如果可以用 \m{k} 次乘法運算(假定乘法的交換律不成立)完成兩個 \m{3 \times 3} 矩陣相乘,
那麼就可以在 \m{o(n^{\lg 7})} 時間內完成 \m{n \times n} 矩陣相乘,
滿足這一條件的 \m{k} 最大是多少?
此算法運行時間如何?
\stopEXERCISE

\startANSWER
由分治法可得遞迴算法運行時間爲:
\startformula
T(n) = k T(n/3) + \Theta(n)
\stopformula
如果 \m{k} 足夠大,則求解得 \m{T(n) = \Theta(n^{\log_3^k})}。則:
\startformula\startmathalignment[n=1,align=center]
\NC n^{\log_3 k} < n^{\lg7} \NR
\NC \Downarrow \NR
\NC \log_3k < \lg7 \NR
\NC \Downarrow \NR
\NC k < 3^{\lg7} \approx 21.84986\NR
\stopmathalignment\stopformula
\stopANSWER

\startEXERCISE
V.Pan 發現一種方法可以用 \m{132464} 次乘法操作完成 \m{68\times 68} 的矩陣乘法;
還發現一種方法可以用 \m{143640} 次乘法操作完成 \m{70\times 70} 的矩陣乘法;
還發現一種方法可以用 \m{155424} 次乘法操作完成 \m{72\times 72} 的矩陣乘法。
若用於矩陣相乘的分治算法,上述哪種方法會得到最佳的漸進運行時間?
與 Strassen 算法相比,性能如何?
\stopEXERCISE
\startANSWER
\startformula\startalign
\log_{68} 132464 \approx 2.795128 \NR
\log_{70} 143640 \approx 2.795122 \NR
\log_{72} 155424 \approx 2.795147 \NR
\stopalign\stopformula
顯然 \m{70 \times 70} 的方法漸進運行時間最佳。比 Strassen 算法要好。
\stopANSWER

\startEXERCISE
用 Strassen 算法作爲子例程求解 \m{kn\times n} 和 \m{n \times kn} 兩個矩陣相乘,
最快需要花費多少時間?
如果兩個輸入矩陣規模互換,又需要多長時間?
\stopEXERCISE

\startANSWER
\m{(kn \times n)(n \times kn)} 生成 \m{kn \times kn} 矩陣,需要 \m{k^2} 次 \m{n\times n} 矩陣相乘。

\m{(n \times kn)(kn \times n)} 生成 \m{n \times n} 矩陣,需要 \m{k} 次乘法和 \m{k-1} 次加法。
\stopANSWER

\startEXERCISE
設計算法,僅使用三次實數乘法即可完成復數 \m{a + bi} 和 \m{c + di} 相乘。
算法輸入爲 \m{a}、 \m{b}、 \m{c}、 \m{d},分別生成實部 \m{ac - bd} 和虛部 \m{ad + bc}。
\stopEXERCISE

\startANSWER
三次乘法分別爲:
\startformula\startalign
\NC A \NC = (a+b)(c+d) = ac + ad + bc + bd \NR
\NC B \NC = ac \NR
\NC C \NC = bd \NR
\stopalign\stopformula
結果爲:
\startformula
(B - C) + (A - B - C)i
\stopformula
\stopANSWER

\section{The substitution method for solving recurrences}

\startEXERCISE
證明 \m{T(n)  = T(n - 1) + n} 的解爲 \m{O(n^2)}。
\stopEXERCISE
\startANSWER
猜測 \m{T(n) \leq cn^2},則:
\startformula\startalign
\NC T(n) \NC \leq (n - 1)^2 + n \NR
\NC      \NC = n^2 - 2n + 1 + n \NR
\NC      \NC = n^2 - n + 1 \NR
\NC      \NC \leq cn^2 \NR
\stopalign\stopformula
其中 \m{c\geq 1}, \m{n\geq 1}。
\stopANSWER

\startEXERCISE
證明 \m{T(n) = T(\lceil n/2 \rceil) + 1} 的解爲 \m{O(\lg n)}。
\stopEXERCISE
\startANSWER
猜測 \m{T(n) \leq c\lg(n - 2)},則:
\startformula\startalign
\NC T(n) \NC \le c\lg(\lceil n/2 \rceil - 2) + 1 \NR
\NC      \NC \le c\lg(n/2 + 1 - 2) + 1 \NR
\NC      \NC \le c\lg((n - 2)/2) + 1 \NR
\NC      \NC \le c\lg(n - 2) - c\lg2 + 1 \NR
\NC      \NC \le c\lg(n - 2) \NR
\stopalign\stopformula
\stopANSWER

\startEXERCISE
我們已經知道 \m{T(n) = 2T(\lfloor n/2 \rfloor) + n} 的解爲 \m{O(n\lg n)}。
證明 \m{\Omega(n\lg n)} 也是他的解。
從而可得結論:解爲 \m{\Theta(n\lg n)}。
\stopEXERCISE
\startANSWER
猜測 \m{T(n) \le cn\lg n},則:
\startformula\startalign
\NC T(n) \NC \le 2c\lfloor n/2 \rfloor\lg{\lfloor n/2 \rfloor} + n \NR
\NC      \NC \le cn\lg(n/2) + n \NR
\NC      \NC \le cn\lg{n} - cn\lg{2} + n \NR
\NC      \NC \le cn\lg{n} + (1 - c)n \qquad \text{for } c \ge 1 \NR
\NC      \NC \le cn\lg{n} \NR
\stopalign\stopformula
猜測 \m{T(n) \ge c(n+2)\lg(n+2)},則:
\startformula\startalign
\NC T(n) \NC \ge 2c(\lfloor n/2 \rfloor + 2)(\lg(\lfloor n/2 \rfloor + 2) + n \NR
\NC      \NC \ge 2c(n/2 - 1 + 2)(\lg(n/2 - 1 + 2) + n \NR
\NC      \NC \ge 2c\frac{n+2}{2}\lg\frac{n+2}{2} + n \NR
\NC      \NC \ge c(n+2)\lg(n+2) - c(n+2)\lg2 + n \NR
\NC      \NC \ge c(n+2)\lg(n+2) + (1 - c)n - 2c \qquad \text{for } n \ge 2c/(1-c), 0 < c < 1 \NR
\NC      \NC \ge c(n+2)\lg(n+2) \NR
\stopalign\stopformula
\stopANSWER

\startEXERCISE
證明:通過做出不同的歸納假設,我們不必調整歸納證明中的邊界條件,
即可克服遞迴式(4.19)中邊界條件 \m{T(1)=1} 所帶來的困難。
\stopEXERCISE
\startANSWER
猜測 \m{T(n) \le n\lg{n} + n},則:
\startformula\startalign
\NC T(n) \NC \le 2(c\lfloor n/2 \rfloor\lg{\lfloor n/2 \rfloor} + \lfloor n/2 \rfloor) + n \NR
\NC      \NC \le 2c(n/2)\lg(n/2) + 2(n/2) + n \NR
\NC      \NC \le cn\lg(n/2) + 2n \NR
\NC      \NC \le cn\lg(n/2) + 2n \NR
\NC      \NC \le cn\lg{n} - cn\lg{2} + 2n \NR
\NC      \NC \le cn\lg{n} + (2 - c)n \qquad (c \ge 1)\NR
\NC      \NC \le cn\lg{n} + n \NR
\stopalign\stopformula
這樣邊界條件爲:
\startformula
T(1) = 1 \le cn\lg{n} + n = 0 + 1 = 1
\stopformula
\stopANSWER

\startEXERCISE
證明:歸並排序的“嚴格”遞迴式(4.3)的解爲 \m{\Theta(n\lg n)}。
\stopEXERCISE
\startANSWER
遞迴式爲:
\startformula
T(n) = T(\lfloor n/2 \rfloor) + T(\lceil n/2 \rceil) + \Theta(n)
\stopformula

猜測 \m{T(n) \le c(n - 2)\lg(n -2)},則:
\startformula\startalign
\NC T(n) \NC \le c(\lfloor n/2 \rfloor - 2)\lg(\lfloor n/2 \rfloor - 2) + c(\lceil n/2 \rceil -2 )\lg(\lceil n/2 \rceil - 2) + dn \NR
\NC      \NC \le c(n/2 - 2)\lg(n/2 - 2) + c(n/2 + 1 -2 )\lg(n/2 + 1 - 2) + dn \NR
\NC      \NC \le c(n/2 - 1)\lg(n/2 - 1) + c(n/2 - 1 )\lg(n/2 - 1) + dn \NR
\NC      \NC \le c\frac{n-2}{2}\lg\frac{n-2}{2} + c\frac{n-2}{2}\lg\frac{n-2}{2} + dn \NR
\NC      \NC \le c(n-2)\lg\frac{n-2}{2} + dn \NR
\NC      \NC \le c(n-2)\lg(n-2) - c(n-2) + dn \NR
\NC      \NC \le c(n-2)\lg(n-2) + (d - c)n + 2c \qquad (c > d, n > 2c)\NR
\NC      \NC \le c(n-2)\lg(n-2) \NR
\stopalign\stopformula
這就是 \m{\Theta(n\lg n)}。 \m{\Omega(n\lg n)} 類似。
\stopANSWER

\startEXERCISE
證明: \m{T(n) = 2T(\lfloor n/2 \rfloor + 17) + n} 的解爲 \m{O(n\lg n)}。
\stopEXERCISE
\startANSWER
猜測 \m{T(n) \le c(n-a)\lg(n-a)},則:
\startformula\startmathalignment[n=3]
\NC T(n) \NC \le 2c(\lfloor n/2 \rfloor + 17 - a)\lg(\lfloor n/2 \rfloor + 17 - a) + n \NC \NR
\NC      \NC \le 2c(n/2 + 1 + 17 - a)\lg(n/2 + 1 + 17 - a) + n \NC \NR
\NC      \NC \le c(n + 36 - 2a)\lg\frac{n + 36 - 2a}{2} + n \NC \NR
\NC      \NC \le c(n + 36 - 2a)\lg(n + 36 - 2a) - c(n + 36 - 2a) + n \qquad\NC (c > 1, n > n_0 = f(a))\NR
\NC      \NC \le c(n + 36 - 2a)\lg(n + 36 - 2a)                      \NC (a \ge 36) \NR
\NC      \NC \le c(n - a)\lg(n - a) \NC \NR
\stopmathalignment\stopformula
\stopANSWER

\startEXERCISE
使用 4.5 節中的方法,可以證明 \m{T(n) = 4T(n/3) + n} 的解爲 \m{T(n) = \Theta(n^{\log_3^4})}。
說明基於假設 \m{T(n) \le cn^{\log_3^4}} 的代入法不能證明這一結論。
然後說明如何通過減去一個低階項完成代入法證明。
\stopEXERCISE
\startANSWER
猜測 \m{T(n) \le cn^{\log_3{4}}},則:
\startformula\startmathalignment
\NC T(n) \NC \le 4c(n/3)^{\log_3{4}} + n \NR
\NC      \NC \le cn^{\log_3{4}} + n \NR
\stopmathalignment\stopformula
與猜測不一致,證明失敗。

猜測 \m{T(n) \le cn^{\log_3{4}} - n},則:
\startformula\startmathalignment
\NC T(n) \NC \le 4\Big(c(n/3)^{\log_3{4}} - n\Big) + n \NR
\NC      \NC \le cn^{\log_3{4}} - 4n + n \NR
\NC      \NC \le cn^{\log_3{4}} - 3n \NR
\NC      \NC \le cn^{\log_3{4}} - n \NR
\stopmathalignment\stopformula
\stopANSWER

\startEXERCISE
使用 4.5 節中的方法,可以證明 \m{T(n) = 4T(n/2) + n} 的解爲 \m{T(n) = \Theta(n^2)}。
說明基於假設 \m{T(n) \le cn^2} 的代入法不能證明這一結論。
然後說明如何通過減去一個低階項完成代入法證明。
\stopEXERCISE
\startANSWER
猜測 \m{T(n) \le cn^2},則:
\startformula\startmathalignment
\NC T(n) \NC \le 4c(n/2)^2 + n \NR
\NC      \NC \le cn^2 + n \NR
\NC      \NC \le (c+1)n^2 \NR
\stopmathalignment\stopformula
與猜測不一致,證明失敗。

猜測 \m{T(n) \le cn^2 - n},則:
\startformula\startmathalignment
\NC T(n) \NC \le 4\Big(c(n/2)^2 - n/2\Big) + n \NR
\NC      \NC \le cn^2 - 2n + n \NR
\NC      \NC \le cn^2 - n \NR
\stopmathalignment\stopformula
\stopANSWER

\startEXERCISE
利用改變變量的方法求解遞迴式 \m{T(n) = 3 T(\sqrt{n}) + \log{n}}。
你的解應該是漸進緊確的。不必擔心數值是否是整數。
\stopEXERCISE
\startANSWER
假設 \m{m = \lg n}, \m{S(m) = T(2^m)},則:
\startformula\startmathalignment
\NC T(n)   \NC = 3T(\sqrt{n}) + \lg{n} \NR
\NC T(2^m) \NC = 3T(2^{m/2}) + m \NR
\NC S(m)   \NC = 3S(m/2) + m \NR
\stopmathalignment\stopformula

猜測 \m{S(m)\le cm^{\lg 3} + dm},則:
\startformula\startmathalignment[n=3]
\NC S(m) \NC \le 3\Big(c(m/2)^{\lg{3}} + d(m/2)\Big) + m \NC \NR
\NC      \NC \le cm^{\lg{3}} + (\frac{3}{2}d + 1)m       \NC (d \le -2) \NR
\NC      \NC \le cm^{\lg{3}} + dm \NC \NR
\stopmathalignment\stopformula

猜測 \m{S(m)\ge cm^{\lg 3} + dm},則:
\startformula\startmathalignment[n=3]
\NC S(m) \NC \ge 3\Big(c(m/2)^{\lg{3}} + d(m/2)\Big) + m \NC \NR
\NC      \NC \ge cm^{\lg{3}} + (\frac{3}{2}d + 1)m       \NC (d \ge -2) \NR
\NC      \NC \ge cm^{\lg{3}} + dm \NC \NR
\stopmathalignment\stopformula

因此:
\startformula\startmathalignment
\NC S(m) \NC = \Theta(m^{\lg{3}}) \NR
\NC T(n) \NC = \Theta(\lg^{\lg{3}}{n}) \NR
\stopmathalignment\stopformula
\stopANSWER

\section{The recursion-tree method for solving recurrences}

\startEXERCISE
對遞迴式 \m{T(n) = 3T(\lfloor n/2 \rfloor) + n},
利用遞迴樹確定一個漸進上界,用代入法進行驗證。
\stopEXERCISE
\startANSWER
忽略向下取整,結果沒有區別。

樹深 \m{\lg n},有 \m{\Theta(3^{\lg 2}) = \Theta(2^{\lg n})} 片葉子,因此:
\startformula\startmathalignment
\NC T(n) \NC = \sum_{i=0}^{\lg{n}-1}\Big(\frac{3}{2}\Big)^i n + \Theta(n^{\lg3}) \NR
\NC      \NC = n\frac{(3/2)^{\lg{n}} - 1}{(3/2) - 1} + \Theta(n^{\lg3}) \NR
\NC      \NC = n\Theta(n^{\lg3 - 1}) + \Theta(n^{\lg3}) \NR
\NC      \NC = \Theta(n^{\lg3}) \NR
\stopmathalignment\stopformula

套用代入法,猜測 \m{T(n) \le cn^{\lg3} + 2n} (忽略向下取整):
\startformula\startmathalignment
\NC T(n) \NC \le 3c(n/2)^{\lg3} + 2n/2 + n \NR
\NC      \NC \le cn^{\lg3} + 2n \NR
\NC      \NC = \Theta(n^{\lg3}) \NR
\stopmathalignment\stopformula

	\startMPcode
		input reccursion_tree
		string A[];
		string B[];
		A[3] := "$n$";		B[3] := "$n$";
		A[2] := "$n/2$";	B[2] := "$3n/2$";
		A[1] := "$n/4$";	B[1] := "$9n/4$";
		A[0] := "$T(1)$";	B[0] := "$(3/2)^i n$";
		rectree(A, B)(3, 3, 30, 30, 6);
	\stopMPcode
\stopANSWER

\startEXERCISE
對遞迴式 \m{T(n) = T(n/2) + n^2},
利用遞迴樹確定一個漸進上界,用代入法進行驗證。
\stopEXERCISE
\startANSWER
樹的每一層爲 \m{n^2/4^i},有 \m{\lg n} 層和 \m{1} 個葉子。因此:
\startformula\startmathalignment
\NC T(n) \NC = \sum_{i=0}^{\lg{n}-1}\Big(\frac{1}{4}\Big)^i n^2 + 1 \NR
\NC      \NC < n^2 \sum_{i=0}^{\infty}\Big(\frac{1}{4}\Big)^i + 1 \NR
\NC      \NC = n^2 \frac{1}{1-1/4} + 1 \NR
\NC      \NC = \Theta(n^2) \NR
\stopmathalignment\stopformula

猜測 \m{T(n) \le cn^2},有:
\startformula\startmathalignment
\NC T(n) \NC \le c(n/2)^2 + n^2 \NR
\NC      \NC \le cn^2/4 + n^2 \NR
\NC      \NC \le (c/4 + 1)n^2 \qquad (c > 4/3) \NR
\NC      \NC \le cn^2 \NR
\stopmathalignment\stopformula

	\startMPcode
		input reccursion_tree
		string A[];
		string B[];
		A[3] := "$n^2$";	B[3] := "$n^2$";
		A[2] := "$n^2/4$";	B[2] := "$n^2/4$";
		A[1] := "$n^2/16$";	B[1] := "$n^2/16$";
		A[0] := "$T(1)$";	B[0] := "$n^2/4^i$";
		rectree(A, B)(1, 3, 30, 30, 6);
	\stopMPcode
\stopANSWER

\startEXERCISE
對遞迴式 \m{T(n) = 4T(n/2+2) + n},
利用遞迴樹確定一個漸進上界,用代入法進行驗證。
\stopEXERCISE
\startANSWER
簡化後,樹高 \m{\lg n},每層和爲 \m{2^i n + 2^{1-i}},有 \m{4^{\lg n} = n^2} 個葉子。則:
\startformula\startmathalignment
\NC T(n) \NC = \sum_{i=0}^{\lg{n}-1}\Big(2^i n + 2^{1-i}) + \Theta(n^2) \NR
\NC      \NC = \sum_{i=0}^{\lg{n}-1}2^i n + \sum_{i=0}^{\lg{n}-1}2^{1-i} + \Theta(n^2) \NR
\NC      \NC = \frac{2^{\lg{n}} - 1}{2 - 1} + 2\sum_{i=0}^{\lg{n}-1}\Big(\frac{1}{2}\Big)^i + \Theta(n^2) \NR
\NC      \NC \le n - 1 + 2\sum_{i=0}^{\infty}\Big(\frac{1}{2}\Big)^i + \Theta(n^2) \NR
\NC      \NC = n - 1 + 2\frac{1}{1-1/2} + \Theta(n^2) \NR
\NC      \NC = \Theta(n^2) + n + 3 \NR
\NC      \NC = \Theta(n^2) \NR
\stopmathalignment\stopformula

猜測 \m{T(n) \le cn^2 + 2n}:
\startformula\startmathalignment
\NC T(n) \NC \le 4c(n/2)^2 + 2n/2 + n \NR
\NC      \NC \le cn^2 + 2n \NR
\NC      \NC = \Theta(n^2) \NR
\stopmathalignment\stopformula

	\startMPcode
		input reccursion_tree
		string A[];
		string B[];
		A[3] := "$n$";		B[3] := "$n$";
		A[2] := "$n/2+2$";	B[2] := "$2n$";
		A[1] := "$n/4 + 1$";	B[1] := "$4n$";
		A[0] := "$T(1)$";	B[0] := "$2^i n + 2^{1-i}$";
		rectree_vertical(A, B)(4, 3, 35, 15, 2);
	\stopMPcode
\stopANSWER

\startEXERCISE
對遞迴式 \m{T(n) = T(n-1) + 1},
利用遞迴樹確定一個漸進上界,用代入法進行驗證。
\stopEXERCISE
\startANSWER
深度爲 \m{n},每一層爲 \m{2^i},有 \m{2^n} 個葉子。因此:
\startformula\startmathalignment
\NC T(n) \NC = \sum_{i=0}^{n-1}2^i + \Theta(2^n) \NR
\NC      \NC = \frac{2^n - 1}{2 - 1} + \Theta(2^n) \NR
\NC      \NC = \Theta(2^n) + 2^n - 1 \NR
\NC      \NC = \Theta(2^n) \NR
\stopmathalignment\stopformula

猜測 \m{T(n) \le c 2^n + n}。因此:
\startformula\startmathalignment
\NC T(n) \NC \le 2c2^{n-1} + (n - 1) + 1 \NR
\NC      \NC \le c2^n + n \NR
\NC      \NC = O(2^n) \NR
\stopmathalignment\stopformula
	\startMPcode
		input reccursion_tree
		string A[];
		string B[];
		A[3] := "$1$";		B[3] := "$1$";
		A[2] := "$1$";		B[2] := "$2$";
		A[1] := "$1$";		B[1] := "$4$";
		A[0] := "$T(1)$";	B[0] := "$2^i$";
		rectree(A, B)(2, 3, 30, 30, 6);
	\stopMPcode
\stopANSWER

\startEXERCISE
對遞迴式 \m{T(n) = T(n-1) + T(n/2) + n},
利用遞迴樹確定一個漸進上界,用代入法進行驗證。
\stopEXERCISE
\startANSWER
此題有點怪異。遞迴樹表明最差也就是指數級。
同時遞迴樹不是高度爲 \m{n} 的完全二叉樹,但也不是多項式。

猜測 \m{T(n) \le c 2^n - 4n},則:
\startformula\startmathalignment[n=3]
\NC T(n) \NC \le c2^{n-1} - 4(n-1) + c2^{n/2} - 4n/2 + n \NC \NR
\NC      \NC \le c(2^{n-1} + 2^{n/2}) - 5n + 1 \NC (n > 1/4) \NR
\NC      \NC \le c(2^{n-1} + 2^{n/2}) - 4n \NC (n > 2)\NR
\NC      \NC \le c(2^{n-1} + 2^{n-1}) - 4n \NC \NR
\NC      \NC \le c2^n - 4n \NC \NR
\NC      \NC = O(2^n) \NC \NR
\stopmathalignment\stopformula

猜測 \m{T(n) \ge c n^2},則:
\startformula\startmathalignment[n=3]
\NC T(n) \NC \ge c(n - 1)^2 + c(n/2)^2 + n \NC\NR
\NC      \NC \ge cn^2 - 2cn + 1 + cn^2/4 + n \NC\NR
\NC      \NC \ge (5/4)cn^2 + (1 - 2c)n + 1 \NC\NR
\NC      \NC \ge cn^2 + (1 - 2c)n + 1 \NC (c < 1/2)\NR
\NC      \NC \ge cn^2 \NC\NR
\NC      \NC = O(n^2) \NC\NR
\stopmathalignment\stopformula
\stopANSWER

\startEXERCISE
對遞迴式 \m{T(n) = T(n/3) + T(2n/3) + cn}(\m{c}爲常數),
利用遞迴樹論證其解爲 \m{\Omega(n\lg{n})}。
\stopEXERCISE
\startANSWER
到一個葉子節點的最短路徑爲 \m{\log_3^n}。
每層的葉子節點和爲 \m{c n}。

猜測 \m{T(n) \ge k n \log_3{n}},則:
\startformula\startmathalignment[n=3]
\NC T(n) \NC \ge k(n/3)\log_3(n/3) + k(2n/3)\log_3{(2n/3)} + cn \NC \NR
\NC      \NC = k(n/3)(\log_3{n} - 1) + k(2n/3)(\log_3{2} + \log_3{n} - 1) + cn \NC \NR
\NC      \NC = kn\log_3{n} + k(2n/3)\log_3{2} - kn + cn \NC (k < \frac{3c}{3 - 2\log_3{2}}) \NR
\NC      \NC \ge kn \log_3{n} \NC \NR
\stopmathalignment\stopformula
\stopANSWER

\startEXERCISE
對遞迴式 \m{T(n) = 4T(\lfloor n/2 \rfloor) + cn} (\m{c}爲常數),
利用遞迴樹給出其解的一個漸進緊確界,用代入法進行驗證。
\stopEXERCISE
\startANSWER
忽略向下取整。上個練習已經說明了如何處理。
樹深爲 \m{\lg n},每層爲 \m{2^{i}cn},有 \m{4^{\lg{n}} = n^2} 個葉子。
因此:
\startformula\startmathalignment
\NC T(n) \NC = \sum_{i=0}^{\lg{n}- 1}2^icn + \Theta(n^2) \NR
\NC      \NC = cn \sum_{i=0}^{\lg{n}-1}2^i + \Theta(n^2) \NR
\NC      \NC = cn \frac{2^{\lg{n}} - 1}{2 - 1} + \Theta(n^2) \NR
\NC      \NC = \Theta(n^2) \NR
\stopmathalignment\stopformula

猜測 \m{T(n) \le c n^2 + 2cn},則:
\startformula
T(n) \le 4c(n/2)^2 + 2cn/2 + cn
        \le cn^2 + 2cn
\stopformula

猜測 \m{T(n) \ge c n^2 + 2cn},則:
\startformula
T(n) \ge 4c(n/2)^2 + 2cn/2 + cn
        \ge cn^2 + 2cn
\stopformula
	\startMPcode
		input reccursion_tree
		string A[];
		string B[];
		A[3] := "$cn$";		B[3] := "$cn$";
		A[2] := "$cn/2$";	B[2] := "$2cn$";
		A[1] := "$cn/4$";	B[1] := "$4cn$";
		A[0] := "$T(1)$";	B[0] := "$2^icn$";
		rectree_vertical(A, B)(4, 3, 35, 15, 2);
	\stopMPcode
\stopANSWER

\startEXERCISE
對遞迴式 \m{T(n) = T(n-a) + T(a) + cn} (其中 \m{a\ge 1} 和 \m{c > 0}爲常數),
利用遞迴樹給出其解的一個漸進緊確界。
\stopEXERCISE
\startANSWER
樹高 \m{n/a},每層爲 \m{c(n - ia)}。有
\startformula\startmathalignment
\NC T(n) \NC = \sum_{i=0}^{n/a}c(n-ia) + (n/a)ca \NR
\NC      \NC = \sum_{i=0}^{n/a}cn - \sum_{i=0}^{n/a}cia + (n/a)ca \NR
\NC      \NC = cn^2/a - \Theta(n) + \Theta(n) \NR
\NC      \NC = \Theta(n^2) \NR
\stopmathalignment\stopformula

另一種方式:
\startformula\startmathalignment
\NC T(n) \NC = cn + T(a) + T(n - a) + T(a) \NR
\NC      \NC = cn + ca + c(n-a) + T(a) + T(n - 2a) \NR
\NC      \NC = cn + c(n-a) + 2ca + c(n - 2a) + T(a) + T(n - 3a) \NR
\NC      \NC = cn + c(n-a) + c(n - 2a) + c(n - 3a) + T(n - 4a) + 3ca + T(a) \NR
\NC      \NC = \frac{n(n+1)}{2a} + cn \NR
\NC      \NC = \Theta(n^2) \NR
\stopmathalignment\stopformula

猜測 \m{T(n) \le c n^2},則:
\startformula\startmathalignment[n=3]
\NC T(n) \NC \le c(n-a)^2 + ca + cn \NC \NR
\NC      \NC \le cn^2 - 2acn + ca + cn \NC \NR
\NC      \NC \le cn^2 - c(2an - a - n) \NC (a > 1/2, n > 2a) \NR
\NC      \NC \le cn^2 - cn \NC \NR
\NC      \NC \le cn^2 \NC \NR
\NC      \NC = \Theta(n^2) \NC \NR
\stopmathalignment\stopformula

猜測 \m{T(n) \ge c n^2},則:
\startformula\startmathalignment[n=3]
\NC T(n) \NC \ge c(n-a)^2 + ca + cn \NC \NR
\NC      \NC \ge cn^2 - 2acn + ca + cn \NC \NR
\NC      \NC \ge cn^2 - c(2an - a - n) \NC (a < 1/2, n > 2a) \NR
\NC      \NC \ge cn^2 + cn \NC \NR
\NC      \NC \ge cn^2 \NC \NR
\NC      \NC = \Theta(n^2) \NC \NR
\stopmathalignment\stopformula

	\startMPcode
		input reccursion_tree
		string A[];
		string B[];
		string S;
		A[4] := "$cn$";		B[4] := "$cn$";
		A[3] := "$c(n-a)$";	B[3] := "$cn$";
		A[2] := "$c(n-2a)$";	B[2] := "$c(n-a)$";
		A[1] := "$c(n-3a)$";	B[1] := "$c(n-2a)$";
		A[0] := "$c(n-ia)$";	B[0] := "$c(n-(i-1)a)$";
		S := "$ca$";
		rectree_single_side_binary(A, B, S)(2, 4, 30, 30, 6);
	\stopMPcode
\stopANSWER

%e4.4-9
\startEXERCISE[exercise:partition_alpha]
對遞迴式 \m{T(n) = T(\alpha n) + T((1-\alpha)n) + cn} (其中 \m{0 < \alpha < 1} 和 \m{c > 0}爲常數),
利用遞迴樹給出其解的一個漸進緊確界。
\stopEXERCISE
\startANSWER
我們可以假設 \m{\alpha \le 1/2},因爲我們可以讓 \m{\beta = 1 - \alpha} 來代替 \m{\alpha}。

由此,樹的深度爲 \m{\log_{1/\alpha}n},每一層爲 \m{cn}。
葉子節點不明朗,但我們可以假設爲 \m{\Theta(n)}。
\startformula\startmathalignment
\NC T(n) \NC = \sum_{i=0}^{\log_{1/\alpha}n}cn + \Theta(n) \NR
\NC      \NC = cn\log_{1/\alpha}n + \Theta(n) \NR
\NC      \NC = \Theta(n\lg{n}) \NR
\stopmathalignment\stopformula

另一種方式,設 \m{\beta = 1 - \alpha},則:
\startformula\startmathalignment
\NC T(n) \NC = T(\alpha n) + T(\beta n) + cn \NR
\NC      \NC = T(\alpha^2 n) + 2T(\alpha \beta n) + T(\beta^2 n) + cn + c \alpha n  + c \beta n \NR
\NC      \NC = T(\alpha^2 n) + 2T(\alpha \beta n) + T(\beta^2 n) + 2cn \NR
\NC      \NC = T(\alpha^3 n) + T(\alpha^2 \beta n) + c\alpha^2 n +
               2T(\alpha^2 \beta n) + 2T(\alpha \beta^2 n) + 2c\alpha\beta n +
               T(\alpha \beta^2 n) + T(\beta ^ 3 n) + c\beta ^ 2 n + 2cn \NR
\NC      \NC = T(\alpha^3 n) + 3T(\alpha^2 \beta n) + 3T(\alpha \beta^2 n) + T(\beta^3 n) +
               c \alpha^2 n + 2c \alpha \beta n + c \beta ^ 2 n + 2cn \NR
\NC      \NC = T(\alpha^3 n) + 3T(\alpha^2 \beta n) + 3T(\alpha \beta^2 n) + T(\beta^3 n) + 3cn \NR
\NC      \NC = \ldots \NR
\stopmathalignment\stopformula

一直到 \m{\alpha^k n \le 1},最後得到 \m{T(n) = \cal{O}(1) + ckn}。
\startformula\startmathalignment
\NC \NC \alpha^k = \frac{1}{n} \NR
\NC \Rightarrow \NC \log{\alpha^k} = \log\frac{1}{n} \NR
\NC \Rightarrow \NC k\log\alpha = - \log{n} \NR
\NC \Rightarrow \NC k = \frac{-\log{n}}{\log\alpha} = \frac{\log{n}}{\log(1/\alpha)} = \log_{1/\alpha}n \NR
\stopmathalignment\stopformula

用代入法進行驗證。猜測 \m{T(n) \le dn\lg{n}}:
\startformula\startmathalignment[n=3]
\NC T(n) \NC \le d \alpha n \lg(\alpha n) + c \beta n \lg(\beta n) + cn \NC \NR
\NC      \NC \le d \alpha n \lg{n} + d \beta n \lg{n} + d \alpha n \lg\alpha + d \beta n \lg\beta + cn \NC \NR
\NC      \NC \le d n \lg{n} + \big(d (\alpha \lg\alpha + \beta \lg\beta) + c\big)n \NC (d(\alpha\lg\alpha + \beta\lg\beta) + c \le 0)\NR
\NC      \NC \le d n \lg{n} \NC \NR
\stopmathalignment\stopformula

由於 \m{1/2 \le \alpha < 1} 並且 \m{0 < 1 - \alpha \le 1/2},有 \m{\lg\alpha < 0} 以及 \m{\lg(1-\alpha)<0}。
因此 \m{\alpha\lg\alpha + \beta\lg\beta < 0},因此:
\startformula
d \ge -\frac{c}{\alpha\lg\alpha + (1-\alpha)\lg(1-\alpha)}
\stopformula
不等式右側爲一個正常數,只要 \m{d} 滿足此不等式, \m{T(n) \le dn\lg{n}} 就成立。

同理若要 \m{T(n) \ge dn\lg{n}},只需 \m{d(\alpha\lg\alpha + \beta\lg\beta) + c \ge 0} 即可,
即:
\startformula
0 \le d \le -\frac{c}{\alpha\lg\alpha + (1-\alpha)\lg(1-\alpha)}
\stopformula

因此 \m{T(n) = \Theta(n\lg n)}。
\stopANSWER

\section{The master method for solving recurrences}

\startEXERCISE
用主定理求下列遞迴式的漸進緊確界。
\startigBase[a]
\item \m{T(n) = 2T(n/4) + 1}
\item \m{T(n) = 2T(n/4) + \sqrt{n}}
\item \m{T(n) = 2T(n/4) + n}
\item \m{T(n) = 2T(n/4) + n^2}
\stopigBase
\stopEXERCISE
\startANSWER
\startigBase[a]
\item \m{\Theta(n^{\log_4{2}}) = \Theta(\sqrt{n})}
\item \m{\Theta(n^{\log_4{2}}\lg{n}) = \Theta(\sqrt{n}\lg{n})}
\item \m{\Theta(n)}
\item \m{\Theta(n^2)}
\stopigBase
\stopANSWER

\startEXERCISE
Caesar 教授想設計一個漸進優於 Strassen 算法的矩陣相乘算法。
他的算法使用分治策略,將每個矩陣分解爲 \m{n/4 \times n/4} 的子矩陣,
分解和合並步驟共花費時間爲 \m{\Theta(n^2)}。
他需要確定,這個算法需要創建多少個子問題,才能擊敗 Strassen 算法。
如果他的算法創建 \m{a} 個子問題,則描述運行時間 \m{T(n)} 的遞迴式爲
\m{T(n) = a T(n/4) + \Theta(n^2)}。
Caesar 教授的算法如果要漸進優於 Strassen 算法, \m{a} 的最大整數值應爲多少?
\stopEXERCISE
\startANSWER
當 \m{a < 16} 時適用於主定理的第一種情況。
此時,此算法 \m{T(n) = \Theta(n^{\log_4{a}})}。
要想更快,需 \m{\log_4{a} < \log_{2}7},其中 \m{a < 14}。
因此,最大整數值爲 \m{13}。
\stopANSWER

\startEXERCISE
用主定理證明:而分查找遞迴式 \m{T(n) = T(n/2) + \Theta(1)} 的解爲 \m{T(n) = \Theta(\lg n)}。
(二分查找的描述參見\refexercise{bin_search})
\stopEXERCISE
\startANSWER
\startformula\startmathalignment
\NC a =\NC 1\NR
\NC b =\NC 2\NR
\NC f(n) =\NC \Theta(n^{\log_2^1}) = \Theta(1) \NR
\NC T(n) =\NC \Theta(\lg n) \NR
\stopmathalignment\stopformula
\stopANSWER

\startEXERCISE
主定理是否可用於遞迴式 \m{T(n) = 4 T(n/2) + n^2 \lg n}?
請說明爲什麼。
請給出此遞迴式的漸進上界。
\stopEXERCISE
\startANSWER
根據主定理, 有 \m{a = 4}、 \m{b = 2},
而 \m{f(n) = n^2\lg{n} \ne \cal{O}(n^{2-\epsilon}) \ne \Omega(n^{2-\epsilon})},
因此無法使用主定理。

猜測 \m{\Theta(n^2\lg^2{n})}:
\startformula\startmathalignment
\NC T(n) \NC \le 4T(n/2) + n^2\lg{n} \NR
\NC      \NC \le 4c(n/2)^2\lg^2(n/2) + n^2\lg{n} \NR
\NC      \NC \le cn^2\lg(n/2)\lg{n} - cn^2\lg(n/2)\lg{2} + n^2\lg{n} \NR
\NC      \NC \le cn^2\lg^2{n} - cn^2\lg{n}\lg{2} - cn^2\lg(n/2) + n^2\lg{n} \NR
\NC      \NC \le cn^2\lg^2{n} + (1 - c)n^2\lg{n} - cn^2\lg(n/2) \qquad (c > 1) \NR
\NC      \NC \le cn^2\lg^2{n} - cn^2\lg(n/2) \NR
\NC      \NC \le cn^2\lg^2{n} \NR
\stopmathalignment\stopformula

推廣結論參見 \refexercise{master_method_generalized}。
\stopANSWER

\startEXERCISE \DIFFICULT
考慮主定理第三種情況中,對於某個常數 \m{c < 1},考慮其正則條件 \m{a f(n/b) \le c f(n)} 是否成立。
給出例子,使常數 \m{a \ge 1}、 \m{b > 1},且函數 \m{f(n)} 滿足主定理第三種情況中除正則條件外的所有條件。
\stopEXERCISE
\startANSWER
\startformula\startmathalignment
\NC a \NC = 1 \NR
\NC b \NC = 2 \NR
\NC f(n) \NC = n (2 - \cos n) \NR
\stopmathalignment\stopformula

如果想要證明:
\startformula\startmathalignment[n=1]
\NC \frac{n}{2}(2 - \cos \frac{n}{2}) < c n \NR
\NC \frac{1 - \cos(n/2)}{2} < c \NR
\NC c \ge 1 - \frac{\cos(n/2)}{2} \NR
\stopmathalignment\stopformula
由於 \m{\min \cos(n/2) = -1},因此需要 \m{c \ge \frac{3}{2}},但是 \m{c < 1}。
\stopANSWER

\section{Proof of the master theorem}

\startEXERCISE \DIFFICULT
對 \m{b} 時正整數而非任意實數的情況,給出公式(4.27)中 \m{n_j} 的表達式,要求既簡單又準確。
附公式(4.27):
\startformula
 n_j = \startmathcases
   \NC n \NC 若 \m{j = 0} \NR
   \NC \lceil n_{j-1}/b \rceil \NC 若 \m{j > 0} \NR
\stopmathcases
\stopformula
\stopEXERCISE
\startANSWER
若 \m{n} 爲正整數,則解爲:
\startformula
n_j = \lceil n / b^j \rceil
\stopformula

\goto{stackexchange}[url(http://math.stackexchange.com/questions/509862/closed-form-expression-for-n-j-defined-by-n-j-lceil-n-j-1-b-rceil)] 上有如下證明:

若 \m{a} 和 \m{b} 均爲正整數, \m{x} 爲任意實數,則有:
\startformula
\lceil \frac{x}{a} \rceil - 1
< \frac{x}{a}
\le \lceil \frac{x}{a} \rceil
\stopformula

兩邊同除以 \m{b},則:
\startformula
\frac{1}{b}(\lceil \frac{x}{a} \rceil - 1)
< \frac{x}{ab}
\le \frac{1}{b} \lceil \frac{x}{a} \rceil
\stopformula

令 \m{P = \lceil \frac{x}{a} \rceil},記 \m{P = q \cdot b - r},
其中 \m{0 \le r < b}, \m{r}、 \m{b} 均爲整數。
則由 \m{P > (q - 1) \cdot b} 可知 \m{P - 1 \ge (q - 1) \cdot b},因此:
\startformula
q - 1 \le \frac{P-1}{b} < \frac{x}{ab} \le \frac{P}{b} \le q
\stopformula

由此可知:
\startformula
\lceil \frac{x}{ab} \rceil = \lceil \frac{1}{b} \lceil \frac{x}{a} \rceil \rceil
\stopformula

顯然,推廣後可以執行任意次除法,也可以將 \m{\lceil \rceil} 換乘 \m{\lfloor \rfloor},
但是不能混合使用。因此
\startformula
n_j = \lceil \frac{n_{j-1}}{b} \rceil
    = \lceil \frac{1}{b} \lceil \frac{n}{b^{j-1}} \rceil \rceil
    = \lceil \frac{n}{b^j} \rceil
\stopformula

如果 \m{b} 是任意正實數,則結論不成立,
因爲無法由 \m{P > (q-1)\cdot b} 得出 \m{P-1 \ge (q-1) \cdot b}。
\stopANSWER

\startEXERCISE[exercise:master_method_generalized] \DIFFICULT
證明:如果 \m{f(n) = \Theta(n^{\log_b{a}}\lg^k{n})},其中 \m{k\ge 0},
則主遞迴式的解爲 \m{T(n) = \Theta(n^{\log_b{a}}\lg^{k+1}n)}。
簡單起見,假定 \m{n} 是 \m{b} 的冪。(\m{T(n)=aT(n/b)+f(n)})
\stopEXERCISE
\startANSWER
\startformula\startmathalignment[n=2]
\NC g(n) \NC= \sum_{j=0}^{\log_b{n}-1}a^jf(n/b^j) \NR
\NC f(n/b^j) \NC= \Theta\Big((n/b^j)^{\log_b{a}}\lg^k(n/b^j)\Big) \NR
\NC g(n) \NC= \Theta\Big(\sum_{j=0}^{\log_b{n}-1}a^j\big(\frac{n}{b^j}\big)^{\log_b{a}}\lg^k\big(\frac{n}{b^j}\big)\Big) = \Theta(A) \NR
\NC A \NC= \sum_{j=0}^{\log_b{n}-1}a^j\big(\frac{n}{b^j}\big)^{\log_b{a}}\lg^k\frac{n}{b^j} \NR
\NC   \NC= n^{\log_b{a}}\sum_{j=0}^{\log_b{n}-1}\Big(\frac{a}{b^{\log_b{a}}}\Big)^j\lg^k\frac{n}{b^j} \NR
\NC   \NC= n^{\log_b{a}}\sum_{j=0}^{\log_b{n}-1}\lg^k\frac{n}{b^j} \NR
\NC   \NC= n^{\log_b{a}}B \NR
\NC \lg^k\frac{n}{d} \NC= (\lg{n} - \lg{d})^k = \lg^k{n} + o(\lg^k{n}) \NR
\NC B \NC= \sum_{j=0}^{\log_b{n}-1}\lg^k\frac{n}{b^j} \NR
\NC   \NC= \sum_{j=0}^{\log_b{n}-1}\Big(\lg^k{n} - o(\lg^k{n})\Big) \NR
\NC   \NC= \log_b{n}\lg^k{n} + \log_b{n} \cdot o(\lg^k{n}) \NR
\NC   \NC= \Theta(\log_b{n}\lg^k{n}) \NR
\NC   \NC= \Theta(\lg^{k+1}{n}) \NR
\NC g(n) \NC= \Theta(A) = \Theta(n^{\log_b{a}}B) = \Theta(n^{\log_b{a}}\lg^{k+1}{n}) \NR
\stopmathalignment\stopformula
\stopANSWER

\startEXERCISE \DIFFICULT
證明:主定理中第三種情況被過分強調了,因爲對於某個常數 \m{c<1},
正則條件 \m{af(n/b)\le cf(n)} 成立本身就意味着存在常數 \m{\epsilon > 0},
使得 \m{f(n) = \Omega(n^{\log_b{a} + \epsilon})}。
\stopEXERCISE
\startANSWER
證明如下,參見\goto{stackexchange}[url(http://math.stackexchange.com/questions/510897/why-does-afn-b-cfn-for-c-1-imply-that-fn-omegan-log-ba-ep)]:
\startformula\startmathalignment[n=1]
\NC a f(n/b) \le cf(n) \NR
\NC \alpha f(n/b) \le f(n) \quad \alpha = a/c \NR
\NC \alpha f(n) \le f(nb) \NR
\NC \alpha^i f(1) \le f(b^i) \NR
\NC n = b^i \Rightarrow i = \log_{b}n \Rightarrow f(n) \ge \alpha^{\log_b{n}}f(1) = n^{\log_{b}\alpha} \NR
\NC \alpha > a \Rightarrow \alpha = a + d \quad (c < 1, d > 0) \NR
\NC \Rightarrow f(n) = n^{\log_b{a} + log_b{d}} = n^{\log_b{a}+\epsilon} \quad (\epsilon = \log_{b}d) \NR
\stopmathalignment\stopformula
\stopANSWER

\startPROBLEM(遞迴式例子)
對於下列遞迴式,給出 \m{T(n)} 的漸進上界和下界。
假定 \m{n\le 2} 時 \m{T(n)} 是常數。給出盡量緊確的界,並驗證其正確性。
\startigBase[n]
\item \m{T(n) = 2 T(n/2) + n^4}

\startANSWER
\m{\Theta(n^4)} (主定理)
\stopANSWER

\item \m{T(n) = T(7n/10) + n}

\startANSWER
\m{\Theta(n)} (主定理, \m{\log_{10/7} 1 = 0})
\stopANSWER

\item \m{T(n) = 16 T(n/4) + n^2}

\startANSWER
\m{\Theta(n^2\lg n)} (主定理)
\stopANSWER

\item \m{T(n) = 7 T(n/3) + n^2}

\startANSWER
\m{\Theta(n^2)} (主定理)
\stopANSWER

\item \m{T(n) = 7 T(n/2) + n^2}

\startANSWER
\m{\Theta(n^{\log_2 7})} (主定理)
\stopANSWER

\item \m{T(n) = 2 T(n/4) + n^2}

\startANSWER
\m{\Theta(\sqrt{n}\log_n)} (主定理)
\stopANSWER

\item \m{T(n) = T(n-2) + n^2}

\startANSWER
\m{T(n) = n^2 + T(n-2) = n^2 + (n-2)^2 + T(n-4) = \sum_{i=0}^{n/2}(n-2i)^2 = \Theta(n^3)}
\stopANSWER
\stopigBase
\stopPROBLEM

\startPROBLEM(參數傳遞代價)
本書中自始至終假設:過程調用中的參數傳遞花費常量時間,
即使傳遞一個 \m{N} 個元素的數列亦是如此。
在大多數系統中,這個假設是成立的,因爲傳遞的是數列的指標,而非數列本身。
本題討論三種參數傳遞策略:
\startigBase[n]
\item 通過指標傳遞數列。時間爲 \m{\Theta(1)}。
\item 通過復制元素傳遞數列。時間爲 \m{\Theta(N)},其中 \m{N} 是數列的規模。
\item 傳遞數列時,只復制過程可能存取的子區域。若子數列 \m{A[p..q]} 被傳遞,則時間爲 \m{\Theta(q-p+1)}。
\stopigBase
\startigBase[a]
\item[parameter_pass_cost_item]
考慮在有序數列中查找元素的遞迴二分查找算法(參見\refexercise{bin_search})。
分別給出上述三種策略下,二分查找最壞情況運行時間的遞迴式,並給出遞迴式解的好的上界。
令 \m{N} 爲原問題的規模, \m{n} 爲子問題的規模。
\item 對節 2.3.1 中的 MERGE-SORT 算法重做上一項。
\stopigBase

\startANSWER
二分查找:
\startigBase[n]
\item \m{T(n) = T(n/2) + c = \Theta(\lg n)} (主定理)
\item \m{T(n) = T(n/2) + cN = T(n/4) + 2cN = \sum_{i=0}^{\lg n - 1}(2^i cN/2^i) = cN\lg n = \Theta(n\lg n)}
\item \m{T(n) = T(n/2) + cn = \Theta(n)} (主定理)
\stopigBase
歸並排序:
\startigBase[n]
\item \m{T(n) = 2T(n/2) + cn = \Theta(n\lg{n})} (主定理)
\item
\startformula\startmathalignment
\NC T(n) \NC= 2T(n/2) + cn + 2N = 4N + cn + 2c(n/2) + 4T(n/4) \NR
\NC      \NC= 8N + 2cn + 4c(n/4) + 8T(n/8) \NR
\NC      \NC= \sum_{i=0}^{\lg{n}-1}(cn + 2^iN) \NR
\NC      \NC= \sum_{i=0}^{\lg{n}-1}cn + N\sum_{i=0}^{\lg{n}-1}2^i \NR
\NC      \NC= cn\lg{n} + N\frac{2^{\lg{n}} - 1}{2-1} \NR
\NC      \NC= cn\lg{n} + nN - N \NR
\NC      \NC= \Theta(nN) \NR
\NC      \NC= \Theta(n^2) \NR
\stopmathalignment\stopformula
\item \m{T(n) = 2T(n/2) + cn + 2n/2 = 2T(n/2) + (c+1)n = \Theta(n\lg{n})} (主定理)
\stopigBase
\stopANSWER
\stopPROBLEM

\startPROBLEM(更多遞迴式例子)
對於下列遞迴式,給出 \m{T(n)} 的漸進上界和下界。
假定對足夠小的 \m{n}, \m{T(n)} 是常數。給出盡量緊確的界,並驗證其正確性。
\startigBase[n]
\item \m{T(n) = 4T(n/3) + n\lg{n}}

\startANSWER
\m{\Theta(n^{\log_3 4})} (主定理)
\stopANSWER

\item \m{T(n) = 3T(n/3) + n/\lg{n}}

\startANSWER
\m{\Theta(n \lg \lg n)} (參見\refitem{Tnlgn})
\stopANSWER

\item \m{T(n) = 4T(n/2) + n^2\sqrt{n}}

\startANSWER
\m{\Theta(n^2\sqrt{n}) = \Theta(n^{2.5})} (主定理 \m{\log_2{4} = 2 < 2.5})
\stopANSWER

\item \m{T(n) = 3T(n/3 - 2) + n/2}

\startANSWER
可以忽略 \m{-2},用主定理可得 \m{\Theta(n\lg{n})}。
\stopANSWER

\item[item:Tnlgn] \m{T(n) = 2T(n/2) + n/\lg{n}}

\startANSWER
\startformula\startmathalignment
\NC T(n) \NC= 2T(n/2) + \frac{n}{\lg{n}} \NR
\NC      \NC= 4(n/4) + 2\frac{n/2}{\lg(n/2)} + \frac{n}{\lg{n}} \NR
\NC      \NC= 4T(n/4) + \frac{n}{\lg{n} - 1} + \frac{n}{\lg{n}} \NR
\NC      \NC= nT(1) + \sum_{i=0}^{\lg{n} - 1}\frac{n}{\lg{n}-i} \NR
\NC      \NC= nT(1) + n\sum_{i=1}^{\lg{n}}\frac{1}{\lg{n}} \NR
\NC      \NC= \Theta(n\lg\lg{n}) \NR
\stopmathalignment\stopformula
\stopANSWER

\item \m{T(n) = T(n/2) + T(n/4) + T(n/8) + n}

\startANSWER
猜測 \m{\Theta(n)}:
\startformula\startmathalignment
\NC T(n) \NC = cn/2 + cn/4 + cn/8 + n \le (7/8)cn + n \le cn = O(n) \quad (c \ge 8) \NR
\NC T(n) \NC = cn/2 + cn/4 + cn/8 + n \ge (7/8)cn + n \ge cn = \Omega(n) \quad (c \le 8) \NR
\stopmathalignment\stopformula
\stopANSWER

\item \m{T(n) = T(n - 1) + 1/n}

\startANSWER
\startformula\startmathalignment
\NC T(n) \NC= T(n-1) + 1/n \NR
\NC      \NC= \frac{1}{n} + \frac{1}{n-1} + T(n-2) \NR
\NC      \NC= \frac{1}{n} + \frac{1}{n-1} + \frac{1}{n-2} + T(n-3) \NR
\NC      \NC= \sum_{i=0}^{n-1}\frac{1}{n-i} \NR
\NC      \NC= \sum_{i=1}^n\frac{1}{i} \NR
\NC      \NC= \Theta(\lg{n})
\stopmathalignment\stopformula
\stopANSWER

\item \m{T(n) = T(n - 1) + \lg{n}}

\startANSWER
\startformula\startmathalignment
\NC T(n) \NC= \lg{n} + T(n-1) \NR
\NC      \NC= \lg{n} + \lg{n-1} + T(n-2) \NR
\NC      \NC= \sum_{i=0}^{n-1}\lg(n - i) \NR
\NC      \NC= \sum_{i=1}^{n}\lg{i} \NR
\NC      \NC= \lg(n!) \le \lg{n^n} \NR
\NC      \NC= n\lg{n} \NR
\NC      \NC= \Theta(n\lg{n}) \NR
\stopmathalignment\stopformula
\stopANSWER

\item \m{T(n) = T(n - 2) + 1/\lg{n}}

\startANSWER
\startformula\startmathalignment
\NC T(n) \NC= \frac{1}{\lg{n}} + \frac{1}{\lg{n-2}} + \ldots \NR
\NC      \NC= \sum_{i=1}^{n/2}\frac{1}{\lg(2i)} \NR
\NC      \NC= \sum_{i=1}^{\infty}\frac{1}{\lg{i}} \NR
\NC      \NC= \Theta(\lg\lg{n}) \NR
\stopmathalignment\stopformula
\stopANSWER

\item \m{T(n) = \sqrt{n}T(\sqrt{n}) + n}

\startANSWER
猜測 \m{T(n) \le cn\lg{n}\lg{n}}:
\startformula\startmathalignment
\NC T(n) \NC\le \sqrt{n}c\sqrt{n}\lg\lg\sqrt{n} + n \NR
\NC      \NC= cn\lg\lg\sqrt{n} + n \NR
\NC      \NC= cn\lg\frac{\lg{n}}{2} + n \NR
\NC      \NC= cn\lg\lg{n} - cn\lg{2} + n \NR
\NC      \NC= cn\lg\lg{n} + (1 - c)n \quad (c > 1) \NR
\NC      \NC\le cn\lg\lg{n} \NR
\NC      \NC= \Theta(n\lg\lg{n}) \NR
\stopmathalignment\stopformula
\stopANSWER

\stopigBase
\stopPROBLEM

\startPROBLEM[problem:generating_function] (Fibonacci 數)
本題討論遞迴式(3.22)所定義的 Fibonacci 數的性質。
我們將使用生成函數技術來求解 Fibonacci 遞迴式。
{\EMP 生成函數 generating function}(又稱{\EMP 形式冪級數 formal power series}) \m{\cal{F}} 定義爲:
\startformula\startmathalignment
\NC \cal{F}(z) \NC= \sum_{i=0}^{\infty}F_iz^i \NR
\NC                \NC= 0 + z + z^2 + 2z^3 + 3z^4 + 5z^5 + 8z^6 + 13z^7 + 21z^8 + \ldots \NR
\stopmathalignment\stopformula
其中 \m{F_i} 是第 \m{i} 個 Fibonacci 數。

附遞迴式(3.22):
\startformula\startmathalignment
\NC F_0 \NC = 0 \NR
\NC F_1 \NC = 1 \NR
\NC F_i \NC = F_{i-1} + F_{i-2} \qquad 若 \m{i \ge 2} \NR
\stopmathalignment\stopformula

\startigBase[a]
\item 證明: \m{\cal{F}(z) = z + z\cal{F}(z) + z^2\cal{F}(z)}。

\startANSWER
\startformula\startmathalignment
\NC  \NC z + z\cal{F}(z) + z^2\cal{F}(Z) \NR
\NC =\NC z + z\sum_{i=0}^{\infty}F_iz^i + z^2\sum_{i=0}^{\infty}F_iz^i \NR
\NC =\NC z + \sum_{i=1}^{\infty}F_{i-1}z^i + \sum_{i=2}^{\infty}F_{i-2}z^i \NR
\NC =\NC z + F_1z + \sum_{i=2}^{\infty}(F_{i-1} + F_{i-2})z^i \NR
\NC =\NC z + F_1z + \sum_{i=2}^{\infty}F_iz^i \NR
\NC =\NC \cal{F}(z) \NR
\stopmathalignment\stopformula
\stopANSWER

\startitem 證明:\startformula\startmathalignment
\NC \cal{F}(z) \NC= \frac{z}{1 - z - z^2} \NR
\NC                \NC= \frac{z}{(1 - \phi z)(1 - \hat\phi z)} \NR
\NC                \NC= \frac{1}{\sqrt5}\Big(\frac{1}{1 - \phi z} - \frac{1}{1 - \hat{\phi} z}\Big) \NR
\stopmathalignment\stopformula

其中\startformula\startmathalignment
\NC     \phi \NC= \frac{1 + \sqrt5}{2} = 1.61803\ldots \NR
\NC \hat\phi \NC= \frac{1 - \sqrt5}{2} = -0.61803\ldots \NR
\stopmathalignment\stopformula
\stopitem

\startANSWER
只需注意到 \m{\phi - \hat\phi = \sqrt5}, \m{\phi + \hat\phi = 1}, \m{\phi\hat\phi = -1}:
\startformula\startmathalignment
\NC  \NC \cal{F}(z) \NR
\NC =\NC \frac{\cal{F}(z)(1 - z - z^2)}{1 - z - z^2} \NR
\NC =\NC \frac{\cal{F}(z) - z\cal{F}(z) - z^2\cal{F}(z) - z + z}{1 - z - z^2} \NR
\NC =\NC \frac{\cal{F}(z) - \cal{F}(z) + z}{1 - z - z^2} \NR
\NC =\NC \frac{z}{1 - z - z^2} \NR
\NC =\NC \frac{z}{1 - (\phi + \hat\phi)z + \phi\hat\phi z^2} \NR
\NC =\NC \frac{z}{(1 - \phi z)(1 - \hat\phi z)} \NR
\NC =\NC \frac{\sqrt5 z}{\sqrt5 (1 - \phi z)(1 - \hat\phi z)} \NR
\NC =\NC \frac{(\phi - \hat\phi)z + 1 - 1}{\sqrt5 (1 - \phi z)(1 - \hat\phi z)} \NR
\NC =\NC \frac{(1 - \hat\phi z) - (1 - \phi z)}{\sqrt5 (1 - \phi z)(1 - \hat\phi z)} \NR
\NC =\NC \frac{1}{\sqrt5}\Big(\frac{1}{1 - \phi z} - \frac{1}{1 - \hat\phi z}\Big) \NR
\stopmathalignment\stopformula
\stopANSWER

\item 證明: \m{\cal{F}(z) = \sum_{i=0}^{\infty}\frac{1}{\sqrt5}(\phi^i - \hat{\phi}^i)z^i}。

\startANSWER
如果 \m{|x|<1},則 \m{\frac{1}{1 - x} = \sum_{k=0}^{\infty}x^k}。因此:
\startformula\startmathalignment
\NC \cal{F}(n) \NC= \frac{1}{\sqrt5}\Big(\frac{1}{1 - \phi z} - \frac{1}{1 - \hat\phi z}\Big) \NR
\NC                \NC= \frac{1}{\sqrt5}\Big(\sum_{i=0}^{\infty}\phi^i z^i - \sum_{i=0}^{\infty}\hat{\phi}^i z^i\Big) \NR
\NC                \NC= \sum_{i=0}^{\infty}\frac{1}{\sqrt5}(\phi^i - \hat{\phi}^i) z^i\NR
\stopmathalignment\stopformula
\stopANSWER

\item 利用上一項的結果證明:對 \m{i>0}, \m{F_i = \phi^i / \sqrt5},結果舍入到最近整數。
(\hint \m{|\hat\phi|<1})

\startANSWER
\startformula
\cal{F}(z) = \sum_{i=0}^{\infty}\alpha_iz^i \quad\text{其中} \alpha_i = \frac{\phi^i - \hat{\phi}^i}{\sqrt5}
\stopformula

由於 \m{\alpha_i = F_i},即:
\startformula
F_i = \frac{\phi^i - \hat{\phi}^i}{\sqrt5}  = \frac{\phi^i}{\sqrt5} - \frac{\hat{\phi}^i}{\sqrt5}
\stopformula

對於 \m{i = 0}, \m{\phi^i/\sqrt5 = (\sqrt5 + 5)/10 > 0.5};
對於 \m{i > 2}, \m{|{\hat\phi}^i| < 0.5}。
\stopANSWER

\stopigBase
\stopPROBLEM

\startPROBLEM(芯片檢測)
Diogenes教授有 \m{n} 片可能完全一樣的集成電路芯片,原理上可以用來相互檢測。
教授的測試夾具同時只能容納兩塊芯片。當家具裝載上時,每塊芯片都檢測兩一塊,並報告它是好是壞。
一塊好的芯片總能準確報告另一塊芯片的好壞,但教授不能信任壞芯片報告的結果。
因此, 4 種可能的檢測結果如下:
\bTABLE[align=center]
\bTABLEhead\bTR
	\bTH 芯片 A 報告 \eTH
	\bTH 芯片 B 報告 \eTH
	\bTH 結論 \eTH
\eTR\eTABLEhead
\bTABLEbody\bTR
	\bTD B 是好的 \eTD
	\bTD A 是好的 \eTD
	\bTD 兩片都是好的,或都是壞的 \eTD
\eTR\bTR
	\bTD B 是好的 \eTD
	\bTD A 是壞的 \eTD
	\bTD 至少一塊是壞的 \eTD
\eTR\bTR
	\bTD B 是壞的 \eTD
	\bTD A 是好的 \eTD
	\bTD 至少一塊是壞的 \eTD
\eTR\bTR
	\bTD B 是壞的 \eTD
	\bTD A 是壞的 \eTD
	\bTD 至少一塊是壞的 \eTD
\eTR\eTABLEbody
\eTABLE
\startigBase[a]
\item 證明:如果超過 \m{n/2} 塊芯片是壞的,使用任何基於這種結對檢測的策略都不能確定哪些芯片是好的。
假定壞芯片可以合謀欺騙教授。

\startANSWER
假定有 \m{g < n/2} 塊好芯片。其餘同樣數量的壞芯片可以表現出與好芯片類似的行爲。
也就是說,他們可以相互識別爲好芯片,並將其餘芯片識別爲壞的。
由於這兩組芯片表現出來的行爲一樣,無從區分,也就無法識別哪塊是好芯片。
\stopANSWER

\item 考慮從 \m{n} 塊芯片中尋找一塊好芯片的問題,假定超過 \m{n/2} 塊芯片是好的。
證明:進行 \m{\lfloor n/2 \rfloor} 次結對檢測足以將問題規模減半。

\startANSWER
將芯片分成兩組進行比較。如果結果是第一種(都是好的或都是壞的)我們可以取其一,否則兩塊都棄用。
當兩塊都棄用時,我們至少移除了一塊壞芯片,當然可能同時移除了一塊好芯片。
而當我們選用其中一塊時,好芯片數目多於壞芯片(兩塊都是好芯片的對數更多,因爲好芯片多於壞芯片)。
現在我們最多有 \m{n/2} 塊芯片,其中至少一半是好的。
\stopANSWER

\item 假定超過 \m{n/2} 塊芯片是好的,證明:可以用 \m{\Theta(n)} 次結對檢測找到好的芯片。
給出描述檢測次數的遞迴式,並求解。

\startANSWER
遞迴式爲 \m{T(n) = T(n/2) + n/2}。
由主定理可知其解爲 \m{\Theta(n)}。
找到一個好的後,我們可以用他與其他芯片進行結對檢測,即進行 \m{\Theta(n)} 次運算。
\stopANSWER

\stopigBase
\stopPROBLEM

\startPROBLEM(Monge 陣列)
對一個 \m{m\times n} 的實數陣列 \m{A},
若對所有滿足 \m{1\le i < k \le m} 和 \m{1\le j < l \le n} 的 \m{i}、 \m{j}、 \m{k} 和 \m{l} 有:
\startformula
A[i, j] + A[k, l] \le A[i, l] + A[k, j]
\stopformula
則稱 \m{A} 爲 {\EMP Monge 陣列}。
還句話說,在 Monge 陣列中任選兩行和兩列,對於交叉點上的 4 個元素,
左上角和右下角兩個元素之和總是小於等於左下角和右上角元素之和。
例如,下面就是一個 Monge 陣列:
\startformula\startmatrix
\NC 10 \NC 17 \NC 13 \NC 28 \NC 23 \NR
\NC 17 \NC 22 \NC 16 \NC 29 \NC 23 \NR
\NC 24 \NC 28 \NC 22 \NC 34 \NC 24 \NR
\NC 11 \NC 13 \NC  6 \NC 17 \NC  7 \NR
\NC 45 \NC 44 \NC 32 \NC 37 \NC 23 \NR
\NC 36 \NC 33 \NC 19 \NC 21 \NC  6 \NR
\NC 75 \NC 66 \NC 51 \NC 53 \NC 34 \NR
\stopmatrix\stopformula
\startigBase[a]
\startitem 證明:若一個陣列是 Monge 陣列,當且儘當對所有 \m{i=1,2,\ldots,m-1} 和 {j = 1,2,\ldots,n-1} 都有
\startformula
A[i,j] + A[i+1,j+1] \le A[i,j+1] + A[i+1,j]
\stopformula
(\hint 對於“當”的部分,分別對行和列使用歸納法)
\stopitem

\startANSWER
“儘當”的部分是顯然的,由 Monge 陣列的定義可知。
而對於“當”的部分,先用歸納法證明:
\startformula\startmathalignment[n=1]
\NC A[i,j] + A[i+1, j+1] \le A[i,j+1] + A[i+1, j] \NR
\NC \Downarrow \NR
\NC A[i,j] + A[k, j+1] \le A[i, j+1] + A[k,j] \NR
\stopmathalignment\stopformula
其中 \m{i<k}。

第一步, \m{k=i+1},由已知條件顯然成立。
歸納時,我們假設對於 \m{k=i+n} 成立,我們需要證明對於 \m{k+1=i+n+1} 仍然成立。
\startformula\startmathalignment[n=1]
\NC A[i, j] + A[k, j+1] \le A[i, j+1] + A[k, j] \quad (assumption) \NR
\NC A[k, j] + A[k+1, j+1] \le A[k, j+1] + A[k+1, j] \quad (given) \NR
\NC \Downarrow \NR
\NC A[i, j] + A[k, j+1] + A[k, j] + A[k+1, j+1] \le A[i, j+1] + A[k, j] + A[k, j+1] + A[k+1, j] \NR
\NC \Downarrow \NR
\NC A[i, j] + A[k+1, j+1] \le A[i, j+1] + A[k+1, j] \NR
\stopmathalignment\stopformula
\stopANSWER

\startitem
如下陣列並不是 Monge 陣列,改變其中一個元素,使其成爲 Monge 陣列。(\hint 用上一項的結果)
\startformula\startmatrix
\NC 37 \NC 23 \NC 22 \NC 32 \NR
\NC 21 \NC  6 \NC  7 \NC 10 \NR
\NC 53 \NC 34 \NC 30 \NC 31 \NR
\NC 32 \NC 13 \NC  9 \NC  6 \NR
\NC 43 \NC 21 \NC 15 \NC  8 \NR
\stopmatrix\stopformula
\stopitem

\startANSWER

\startformula\startmatrix
\NC 37 \NC 23 \NC \bf 24 \NC 32 \NR
\NC 21 \NC  6 \NC  7 \NC 10 \NR
\NC 53 \NC 34 \NC 30 \NC 31 \NR
\NC 32 \NC 13 \NC  9 \NC  6 \NR
\NC 43 \NC 21 \NC 15 \NC  8 \NR
\stopmatrix\stopformula
\stopANSWER

\item 令 \m{f(i)} 表示第 \m{i} 行中最左最小元素的列下標。
證明:對任意 \m{m\times n} 的 Monge 陣列, \m{f(1)\le f(2)\le\ldots\le f(m)}。

\startANSWER
令 \m{i} 和 \m{j} 分別爲行 \m{a} 和 \m{b} 中的最左最小元素的列下標,其中 \m{a < b}。
假設 \m{i>j},則:
\startformula
A[j, a] + A[i, b] \le A[i, a] + A[j, b]
\stopformula
但是\startformula\startmathalignment
\NC A[j, a] \ge A[i, a] \NC \quad (A[i,a] \text{ 最小}) \NR
\NC A[i, b] \ge A[j, b] \NC \quad (A[j,b] \text{ 最小}) \NR
\stopmathalignment\stopformula
隱含\startformula\startmathalignment
\NC A[j, a] + A[i, b] \ge A[i, a] + A[j, b] \NR
\NC \Downarrow \NR
\NC A[j, a] + A[i, b] = A[i, a] + A[j, b] \NR
\stopmathalignment\stopformula
又分別隱含\startformula\startmathalignment
\NC A[j, b] < A[i, b] \Rightarrow A[i, a] > A[j, a] \NC \Rightarrow A[i,a] \text{ 不是最小} \NR
\NC A[j, b] = A[i, b] \NC \Rightarrow A[j,b] \text{ 不是最左最小} \NR
\stopmathalignment\stopformula

推出矛盾,因此假設 \m{i>j} 不成立,即 \m{i\le j}。
可以得出結論:
\startformula
a < b \Rightarrow f(a) \le f(b)
\stopformula
即 \m{f(1)\le f(2)\le\ldots\le f(m)}。

\stopANSWER

\startitem
下面所描述的分治算法可用於找出 \m{m\times n} Monge 陣列 \m{A} 中每行的最左最小元素:

提取 \m{A} 的偶數行構造其子矩陣 \m{A'}。
遞迴確定 \m{A'} 每行的最左最小元素。
然後找出 \m{A} 的奇數行最左最小元素。

如何在 \m{O(m+n)} 時間內找出 \m{A} 的奇數行最左最小元素?
(假定已知偶數行的最左最小元素)
\stopitem

\startANSWER
如果 \m{\mu_i} 是第 \m{i} 行最左最小元素的列坐標,由上一項可知 \m{\mu_{i-1} \le \mu_i \le \mu_{i+1}}。

對於 \m{i = 2k+1},其中 \m{k\ge 0},
我們只需要比較此行中列坐標從 \m{\mu_{i-1}} 到 \m{\mu_{i+1}} 中的這些元素,
最多 \m{\mu_{i+1}-\mu_{i-1} + 1} 步就可以找出 \m{\mu_i}。
\startformula\startmathalignment
\NC T(m, n) \NC= \sum_{i=0}^{m/2-1}\Big(\mu_{2i + 2} - \mu_{2i} + 1\Big) \NR
\NC \NC= \sum_{i=0}^{m/2-1}\mu_{2i+2} - \sum_{i=0}^{m/2-1}\mu_{2i} + m/2 \NR
\NC \NC= \sum_{i=1}^{m/2}\mu{2i} - \sum_{i=0}^{m/2-1}\mu{2i} + m/2 \NR
\NC \NC= \mu_m - \mu_0 + m/2 \NR
\NC \NC= n + m/2 \NR
\NC \NC= O(m + n) \NR
\stopmathalignment\stopformula
\stopANSWER

\item 給出上一項中所述算法的運行時間的遞迴式。證明其解爲 \m{O(m+n\log{m})}。

\startANSWER
“分”所用時間爲 \m{O(1)},“治”所用時間爲 \m{T(m/2)},合並所用時間爲 \m{O(m+n)}。
\startformula\startmathalignment
\NC T(m) \NC= T(m/2) + cn + dm \NR
\NC      \NC= cn + dm + cn + dm/2 + cn + dm/4 + \ldots \NR
\NC      \NC= \sum_{i=0}^{\lg{m}-1}cn + \sum_{i=0}^{\lg{m}-1}\frac{dm}{2^i} \NR
\NC      \NC= cn\lg{m} + dm\sum_{i=0}^{\lg{m} - 1}\frac{1}{2^i} \NR
\NC      \NC< cn\lg{m} + 2dm \NR
\NC      \NC= O(n\lg{m} + m) \NR
\stopmathalignment\stopformula
\stopANSWER

\stopigBase
\stopPROBLEM

\stopcomponent
