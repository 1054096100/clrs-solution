\startsection[
  title={Functions},
]

%eB.3-1
\startEXERCISE
令 \m{A} 和 \m{B} 是有限集合, \m{f:A\rightarrow B} 是一個函數,證明:

\startigBase[a]\startitem
若 \m{f} 是單射,則 \m{|A|\le |B|};
\stopitem\stopigBase

\startigBase[continue]\startitem
若 \m{f} 是滿射,則 \m{|A|\ge |B|}。
\stopitem\stopigBase
\stopEXERCISE

\startANSWER
\TODO{略。}
\stopANSWER

%eB.3-2
\startEXERCISE
請問函數 \m{f(x)=x+1} 是從 \m{\naturalnumbers} 到 \m{\naturalnumbers} 的雙射嗎?
他是從 \m{\integers} 到 \m{\integers} 的雙射嗎?
\stopEXERCISE

\startANSWER
\TODO{略。}
\stopANSWER

%eB.3-3
\startEXERCISE
請爲二元關係的逆給出一個自然的定義,
滿足如下條件:
如果一個關係實際上是雙射函數,
那麼其關係逆即是其函數逆。
\stopEXERCISE

\startANSWER
\TODO{略。}
\stopANSWER

%eB.3-4
\startEXERCISE\DIFFICULT
請舉出一個從 \m{\integers} 到 \m{\integers\times\integers} 的雙射例子。
\stopEXERCISE

\startANSWER
\TODO{略。}
\stopANSWER

\stopsection
