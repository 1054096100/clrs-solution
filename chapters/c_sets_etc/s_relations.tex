\startsection[
  title={Relations},
]

%eB.2-1
\startEXERCISE
證明:集合 \m{\integers} 的所有子集上的子集關係”\m{\subseteq}“是偏序關係,
但不是全序關係。
\stopEXERCISE

\startANSWER
\TODO{略。}
\stopANSWER

%eB.2-2
\startEXERCISE
證明:對於任意正整數 \m{n},
關係”模 \m{n} 等價“是整數集上的等價關係。
(如果存在一個整數 \m{q},使得 \m{a-b=qn},則有 \m{a\equiv b(\mod n)}。)
這個關係將整數劃分爲哪些等價類?
\stopEXERCISE

\startANSWER
\TODO{略。}
\stopANSWER

%eB.2-3
\startEXERCISE
給出符合如下條件的關係的例子:

\startigBase[a]\startitem
具有自反性和對稱性,但不具有傳遞性。
\stopitem\stopigBase

\startigBase[continue]\startitem
具有自反性和傳遞性,但不具有對稱性。
\stopitem\stopigBase

\startigBase[continue]\startitem
具有對稱性和傳遞性,但不具有自反性。
\stopitem\stopigBase
\stopEXERCISE

\startANSWER
\TODO{略。}
\stopANSWER

%eB.2-4
\startEXERCISE
設 \m{S} 是一個有限集, \m{R} 是 \m{S\times S} 上的一個等價關係。
證明:如果 \m{R} 同時有反對稱性,
則 \m{S} 關於 \m{R} 劃分出的等價類是單元素集合。
\stopEXERCISE

\startANSWER
\TODO{略。}
\stopANSWER

%eB.2-5
\startEXERCISE
Narcissus 教授聲稱:
如果關係 \m{R} 具有對稱性和傳遞性,則也具有自反性。
他給出了如下證明:
由對稱性, \m{aRb} 蘊含 \m{bRa},
因此,由傳遞性可得 \m{aRa},自反性得證。
請問 Narcissus 教授的證明正確嗎?
\stopEXERCISE

\startANSWER
\TODO{略。}
\stopANSWER

\stopsection
