\startsection[
  title={Sets},
]

%eB.1-1
\startEXERCISE
用 Venn 圖描述分配律中第一條定律(B.1)。
\startformula
A\cap(B\cap C) = (A\cap B) \cap C
\stopformula
\stopEXERCISE

\startANSWER
\TODO{略。}
\stopANSWER

%eB.1-2
\startEXERCISE
證明推廣到任意有限數目集合的廣義 DeMorgan 定律:
\startformula\startmathalignment
\NC \overbar{A_1\cap A_2\cap\ldots\cap A_n}
    \NC = \overbar{A_1}\cup \overbar{A_2}\cup \ldots \cup \overbar{A_n} \NR
\NC \overbar{A_1\cup A_2\cup\ldots\cup A_n}
    \NC = \overbar{A_1}\cap \overbar{A_2}\cap \ldots \cap \overbar{A_n} \NR
\stopmathalignment\stopformula
\stopEXERCISE

\startANSWER
\TODO{略。}
\stopANSWER

%eB.1-1
\startEXERCISE
證明等式(B.3)的推廣,即{\EMP 容斥原理}:
\startformula\startmathalignment[n=3]
\NC |A_1\cup A_2\cup\ldots\cup A_n| =
    \NC |A_1| + |A_2| + \ldots + |A_n| \NC \NR
\NC \NC - |A_1\cap A_2| - |A_1\cap A_3| - \ldots \NC \qquad \text{(所有二元組)} \NR
\NC \NC + |A_1\cap A_2\cap A_3| + \ldots \NC \qquad \text{(所有三元組)} \NR
\NC \NC \vdots \NC \NR
\NC \NC + (-1)^{n-1}|A_1\cap A_2\cap \ldots \cap A_n| \NC \NR
\stopmathalignment\stopformula
\stopEXERCISE

\startANSWER
\TODO{略。}
\stopANSWER

%eB.1-4
\startEXERCISE
證明:奇自然數集合是可數的。
\stopEXERCISE

\startANSWER
\TODO{略。}
\stopANSWER

%eB.1-5
\startEXERCISE
證明:對於任意有限集 \m{S},
其冪集 \m{2^S} 有 \m{2^{|S|}} 個元素(即 \m{S} 存在 \m{2^{|S|}} 個不同的子集)。
\stopEXERCISE

\startANSWER
\TODO{略。}
\stopANSWER

%eB.1-6
\startEXERCISE
請通過擴展有序對的集合論定義來給出 \m{n} 元組的一個歸納定義。
\stopEXERCISE

\startANSWER
\TODO{略。}
\stopANSWER

\stopsection
