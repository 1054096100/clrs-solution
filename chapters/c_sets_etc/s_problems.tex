\startsubject[
  title={Problems},
]

%pB-1
\startPROBLEM
(Graph coloring)
給定一個無向圖 \m{G=(V,E)},
 \m{G} 的 {\EMP k 着色}是一個函數 \m{c:V\rightarrow\{0,1,\ldots,k-1\}},
滿足對於每條邊 \m{(u,v)\in E},有 \m{c(u)\ne c(v)}。
換句話說,數字 \m{0,1,\ldots,k-1} 代表了 \m{k} 種顏色,
鄰接的頂點不能具有相同顏色。

\startigBase[a]\startitem
證明:所有樹都可 2 着色。
\stopitem\stopigBase

\startANSWER
\TODO{略。}
\stopANSWER

\startigBase[continue]\startitem
證明下列三條描述是等價的:
 1) \m{G} 是二分圖;
 2) \m{G} 可 2 着色;
 3) \m{G} 沒有奇數長度的環。
\stopitem\stopigBase

\startANSWER
\TODO{略。}
\stopANSWER

\startigBase[continue]\startitem
令 \m{d} 爲圖 \m{G} 中任意頂點的最大度數。
證明: \m{G} 可以用 \m{d+1} 種顏色着色。
\stopitem\stopigBase

\startANSWER
\TODO{略。}
\stopANSWER

\startigBase[continue]\startitem
證明:如果 \m{G} 有 \m{O(|V|)} 條邊,
那麼 \m{G} 可以用 \m{O(\sqrt{|V|})} 種顏色着色。
\stopitem\stopigBase

\startANSWER
\TODO{略。}
\stopANSWER
\stopPROBLEM

%pB-2
\startPROBLEM
(Friendly graphs)
將下列描述改寫成關於無向圖的定理,並給出證明。
假定友誼是對稱的,但不是自反的。

\startigBase[a]\startitem
如果一組人不少於兩個,
那麼其中至少有兩人在組內有相同數目的朋友。
\stopitem\stopigBase

\startANSWER
\TODO{略。}
\stopANSWER

\startigBase[continue]\startitem
任何 6 人組要麼至少有三個人互爲朋友,
要麼至少有三人互不相識。
\stopitem\stopigBase

\startANSWER
\TODO{略。}
\stopANSWER

\startigBase[continue]\startitem
每組人都可以分成兩組,
其中每個人都至少有一半的朋友在另外一個子組中。
\stopitem\stopigBase

\startANSWER
\TODO{略。}
\stopANSWER

\startigBase[continue]\startitem
如果一組人中,每個人都至少是該組內一半人的朋友,
那麼該組可以按如下方式安排組員坐在圓桌周圍:
每個人左右兩側都是他的朋友。
\stopitem\stopigBase

\startANSWER
\TODO{略。}
\stopANSWER
\stopPROBLEM

%pB-3
\startPROBLEM
(Bisecting trees)
許多圖的分治算法要求圖被等分爲兩個大小基本相同的子圖。
這可以通過在劃分頂點後,求取兩個點集的導出子圖來實現。
本題將研究通過移除一小部分邊來將樹二等分的方法。
要求如果在移除邊後兩個頂點落在了同一棵子樹中,
則他們一定在同一個劃分內。

\startigBase[a]\startitem
證明:任意 \m{n} 個頂點二叉樹的頂點均可通過移除一條邊
被劃分爲兩個集合 \m{A} 和 \m{B},
滿足 \m{|A|\le 3n/4}, \m{|B|\le 3n/4}。
\stopitem\stopigBase

\startANSWER
\TODO{略。}
\stopANSWER

\startigBase[continue]\startitem
通過給出一個例子:
一棵簡單二叉樹在移除一條邊後,
其最均勻平衡的劃分滿足 \m{|A|= 3n/4},
來證明(a)中的常數 \m{3/4} 在最壞情況下是最優的。
\stopitem\stopigBase

\startANSWER
\TODO{略。}
\stopANSWER

\startigBase[continue]\startitem
證明:通過移除最多 \m{O(\lg n)} 條邊,
可以將任意 \m{n} 頂點二叉樹的頂點劃分爲兩個集合 \m{A} 和 \m{B},
滿足 \m{|A|=\left\lfloor n/2\right\rfloor} 和 \m{|B|=\left\lceil n/2\right\rceil}。
\stopitem\stopigBase

\startANSWER
\TODO{略。}
\stopANSWER
\stopPROBLEM

\stopsubject%Problems
