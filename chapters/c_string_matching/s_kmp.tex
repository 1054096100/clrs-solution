\startsection[
  title={The Knuth-Morris-Pratt algorithm},
]

%e32.4-1
\startEXERCISE
計算對應於模式 \m{ababbabbabbababbabb} 的前綴函數 \m{\pi}。
\stopEXERCISE

\startANSWER
\TODO{略。}
\stopANSWER

%e32.4-2
\startEXERCISE
給出關於 \m{q} 的函數 \m{\pi^*[q]} 的規模的上界。
舉例說明所給出的上界是嚴格的。
\stopEXERCISE

\startANSWER
\TODO{略。}
\stopANSWER

%e32.4-3
\startEXERCISE
試說明如何通過檢查字串 \m{PT} (
由 \m{P} 和 \m{T} 連接形成的長度爲 \m{m+n} 的字串
)的 \m{\pi} 函數來確定模式 \m{P} 在文本 \m{T} 中的出現位置。
\stopEXERCISE

\startANSWER
\TODO{略。}
\stopANSWER

%e32.4-4
\startEXERCISE
用聚合分析方法證明 \ALGO{KMP-MATCHER} 的運行時間是 \m{\Theta(n)}。
\stopEXERCISE

\startANSWER
\TODO{略。}
\stopANSWER

%e32.4-5
\startEXERCISE
用勢函數證明 \ALGO{KMP-MATCHER} 的運行時間是 \m{\Theta(n)}。
\stopEXERCISE

\startANSWER
\TODO{略。}
\stopANSWER

%e32.4-6
\startEXERCISE
試說明如何通過以下方式對過程 \ALGO{KMP-MATCHER} 進行改進:
把第 7 行(不是第 12 行)出現的 \m{\pi} 替換爲 \m{\pi'},
其中對於 \m{q=1,2,\ldots,m-1}, \m{\pi'} 遞歸定義如下:
\startformula
\pi'[q]=\startcases
\NC 0 \NC \text{如果 \m{\pi[q] = 0}} \NR
\NC \pi'[\pi[q]] \NC \text{如果 \m{\pi[q] \ne 0} 且 \m{P[\pi[q]+1] = P[q+1]}} \NR
\NC \pi[q] \NC \text{如果 \m{\pi[q] \ne 0} 且 \m{P[\pi[q]+1] \ne P[q+1]}} \NR
\stopcases
\stopformula
說明修改後的算法爲什麼是正確的,
並說明在何種意義上,這一修改是對原算法的改進。
\stopEXERCISE

\startANSWER
\TODO{略。}
\stopANSWER

%e32.4-7
\startEXERCISE
寫出一個線性時間的算法,以確定文本 \m{T} 是否是另一個字串 \m{T'} 的循環旋轉。
例如 \m{arc} 和 \m{car} 是彼此的循環旋轉。
\stopEXERCISE

\startANSWER
\TODO{略。}
\stopANSWER

%e32.4-8
\startEXERCISE\DIFFICULT
給出一個有效算法,
計算出相應於某給定模式 \m{P} 的字串匹配自動機的轉移函數 \m{\delta}。
所給出的算法運行時間應爲 \m{O(m|\sum|)}。
(\hint 證明如果 \m{q=m} 或 \m{P[q+1]\ne a},
則 \m{\delta(q,a)=\delta(\pi[q],a)}。)
\stopEXERCISE

\startANSWER
\TODO{略。}
\stopANSWER

\stopsection
