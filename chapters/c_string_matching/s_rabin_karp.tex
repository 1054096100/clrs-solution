\startsection[
  title={The Rabin-Karp algorithm},
]

%e32.2-1
\startEXERCISE
如果模 \m{q=11},
那麼當 Rabin-Karp 匹配算法在文本 \m{T=3 141 592 653 589 793} 中
搜尋模式 \m{P=26} 時,會遇到多少個僞命中點?
\stopEXERCISE

\startANSWER
\TODO{略。}
\stopANSWER

%e32.2-2
\startEXERCISE
如何擴展 Rabin-Karp 算法,
使其能解決如下問題:
如何在文本字串中搜尋出給定的 \m{k} 個模式中的任何一個出現?
起初假設所有 \m{k} 個模式都是等長的,
然後擴展你的算法以適用於不同長度的模式。
\stopEXERCISE

\startANSWER
\TODO{略。}
\stopANSWER

%e32.2-3
\startEXERCISE
試說明如何擴展 Rabin-Karp 算法用於處理一下問題:
在一個 \m{n\times n} 的二維字符數列中搜索一個給定的 \m{m\times m} 的模式。
(該模式可以在水平方向和垂直方向移動,
但是不可以旋轉。)
\stopEXERCISE

\startANSWER
\TODO{略。}
\stopANSWER

%e32.2-4
\startEXERCISE
Alice 有一份很長的 \m{n} 位文件複印件 \m{A=\langle a_{n-1},a_{n-2},\ldots,a_0\rangle},
 Bob 也有一份類似的文件 \m{B=\langle b_{n-1},b_{n-2},\ldots,b_0\rangle}。
 Alice 和 Bob 都希望直到他們的文件是否一樣。
爲了避免傳送整個文件 A 或 B,
他們運用下列快速的概率檢查方法。
他們共同選擇一個素數 \m{q>1000n},
並從 \m{\{0,1,\ldots,q-1\}} 中隨機選取一個整數 \m{x}。
然後, Alice 求出
\startformula
A(x)=\left(\sum_{i=0}^{n-1} a_i x^i \right)\mod q
\stopformula
的值, Bob 也用類似方法計算出 \m{B(x)}。
證明:如果 \m{A\ne B},
則 \m{A(x)=B(x)} 的概率至多爲 \m{1/1000};
如果兩個文件相同,則 \m{A(x)} 的值必定等於 \m{B(x)} 的值。
(\hint 參見\refexercise{model_zero}。)
\stopEXERCISE

\startANSWER
\TODO{略。}
\stopANSWER

\stopsection
