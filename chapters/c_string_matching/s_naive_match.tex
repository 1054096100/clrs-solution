\startsection[
  title={The naive string-matching algorithm},
]

%e32.1-1
\startEXERCISE
試說明當模式 \m{P=0001},文本 \m{T=000010001010001} 時,
樸素字串匹配所執行的比較。
\stopEXERCISE

\startANSWER
\TODO{略。}
\stopANSWER

%e32.1-2
\startEXERCISE
假設在模式 \m{P} 中所有字符都不相同。
試說明如何對一段 \m{n} 個字符的文本 \m{T} 加速
 \ALGO{NAIVE-STRING-MATCHER} 的執行速度,
使其運行時間達到 \m{O(n)}。
\stopEXERCISE

\startANSWER
從第一個不匹配的字符開始下一次搜索。
\stopANSWER

%e32.1-3
\startEXERCISE
假設模式 \m{P} 和文本 \m{T} 是長度分別爲 \m{m} 和 \m{n} 的隨機選取的字串。
其字符分別來自含有 \m{d} 個元素的字母表:
\startformula
\sum_d = \{0,1,\ldots,d-1\}
\stopformula
,其中 \m{d\ge 2}。
證明樸素算法第 4 行中隱含的循環所執行的字符比較的{\EMP 期望}次數爲:
\startformula
(n-m+1)\frac{1-d^{-m}}{1-d^{-1}} \le 2 (n-m+1)
\stopformula
直到這次循環結束。
假設對於一個給定的偏移,
當有一個字符不匹配或者整個模式已被匹配時,
樸素算法將終止字符比較。
因此,對任意隨機選取的字串,樸素算法都是有效的。
\stopEXERCISE

\startANSWER
\TODO{略。}
\stopANSWER

%e32.1-4
\startEXERCISE[exercise:32.1-4]
假設允許模式 \m{P} 中包含一個{\EMP 間隔符} \m{\diamondsuit},
他可以和{\EMP 任意}字串匹配(甚至可以和長度爲 0 的字串匹配)。
例如,模式 \m{ab\diamondsuit ba \diamondsuit c} 與文本 \m{cabccbacbacab} 的匹配:
\startformula\startmathalignment[n=7,
  align={middle,middle,middle,middle,middle,middle,middle}]
\NC c \NC \underbrace{ab} \NC \underbrace{cc} \NC \underbrace{ba}
  \NC \underbrace{cba} \NC \underbrace{c} \NC ab \NR
\NC   \NC ab \NC \diamondsuit \NC ba \NC \diamondsuit \NC c \NR
\stopmathalignment\stopformula
和
\startformula\startmathalignment[n=7,
  align={middle,middle,middle,middle,middle,middle,middle}]
\NC c \NC \underbrace{ab} \NC \underbrace{ccbac} \NC \underbrace{ba}
  \NC \ \underbrace{\ } \  \NC \underbrace{c} \NC ab \NR
\NC   \NC ab \NC \diamondsuit \NC ba \NC \diamondsuit \NC c \NR
\stopmathalignment\stopformula
注意,間隔符可以在模式中出現任意次,
但是不能在文本中出現。
給出一個多項式時間算法,
以確定這樣的模式 \m{P} 是否在給定的文本 \m{T} 中出現,
並分析算法的運行時間。
\stopEXERCISE

\startANSWER
\TODO{略。}
\stopANSWER

\stopsection
