% NOTE: if you want to use specific font:
% \definefontsynonym [MartinVogel] [marvosym.ttf]
% \definesymbol [Sun]                [\getnamedglyphdirect{MartinVogel}{Sun}]
\definesymbol [1] [*]
\definesymbol [2] [-]

\defineenumeration[example][
  text={例},
  headstyle=\rm\bf,	%normal bold slanted boldslanted type cap small... command
%  headcolor=blue:7,
%style	normal bold slanted boldslanted type cap small... command
%color	name
%width	fit broad dimension
%distance	dimension
%sample	text
%text	text
%align	flushleft middle flushright
%margin	standard yes no dimension
  alternative=top,	%left right top serried inmargin inleft inright hanging
%headcommand	command
%hang	fit broad number
%before	command
%inbetween	command
%after	command
%indentnext	yes no
%indenting	never not no yes always first next
  prefix=yes,
  prefixsegments=chapter, %chapter:section
]

\defineenumeration[QUESTION][
  text={問題},
  %right={:},
  headstyle=\rm\bf,	%normal bold slanted boldslanted type cap small... command
  headcolor=red,
  %style=\rm\bf,		%normal bold slanted boldslanted type cap small... command
  color=red,	%name
  width=fit,	%fit broad dimension
  distance=0em,		%dimension
%sample	text
  %align=flushleft,	%flushleft middle flushright
  %margin=no,	%standard yes no dimension
  alternative=serried,	%left right top serried inmargin inleft inright hanging
  %hang=2,		%fit broad number
%headcommand	command
%before	command
%inbetween={\startEnglish},	%command
%after={\stopEnglish},	%command
%indentnext	yes no
%indenting	never not no yes always first next
  prefix=yes,
  prefixsegments=chapter:section, %chapter:section
  prefixconnector={-},
  numberstopper={:},
  way=bysection,
]

\definestartstop[ANSWER][%
  before={\startcolor[black]},
  after={\stopcolor},
]

% no number
\defineitemgroup[igBase][levels=2]
\setupitemgroup[igBase]
[1]
[1, packed,joinedup,intext]
%standard broad serried packed unpacked stopper joinedup atmargin inmargin autointro loose repeat section paragraph intext random columns
%standard: default setup
%n*broad:  extra horizontal white space after symbol
%n*serried:little horizontal white space after symbol
%packed:   no whitespace between items
%stopper:  punctuation after item separator
%joinedup: no white space before and after itemization
%atmargin: item separator at the margin
%inmargin: item separator in margin
[
%margin=2em,	%no standard dimension
  leftmargin=2em,	%no standard dimension
%rightmargin	no standard dimension
  %width=2em,	%dimension
%distance	dimension
%factor	number
%items	number
%start	number
  before=,	%command
  inbetween=,	%command
  after={\blank[.5ex]},	%command
%left	text
%right	text
%beforehead	command
%afterhead	command
%headstyle=boldslanted, %normal bold slanted boldslanted type cap small... command
%marstyle=boldslanted, %normal bold slanted boldslanted type cap small... command
%symstyle=boldslanted, %normal bold slanted boldslanted type cap small... command
%stopper	text
%n	number
%  symbol=1,	%number
%align	left right normal
  indenting=next,	%never not no yes always first next
  indentnext=auto,	%yes no
]
\setupitemgroup[igBase][2]
[2, packed,joinedup,intext][
  leftmargin=4em,	%no standard dimension
  %width=2em,	%dimension
  before=,	%command
  inbetween=,	%command
  after=,	%command
%  symbol=2,
  indenting=next,
  indentnext=auto,
]

\setupitemgroup[igBase][each][][leftmargin=2em]
%m	A numbered list, with lowercase (“medieval”, aka “oldstyle”) numbers.
%1 … 8	Different kinds of bullets. All items get the same symbol.
%a	Items are numbered a., b., c., …
%A	Items are numbered A., B., C., …
%AK	Items are numbered A., B., C., …, in small caps.
%r	Items are numbered in lowercase Roman numerals.
%R	Items are numbered in uppercase Roman numerals.
%KR	Items are numbered in uppercase Roman numerals, small caps style.

% with number
\defineitemgroup[igNum][levels=2]
\setupitemgroup[igNum]
[1]
[a, packed,joinedup,intext]
[
  leftmargin=2em,	%no standard dimension
  %width=2em,	%dimension
  before=,	%command
  inbetween=,	%command
  after={\blank[.5ex]},	%command
  indenting=next,
  indentnext=auto,
]
\setupitemgroup[igNum]
[2]
[a, packed,joinedup,intext]
[
  leftmargin=4em,	%no standard dimension
  %width=2em,	%dimension
  before=,	%command
  inbetween=,	%command
  after={\blank[.5ex]},	%command
  indenting=next,
  indentnext=auto,
]
