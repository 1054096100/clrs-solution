%两个quad空格	a \qquad b
%quad空格	a \quad b
%大空格		a\ b
%中等空格	a\;b
%小空格		a\,b
%没有空格	ab
%紧贴		a\!b

% page number
\definestructureconversionset[frontpart:pagenumber][][romannumerals]
\definestructureconversionset [bodypart:pagenumber][][numbers]
\definestructureconversionset [appendix:pagenumber][][numbers]
\definestructureconversionset [backpart:pagenumber][][numbers]
\setuppagenumber[numberconverionset=pagenumber]

% 用來控制章節編號的復位 part、chapter獨立編號,section則每章重新編號
\definestructureresetset[frontpart:default][0,0,1][0] % don't reset part, chapter, but section
\definestructureresetset [bodypart:default][0,0,1][0] % don't reset part, chapter, but section
\definestructureresetset [appendix:default][0,0,1][0] % don't reset part, chapter, but section
\definestructureresetset [backpart:default][0,0,1][0] % don't reset part, chapter, but section
\setuphead[sectionresetset=default]

% 用來控制編號格式
\definestructureconversionset[frontpart:sectionnumber][][]
\definestructureconversionset [bodypart:sectionnumber][chinesecapnumerals, numbers, numbers][numbers]
\definestructureconversionset [appendix:sectionnumber][chinesecapnumerals, Characters, numbers][numbers]
\definestructureconversionset [backpart:sectionnumber][][]
\setuphead[sectionconversionset=sectionnumber]

% 用來控制章節編號間的串聯
% can't remove partnumber
%\definestructureprefixset [myset] [section-1,section-2,section-3,section-4,section-5,section-6][]
%\setuphead[sectionset=myset]
\setuphead[part][sectionsegments=1]
\setuphead[chapter][sectionsegments=2:*]
\setuphead[section][sectionsegments=2:*]
\setuphead[subsection][sectionsegments=2:*]
\setuphead[subsubsection][sectionsegments=2:*]
\setuphead[subsubsubsection][sectionsegments=2:*]

% 用來控制編號間隔符
\defineseparatorset[frontpart:sepset][!,.,.][.]
\defineseparatorset [bodypart:sepset][!,.,.][.]
\defineseparatorset [appendix:sepset][!,-,.][.]
\defineseparatorset [backpart:sepset][!,.,.][.]
\setuphead[sectionseparatorset=sepset]

% 用來控制label
\setuplabeltext[cn][frontpart={第\;,\;卷},bodypart={第\;,\;卷},appendpart={第\;,\;卷},backpart={第\;,\;卷}]
\setuplabeltext[cn][frontchapter={第\;,\;章},bodychapter={第\;,\;章},appendchapter={附錄\;,},backchapter={第\;,\;章}]
\setuplabeltext[cn][frontsection={節\;,},bodysection={節\;,},appendsection={,},backsection={節\;,}]
\setuphead[part][
  frontpartlabel=frontpart,
  bodypartlabel=bodypart,
  appendixlabel=appendpart,
  backpartlabel=backpart,
]
\setuphead[chapter][
  frontpartlabel=frontchapter,
  bodypartlabel=bodychapter,
  appendixlabel=appendchapter,
  backpartlabel=backchapter,
]
\setuphead[section][
  frontpartlabel=frontsection,
  bodypartlabel=bodysection,
  appendixlabel=appendsection,
  backpartlabel=backsection,
]

\setuphead[part][
  textstyle={\rm\bfd}, %normal bold slanted boldslanted % 標題內容
  numberstyle={\rm\bfd}, %normal bold slanted boldslanted % 編號
%  numbercommand=\PartNbr,
  header=high, %none empty high nomarking %章節首頁無頁眉
  footer=high, %none empty high nomarking
  before={\blank[2em]}, %COMMAND
  after={\blank[1em]}, %COMMAND
  alternative=middle, %normal inmargin middle TEXT
  placehead=yes,
]

\setuphead[chapter][
  indentnext=yes,
  textstyle={\rm\bfc}, %normal bold slanted boldslanted % 標題內容
  numberstyle={\rm\bfc}, %normal bold slanted boldslanted % 編號
  number=yes, %yes no % 是否帶編號
  ownnumber=no, %yes no % 是否指定編號,如果是,則第一個參數就是編號
  page=yes, %left right yes % 是否固定於左頁或右頁
%  continue=no, %yes no % 第一個是否緊接上一層,優先於*page*
  header=high, %none empty high nomarking %章節首頁無頁眉
%  text=nomarking, %none empty high nomarking
  footer=high, %none empty high nomarking
  before={\blank[2em]}, %COMMAND
%  inbetween=, %COMMAND
  after={\blank[1em]}, %COMMAND
  alternative=middle, %normal inmargin middle TEXT
%  sectionnumber=no,
]

\setuphead[section][
  indentnext=yes,
  textstyle={\rm\bfb}, %normal bold slanted boldslanted % 標題內容
  numberstyle={\rm\bfb}, %normal bold slanted boldslanted % 編號
  number=yes, %yes no % 是否帶編號
  ownnumber=no, %yes no % 是否指定編號,如果是,則第一個參數就是編號
  before={\blank[1.5em]}, %COMMAND
  after={\blank[1em]}, %COMMAND
  alternative=inmargin, %normal inmargin middle TEXT
]

\setuphead[subsection][
  indentnext=yes,
  textstyle={\rm\bfa}, %normal bold slanted boldslanted % 標題內容
  numberstyle={\rm\bfa}, %normal bold slanted boldslanted % 編號
  number=yes, %yes no % 是否帶編號
  ownnumber=no, %yes no % 是否指定編號,如果是,則第一個參數就是編號
  before={\blank[1em]}, %COMMAND
  after={\blank[1em]}, %COMMAND
  alternative=inmargin, %normal inmargin middle TEXT
]

\setuphead[subsubsection][
  indentnext=yes,
  textstyle={\rm\ita}, %normal bold slanted boldslanted % 標題內容
  numberstyle={\rm\ita}, %normal bold slanted boldslanted % 編號
  number=yes, %yes no % 是否帶編號
  ownnumber=no, %yes no % 是否指定編號,如果是,則第一個參數就是編號
  before={\blank[.5em]}, %COMMAND
  after={\blank[.5em]}, %COMMAND
  alternative=inmargin, %normal inmargin middle TEXT
]

\setuplist[
  alternative=c,	%a 空格
			%b 右对齐
			%c ...page
			%d 大空格 紧接下一个list
			%e 编号加方框,标题向左缩
			%f
			%g 标题居中... none command
  %label=yes,		% bug: 如果設置此項,則編譯參考文獻時會失敗
  interaction=all,
%  aligntitle=yes,
%  align=flushleft,
]
\setuplist[chapter][
  width=4em,	% setuplist中的設置無效,可能是bug
  margin=0em,	%dimension % 左侧缩进
  label=yes,
]

\setuplist[section][
  width=4em,
  margin=2em,	%dimension % 左侧缩进
  label=yes,
]

\setuplist[figure][
  width=4em,
  label=figure, % label=yes 無效,這可能是bug
  margin=0em,	%dimension % 左侧缩进
]

\setuplist[table][
  width=4em,
  label=table, % label=yes 無效,這可能是bug
  margin=0em,	%dimension % 左侧缩进
]

\startsectionblockenvironment[frontpart]
\resetpagenumber

\setupfootertexts[text]
[]
[pagenumber]
[pagenumber]
[]

\setuphead[title][
  indentnext=yes,
  textstyle={\rm\bfb}, %normal bold slanted boldslanted % 標題內容
  page=yes, %left right yes % 是否固定於左頁或右頁
  continue=no, %yes no % 第一個是否緊接上一層,優先於*page*
  header=high, %none empty high nomarking %章節首頁無頁眉
%  text=nomarking, %none empty high nomarking
  footer=normal, %none empty high nomarking
  before={\blank[2em]}, %COMMAND
%  inbetween=, %COMMAND
  after={\blank[1em]}, %COMMAND
  alternative=middle, %normal inmargin middle TEXT
  sectionnumber=no,
]

\stopsectionblockenvironment

\startsectionblockenvironment[bodypart]
\resetpagenumber
%\setupheader[state=none]
\define\BodyTextRightHeader{%
\framedtext[frame=off,bottomframe=on,width=broad,offset=none,frameoffset=2pt,]{%
\rlap{} \hfill {第\;\headnumber[chapter]\;章\;\;\getmarking[chapter]}\hfill \llap{}%
}%
}
% 由於可能某頁中沒有任何 section,所以不打印節編號,只打印節名。 no workaround
\define\BodyTextLeftHeader{%
\framedtext[frame=off,bottomframe=on,width=broad,offset=none,frameoffset=2pt,]{%
\rlap{} \hfill {\getmarking[section][first]}\hfill \llap{}%
}%
}

\setupheader[text][before={\vskip-4pt},]

\setupheadertexts[text]
[]
[\BodyTextRightHeader]
[]
[\BodyTextLeftHeader]

\setupfootertexts[text]
[]
[\centerline{第\;\pagenumber\;頁}]
[]
[\centerline{第\;\pagenumber\;頁}]

\define\BodyEdgeLeftFooter{\rotate[%
  rotation=-90,
  width=fit,
  height=fit,
  frame=off,
  offset=2pt,
  background=color,
  backgroundcolor=darkgray,
  foregroundcolor=white,
  corner=round,
]{\prdname —— \from[authorEmail]}}
\define\BodyEdgeRightFooter{\rotate[%
  rotation=90,
  width=fit,
  height=fit,
  frame=off,
  offset=2pt,
  background=color,
  backgroundcolor=darkgray,
  foregroundcolor=white,
  corner=round,
]{\prdname —— \from[authorBlog]}}

\setupfootertexts[edge]
[]
[\BodyEdgeLeftFooter]
[\BodyEdgeRightFooter]
[]
\stopsectionblockenvironment

\startsectionblockenvironment[appendix]
\setupfootertexts[text]
[]
[pagenumber]
[pagenumber]
[]
\stopsectionblockenvironment

\startsectionblockenvironment[backpart]
\resetpagenumber
\stopsectionblockenvironment

