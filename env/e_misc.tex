% chinese date
\startluacode
local function tochineseYear(n)
	local cap = {
		["0"] = "零",
		["1"] = "壹",
		["2"] = "贰",
		["3"] = "叁",
		["4"] = "肆",
		["5"] = "伍",
		["6"] = "陆",
		["7"] = "柒",
		["8"] = "捌",
		["9"] = "玖",
	}

	s, p = string.gsub(string.format("%d",n), "(.)", function(s) return cap[s] end)
	return s
end

function commands.chinese_date() -- wrong namespace
	local temp = os.date("*t")
	local day = temp.day

	tex.sprint(tex.ctxcatcodes,tochineseYear(temp.year))
	tex.sprint(tex.ctxcatcodes,"年")
	tex.sprint(tex.ctxcatcodes, converters.chinesecapnumerals(temp.month))
	tex.sprint(tex.ctxcatcodes,"月")

	if day % 10 == 0 then
		converters.chinesecapnumerals(day)
	else
		if day > 30 then
			tex.sprint(tex.ctxcatcodes,"卅")
		elseif day > 20 then
			tex.sprint(tex.ctxcatcodes,"廿")
		elseif day > 10 then
			tex.sprint(tex.ctxcatcodes,"拾")
		end
		day = day % 10
		if day > 0 then
			tex.sprint(tex.ctxcatcodes, converters.chinesecapnumerals(day))
		end
	end
	tex.sprint(tex.ctxcatcodes,"日")
end
\stopluacode

\define\COMPILEDATE{%
\ctxlua{commands.chinese_date()}%
}
%\setuplanguage[cn][date={year,年,month,月,day,日}]
